\chapter{Tame path algebras are green sequence-finite}\label{C2}
\section{Introduction}
 In Br\"ustle-Dupont-P\'erotin \cite{BDP13} and the paper by Br\"ustle, Hermes, Igusa and Todorov \cite{BHIT15} it is proven that there are finitely maximal green sequences when the quiver is of finite, tame type or the quiver is mutation equivalent to a quiver of finite or tame type. Furthermore in \cite{BHIT15} it is proven that any tame quiver has finitely many $k$-reddening sequences.\\
\indent When we restrict our attention to the case where the algebra is basic, connected and hereditary it is a path algebra of a quiver \cite{ASS06}. In this chapter when we say an $m$-maximal green sequence of an algebra we mean an $m$-maximal green sequence of its path algebra. Here is the main theorem we have proven.\\
\begin{theorem}
Any tame quiver has finitely many $m$-maximal green sequences.\label{T:C3T}
\end{theorem}
\indent To prove this theorem we only need to prove that only finitely many indecomposable objects can appear as summands of silting objects that can appear in $m$-maximal green sequences of tame quivers $Q$. Since all indecomposable objects of a basic tame path algebra have to be transjective or regular, only finitely many rigid regular objects between $\Lambda$ and $\Lambda[m]$ in $D_b(\Lambda)$, namely the modules on the nonhomogeneous tubes $\mathbb{Z}A_\infty/\langle\tau^k\rangle$ with no repeating composition factors and their shifts. Hence the problem is reduced to proving that only finitely many indecomposable transjective objects between $\Lambda$ and $\Lambda[m]$ can appear in $m$-maximal green sequences.\\
\indent To prove this theorem we need two lemmas.\\
\begin{lemma}\label{def:C3L1}
For a tame quiver $Q$ any silting object in $D^b(kQ)$ contains at most $n-2$ regular summands. In other words, at least 2 summands have to be transjective. 
\end{lemma}
\begin{lemma}\label{def:C3L2}
For a tame quiver $Q$ there is a uniform bound, depending only on $Q$ and $m$, on the transjective degree of any transjective summand in any silting object in any $m$-maximal green sequence $D^b(kQ)$.
\end{lemma}
\indent It is easy to see why Lemma \ref{def:C3L2} implies the theorem. Here the \textit{transjective degree} of an indecomposable transjective object $\tau^iP_j[k]$ is defined as $deg(\tau^iP_j[k])=i$. The \textit{maximal transjective degree} and \textit{minimal transjective degree} of a silting object are defined as the highest/lowest transjective degree of its indecomposable transjective summands respectively.\\
\indent In Section 2 we prove Lemma \ref{def:C3L1}. In Section 3 we prove Lemma \ref{def:C3L2}. In Section 4 we further generalize the theorem to arbitrary finite mutation sequences with finitely many forward/green or backward/red mutations.\\
\section{Proof of Lemma \ref{def:C3L1}}
\indent To prove Lemma \ref{def:C3L1} we need to understand regular components of Auslander-Reiten quivers of $D^b(kQ)$ for tame quivers. Regular components of Auslander-Reiten quivers of tame path algebras are all standard stable tubes with at most three tubes nonhomogeneous (see \cite{DR76} and Chapter X of \cite{SS06}). Note that no object on a homogeneous tube is rigid so no object there can appear in a silting object of $D^b(kQ)$. Hence we only need to discuss the nonhomogeneous tubes.\\
\indent It is easy to see that in an indecomposable object in a standard stable tube $\mathcal{T}$ of size $n$, $M$ and any of its shifts can not be in the same pre-silting object.\\ 
\begin{definition}
If $\{M_i\}_{i\in I}$ are a family of indecomposable objects of $D^b(kQ)$ and $\Pi_{i\in I}M_i[n_i]$ is not pre-silting for any $\{n_i\}_{i\in I}$ We say that $\{M_i\}_{i\in I}$ is \textit{silting-incompatible}. Otherwise we say that it is \textit{silting-compatible}.
\end{definition}
\indent From now on in this proof we identify $[n]$ with $\mathbb{Z}/n\mathbb{Z}$ and hence will no longer differentiate between $0$ and $n$ which we usually denote as $n$. It is also clear that it makes sense to define $[a_1, a_2,\cdots, a_k]$ on $[n]$ in the sense of cyclic orders. Ex: $[1,2,3]$ and $[2,3,4]$ hold in $[4]$.\\
\begin{definition}
Let $M_i$ be the quasi-simples of the tube such that $\tau M_i=M_{i-1}$. a regular module in $D^b(kQ)$ \textit{regular sincere} if its composition series contain all quasi-simples.
\end{definition}
\indent No indecomposable regular sincere modules or their shifts can appear as summands in any silting object because they are not rigid. (See Corollary X.2.7 of \cite{SS06}).As for the remaining $n(n-1)$ indecomposable regular modules that are actually rigid we can unambiguously label them as $M_{ij}$ if the quasi-top and quasi-socle of the object are $M_j$ and $M_i$ respectively. Note that $M_i=M_{ii}$. It is clear that $\tau M_{ij}=M_{i-1,j-1}$ and $\tau^{-1} M_{ij}=M_{i+1,j+1}$.\\
\indent Now let's prove two easy lemmas on what can not appear in a pre-silting object in a regular component of the Auslander-Reiten quiver of $D^b(kQ)$.\\
\begin{lemma}
\begin{enumerate}
\item If $M$ and $N$ are regular modules in a nonhomogeneous tube in the Auslander-Reiten quiver of $kQ$. If $Hom(M,N)\neq 0$ and $Ext^1(N,M)\neq 0$, then $M$ and $N$ are silting-incompatible.\\
\item If $M_1,\cdots, M_k$ are regular modules in a nonhomogeneous tube in the Auslander-Reiten quiver of $kQ$. If $Ext^1(M_i,M_{i+1})\neq 0$ for any $1\leq i<k$  and $Ext^1(M_k,M_1)\neq 0$, then $\{M_i\}$  is silting-incompatible.\\
\end{enumerate}
\end{lemma}
\begin{proof}
\indent For (1) since $Hom(M,N)\neq 0$ if $i>j$ we have $Ext^{i-j}(M[i],N[j])\neq 0$ . Since $Ext^1(N,M)\neq 0$ if $i\leq j$ it is true that $Ext^{j-i+1}(N[j],M[i])\neq 0$ . Hence $M[i]\oplus N[j]$ is not pre-silting for any arbitrary $i$ and $j$.\\
\indent For (2) for arbitrary $n_1,\cdots n_k$ use the argument above it is easy to see that if $\oplus_{i=1}^kM_i[n_i]$ is pre-silting, then $n_2>n_1$, $n_3>n_2$, $\cdots, n_1>n_k$ which is impossible. Hence $\{M_i\}$  is silting-incompatible.
\end{proof}
\begin{lemma}\label{def:C3L3}
Any pre-silting object in a standard stable tube of size $n$ contains at most $n-1$ summands.
\end{lemma}
\indent To prove this lemma we need the following lemma.
\begin{lemma}\label{lem:C3L4}
Any pre-silting object in a standard stable tube of size $n$ can not be regular sincere.
\end{lemma}
\begin{proof}
\indent Assume that a pre-silting object $T=\oplus_{i=1}^k T_i$ in a standard stable tube of size $n$ is regular sincere. Let $T_{l_1},\cdots, T_{l_m}$ be a minimal set of indecomposable summands of $T$ such that their direct sum $T'=\oplus_{i=1}^m T_{l_i}$ is regular sincere. Note that if $M_{ij}$ and $M_{kl}$ are both summands of $T$, $[i,k,l,j]$ holds $M_{kl}$ and $M_{ij}$ can not both be summands of $T'$ due to minimality. If $m=1$ then $T'$ is a regular sincere indecomposable regular object which contradicts the fact that $T'$ is pre-silting. If $m>1$ without loss of generality assume that $T_{l_1}=M_{1p}$ for some $p\neq n$. Any indecomposable object with its quasi-socle $M_i, 1\leq i\leq p$ can not be a summand of $T'$ either due to silting incompatibility or minimality. Hence there has to be a summand of $T'$ with its quasi-socle $p+1$. Repeat this procedure it's easy to see that $T'=\oplus_{i=1}^m M_{(t_{i-1}+1)t_i}$ with $t_0=t_m=n$. In this case by Lemma \ref{lem:C3L4} the object can not be pre-silting.
\end{proof}
\indent Now we can prove Lemma \ref{def:C3L3}.
\begin{proof}
\indent Since any pre-silting object in a standard stable tube of size $n$ can not be regular sincere, without loss of generality it is a pre-silting object in the exact subcategory of $\mathcal{T}$ closed under extensions such that $M_1,\cdots M_{n-1}$ are the only simple objects. It is easy to see using the condition that the tube is standard stable which is a result of Theorem \ref{Tame}(5) that the category $add(\{M_{ij}\}_{1\leq i < j\leq n-1})$ is isomorphic to the module category of $kA_{n-1}$ with straight orientation and as a result any pre-silting object with all indecomposable summands in it or its shifts has at most $n-1$ summands. 
\end{proof}
\indent Finally we can prove Lemma \ref{def:C3L1}.
\begin{proof}
\indent Due to Lemma \ref{def:C3L3} and \cite{DR76} there are at most $n-2$ regular components in $D^b(kQ)$ when $Q$ is a tame quiver. This is true for each type so this is true for all tame quivers.
\end{proof}
\section{Proof of Lemma \ref{def:C3L2}}
\indent To prove \ref{def:C3L2} we need to rephrase an argument in \cite{BDP13} using degrees.
\begin{lemma}
(\cite{BDP13}, Lemma 10.1) Let $H$ be a representation-infinite connected hereditary algebra. Then there exists $N\geq 0$ such that for any $k\geq N$, for any projective $H$-module $P$, the $H$-modules $\tau^{-k}P$ and $\tau^kP[1]$ are sincere.
\end{lemma}
\begin{lemma}
(\cite{BDP13}) Let $Q$ be a tame quiver and $M_1,M_2$ two transjective modules of $kQ$. If $\{M_1,M_2\}$ is silting-compatible, then $|deg(M_1)-deg(M_2)|\leq N$ 
\end{lemma}
\begin{proof}
If $k-l>N$ we need to prove that $\tau^kP_a$ and $\tau^lP_b$ are silting-incompatible. If $i\leq j$ $Ext^{j-i+1}(\tau^lP_b[j],\tau^kP_a[i])=Ext^1(\tau^lP_b,\tau^kP_a)=Hom(\tau^{k-1}P_a,\tau^l P_b)=Hom(P_a,\tau^{l-k+1}P_b)\neq 0$ since $\tau^{l-k+1}P_b$ is a sincere preprojective module. If $i>j$ $Ext^{i-j}(\tau^kP_a[i],\tau^lP_b[j])=Ext^1(\tau^kP_a[1],\tau^lP_b)=Hom(\tau^{l-1}P_a,\tau^k P_b[1])=Hom(P_a,\tau^{k-l+1}P_b[1])\neq 0$ since $\tau^{k-l+1}P_b[1]$ is a sincere preinjective module. Hence $\tau^kP_a$ and $\tau^lP_b$ are silting-incompatible. Exchange the objects if $k-l<-N$. Hence the lemma has been proven.
\end{proof}
\indent Now we can prove Lemma \ref{def:C3L2} following a modified version of the argument in \cite{BDP13}.
\begin{proof}
\indent We only need to prove that there is a lower bound of minimal transjective degrees of silting objects that can appear in $m$-maximal green sequences. Assume that $\tau_kP_i[j]$ is in a silting object in an $m$-maximal green sequence of $kQ$. Note that due to Lemma \ref{def:C3L1} there are at least 2 transjective components in any silting object in $D^b(kQ)$. Note that each mutation on a transjective object $T$ in $\mathcal{P}_i$ can result in a transjective object in $\mathcal{P}_{i+1}$, a transjective object in $\mathcal{P}_i$ with degree less than or equal to $deg(T)$ or a regular object in $\mathcal{R}_i$. Each mutation on a regular object $T'$ in $\mathcal{R}_i$ can result in an object of $\mathcal{R}_i$, an object of $\mathcal{P}_{i+1}$ or an object of $\mathcal{R}_{i+1}$. Let $L$ be the minimal transjective degree of a silting object. No green mutation within a component or green mutation from a regular component to another one can increase $L$. All other green mutations may increase $L$ by at most $N$. However there are only $n$ summands of a silting object, $m+1$ transjective components and $m$ regular components so the amount of mutations that can increase $L$ is finite. To reach $\Lambda[m]$ which is of degree 0 $L$ has to be at least $-2mnN$. As a result no indecomposable transjective object in any silting object in an $m$-maximal green sequence can have a degree less than $-2mnN$. Similarly silting objects in $m$-maximal green sequences can not have maximal transjective degree higher than $2mnN$ or it can not start from $\Lambda$.
\end{proof}
\section{Almost morphism finiteness}
\indent Using the same method we can prove a stronger result.
\begin{theorem}\label{C3T2}
If $Q$ is a Dynkin or tame quiver and $T_1$, $T_2$ are silting objects of $D^b(kQ)$ then there are finitely many $k$-red and finitely many $k$-green mutation sequences from $T_1$ to $T_2$ for any $k$.
\end{theorem}
\indent Note that we only need to prove that part of the statement about $k$-red sequences. To prove the theorem we first need to prove the following lemma which is a generalization of Lemma 4.4.2 in \cite{BHIT15}.
\begin{lemma}\label{C3L4}
\begin{enumerate}
\item Any $k$-red sequence from $T_1$ to $T_2$ can go through any silting object at most $r+1$ times.
\item Any $k$-green sequence from $T_1$ to $T_2$ can go through any silting object at most $r+1$ times.
\end{enumerate}
\end{lemma}
\begin{proof}
We only need to prove (1). It is clear from the definition of mutations that a green sequence can go through any silting object at most once. (See \cite{BY13} and \cite{KY12} for more details.) Let's define a \textit{green arm} of a mutation sequence as a maximal subsequence of the mutation sequence that is green. Similarly we can define what is a \textit{red arm}. Assume that an $k$-red sequence $\{T_i\}$ has $n_r$ red arms and $n_g$ green arms. $n_r\leq k$. $n_g\leq n_r+1$. Let $n_1$ be the number of red arms of length 1 and $n_2$ the number of red arms of length at least 2. It's clear that $n_r=n_1+n_2$ and $n_1+2n_2\leq k$. Note that any silting object on a red arm of length 1 is on a green arm. Hence $\{T_i\}$ can go through any silting object at most $n_g+n_2\leq n_1+2n_2+1\leq k+1$ times.
\end{proof}
\indent It is easy to see that the bounds established in the lemma are optimal. Now we can prove the theorem. Note that the lemma above implies that in the Euclidean case if we can prove that for any $k$ if there are finitely many rigid objects that ca$k$-redn appear as summands of silting objects in $k$-red sequences Theorem \ref{C3T2} will been proven. 
\begin{proof}
\indent As we said above we will only prove the part about $k$-red sequences. Assume that all indecomposable summands of $T_1$ and $T_2$ are between $\Lambda[i]$ and $\Lambda[j]$. Since there are only $k$ red mutations, all indecomposable summands that appear in $k$-red sequences from $T_1$ to $T_2$ have to be between $\Lambda[i-k]$ and $\Lambda[j+k]$.\\ 
\indent If $Q$ is Dynkin there are only finitely many indecomposable objects between $\Lambda[i-k]$ and $\Lambda[j+k]$ and hence only finitely many silting objects can exist on an $k$-red sequence. Due to Lemma \ref{C3L4} there are finitely many $k$-red sequences. \\
\indent From now on we assume that $Q$ is Euclidean. There are only finitely many regular rigid indecomposable objects between $\Lambda[i-k]$ and $\Lambda[j+k]$ so the problem has been reduced to proving that only finitely many transjective indecomposable components can appear in silting objects in $k$-red sequences.\\
\indent Let the minimal degree of $T_2$ be $L$. Note that a red mutation can increase the minimal degree of a silting object by at most $N$. Use an argument similar to that one used to prove Theorem \ref{C3T} we can prove that no indecomposable transjective object with degree less than $L-2nN(2k+j-i)-kN$ can appear in any $k$-red sequences from $T_1$ to $T_2$. Similarly let the maximal degree of $T_1$ be $U$. No indecomposable transjective object with degree less than $U+2nN(2k+j-i)+kN$ can appear in any $k$-red sequence from $T_1$ to $T_2$. Hence there are only finitely many indecomposable transjective objects can appear in any $k$-red sequence from $T_1$ to $T_2$ and the theorem is proven.
\end{proof}
\indent Note that the bounds of transjective degrees in the proofs of Theorem \ref{T:C3T} and Theorem \ref{C3T2} above are very crude. In the future we will try to find better bounds.\\
\indent Finally let's define a new term to characterize finite dimensional algebras that satisfy the conditions of Theorem \ref{C3T2}.\\
\begin{definition}
\begin{enumerate}
\item A finite dimensional algebra $\Lambda$ of finite global dimension such that it has finitely many $k$-red sequences from any silting object $T_1$ to any silting object $T_2$ for any $k$ is \textit{almost morphism finite}.
\item A finite dimensional algebra $\Lambda$ of finite global dimension such that it has finitely many green sequences from any silting object $T_1$ to any silting object $T_2$ for any $m$ is \textit{green sequence finite}.
\end{enumerate}
\end{definition}
\indent Hence we can rephrase Theorem \ref{C3T2} as the following:
\begin{theorem}
If $\Lambda$ is the path algebra of a quiver of finite or tame type, then $\Lambda$ is almost morphism finite.
\end{theorem}
\indent Note that the condition of an algebra being almost morphism finite is stronger than the condition that it is green sequence finite which is stronger than the condition that there are finitely many $m$-maximal green sequences for any $m$. An almost morphism finite algebra has finitely many $k$-red sequences for any $k$ hence it has finitely many green-to-red sequences with $k$ red mutations.\\