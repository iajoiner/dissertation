\chapter{Two alternative definitions of $m$-maximal green sequences}\label{C3}
\section{Introduction}
\indent In Chapter \ref{CB} we introduced a result by Igusa, namely Theorem \ref{thm:3} namely there are new alternative definitions of maximal green sequences. How the result can possibly be generalized to $m$-maximal green sequences in general is an interesting question that we have mostly solved.\\
\begin{theorem}\label{C3T}
%\indent (Theorem \ref{C3TB}) Let $\Lambda$ be a finite dimensional hereditary algebra. Let $(C^{\leq 0}, C^{\geq 0}), (C'^{\leq 0}, C'^{\geq 0})$ be two $t$-structures such that there exists at least one green sequence from $(C^{\leq 0}, C^{\geq 0})$ to $(C'^{\leq 0}, C'^{\geq 0})$. Let $\mathcal{T} = C^{\leq 0}\cap C'^{\geq 0}$.  Let $M_1,\cdots, M_n$ be a finite sequence in $\mathcal{T}$. The following are equivalent.
%\begin{theorem}\label{C3TB}
%\indent Let $\Lambda$ be a finite dimensional hereditary algebra. Let $(C^{\leq 0}, C^{\geq 0}), (C'^{\leq 0}, C'^{\geq 0})$ be two $t$-structures such that there exists at least one green sequence from $(C^{\leq 0}, C^{\geq 0})$ to $(C'^{\leq 0}, C'^{\geq 0})$. Let $\mathcal{T} = C^{\leq 0}\cap C'^{\geq 0}$.  Let $M_1,\cdots, M_n$ be a finite sequence in $\mathcal{T}$. The following are equivalent.
\indent (Theorem \ref{C3TB}) Let $\Lambda$ be a finite dimensional hereditary algebra. Let $\mathcal{T} = add(\cup_{i=0}^{m-1} (mod\,\Lambda)[i])$. Let $\{M_1,\cdots, M_n\}$ be a finite sequence of nonzero objects in $\mathcal{T}$. The following are equivalent:
\begin{enumerate}
\item The sequence is a maximal sequence of backward $Hom^{\leq 0}$-orthogonal Schurian objects $\{M_n\}$ in $\mathcal{T}$.
\item The sequence is a finite sequence in $\mathcal{T}$ that forms a finite HN system for $\mathcal{T}$.
\item The sequence is a sequence of simples from the simple-minded collection $\{S_1,\cdots, S_n\}$ to  $\{S_1[m],\cdots, S_n[m]\}$. that is, it is an $m$-maximal green sequence..
\end{enumerate}
\end{theorem}

%\begin{enumerate}
%\item The sequence is a maximal sequence of backward $Hom^{\leq 0}$-orthogonal Schurian objects $\{M_n\}$ on $\mathcal{T}$.
%\item The sequence is a finite $c$-green sequence on $\mathcal{T}$.
%\item The sequence is a green sequence from a simple-minded collection $\{X_i\}$ to another one $\{Y_i\}$.
%\end{enumerate}
%\end{theorem}
\indent In Section 2 we will discuss an alternative definition of the stability condition on module categoris. In Section 3 we will discuss maximal backward-$Hom^{\leq 0}$ orthogonal sequences. In Section 4 we will discuss Harder-Narasimhan filtrations. In Section 5 we will establish the fact that the two alternative definitions are equivalent to the original ones.\\
\section{Alternative definition of the stability condition on module categories}
\indent This section is not used in the rest of the chapter. However it does provide new definitions of stability and semistability that are previously underdiscussed. Moreover the ideas in this section are related to $Hom^{\leq 0}$-backward orthogonal sequences.\\
\indent In general in a triangulated category the concepts of monomorphisms and epimorphisms are less important because there are almost no nontrivial ones. Instead the concept of homotopy kernels and homotopy cokernels are much more important.\\
\indent Before we can generalize the idea of a maximal green sequence we first need to generalize the idea of a stability condition without always relying on monomorphisms and epimorphisms.\\
\begin{theorem}
If $\Lambda$ is a finite dimensional hereditary algebra for an indecomposable module $M$ in a module category $mod \Lambda$ for a stability condition $\phi$ the following are equivalent:
\begin{enumerate}
\item $M$ is stable. That is, for any proper submodule $N$ of $M$ it is true that $\phi(N)<\phi(M)$.
\item For any proper quotient module $N$ of $M$ it is true that $\phi(M)<\phi(N)$.
\item For any indecomposable stable module $N\not\cong M$ such that $(M,N)\neq 0$ it is true that $\phi(M)<\phi(N)$.
\item For any indecomposable stable module $N\not\cong M$ such that $(N,M)\neq 0$ it is true that $\phi(N)<\phi(M)$.
\end{enumerate}
\end{theorem}
\begin{proof}
\indent (1)$\to$(2) If $N$ is a quotient module of $M$, we have the short exact sequence $0\to\sum_{i=1}^kR_i\to M\to N\to 0$. Since $M$ is stable $\phi(\sum_{i=1}^kR_i)<\phi(M)$ hence $\phi(N)>\phi(M)$.\\
\indent (2)$\to$(1) This proof is analogous to the proof of (1)$\to$(2).\\
\indent (1),(2)$\to$(3) If $M$ is a submodule of $N$ or $N$ is a quotient module of $M$ then due to (1) and (2) the statement is trivially true. Assume that there exists neither monomorphisms nor epimorphisms from $M$ to $N$. Assume that there exists $0\neq f\in(M,N)$ it is easy to see that $Im f\in mod\Lambda$. Take one of its indecomposable summand, $L$. It is easy to see that $L$ is a proper submodule of $N$ and a proper quotient module of $M$ at the same time. Since $M,N$ are stable we have $\phi(M)<\phi(L)<\phi(N)$.\\
\indent (1),(2)$\to$(4) This proof is analogous to the proof of (1),(2)$\to$(3).\\
\indent (4)$\to$(1) Use induction. If $M$ is simple it is of course stable. Hence (4)$\to$(1) is trivially true in the case of simples. Otherwise assume that (4)$\to$(1) is already true for all indecomposable modules with dimension less than $dim(M)$. If $M$ satisfies condition (4) then for any of its stable submodule $N$ we already have $\phi(N)<\phi(M)$ so we only need to focus on the non-stable ones. Assume that $L$ is one of its minimal non-stable indecomposable proper submodules such that $\phi(L)\geq\phi(M)$. By induction since $\lnot$(1) $\to$ $\lnot$(4) holds for $L$ there has to be an indecomposable stable module $N\not\cong L$ such that $(N,L)\neq 0$ and $\phi(N)\geq \phi(L)$. Hence $\phi(N)\geq\phi(L)\geq\phi(M)$ and $(N,M)\neq 0$. Hence $M$ does not satisfy condition $(4)$ and we have reached a contradiction. As a result (4)$\to$(1) is proven.\\
\indent (3)$\to$(2) This proof is analogous to the proof of (4)$\to$(1). Use induction. If $M$ is simple (2) of course holds. Hence (3)$\to$(2) is trivially true in the case of simples. Otherwise assume that (3)$\to$(2) is already true for all indecomposable modules with dimension less than $dim(M)$. If $M$ satisfies condition (3) then for any of its stable quotient submodule $N$ we have $\phi(M)<\phi(N)$ so we only need to focus on the non-stable ones. Assume that $L$ is one of its minimal non-stable indecomposable proper quotient modules such that $\phi(M)\geq\phi(L)$. By induction since $\lnot$(2) $\to$ $\lnot$(3) holds for $L$ there has to be an indecomposable stable module $N\not\cong L$ such that $(L,N)\neq 0$ and $\phi(L)\geq \phi(N)$. Hence $\phi(M)\geq\phi(L)\geq\phi(N)$ and $(M,N)\neq 0$. Hence $M$ does not satisfy condition $(3)$ and we have reached a contradiction. As a result (3)$\to$(2) is proven.\\
\end{proof}
\indent We can obtain a similar result in the case of semistability.\\
\begin{theorem}
If $\Lambda$ is a finite dimensional algebra for an indecomposable module $M$ in a module category $mod \Lambda$ for a stability condition $\phi$ the following are equivalent:
\begin{enumerate}
\item $M$ is semistable.
\item For any proper quotient module $N$ of $M$ it is true that $\phi(M)\leq\phi(N)$.
\item For any indecomposable stable module $N$ such that $(M,N)\neq 0$ it is true that $\phi(M)\leq\phi(N)$.
\item For any indecomposable stable module $N$ such that $(N,M)\neq 0$ it is true that $\phi(N)\leq\phi(M)$.
\end{enumerate}
\end{theorem}
\section{Maximal backward $Hom^{\leq 0}$ orthogonal sequences}
\indent In order to discuss maximal backward $Hom^{\leq 0}$ orthogonal sequences we first need to define them.\\
\begin{definition}
$M_1,M_2,\cdots, M_k$ is a \textit{backward $Hom^{\leq 0}$ orthogonal sequence} of Schur objects if $(M_i[\geq 0], M_j) = 0$ for all $i>j$ and all $M_i$s are non-zero.
\end{definition}
\begin{definition}
$M_1,M_2,\cdots, M_k\in \mathcal{T}$ is a \textit{maximal backward $Hom^{\leq 0}$ orthogonal sequence} of Schur objects on $\mathcal{T}$ if $(M_i[\geq 0], M_j) = 0$ for all $i>j$, all $M_i$ are Schur and that for any other Schur object $M'\in \mathcal{T}$ if it is inserted anywhere in the sequence it will no longer be backward $Hom^{\leq 0}$ orthogonal.
\end{definition}
\indent Now let's prove a crucial lemma.\\
%\indent Our goal is to prove that if $M_1,M_2,\cdots, M_k$ is a maximal $Hom^{\leq 0}$-backward orthogonal sequence of Schur objects we can have $E_0(M_1,\cdots, M_k)=\mathcal{T}$.
\begin{lemma}\label{lem:C3L1}
If $H_1, H_2$ are two hearts of $t$-structures $(C^{\leq 0}, C^{\geq 0})$ and $(C'^{\leq 0}, C'^{\geq 0})$ respectively, there exists a maximal backward $Hom^{\leq 0}$-orthogonal sequence from $H_1$ to $H_2$ then the first term of the sequence has to be a simple of $H_1$ that is not in $H_2$.
\end{lemma}
\begin{proof}
\indent Let's first assume that $M\in H_1[l]$ with $l\geq 0$ is the first term of the maximal backward $Hom^{\leq 0}$-orthogonal sequence. Let the truncation functors of $(C^{\leq 0}, C^{\geq 0})$ be $\tau_{\geq n}$ and $\tau_{\leq n}$ respectively. There exists some simple $S\in H_1$ such that $S[l]$ is a subobject of $M$ in $H_1[l]$ which is Abelian. If there exists no non-initial term $N$ in the sequence such that $(N,S) \neq 0$ (note that it is impossible to have $(N[i],S)\neq 0$ for positive $i$ due to $N=\tau_{\leq 0}N$) then the sequence is not maximal because $S$ can be inserted before $M$. Hence we assume that such an $N$ exists, In this case $(N[l],M)= 0$ or the sequence would have no longer been backward $Hom^{\leq 0}$-orthogonal. Let $N' = \tau_{\geq 0}N$ and $N'' = \tau_{<0} N$ . So we have the canonical triangle $N''\to N\to N'\to N''[1]$. Due to $(N[l],M) = 0$ it is obvious that $(N'[l],M) = 0$. Note that $M, N'[l], S[l]\in H_1[l]$. $(N'[l],S[l]) = 0$ since $S[l]$ is a subobject of $M$ in an Abelian category. As a result $(N',S) = 0$ and hence $(N,S) = 0$ which contradicts the assumption that $(N,S)\neq 0$.\\
\indent Now let's assume that $M\in \tau_{\geq -l}$ but not $\tau_{\geq -l+1}$. Let $M' = \tau_{\leq -l}M$. Let $S[l]$ be a subobject of $M'$ in $H_1[l]$ which is Abelian. Note that $(S[l],M) = (S[l],M')\neq 0$ because $(S[l], M/M'[-1]) = 0$ since $M/M'[-1]\in \tau_{\geq -l+1}$. If there exists no non-initial term $N$ in the sequence such that $(N,S) \neq 0$ (note that again it is impossible to have $(N[i],S)\neq 0$ for positive $i$ due to $N=\tau_{\leq 0}N$) then the sequence is not maximal because $S$ can be inserted before $M$. Hence we assume that such an $N$ exists, In this case $(N[l],M)= 0$ or the sequence would have no longer been backward $Hom_{\leq 0}$-orthogonal. Let $N' = \tau_{\geq 0}N$ and $N'' = \tau_{<0} N$ . So we have the canonical triangle $N''\to N\to N'\to N''[1]$. Due to $(N[l],M) = 0$ it is obvious that $(N'[l],M) = 0$ and $(N'[l],M')=0$. Note that $M', N'[l], S[l]\in H_1[l]$. $(N'[l],S[l]) = 0$ since $S[l]$ is a subobject of $M'$ in an Abelian category. As a result $(N',S) = 0$ and hence $(N,S) = 0$ which contradicts the assumption that $(N,S)\neq 0$.\\
\end{proof}
%%%%%%%%%%%%%%%%%%%%%%%%%%%%%%%%%%%%%%%%%%%%%%%
\section{Harder-Narasimhan filtration}
\indent Now we need to introduce Harder-Narasimhan (HN) filtrations.\\
\begin{definition}
Let $\catt$ be a subcategory of a triangulated category. $M_1,\cdots, M_k\in \catt$ are a finite sequence of nonzero objects.  An \textit{HN filtration of object $X\in\catt$ with respect to $\{M_i\}$} aka an HN filtration of an object $X$ is the following diagram:\\
$\begin{tikzcd}
0 = X_k\arrow[rr] &                               & X_{k-1}\arrow[ld]\arrow[rr]&                & X_{k-2}\arrow[ld]& \cdots & X_0 = X\arrow[ld]\arrow[l,leftarrow]\\
                            & X_{k-1}/X_k\arrow[lu, dashrightarrow] &        & X_{k-2}/X_{k-1}\arrow[lu, dashrightarrow] & \cdots & X_0/X_1&
\end{tikzcd}$
where $X_{i-1}/X_i$ is a self-extension of $M_i$. 
\end{definition}
\indent If an object has an HN filtration with respect to $\{M_i\}_{i\in[N]}$ then it makes sense to define its \textit{lowest and highest indices} with respect to the filtration.\\
\begin{definition}
Let $\catt$ be a subcategory of a triangulated category. $M_1,\cdots, M_k\in \catt$ are a finite sequence of nonzero objects. If some object $X\in\catt$ has an HN filtration $\begin{tikzcd}
0 = X_k\arrow[rr] &                               & X_{k-1}\arrow[ld]\arrow[rr]&                & X_{k-2}\arrow[ld]& \cdots & X_0 = X\arrow[ld]\arrow[l,leftarrow]\\
                            & X_{k-1}/X_k\arrow[lu, dashrightarrow] &        & X_{k-2}/X_{k-1}\arrow[lu, dashrightarrow] & \cdots & X_0/X_1&
\end{tikzcd}$ with respect to $M_1,\cdots, M_k$ then we can define the following:
\begin{enumerate}
\item The \textit{lowest index} in the filtration is defined as the smallest $i$ such that $X_{i-1}/X_i\neq 0$.
\item The \textit{highest index} in the filtration is defined as the largest $i$ such that $X_{i-1}/X_i\neq 0$.
\end{enumerate}
\end{definition}
\indent If $X$ has a unique HN filtration with respect to $M_1,\cdots, M_k$ then the lowest and highest indices of $X$ are only dependent on $M_1,\cdots, M_k$. In this case we can refer to them as $l_X$ and $h_X$ without any ambiguity.\\
\indent We may sometimes abuse notations and refer to the unique HN filtration of some $X$ as $0\to X_{h_X-1}\to\cdots\to X_{l_X} = X$.\\
\begin{definition}
Let $\catt$ be a subcategory of a triangulated category. $M_1,\cdots, M_k\in \catt$ are a finite sequence of nonzero objects. $M_1,\cdots, M_k$ form a \textit{finite HN system for $\catt$} if any object $X\in\catt$ has a unique HN filtration with respect to $M_1,\cdots, M_k$.
\end{definition}
%\indent The fact that a category accepts a unique Harder-Narasimhan (HN) filtration is a very strong condition. In this case we can define the \textit{degree} of any object in $D^b(\Lambda)$ as the composition length of the object when decomposed using the HN filtration.\\
\begin{lemma}\label{lem:C3L3}
Let $\Lambda$ be a finite dimensional hereditary algebra. Let $\mathcal{T} = add(\cup_{i=0}^{m-1} (mod\,\Lambda)[i])$. Let $M_1,\cdots, M_n$ be a finite sequence of nonzero objects in $\mathcal{T}$. If any object $Y$ in $\catt$ accept a unique HN filtration $0\to Y_m\to\cdots\to Y_1=Y$ with $Y_i/Y_{i+1}\in \mathcal{E}(M_i)$, the following holds.
\begin{enumerate}
%\item For any $i$ $M_i\in mod\, \Lambda[k]$ for a certain $k$.
\item For any $i$ it is true that $M_i$ is indecomposable.
\item For any $i$ it is true that $M_i$ is Schur.
\end{enumerate}
\end{lemma}
\begin{proof}
%\indent For (1) assume that $M_i = \oplus_{k}M_i^k$ such that $M_i^k\in\, mod\, \Lambda[k]$. Define the generalized dimension vector of $M_i$ as $dim\, M_i\, =\, \Sigma_k (-1)^k\,dim\,M_i^k$. Any object $X$ that only have $i$-th entry in its HN filtration has to satisfy $dim\,X = k\,dim\,M_i$ for some nonnegative $k$.  If not all entries of $dim\, M_i$ are non-negative then $M_i^k = 0$ for all odd $k$ for otherwise the unique HN filtration of $M_i^k$ can not only have the $i$-th entry which violates the condition. Similarly if not all entries of $dim\, M_i$ are non-positive then $M_i^k = 0$ for all even $k$. If $dim\, M_i = 0$ then all $M_i^k$ has to be 0 because otherwise $dim\,M_i^k$ could never be a multiple of $dim\,M_i$. Hence $dim\, M_i$ has to be either a positive vector or a negative vector. Without loss of generality let's assume that it is a positive one. In this case $M_i^k = 0$ for all odd $k$. If $M_i^k \neq 0$ and $M_i^l\neq 0$ then $dim\,M_i^k\,<\,dim\,M_i$. This can not happen either.\\
\indent For (1) assume that $M_i$ is decomposable. Then $M_i = A\oplus B$ with $A\neq 0$ and $B\neq 0$. In this case $A$ and $B$ have nontrivial HN filtrations and adding them up we should obtain an HN filtration for $M_i$ that isn't the canonical one which contradicts the fact that $M_i$ has a unique HN filtration.\\
\indent For (2) assume that $M_i$ is not Schur. Without loss of generality we can assume that $M_i\in mod\Lambda$. Then there exists $f\in End(M_i)$ such that $Im f\neq M_i$. Take an indecomposable direct summand of $Im f$, $Q$. Since $Q$ is a quotient module of $M_i$ and $\Lambda$ is hereditary $Ext^1(M_i/Q, M_i)\to Ext^1(M_i/Q, Q) \neq 0$ is a surjection. We have the following diagram.\\
$\begin{tikzcd}
0\arrow[r] & M_i\arrow[r]\arrow[d] & X\arrow[r]\arrow[d] & M_i/Q \arrow[r]\arrow[d,equal] & 0\\
0\arrow[r] & Q \arrow[r] & M_i\arrow[r] & M_i/Q \arrow[r] & 0\\
\end{tikzcd}$\\
\indent Due to $0\to M_i\to Q\oplus X\to M_i\to 0$ being a short exact sequence with is a consequence of a pushout diagram on the left and $M_i\rightarrowtail X$ being monomorphic, $Q\oplus X$ has two HN-filtrations, one containing the $i-$th entry only while the other definitely contain what is not in the $i$-th entry because $Q$ can not only have the $i$-th entry.\\
\end{proof}
%\begin{lemma}\label{lem:C3L3}
%Let $(C^{\leq 0}, C^{\geq 0}), (C'^{\leq 0}, C'^{\geq 0})$ be two $t$-structures such that there exists at least one green sequence from $(C^{\leq 0}, C^{\geq 0})$ to $(C'^{\leq 0}, C'^{\geq 0})$. Let $\mathcal{T} = C^{\leq 0}\cap C'^{\geq 0}$.  If any object $Y$ in $\catt$ accept a unique HN filtration $0\to Y_m\to\cdots\to Y_1=Y$ with $Y_i/Y_{i+1}\in \mathcal{E}(M_i)$, the following holds.
%\begin{enumerate}
%\item For any $i$ $M_i\in mod\, \Lambda[k]$ for a certain $k$.
%\item For any $i$ $M_i$ is indecomposable.
%\item For any $i$ $M_i$ is Schur.
%\end{enumerate}
%\end{lemma}
%\indent Now we need to restrict the situation to $\mathcal{T} = C^{\leq 0}\cap C'^{\geq 0}$ for $t$-structures $(C^{\leq 0}, C^{\geq 0})$ and $(C'^{\leq 0}, C'^{\geq 0})$.
%\begin{proof}
%\indent (1), (2) and (3) are all trivially true because $\catt$ is closed under direct summands and extensions.
%\end{proof}
\begin{lemma}\label{lem:C3L2}
Let $\Lambda$ be a finite dimensional hereditary algebra. Let $\mathcal{T} = add(\cup_{i=0}^{m-1} (mod\,\Lambda)[i])$. Let $M_1,\cdots, M_n$ be a finite sequence of nonzero objects in $\mathcal{T}$. If any object $Y$ in $D^b(\Lambda)$ accept a unique HN filtration $0\to Y_N\to\cdots\to Y_1=Y$ with $Y_i/Y_{i+1}\in \mathcal{E}(M_i)$, the following holds.
\begin{enumerate}
\item For any $i>j$ it is true that $Hom(M_i,M_j) = 0$.
\item If $M_j = M_i[1]$ then $j>i$. Moreover if $Y$ is any object, the lowest nonzero entry of the HN filtration of $Y$ has index $l_Y$, can not be higher than the highest nonzero entry of the HN filtration of $Y[1]$ then $l_Y\leq h_{Y[1]}$.
\item If $Y_i\neq 0$ for some $i\in[N]$ then $Hom(Y_i, Y)\neq 0$ .
\item If $Y/Y_i\neq 0$ for some $1<i\leq N$ then $Hom(Y, Y/Y_i)\neq 0$ .
\item For any $i>j$ it is true that $Hom(M_i[1],M_j) = 0$.
\item If $M_j = M_i[m]$ where $m>0$ then $j>i$. Moreover if $Y$ is any object and the lowest nonzero entry of the HN filtration of $Y$ has index can not be higher than the highest nonzero entry of the HN filtration of $Y[m]$ for any positive $m$. 
\item $Hom(M_i[m],M_j) = 0$ where $m>0$ for any $i>j$.
\end{enumerate}
\end{lemma}
\begin{proof}
\indent (1) is true because otherwise we have completely different HN filtrations for $M_i\oplus M_j$, namely $\triangwm{M_i}{(1,0)^t}{M_i\oplus M_j}{(0,1)}{M_j}{0}{M_i[1]}$ and $\triangwm{M_i}{(1,f)^t}{M_i\oplus M_j}{(f,-1)}{M_j}{0}{M_i[1]}$ where $f$ is a nontrivial morphism from $M_i$ to $M_j$.\\
\indent (2) is true because otherwise $M_i\to 0\to M_i[1]\to M_i[1]$ will be an HN filtration of 0 and hence there will be at least two HN filtrations of 0. Similarly if the HN filtration of $Y$ is strictly before the HN filtration of $Y[1]$ $0\to Y\to Y[1]\to 0$ will be an HN filtration of 0.\\
\indent Let $l \geq i$ be any number such that $Y_l/Y_{l+1}\neq 0$. Since $Y_i\neq 0$ such $l$ must exist. (3) is true because $\triang{Y_i}{Y}{Y/Y_i}{Y_i[1]}$ is a triangle. If $Hom(Y_i,Y)= 0$ the triangle splits and $Y/Y_i = Y\oplus Y_i[1]$. Since $Y_i[1]$ has a unique HN filtration $Y/Y_i$ has two HN filtrations, one with the $l$-th entry 0 and one with a nontrivial $l$-th entry. \\
\indent Let $l < i$ be any number such that $Y_l/Y_{l+1}\neq 0$. Since $Y/Y_i\neq 0$ such $l$ must exist. (4) is true because $\triang{Y_i}{Y}{Y/Y_i}{Y_i[1]}$ is a triangle. If $Hom(Y,Y/Y_i)= 0$ the triangle splits and $Y_i = Y\oplus Y_i[-1]$. Since $Y_i[-1]$ has a unique HN filtration $Y_i$ has two HN filtrations, one with the $l$-th entry 0 and one with a nontrivial $l$-th entry. \\
\indent Now let's prove (5). Assume that $i>j$ and $Hom(M_i[1],M_j) \neq 0$. Since $Hom(M_i,M_j[-1])\neq 0$ and that $M_j\in\catt$ it is clear that $M_j[-1]\in\catt$ and hence has a unique HN filtration. Let's first assume that the highest term of the HN filtration of $M_j[-1]$ is a self-extension of $M_i$. If this is not the case assume that the highest term is a self-extension of $M_{i'}$. If $i'\leq j$ then it is clear that $Hom(M_i,M_j[-1])=0$ since the Hom from $M_i$ to all terms in the HN filtration of $M_j[-1]$ is 0 due to (1). Hence $i'>j$ and we can simply use $i'$ instead of $i$ since $Hom(M_{i'},M_j[-1])\neq 0$. So we can indeed assume that the highest entry of the HN filtration of $M_j[-1]$ is a self-extension of $M_i$. Let such an entry be $X_i$. Since $Hom(M_i[1], M_j)\neq 0$ and $M_i\in\catt$ we can see that $M_i[1]\in\catt$. As a result $X_i[1]\in\catt$. Let $h$ be the highest nontrivial index of the HN filtration of $X_i[1]$ then $h$ is not higher than $j$ or the highest nontrivial index of $M_j[-1]/X_i$ both of which are lower than $i$ due to (4) since $M_j[-1]\to M_j[-1]/X_i\to X_i[1]\to M_j$ is a triangle. Apply (4) again to $X_i[1]$ as a self-extension of $M_i[1]$ the highest nontrivial index of the HN filtration of $M_i[1]$ is lower than $i$. Apply (2) to $M_i$ and we can reach a contradiction.\\
\indent As for (6), since (2) is already proven let's assume that the result has been proven for all positive integers below $m$ and use induction. For the first claim it is clear that $M_i[1]$ can not be any $M_l$ or the induction hypothesis would be violated. It is also clear that $M_i[1]\in\catt$ since $M_i, M_i[m]\in\catt$. Take the lowest nonzero entry $Y_k$ of the HN filtration of $M_i[1]$. If $k=i$ then $Hom(M_i[1],M_i)\neq 0$ which can not happen. If $k>i$ then we can apply the induction hypothesis to the HN filtration of $M_i[1]$ and $M_i[m]$ and show that this is false. Hence $k<i$. In this case $Hom(M_i[1],Y_k)\neq 0$. Hence $Hom(M_i[1],M_k)\neq 0$ which is impossible due to (5). Now let's prove the second claim. Here since $Y,Y[m]\in\catt$ so does $Y[1]$. Let the lowest entry of the HN filtration of $Y[1]$ be $N_k\in\mathcal{E}(M_k)$ and let $0\to Y_j\to\cdots Y_l = Y$ be the HN filtration of $Y$. It is easy to see that $k<l$ due to the induction hypothesis applied to $Y[1]$ and $Y[k]$. Hence $Hom(Y[1],M_k)\neq 0$. Hence for some $h>k$ we have $Hom(M_h[1],M_k)\neq 0$ which is impossible due to (5).\\
\indent Finally we need to prove (7). Assume that the result is true for any positive integer below $m$ which is legit because (5) is already proven. Assume that $i>j$ and $Hom(M_i[m],M_j) \neq 0$. Since $Hom(M_i, M_j[-m])\neq 0$ and $M_j\in\catt$ it is clear that $M_j[-m]\in\catt$. Let's first assume that the highest entry of the HN filtration of $M_j[-m]$ is a self-extension of $M_i$. If this is not the case assume that the highest term is a self-extension of $M_{i'}$. If $i'\leq j$ then it is clear that $Hom(M_i,M_j[-m])=0$ since the Hom from $M_i$ to all terms in the HN filtration of $M_j[-m]$ is 0 due to (1). Hence $i'>j$ and we can simply use $i'$ instead of $i$ since $Hom(M_{i'},M_j[-m])\neq 0$. So we can indeed assume that the highest entry of the HN filtration of $M_j[-m]$ is a self-extension of $M_i$ and let such an entry be $X_i$. Since $Hom(M_i[1], M_j[1-m])\neq 0$ and $M_i\in\catt$ we can see that $M_i[1]\in\catt$. As a result $X_i[1]\in\catt$. Let $h$ be the highest nontrivial index of the HN filtration of $X_i[1]$ then $h$ is no higher than the highest nontrivial index of $M_j[-m]/X_i$ or $Hom(M_h,M_j[1-m])\neq 0$ in which case $h\leq j$ by induction since $M_j[-m]\to M_j[-m]/X_i\to X_i[1]\to M_j[1-m]$ is a triangle. Hence in both cases $h<i$. Apply (4) again to $X_i[1]$ as a self-extension of $M_i[1]$ the highest nontrivial index of the HN filtration of $M_i[1]$ is lower than $i$. Apply (2) to $M_i$ and we can reach a contradiction.\\
\end{proof}
%\indent Here Lemma 4.4.2 only applies if the unique HN filtration exists for all objects $Y\in D^b(\Lambda)$. When restricted to the case of actual green sequences most of the arguments remain  the same. However truncation functors need to be used in some cases.\\
%\begin{lemma}\label{lem:C3L4}
%Let $(C^{\leq 0}, C^{\geq 0}), (C'^{\leq 0}, C'^{\geq 0})$ be two $t$-structures such that there exists at least one green sequence from $(C^{\leq 0}, C^{\geq 0})$ to $(C'^{\leq 0}, C'^{\geq 0})$. Let $\mathcal{T} = C^{\leq 0}\cap C'^{\geq 0}$. Let $M_1,\cdots, M_N$ be nonzero objects of $\catt$. If any object $Y$ in $\catt$ accept a unique HN filtration $0\to Y_N\to\cdots\to Y_1=Y$ with $Y_i/Y_{i+1}\in \mathcal{E}(M_i)$, the following holds.
%\begin{enumerate}
%\item $Hom(M_i,M_j) = 0$ for any $i>j$.
%\item If $M_j = M_i[1]$ then $j>i$. Moreover if $Y$ is any object such that $Y[1]\in\catt$ and the lowest nonzero entry of the HN filtration of $Y$ has index can not be higher than the highest nonzero entry of the HN filtration of $Y[1]$. 
%\item If $Y_i\neq 0$ $Hom(Y_i, Y)\neq 0$ for any $i$.
%\item If $Y/Y_i\neq 0$ $Hom(Y, Y/Y_i)\neq 0$ for any $i>1$.
%\item If $Hom(M_i[1],M_j) = 0$ for any $i>j$.
%\item If $M_j = M_i[m]$ where $m>0$ then $j>i$. Moreover if $Y$ is any object and the lowest nonzero entry of the HN filtration of $Y$ has index can not be higher than the highest nonzero entry of the HN filtration of $Y[m]$ for any positive $m$. 
%\item $Hom(M_i[m],M_j) = 0$ where $m>0$ for any $i>j$.
%\end{enumerate}
%\end{lemma}
%\begin{proof}
%\indent (1) is true for the same reason why Lemma \ref{lem:C3L2} (1) is true. In particular $M_i\oplus M_j$ is in $\catt$ because $C'^{\leq 0}$ and $C^{\geq 0}$ are both closed under extensions.\\
%\indent The proof of(2)-(4) are the same as the proof of Lemma \ref{lem:C3L2} (2)-(4).\\
%\indent For (5) we must consider situations where numerous objects we mentioned are not actually in $\catt$. Assume that $i>j$ and $Hom(M_i[1],M_j) \neq 0$. Let $\tilde{M_i}[1]:=\tau'^{\geq 0}(M_i[1])$ and $\bar{M_i}[1]:=\tau'^{< 0}(M_i[1])$. $\bar{M_i}[1] \to M_i[1]\to \tilde{M_i}[1]\to \bar{M_i}[2]$. Since $Hom(\bar{M_i}[1], M_j) = Hom(\bar{M_i}[2],M_j) = 0$ $Hom(\tilde{M_i}[1],M_j) \neq 0$. Let $\tilde{M_j}[-1]:=\tau^{\leq 0}(M_j[-1])$ and $\bar{M_j}[-1]:=\tau^{> 0}(M_j[-1])$. $\tilde{M_j}[-1] \to M_j[-1]\to \bar{M_j}[-1]\to \tilde{M_j}$. Since $Hom(\tilde{M_i},\bar{M_j}[-1]) = Hom(\tilde{M_i},\bar{M_j}[-2]) = 0$ we have $Hom(\tilde{M_i},\tilde{M_j}[-1])\neq 0$.\\
%\indent Let's first assume that the highest term of the HN filtration of $\tilde{M_j}[-1]$ is a self-extension of $M_i$. If this is not the case assume that the highest term is a self-extension of $M_{i'}$. If $i'\leq j$ then it is clear that $Hom(M_i,\tilde{M_j}[-1])=0$ since the Hom from $M_i$ to all terms in the HN filtration of $\tilde{M_j}[-1]$ is 0 due to (1). Hence $i'>j$. Let $\tilde{M_i'}[1]:=\tau'^{\geq 0}(M_i
%[1])$ and $\bar{M_i'}[1]:=\tau'^{< 0}(M_i'[1])$. $\bar{M_{i'}}[1] \to M_{i'}[1]\to \tilde{M_{i'}}[1]\to \bar{M_{i'}}[2]$. Since $Hom(\bar{M_{i'}}[1], M_j) = Hom(\bar{M_{i'}}[2],M_j) = 0$ $Hom(M_{i'}[1],M_j) = Hom(\tilde{M_{i'}}[1],M_j) \neq 0$.  We can simply use $i'$ instead of $i$ since $Hom(M_{i'},\tilde{M_j}[-1])\neq 0$. \\
%\indent So we can indeed assume that the highest entry of the HN filtration of $\tilde{M_j}[-1]$ is a self-extension of $M_i$. Let such an entry be $X_i$. Let $\tilde{X_i}[1]:=\tau'^{\geq 0}(X_i[1])$ and $\bar{X_i}[1]:=\tau'^{< 0}(X_i[1])$. Let $h$ be the highest nontrivial index of the HN filtration of $\tilde{X_i}[1]$ then $h$ is not higher than $j$ or the highest nontrivial index of $\tilde{M_j}[-1]/X_i$ both of which are lower than $i$ due to (4) since $\tilde{M_j}[-1]\to \tilde{M_j}[-1]/X_i\to X_i[1]\to 
%tilde{M_j}$ is a triangle. Apply (4) again to $X_i[1]$ as a self-extension of $M_i[1]$ the highest nontrivial index of the HN filtration of $M_i[1]$ is lower than $i$. Apply (2) to $M_i$ and we can reach a contradiction.\\
%\indent For (6) since if $Y, Y[m]\in\catt$ so is $Y[1]$ the argument in the proof of Lemma \ref{lem:C3L2} (6) does not need to be changed.\\
%\indent For (7) just like for (5) we need to use truncation functors.\\
%\end{proof}
\section{Equivalence of the definitions}
\begin{theorem}\label{C3TB}
%\indent Let $\Lambda$ be a finite dimensional hereditary algebra. Let $(C^{\leq 0}, C^{\geq 0}), (C'^{\leq 0}, C'^{\geq 0})$ be two $t$-structures such that there exists at least one green sequence from $(C^{\leq 0}, C^{\geq 0})$ to $(C'^{\leq 0}, C'^{\geq 0})$. Let $\mathcal{T} = C^{\leq 0}\cap C'^{\geq 0}$.  Let $M_1,\cdots, M_n$ be a finite sequence in $\mathcal{T}$. The following are equivalent.
\indent Let $\Lambda$ be a finite dimensional hereditary algebra. Let $\mathcal{T} = add(\cup_{i=0}^{m-1} (mod\,\Lambda)[i])$. Let $M_1,\cdots, M_n$ be a finite sequence of nonzero objects in $\mathcal{T}$. The following are equivalent:
\begin{enumerate}
\item The sequence is a maximal backward $Hom^{\leq 0}$-orthogonal sequence of Schurian objects $\{M_n\}$ in $\mathcal{T}$.
\item The sequence is a finite sequence in $\mathcal{T}$ that forms a finite HN system for $\mathcal{T}$.
\item The sequence is a sequence of simples from the simple-minded collection $\{S_1,\cdots, S_n\}$ to  $\{S_1[m],\cdots, S_n[m]\}$, that is, it is an $m$-maximal green sequence.
\end{enumerate}
\end{theorem}
\begin{proof}
\indent (1)$\to$(3) By applying Lemma \ref{lem:C3L1} repeatedly it is easy to see that any maximal backward $Hom^{\leq 0}$-orthogonal sequence of Schurian objects $\{M_n\}$ on $\mathcal{T}$ is also a sequence of simples in hearts of $t$-structures related to each other by a finite sequence of forward mutations. Hence (1) implies (3).\\
\indent (2)$\to$(1) Schurness has been proved in Lemma \ref{lem:C3L3}. Backward $Hom^{\leq 0}$ orthogonality has already been proven in Lemma \ref{lem:C3L2}. Maximality holds because of Lemma \ref{lem:C3L2}(3) and (4) due to the reasoning below. Since if the sequence is not maximal then there exists some $M$.such that it can be inserted in the backward $Hom^{\leq 0}$ orthogonal sequence. However such an $M$ must have a unique HN filtration. Hence there exists some $i\leq j$ such that $(M_j, M)\neq 0$ and $(M, M_i)\neq 0$. In this case $M$ can not be inserted in the backward $Hom^{\leq 0}$ orthogonal sequence which proves its maximality.\\
\indent (3)$\to$(2) This is obvious because using truncation functors we can easily show that any object in $\catt$ can be written uniquely as an HN filtration. 
\end{proof}