\chapter{Background}\label{CB}
\section{Notations and conventions}
\indent In this paper $k$ is an algebraically closed field, all algebras will be finitely dimensional $k$-algebras. All modules are assumed to be right modules. The symbol $[n]$ is defined as the set $\{1,2,\cdots, n\}$ which is consistent with how it is usually used in cluster theory. If $\catc$ is an Abelian category then $K_0(\catc)$ is its Grothendieck group.\\
\indent If $\catc$ is a category, $M, N, L\in Ob\,\catc, f\in Hom_{\catc}(M, N), g\in Hom_{\catc}(N, L)$ the composition is written as $gf$. Let $Q$ be a quiver, let $v$ be a path from $i\in Q_0$ to $j\in Q_0$, $w$ be a path from $j$ to $k\in Q_0$ then the composition is written as $vw$ which is the opposite of how we denote morphism compositions.\\
\indent When the category we are discussing is clearly $\mathcal{C}$ then $(M,N)$ will be an abbreviation of $Hom_{\mathcal{C}}(M,N)$. $(M[>0],N)$ is the union of $(M[k],N)$ for all $k>0$. If $\mathcal{P}, \mathcal{Q}$ are subcategories of $D^b(\Lambda)$ then $(\mathcal{P}, \mathcal{Q}):=\bigcup\limits_{M\in\mathcal{P}}\bigcup\limits_{N\in\mathcal{Q}}(M,N)$ and $(\mathcal{P}[>0], \mathcal{Q}):=\bigcup\limits_{k>0}(\mathcal{P}[k], \mathcal{Q})$. Let $S$ be a set of objects in an additive category $\catc$. $add(S)$ is defined as the set of all elements of $\catc$ such that they are summands of finite direct sums of objects in $S$. $\mathcal{E}(X)$ is the extension closure of $X$.\\
\indent The definition of green and red (vertices, mutations, sequences) are consistent with that of \cite{Kel11} and \cite{BDP13}. It is the exact opposite definition of green and red in \cite{BHIT15}.\\
%\indent If $m,n\in\mathbb{R}$ then $sp(m,n):=\begin{cases}
%mn & \text{if }m>0,n>0\\
%-mn & \text{if }m<0,n<0\\
%0 & \text{if } mn\leq 0\end{cases}$\\
\section{Quivers and path algebras}
\indent In this section we will introduce the basics about quivers, path algebras, modules, $c$-vectors, Euler matrices and Euler-Ringel forms. Materials in this section will be used extensively in the rest of the dissertation.
\subsection{Quivers}
\begin{definition}
A \textit{quiver} $Q$ is a quadruple $(Q_0,Q_1,s,t)$ with $Q_0$ and $Q_1$ sets and $s,t$: $Q_1\rightarrow Q_0$. An element of $Q_0$ is a \textit{vertex} of $Q$. An element of $Q_1$ is an \textit{arrow} of $Q$. The map $s$ maps each arrow to its source and $t$ maps each arrow to its target.
\end{definition}
\indent Intuitively we can think of elements of $Q_1$ as oriented edges. Any arrow has a unique source and a unique target both of which are vertices. This is how we obtain the $s$ and $t$ maps. Unless necessary we generally omit the $s$ and $t$ and denote a quiver by $Q = (Q_0, Q_1)$.\\
\begin{example}
The following diagrams are all quivers.\\
$\begin{tikzcd} 
1 \arrow[r] & 2 & 1 & 2 \arrow[l, shift right]\arrow[l, shift left]\arrow[r] & 3 & 1\arrow[r] & 2\arrow[r] & 3\\
\end{tikzcd}$
\end{example}
\indent Now let's define opposite quivers.\\
\begin{definition}
Let $Q = (Q_0, Q_1, s, t)$ be a quiver. The \textit{opposite quiver} $Q^{op}:=(Q^{op}_0, Q^{op}_1, s^{op}, t^{op})$ is a quiver such that $Q^{op}_0 = Q_0$, $Q^{op}_1 = Q_1$ and that for any $x\in Q^{op}_1$ we have $s^{op}(x) = t(x)$ and $t^{op}(x) = s(x)$.
\end{definition}
\indent That is, $Q^{op}$ is formed by keeping all the vertices of $Q$ and reversing all its arrows. Now we need to define subquivers.\\
\begin{example}
The opposite quiver of $Q: 1\to 2\to 3$ is $Q^{op}:1 \leftarrow 2\leftarrow 3$.
\end{example}
\begin{definition}
A \textit{subquiver} $Q' = (Q'_0,Q'_1, s', t')$ in a quiver $Q = (Q_0,Q_1, s, t)$ is a quiver such that $Q'_0\subseteq Q_0$, $Q'_1\subseteq Q_1$, $s|_{Q'_1} = s'$ and $t|_{Q'_1} = t'$.
\end{definition}
\indent From now on we generally do not distinguish between $s$ and $s'$, $t$ and $t'$.\\
\indent Not all quivers are useful for the purpose of this paper. This is why we need to add restrictions. In order to do so we need to introduce several definitions.\\
\begin{definition}
An \textit{oriented cycle} in a quiver $Q$ is a subquiver $Q' = (Q'_0, Q'_1, s, t)$ such that $Q'_0 = \{v_0,v_1,\cdots, v_{k-1}\},Q'_1 = \{a_0,a_1,\cdots, a_{k-1}\}, s(a_i) = v_i$ and $t(a_i) = v_{i+1}$. Here $v_k$ is defined as $v_0$.
\end{definition}
\begin{definition}
A \textit{$k$-cycle} is an oriented cycle with $k$ vertices.
\end{definition}
\begin{definition}
A \textit{loop} in a quiver $Q$ is an arrow from a vertex to itself, that is, a 1-cycle.
\end{definition}
\begin{definition}
A \textit{cluster quiver} is a quiver without loops or 2-cycles. 
\end{definition}
\indent In all but Chapter \ref{C3} and a part of Chapter \ref{C4} all quivers we discuss will be acyclic. Here is the definition of an acyclic quiver.\\
\begin{definition}
An \textit{acyclic quiver} is a quiver without any oriented cycles. 
\end{definition}
\subsection{Path Algebras}
\begin{definition}
A \textit{path} in a quiver $Q$ is a sequence of vertices $\{v_0,\cdots, v_k\}$ and a sequence of arrows $\{a_0,\cdots, a_{k-1}\}$ if $k>0$ such that $t(a_i) = v_{i+1}$, $ s(a_i) = v_i$ for any $i = 0,1,\cdots, k-1$. The source of the path is $v_0$ and the sink is $v_k$.
\end{definition}
\indent Paths with length 0 are known as \textit{trivial paths}. A trivial path only has a single vertex $v_0$ and no arrows at all. All other paths are uniquely determined by their arrows.\\
\indent Now we need to define multiplication of paths. In order to do so we need to define compatibility and concatenation.\\
\begin{definition}
Paths $v,w$ are \textit{compatible} if $t(v) = s(w)$.
\end{definition}
\begin{definition}
The \textit{concatenation} of compatible paths $v = \{a_0,\cdots, a_{k-1}\}$ and $w = \{b_0,\cdots, b_{l-1}\}$ is  $vw = \{a_0,\cdots, a_{k-1}, b_0,\cdots, b_{l-1}\}$.
\end{definition}
\begin{definition}
The \textit{path algebra} of a quiver $Q$ is a $k$-algebra generated by all the paths of the quiver. Multiplication of paths $v$ and $w$ is defined as the concatenation $vw$ if they are compatible and 0 if they aren't, 
\end{definition}
\indent From a homological point of view path algebras are very nice, namely they are \textit{hereditary}. In other words their global dimensions are at most one.
\begin{theorem}
The path algebra $kQ$ of any acyclic quiver $Q$ is hereditary. That is, for any $M,N\in mod kQ$ for all $k>1$ it is true that $Ext^k(M,N)=0$.
\end{theorem}
\indent We mostly only discuss path algebras of acyclic quivers in this thesis because our results are only about finite dimensional algebras.\\
\subsection{Modules in hereditary algebras}
\indent Now let's review some basic concepts about modules. In particular we will review the concepts of bricks and stones.\\
\begin{definition}
\indent An indecomposable module $M$ over an algebra $\Lambda$ is \textit{Schur} or a \textit{brick} if its endomorphism ring $End M$ is a division algebra. In particular if $\Lambda$ is a $k$-algebra where $k$ is an algebraically closed field then a brick is an indecomposable module such that $End M=k$.
\end{definition}
\begin{definition}
\indent An indecomposable module $M$ over an algebra $\Lambda$ is \textit{rigid}, \textit{exceptional} or a \textit{stone} if $Ext^1 (M,M)=0$.
\end{definition}
\indent Here is a well-known result about Schur and rigid modules in hereditary algebras.
\begin{theorem}
\indent \cite{ASS06}Let $k$ be an algebraically closed field, let $\Lambda$ be a finite-dimensional $k$-algebra. Then any rigid $\Lambda-$module $M$ is Schur.
\end{theorem}
\subsection{Euler matrices and the Euler-Ringel form}
\indent Now let's define \textit{Euler matrices} which will be very useful in the understanding of Chapter \ref{C4}.\\
\begin{definition}
Let $Q$ be an acyclic quiver. The \textit{Euler matrix of } $Q$ is defined as the matrix $E = (e_{ij})$ where $e_{ij} = \begin{cases}
1 & \text{if } i=j\\
-k & \text{if there are }k\text{ arrows from }i\text{ to }j\\
0 & \text{in all other cases}\\
\end{cases}$
\end{definition}
\begin{example}
The Euler matrix $E$ of the quiver $Q:1\to 2$ is $E=\begin{bmatrix}1 & -1\\0 & 1\\\end{bmatrix}$.
\end{example}
\begin{example}
The Euler matrix $E$ of the quiver $Q:\begin{tikzcd}1\arrow[r] & 2\rightdoublearrow & 3\end{tikzcd}$ is $E=\begin{bmatrix}1 & -1 & 0\\0 & 1 & -2 \\0 & 0 & 1\\\end{bmatrix}$.
\end{example}
\indent Using the Euler matrix we can define \textit{Euler-Ringel forms}.\\
\begin{definition}
Let $Q$ be an acyclic quiver and $E$ be its Euler matrix. The Euler-Ringel form of $Q$ (or $kQ$) is defined as $\langle x,y\rangle_Q:=x^tEy$.
\end{definition}
\indent When which $Q$ we are talking about isn't ambiguous we can just use $\langle x,y\rangle$ to refer to the Euler-Ringel form of $Q$.
\subsection{Dimension vectors and roots}
\indent The concept of roots originated from Lie theory. Here we will introduce some basic terminologies. Basically a lot of concepts we define using modules can also be defined using their \textit{dimension vectors} which we will define here.\\
\begin{definition}
The \textit{dimension vector} of a module $M$ in a finite dimensional $k$- algebra $\Lambda$ with $n$ indecomposable idempotents $e_1,\cdots, e_n$ is defined as $c_M:=\{c_1,\cdots, c_n\}\in\zz^n$ where $c_i:= dim_k Me_i$.
\end{definition}
\indent Now let's define real roots and imaginary roots. Before that we first need to define sign coherence in vectors.\\
\begin{definition}
A vector $c$ in $\zz^n$ is \textit{sign coherent} if it is nonzero and all its entries are either all nonpositive or all nonnegative.
\end{definition}
\indent The entries of a sign coherent vector are either all nonnegative or all nonpositive. In the former case we say it is \textit{positive}. In the latter case we say it is \textit{negative}.\\
\begin{definition}
Let $Q$ be an acyclic quiver with $n$ vertices. $c\in \nn^n$ is a \textit{real root} if it is sign coherent and $\langle c, c\rangle_Q = 1$.
\end{definition}
\begin{definition}
Let $Q$ be an acyclic quiver with $n$ vertices. $c\in \nn^n$ is an \textit{imaginary root} if it is sign coherent and $\langle c, c\rangle_Q \leq 0$.
\end{definition}
\subsection{Cartan and Euler matrices}
\indent Now we can define Cartan matrices and provide another characterization of Euler matrices.\\
\begin{definition}
Let $Q$ be an acyclic quiver. The \textit{Cartan matrix} of $Q$ is defined as $C:=(c_{ij})$ where $c_{ij}=\#\{\text{paths from }j\text{ to }i\}$.
\end{definition}
\indent Since $Q$ is acyclic for any vertex $i\in Q_0$ there exists one and only one path from $i$ to itself, hence $c_{ii}$ has to be 1.\\
\begin{example}
The Cartan matrix $C$ of the quiver $Q:\begin{tikzcd}1\arrow[r] & 2\rightdoublearrow & 3\end{tikzcd}$ is $C=\begin{bmatrix}1 & 0 & 0\\1 & 1 & 0 \\2 & 2 & 1\\\end{bmatrix}$.
\end{example}
\indent It is easy to see that $C^tE=I$ in the example above. In fact this is something true in general.\\
\begin{theorem}
Let $Q$ be an acyclic quiver, $E$ be its Euler matrix and $C$ be its Cartan matrix. It is true that $C^tE=I$.
\end{theorem}
\begin{proof}
\indent Since $Q$ is acyclic we can relabel elements of $Q_0$ as $1,2,\cdots, n$ to ensure that for any $i, j$ such that $i\geq j$ there are no arrows from $i$ to $j$. Let $C=(c_{ij}), E=(e_{ij})$. Using the definition of Euler matrices for any $i\neq j$ there are $a_{ij}:=-e_{ij}$ arrows from $i$ to $j$. It is easy to see that $C^t$ and $E$ are both upper triangular with all diagonal entries equal to $1$. Hence $C^tE$ is upper triangular with all diagonal entries equal to $1$.\\
\indent Now we need to prove that all other entries of $B:=C^tE=(b_{ij})$ above the diagonal is $0$. Let $n = |Q_0|$. To show that we only need to prove that $b_{1n}=0$ assuming that $n>1$. $b_{1n}=\Sigma_{i = 1}^nc_{i1}e_{in}=c_{n1}-\Sigma_{i = 1}^{n-1}c_{i1}a_{in}$. We know that $c_{i1}$ is the amount of paths from $1$ to $i$and $a_{in}$ is the amount of arrows from $i$ to $n$. Since any path $w$ from $1$ to $n$ is non-trivial it can be uniquely decomposed as $w=w'l$ where $l$ is the last arrow in it. Here $w'$ can be trivial which means $l$ is an arrow from $1$ to $n$. It is clear that $c_{n1}=\Sigma_{i = 1}^{n-1}c_{i1}a_{in}$. Hence $b_{1n}=0$. Similarly we can show that any entry of $B$ above the diagonal is 0. Hence $B=C^tE=I$.
\end{proof}
\indent Now we have the following crucial result about Euler-Ringel form which we will implicitly use extensively in Chapter \ref{C4}.\\
\begin{theorem}
\cite{ASS06} \label{HE}(Prop III.3.13) Let $Q$ be an acyclic quiver. Let $kQ$ be its path algebra. Then for any pair $M,N$ of modules in $mod kQ$ we have $\langle dim M, dim N\rangle = Hom(M,N)-Ext^1(M,N)$.
\end{theorem}
\section{Mutations, mutation sequences and the associated permutation}
\indent In this section we will introduce mutations of quivers and matrices, different kinds of mutation sequences including green sequences, maximal green sequences, reddening sequences and loop sequences. Stability conditions will be introduced. We will also define the associated permutation. Results in this section are mostly used in Chapters \ref{C1} and \ref{C4}.\\
\subsection{Mutation of quivers}
\indent The concept of maximal green sequences has many different equivalent definitions. We will use a simple definition using quiver mutations in this subsection. Later we will introduce other definitions.\\
\begin{definition}
Let $Q$ be a cluster quiver. \textit{Mutation} of $Q$ at vertex $k$ is defined in the following way:
\begin{enumerate}
\item For any pair of arrows $i\to k$ and $k\to j$ add an arrow $i\to j$.
\item Reverse all arrows starting from or ending up in $k$.
\item Delete all 2-cycles that are formed due to process (1) and (2).
\end{enumerate}
\end{definition}
\begin{definition}
\begin{enumerate}
\item The \textit{framed quiver} $\hat{Q}$ of $Q$ is obtained from $Q$ by adding a vertex $i'$ and an arrow $i\rightarrow i'$ for every $i\in Q$.
\item The \textit{coframed quiver} $\breve{Q}$ of $Q$ is obtained from $Q$ by adding a vertex $i'$ and an arrow $i'\rightarrow i$ for every $i\in Q$.
\item An \textit{ice quiver} is a quiver $Q$ where a possibly empty set, $F\subseteq Q_0$, consists of vertices that are not allowed to mutate.
\end{enumerate}
\end{definition}
\indent An ice quiver $(Q,F)$ can not mutate at elements of $F$, so we call them \textit{frozen vertices}.
\begin{definition}
A non-frozen vertex $i$ is \textit{green} if and only if no arrow from a frozen vertex to $i$ exists. Otherwise it is \textit{red}.\cite{Kel11}
\end{definition}
\begin{example}
\indent In this graph below we did a mutation at 1 from the framed quiver of $Q: 1\to 2$. After the mutation the vertex changed from being green to being red.\\
$\begin{tikzcd}
1 \arrow[r] \arrow[green]{d} & 2\arrow[green]{d}\arrow[r,"\mu_1", shift right=3.5ex]  & 1&2\arrow[l]\arrow[green]{d}\\
1' & 2'&1'\arrow[red]{u}& 2'\\
\end{tikzcd}$
\end{example}
\begin{definition}
A \textit{green sequence} is a sequence $\mathbf{i}=(i_1, i_2,\cdots, i_N)$ such that for all $1\leq t\leq N$ the vertex $i_t$ is green in the partially mutated ice quiver $\hat{Q}(\mathbf{i},t)=\mu_{i_{t-1}}\cdots\mu_2\mu_1(\hat{Q})$.
\end{definition}
\begin{definition}
A \textit{maximal green sequence} is a green sequence such that $\hat{Q}(\mathbf{i},N)$ does not have any green vertices.
\end{definition}
\begin{example} For quiver $1\to 2$ here is one of its two maximal green sequences.\\
 $\begin{tikzcd}
1 \arrow[r] \arrow[d,green] & 2\arrow [d,green]\arrow[r,"\mu_1", shift right=3.5ex]  & 1&2\arrow[l]\arrow[d,green]\arrow[r,"\mu_2", shift right=3.5ex]&1\arrow[r] & 2\\
1' & 2'&1'\arrow[u,red]& 2'&1'\arrow[u,red] & 2'\arrow[u,red]\\
\end{tikzcd}$
\end{example}
\indent We also need the definition of reddening sequences which are generalized versions of maximal green sequences in order to discuss the phenomenon of almost morphism finiteness in Chapter \ref{C2}.\\
\begin{definition}
A \textit{red-to-green sequence} or a \textit{reddening sequence}, is a sequence $\mathbf{i}=(i_1, i_2,\cdots, i_N)$ that transforms $\hat{Q}$ to a quiver $\hat{Q}(\mathbf{i},N) = \mu_{i_N}\cdots\mu_2\mu_1(\hat{Q})$ such that $\hat{Q}(\mathbf{i},N)$ does not have any green vertices.\cite{Mul15}\\
\end{definition}
\subsection{Mutation of matrices}
\indent We can also use $c$\textit{-vectors} for this purpose. To do so we need to reinterpret mutations of cluster quivers in terms of mutations of matrices. We recall that cluster quivers correspond to \textit{exchange matrices} as defined below. For more details we recommend \cite{FZ01} and \cite{FZ06}.\\
\begin{definition}
\cite{FZ01} An \textit{exchange matrix} of a cluster quiver $Q$ with $n$ vertices is an $n\times n$ matrix such that $b_{ij}$ is the number of arrows from $i$ to $j$ minus the number of arrows from $j$ to $i$.
\end{definition}
\indent It is easy to see that exchange matrices of cluster quivers are always antisymmetric which is not true in the more general case of \textit{valued quivers} which we won't discuss in this paper. Moreover there is a 1-1 correspondence between antisymmetric exchange matrices and cluster quivers.\\
\indent Mutations of exchange matrices are defined here which exactly agree with mutations of cluster quivers.\\
\begin{definition}
\cite{FZ01} If we mutate an $n\times n$ exchange matrix $B = (b_{ij})$ at $k$ we obtain $B' = (b'_{ij})$ defined here.
$b'_{ij} = \begin{cases}
-b_{ij} & \text{if }i = k\text{ or }j = k\\
b_{ij} + b_{ik}|b_{kj}| & \text{if }b_{ik}b_{kj} > 0\\
b_{ij} & \text{in all other cases}
\end{cases}$
\end{definition}
\indent Each partially mutated ice quiver corresponds to an \textit{extended exchange matrix} defined below.\\
\begin{definition}
The \textit{extended exchange matrix} $B'$ corresponding to a partially mutated ice quiver $Q'$ is an $2n\times n$ matrix with the rows corresponding to vertices $\{1,2,\cdots, n, 1', 2',\cdots n'\}$ while the columns corresponds to the vertices $\{1,2,\cdots, n\}$. Here we use the number $n+i$ to represent $i'$. Here again $b_{ij}$ is the number of arrows from $i$ to $j$ minus the number of arrows from $j$ to $i$.
\end{definition}
\indent An extended exchange matrix $B'$ has an upper and lower square submatrices, $B$ and $C$ respectively. The lower square matrix $C$ is known as the $c$\textit{-matrix}. Column vectors of an $c$-matrix are known as $c$\textit{-vectors}. A $c$-vector is positive if all its entries are non-negative and at least one is positive. A $c$-vector is negative if all its entries are non-positive and at least one is negative. Here is an important result about $c$-vectors, namely \textit{sign coherence}.\\
\begin{definition}
Let $Q$ be an acyclic quiver and $B$ its exchange matrix. A $c$-vector is \textit{reachable} if it is a column vector in some $C$ such that some $\begin{pmatrix}\tilde{B}\\C\end{pmatrix}$ that can be obtained from $\begin{pmatrix}B\\-I_n\end{pmatrix}$ through mutations.
\end{definition}
\begin{theorem}
\cite{GHKK14}\cite{DWZ08}\cite{DWZ10} a reachable $c$-vector is either positive or negative.
\end{theorem}
\indent Moreover $c$-vectors can be completely described due to the following result by Chavez\cite{C15}.\\
\begin{theorem} \label{CV}
Let $Q$ be an acyclic quiver. The set of $c$-vectors associated to $Q$ is equal to the set of real Schur roots associated to $Q$ and their opposites.
\end{theorem}
\indent Now we need to discuss the special case which we will use repeatedly, namely the case of $A_n$ straight orientation.\\
\begin{definition}
Let $Q$ be an acyclic quiver. A representation $(\{V_i\}_{i\in Q_0}, \{\phi_a\}_{a\in Q_1})$ of $Q$ is \textit{thin} if $dim(V_i) \leq 1$ for all $i\in Q_0$.
\end{definition}
\indent If $Q$ is $A_n$ straight orientation, namely quivers of the form $1\to 2\to 3\to\cdots\to n$. Note that for quiver $A_n$ all roots are real and Schur hence a real Schur root is just a root. In $A_n$ in particular since all indecomposable representations are thin, the positive roots are $\beta_{ij}=e_j-e_i$ ($0<i<j<n$, $e_0$ is defined as the zero vector). Hence we have the following lemma.\\
\begin{lemma} \label{CVA}
Any $c$-vector associated to $A_n$ straight orientation is in the form of $\pm\beta_{ij}$.
\end{lemma}
\indent A mutation on vertex $k$ is \textit{green} if the $c$-vector $c_k$ before the mutation is negative.  A mutation on vertex $k$ is \textit{red} if the $c$-vector $c_k$ before the mutation is positive. A \textit{maximal green sequence} is a mutation sequence from $C=-I_n$ to a permuted version of $I_n$ We can use a sequence of $c$-vectors to denote a maximal green sequence because we can use the $c$-vector corresponding to vertex $k$ to represent mutation at vertex $k$ which is possible since all $c$-matrices are invertible.\\
\subsection{Stability conditions}
\indent Here is the definition of stability functions.\\
\begin{definition}
\cite{B07}(Def 2.1) A \textit{stability function} on an abelian category $\catc$ is a group homomorphism $Z : K_0(\catc)\to\cc$ such that for all $0\neq E\in\catc$ the complex number $Z(E)$ lies in the strict upper half-plane $H = \{re^{i\pi\phi} : r > 0\text{ and }0 < \phi\leq 1\}\subseteq\cc$.
\end{definition}
\indent Now let's define the phase of a nonzero object.\\
\begin{definition}
\cite{B07} Let $Z : K_0(\catc)\to\cc$ be a stability function on an Abelian category $\catc$. The \textit{phase} of an object $0\neq E\in\catc$ is defined as $\phi(E):=\dfrac1\pi arg Z(E)$.
\end{definition}
\indent Now we can define what it means for a nonzero object in an Abelian category to be semistable and stable.\\
\begin{definition}
\cite{B07}(Def 2.2) Let $Z : K_0(\catc)\to\cc$ be a stability function on an Abelian category $\catc$. An object $0\neq E\in\catc$ is said to be \textit{semistable} (with respect to Z) if every
subobject $0 \neq A\subseteq E$ satisfies $\phi(A) \leq \phi(E)$.
\end{definition}
\begin{definition}
\cite{B07} Let $Z : K_0(\catc)\to\cc$ be a stability function on an Abelian category $\catc$. An object $0\neq E\in\catc$ is said to be \textit{stable} (with respect to Z) if every
subobject $0 \neq A\subsetneq E$ satisfies $\phi(A) < \phi(E)$.
\end{definition}
\begin{tikzpicture}
\draw [->] (-7,0) -- (-1,0);
\node at (-0.7,0) {$x$};
\draw [->] (-4,-3) -- (-4,3);
\node at (-4,3.3) {$y$};
\draw [->] (-4,0) -- (-2,1);
\node at (-1.7,1) {$P_2$};
\draw [->] (-4,0) -- (-3,1);
\node at (-3,1.3) {$I_1$};
\draw [->] (-4,0) -- (-1,2);
\node at (-0.7,2) {$P_1$};
\draw [->] (1,0) -- (7,0);
\node at (7.3,0) {$x$};
\draw [->] (4,-3) -- (4,3);
\node at (4,3.3) {$y$};
\draw [->] (4,0) -- (6,1);
\node at (6.3,1.15) {$I_1$};
\draw [->] (4,0) -- (3,1);
\node at (2.7,1.3) {$P_2$};
\draw [->] (4,0) -- (5,2);
\node at (5.15,2.3) {$P_1$};
\end{tikzpicture}
\begin{example}
\indent Here are two stability functions of $mod kQ$ where $Q$ is the quiver $1\to 2$. From the picture on the left we can see that $P_2$, $I_1$ and $P_1$ are all stable. $P_2$ and $I_1$ are stable because they are simples. As for $P_1$ it is stable because its submodule $P_2$ satisfies $\phi(P_2)<\phi(P_1)$.\\
From the picture on the right we can observe that $P_2$ and $I_1$ are all stable. $P_2$ and $I_1$ are still stable because they are simples. As for $P_1$ it is unstable because its submodule $P_2$ satisfies $\phi(P_2)\geq \phi(P_1)$.\\
\end{example}
\indent We will discuss stability conditions more in Chapter \ref{C3}.
\subsection{Permutations}
\indent All reddening sequences have associated permutations. When comparing the quivers obtained from transforming the same framed quiver using two different reddening sequences, it is easy to see that they are just one permutation away from each other: If you do a correct permutation of vertices (that means both rows and columns together) you can transform one such matrix into another. In particular any quiver obtained by using a reddening sequence to transform a framed quiver is one permutation away from the coframed quiver.\\
\indent In this subsection if $Q$ is a cluster quiver then $\hat{Q}, \breve{Q}$ are the framed and coframed quiver associated with cluster quiver $Q$ respectively.\\
\indent Here is the formal definition of such a permutation:\\
\begin{definition}
\cite{BDP13} A \textit{permutation} from an ice quiver $(Q,F)$ to $(Q',F)$ is an isomorphism of quivers $Q\rightarrow Q'$ that preserve $F$.\\
\end{definition}
\indent We have a result from \cite{BDP13} which helps us define the permutation:\\
\begin{theorem}
\cite{BDP13} Let $Q$ be a cluster quiver and let $Q'$ be a quiver that is a result of a reddening sequence on $\hat{Q}$, then  $Q'$ equals to a permutation of $\breve{Q}$. That is, for a reddening sequence $\mathbf{i}=(i_1,\cdots, i_N)$, for some $\rho\in S_n$ we have $\mu_{i_N}\cdots\mu_{i_1}\hat{Q}=\rho\breve{Q}$.\\
\end{theorem}
\begin{definition}
\cite{GM14} The \textit{permutation of a reddening sequence} $\mathbf{i}$ is $\rho$ for which $\mu_{i_N}\cdots\mu_{i_1}\hat{Q}=\rho\breve{Q}$.\\
\end{definition}
\indent Here are some simple examples that illustrate the concept of the permutation:\\
\begin{example}
Here is a maximal green sequence of $Q:1\to 2$.\\
$\begin{tikzcd}
1 \arrow[r] \arrow[d,green] & 2\arrow [d,green]\arrow[r,"\mu_1", shift right=3.5ex]  & 1&2\arrow[l]\arrow[d,green]\arrow[r,"\mu_2", shift right=3.5ex]&1\arrow[r] & 2\\
1' & 2'&1'\arrow[u,red]& 2'&1'\arrow[u,red] & 2'\arrow[u,red]\\
\end{tikzcd}$\\
\indent It is clear that this sequence has permutation $id$.
\end{example} 
\begin{example}
Here is another maximal green sequence of $Q:1\to 2$.\\
$\begin{tikzcd}
 1 \arrow[r] \arrow[d,green] & 2\arrow [d,green]\arrow[r,"\mu_2", shift right=3.5ex] & 1\arrow[dr,green]\arrow[d,green]&2\arrow[l]\arrow[r,"\mu_1", shift right=3.5ex]&1\arrow[r]&2\arrow[dl,green]\arrow[r,"\mu_2", shift right=3.5ex]&1&2\arrow[l]\\
 1' & 2'& 1'&2'\arrow[u,red]&1'\arrow[u,red]&2'\arrow[ul,red]&1'\arrow[ur,red]&2'\arrow[ul,red]\\
\end{tikzcd}$\\
\indent It is clear that this sequence has permutation $(12)$.
\end{example}
\begin{example}
Here is a reddening sequence of $Q:1\to 2$ that isn't a maximal green sequence.\\
$\begin{tikzcd}
1 \arrow[r] \arrow[d,green] & 2\arrow [d,green]\arrow[r,"\mu_1", shift right=3.5ex]  & 1&2\arrow[l]\arrow[d,green]\arrow[r,"\mu_2", shift right=3.5ex]&1\arrow[r] & 2 \arrow[r,"\mu_2", shift right=3.5ex] & 1&2\arrow[l]\arrow[d,green]\arrow[r,"\mu_2", shift right=3.5ex] & 1\arrow[r] & 2\\
1' & 2'&1'\arrow[u,red]& 2'&1'\arrow[u,red] & 2'\arrow[u,red] & 1'\arrow[u,red]& 2'&1'\arrow[u,red] & 2'\arrow[u,red]\\
\end{tikzcd}$\\
\indent It is clear that this sequence has permutation $id$.
\end{example}
\indent It is obvious that the result of $\mu_2\mu_1$ and $\mu_2\mu_1\mu_2$ are not identical, though they can be transformed into each other by a single permutation on vertices.\\
\indent We can also define the associated permutation of sequences using extended exchange matrices.\\
\begin{definition}
The \textit{matrix of a permutation}, $\sigma\in S_n$, is defined as the $n\times n$ matrix $P_\sigma=(\delta_{\sigma(i)j})$.\\
\end{definition}
\begin{definition}
\begin{enumerate}
\item For an $m\times n$ matrix $M=(M_1,\cdots, M_n)$ and a permutation $\sigma\in S_n$, if $C=(M_{\sigma(1)},\cdots,M_{\sigma(n)})$ (or equivalently, $(c_{ij})=(m_{i\sigma(j))}$), we denote this as $C=\sigma(M)$.
\item For an $n\times n$ matrix $A=(a_{ij})$ and a permutation $\sigma\in S_n$, if $D=(d_{ij})=(a_{\sigma(i)\sigma(j)})$, we denote this as $D=\tsig(A)$.
\end{enumerate}
\end{definition}
\indent It is easy to see that $C=\sigma(M)$ if and only if $C=M\psiginv$. $D=\tsig(A)$ if and only if $D=\psig A\psiginv$.\\
\indent Now let's define a new concept, namely \textit{loop sequences} which is essential to the discussion about the permutation in Chapter \ref{C1}.\\
\begin{definition} 
A \textit{loop sequence} $w$ is a sequence of mutations $\mu_{i_k}\cdots\mu_{i_1}$ on an ice quiver $(Q,F)$ such that $\mu_{i_k}\cdots\mu_{i_1}(Q) = \rho(Q)$ for some permutation $\rho$.\\
\end{definition}
\begin{example}
\indent Let $Q$ be $1\to 2$. Here is the loop sequence (1,2,1,2,1) with associated permutation $(12)$.\\
$\begin{tikzcd}
1 \arrow[r] \arrow[d,green] & 2\arrow [d,green]\arrow[r,"\mu_1", shift right=3.5ex]  & 1&2\arrow[l]\arrow[d,green]\arrow[r,"\mu_2", shift right=3.5ex]&1\arrow[r] & 2\\
1' \arrow[d,"(12)", shift left=4.5ex]& 2'&1'\arrow[u,red]& 2'&1'\arrow[u,red] & 2'\arrow[u,red]\arrow[d,"\mu_1", shift right=4.5ex]\\
 1\arrow[dr,green]&2\arrow[l]\arrow[dl,green]&1\arrow[r]\arrow[l,"\mu_1", shift left=3.5ex]&2\arrow[dl,green]\arrow[d,green]&1\arrow[l,"\mu_2", shift left=3.5ex]\arrow[d,green]&2\arrow[l]\\
 1'&2'&1'&2'\arrow[ul,red]&1'\arrow[ur,red]&2'\arrow[u,red]\\
\end{tikzcd}$
\end{example}
\begin{definition}
For any loop sequence $w$ the permutation $\rho$ such that $w(\tilde{B})=\rho(\tilde{B})$ is defined as \textit{the associated permutation of the loop sequence $w$}.\\
\end{definition}
\indent In essence for all acyclic quivers, green-to-red sequences in general and maximal green sequences in particular do not have a natural definition of the permutation: The traditional one in essence is the permutation of an associated loop sequence: Take the reddening sequence and then do mutations at sinks only, go over all non-frozen vertices and return to the origin which constitutes the loop sequence we need.\\
\section{Green sequences in bounded derived categories}
\indent In this section we will go over the basics about bounded derived categories, approximations, silting objects, simple-minded collections, torsion classes, $t$-structures and introduce the definition of numerous mutation sequences. This section mostly consists of background for Chapter \ref{C3} and Chapter \ref{C4}.\\
\subsection{Auslander-Reiten quivers}
\indent In this subsection we need to use Auslander-Reiten Theory. However I'm not going to talk about the entire Auslander-Reiten theory even though some parts of it are crucial to the understanding of Chapter \ref{C2}. Here we will just introduce two concepts, namely irreducible morphisms and Auslander-Reiten quivers. For more information on Auslander-Reiten theory we refer the reader to Chapter IV of \cite{ASS06}, \cite{ARS} and \cite{H88}.\\
\indent Moreover we are not going to talk about what triangulated categories and bounded derived categories are in details. For those who want to read about them we recommend Daniel Murfet's notes \cite{MurD1}\cite{MurD2}\cite{MurT1} for introduction and \cite{H88} for its application in the theory of finite dimensional algebras. In particular \cite{H88} is a good source for Auslander-Reiten theory in bounded derived categories which we will use extensively here.\\
\begin{definition}
Let $\catc$ be an Abelian or triangulated category, let $M,N$ be objects of $\catc$. A morphism $f\in Hom(M,N)$ is \textit{irreducible} if for any $L\in\catc$ for any $g\in Hom(M, L), h\in Hom(N, L)$ such that $f = gh$ then either $h$ is a split monomorphism or $g$ is a split epimorphism. 
\end{definition}
\indent The reason why we exclude cases where $h$ is a section and $g$ is a retraction is that these cases are simply trivial since it is very easy to decompose $f$ using $M\overset{\begin{pmatrix}1_M\\0\\\end{pmatrix}}{\longrightarrow}M\oplus K\overset{\begin{pmatrix}f &0\\\end{pmatrix}}{\longrightarrow} N$ or $M\overset{\begin{pmatrix}f\\0\\\end{pmatrix}}{\longrightarrow}N\oplus K\overset{\begin{pmatrix}1_N &0\\\end{pmatrix}}{\longrightarrow} N$. Morally speaking an irreducible morphism can be understood as a morphism that can not be decomposed in any nontrivial way.\\
\begin{example}
Let $Q$ be the quiver $1\to 2\to 3$. The inclusion $P_3\hookrightarrow P_2$ is an irreducible morphism because the morphism does not factor through any other indecomposable module in a nontrivial way.
\end{example}
\begin{example}
Let $Q$ be the quiver $1\to 2\to 3$. The inclusion $P_3\hookrightarrow P_1$ is not an irreducible morphism because the morphism does factor through $P_2$ since $P_3\hookrightarrow P_2\hookrightarrow P_1$.
\end{example}
\indent In order to define Auslander-Reiten quivers in Abelian categories we need several more concepts.\\
\begin{definition}
\cite{ASS06}(Def A.3.3) The (Jacobian) \textit{radical} of an additive $k$-category $\catc$ is the two-sided ideal $rad_{\catc}$ in $\catc$ defined by the formula $rad_{\catc}(X,Y) := \{h\in Hom_{\catc}(X, Y): 1_X-gh\text{ is invertible for any }g\in Hom_{\catc}(Y,X)\}$.
\end{definition}
\indent Note that $rad_{\catc}(X, Y)$ is a two-sided ideal. From now on when $\catc$ is clear from the context we often omit it.\\
\begin{definition}
Let $\catc$ be an additive $k$-category. Let $X, Y\in Ob\,\catc$, let $n$ be a positive integer, $X=X_0, Y=X_n$. We define $rad^n(X,Y):=\{f\in rad(X,Y):\text{for any }i\in [n]\text{ there exist }X_i\in Ob\,\catc, f_i\in rad(X_{i-1}, X_{i})\text{ such that }f=f_1\cdots f_n\}$. 
\end{definition}
\indent Note that $rad^n(X, Y)$ is a two-sided ideal. Hence we can define the space of irreducible morphisms as the following.\\
\begin{definition}
\cite{ASS06}(IV.4) Let $\catc$ be an additive $k$-category. Let $X, Y\in Ob\,\catc$. The \textit{space of irreducible morphisms} $Irr(X,Y)$ is defined as $rad(X,Y)/rad^2(X,Y)$.
\end{definition}
\indent Now we can define Auslander-Reiten quivers in Abelian categories.\\
\begin{definition}
Let $\catc$ be an Abelian category. The \textit{Auslander-Reiten (AR) quiver} $\Gamma(\catc)$ is the quiver with its vertices indecomposable objects of $\catc$ and its arrows $[M]\to[N]$ vectors of a basis of $Irr(M,N)$ as a $k$-vector space.
\end{definition}
\begin{example} Here is the Auslander-Reiten quiver of $mod kQ$ where $Q$ is the quiver $1\to 2$.
$\begin{tikzcd}
&P_1\arrow[rd] &\\
P_2\arrow[ru]& &I_1\\
\end{tikzcd}$ There is an irreducible morphism from $P_2$ to $P_1$ and an irreducible morphism from $P_1$ to $I_1$. There are only 3 indecomposable objects in $mod kQ$.
\end{example}
\indent For triangulated categories the definition is fairly complex. So we refer interested readers to \cite{H88} (I.5.5) and \cite{S09}.
\subsection{Triangulated categories and bounded derived categories}
\indent
\indent Let's recall some basic facts about triangulated categories that we will use a lot in this paper. In a triangulated category $\catt$ there exists an automorphism $[1]$ known as the \textit{translation functor}. $[n]:=([1])^n$ for any integer $n$.\\
\begin{definition}
The \textit{cone} or \textit{homotopy cokernel} of a morphism $A\overset{f}{\to}B\in\catt$ is some $C\in\catt$ such that there exists $g,h\in\catt$ such that $\triangwm{A}{f}{B}{g}{C}{h}{A[1]}$ is a distinguished triangle.
\end{definition}
\begin{definition}
The \textit{homotopy kernel} of a morphism $A\overset{f}{\to}B\in\catt$ is some $C\in\catt$ such that there exists $g,h\in\catt$ such that $\triangwm{C}{g}{A}{f}{B}{h}{C[1]}$ is a distinguished triangle.
\end{definition}
\indent Using axioms of triangulated categories any morphism $A\overset{f}{\to}B\in\catt$ has a homotopy kernel and a homotopy cokernel.\\
\indent Let's recall that bounded derived categories $D^b(\Lambda)$ are obtained by identifying homotopic chain maps in the category of chain complexes $C({\Lambda})$ and then formally invert all quasi-isomorphisms through localization. In bounded derived categories of hereditary algebras the indecomposable objects are of the form $M[i]$ where $M$ is an indecomposable module and $i$ is the amount of shifts we perform. In bounded derived categories it is true that $M,N\in mod\Lambda$ $Hom_{D^b(\Lambda)}(M[i],N[j])=\begin{cases}
Ext_{\Lambda}^{j-i}(M,N) & \text{ if }j\geq i\\
0 & \text{ if }j<i
\end{cases}$.\\
\begin{example}
Let $Q$ be $\begin{tikzcd}1\arrow[r] & 2 \arrow[r] & 3\end{tikzcd}$. Here is the Auslander-Reiten quiver of $D^b(kQ)$.\\
\begin{tikzcd}[cramped,sep=small]
\cdots&I_1[-1]\arrow[rd]& &P_1\arrow[rd] & & P_3[1]\arrow[rd] & & S_2[1]\arrow[rd] & & I_1[1] & \cdots\\
 & \cdots & P_2\arrow[ru]\arrow[rd]& &I_2\arrow[rd]\arrow[ru] & &P_2[1]\arrow[rd]\arrow[ru] & &I_2[1]\arrow[rd]\arrow[ru] & \cdots\\
\cdots& P_3\arrow[ru]& &S_2\arrow[ru]& &I_1\arrow[ru] & &P_1[1]\arrow[ru] & &P_3[2] & \cdots\\
\end{tikzcd}
\end{example}
\subsection{Approximations}
\indent According to \cite{KY12} there are bijections between silting objects, $t$-structures, co-$t$-structures and simple-minded collections in a wide range of cases and such bijections respect mutations. In \cite{BY13} more bijections are mentioned. Here we only need to cover three of them, namely silting objects, simple-minded collections and $t$-structures. To understand their mutations we must first introduce the concept of approximations.\\
\begin{definition}
Let $\catc$ be a category and $\mathcal{X}$ be one of its subcategories. Let $M\in Ob\,\catc, N\in Ob\mathcal{X}$ and $f\in Hom_{\catc}(M,N)$.
\begin{enumerate}
\item $f$ is a \textit{left-$\mathcal{X}$ approximation} if for any $N'\in Ob\mathcal{X}$ and for any $q\in Hom_{\catc}(M,N')$ we have $q$ factors through $f$.
\item $f$ is \textit{left minimal} if for any $g\in End_{\catc} N$ such that $g\circ f = f$ the morphism $g$ is an isomorphism.
\item $f$ is a \textit{minimal left-$\mathcal{X}$ approximation} if it is both left minimal and is a left-$\mathcal{X}$ approximation.
\end{enumerate}
\end{definition}
\begin{tikzcd}
M\arrow[r,"f"]\arrow[rd,"q"] & N\arrow[dashed,d,"l"]\\
 & N'\\
\end{tikzcd}
\begin{definition}
Let $\catc$ be a category and $\mathcal{X}$ be one of its subcategories. Let $M\in Ob\,\catc, N\in Ob\mathcal{X}$ and $f\in Hom_{\catc}(M,N)$.
\begin{enumerate}
\item $f$ is a \textit{right-$\mathcal{X}$ approximation} if for any $M'\in Ob\mathcal{X}$ and for any $q\in Hom_{\catc}(M',N)$ we have $q$ factors through $f$.
\item $f$ is \textit{right minimal} if for any $g\in End_{\catc} M$ such that $f\circ g = f$ the morphism $g$ is an isomorphism.
\item $f$ is a \textit{minimal right-$\mathcal{X}$ approximation} if it is both right minimal and is a right-$\mathcal{X}$ approximation.
\end{enumerate}
\end{definition}
\begin{tikzcd}
M\arrow[r,"f"]& N\\
M'\arrow[ru,"q"]\arrow[u,dashed,"l"]& \\
\end{tikzcd}
\begin{example}
Let $\catc$ be $D^b(\Lambda)$ for some finite dimensional algebra $\Lambda$ and let $\mathcal{X}$ be one of its full subcategories. If $M\in\mathcal{X}$ then $1_M$ is both a minimal left-$\mathcal{X}$ approximation and a minimal right-$\mathcal{X}$ approximation.
\end{example}
\begin{example}
Let $Q$ be $1\to 2$. Let $\catc$ be $D^b(kQ)$. Let $M = P_2$ and $\mathcal{X} = add(P_1)$. The minimal left-$\mathcal{X}$ approximation is the canonical morphism $P_2\to P_1$ induced by the inclusion $P_2\to P_1$ in the module category.
\end{example}
\begin{example}
Let $Q$ be $1\to 2$. Let $\catc$ be $D^b(kQ)$. Let $M = P_2$ and $\mathcal{X} = add(P_1)$. The minimal right-$\mathcal{X}$ approximation is the zero morphism because there is no other morphism from $P_1$ to $P_2$.
\end{example}
\subsection{Silting objects}
\indent Now let's introduce silting objects.\\ %We can think of indecomposable summands of them as indecomposable projectives in an Abelian category.\\
\begin{definition}
Let $\Lambda$ be an algebra with $n$ primitive idempotents. A \textit{silting object} $T$ of $D^b(\Lambda)$ is an object such that $T$ has $n$ direct summands and $(T,T[m])=0$ for all $m>0$. A \textit{pre-silting object} is an object that only has to satisfy the second condition.\\
\end{definition}
\begin{example}
\indent Let's take $A_3$ straight orientation as an example.\\
\begin{tikzcd}
I_1[-1]\arrow[rd]& &P_1\arrow[rd] & & P_3[1]\arrow[rd] & & S_2[1]\arrow[rd] & & I_1[1]\\
& P_2\arrow[ru]\arrow[rd]& &I_2\arrow[rd]\arrow[ru] & &P_2[1]\arrow[rd]\arrow[ru] & &I_2[1]\arrow[rd]\arrow[ru]\\
 P_3\arrow[ru]& &S_2\arrow[ru]& &I_1\arrow[ru] & &P_1[1]\arrow[ru] & &P_3[2]\\
\end{tikzcd}\\
\indent $\Lambda[i]$ is a silting object for any $i$. $T_1=P_3\oplus P_1 \oplus I_1[1]$ is also a silting object.
\end{example}
\indent Now that we already have the definition of silting objects we can discuss their mutations.\\
\begin{definition}
A \textit{forward mutation} on the direct summand $T_i$ of the silting object $T$ is $T'_i\oplus (T/T_i)$ where $T'_i$ is the homotopy cokernel of the minimal left-$add (T/T_i)$ approximation of $T_i$.\\
A \textit{backward mutation} on the direct summand $T_i$ of the silting object $T$ is $T'_i\oplus (T/T_i)$ where $T'_i$ is homotopy kernel of the minimal right-$add (T/T_i)$ approximation of $T_i$.
\end{definition}
\begin{example}
\indent Again let's take $A_3$ straight orientation as an example.\\
\begin{tikzcd}
I_1[-1]\arrow[rd]& &P_1\arrow[rd] & & P_3[1]\arrow[rd] & & S_2[1]\arrow[rd] & & I_1[1]\\
& P_2\arrow[ru]\arrow[rd]& &I_2\arrow[rd]\arrow[ru] & &P_2[1]\arrow[rd]\arrow[ru] & &I_2[1]\arrow[rd]\arrow[ru]\\
 P_3\arrow[ru]& &S_2\arrow[ru]& &I_1\arrow[ru] & &P_1[1]\arrow[ru] & &P_3[2]\\
\end{tikzcd}\\
\indent $\Lambda$ is a silting object. When we do a forward mutation at $P_3$ we get $T'=S_2\oplus P_2\oplus P_1$. When we do a forward mutation at $P_1$ now we get $T''=S_2\oplus P_2\oplus P_1[1]$. When we do another forward mutation at $P_2$ we get $T'''=S_2\oplus P_3[1]\oplus P_1[1]$.
\end{example}
\subsection{Simple-minded collections}
\indent Now let's introduce simple-minded collections. They are simple objects in some Abelian category known as hearts of $t$-structures.\\
\begin{definition}
Let $\Lambda$ be an algebra with $n$ primitive idempotents. A \textit{simple-minded collection} $\{S_i\}_{i\in [n]}$ of $D^b(\Lambda)$ is an $n$-element set such that $(S_i[\geq 0],S_j)=0$ for all $i\neq j$, $(S_i[>0], S_i) = 0$ for all $i$ and $(S_i,S_i)$ is a division algebra.\\
\end{definition}
\begin{example}
\indent As usual our example is $A_3$ straight orientation.\\
\begin{tikzcd}
I_1[-1]\arrow[rd]& &P_1\arrow[rd] & & P_3[1]\arrow[rd] & & S_2[1]\arrow[rd] & & I_1[1]\\
& P_2\arrow[ru]\arrow[rd]& &I_2\arrow[rd]\arrow[ru] & &P_2[1]\arrow[rd]\arrow[ru] & &I_2[1]\arrow[rd]\arrow[ru]\\
 P_3\arrow[ru]& &S_2\arrow[ru]& &I_1\arrow[ru] & &P_1[1]\arrow[ru] & &P_3[2]\\
\end{tikzcd}\\
\indent $\{I_1, S_2, P_3\}$ is a simple-minded collection. $\{P_3[1], P_2, I_1\}$ is also a simple-minded collection.
\end{example}
\begin{definition}
A \textit{forward mutation} on the element $S_i$ of the simple-minded collection $\{S_j\}$ is $\{S'_j\}$ where $S'_i = S_i[1]$ and $S'_j$ ($j\neq i$) is the homotopy cokernel of the minimal left-$add(S_i)$ approximation of $S_j[-1]$.\\
A \textit{backward mutation} on the element $S_i$ of the simple-minded collection $\{S_j\}$ is $\{S'_j\}$ where $S'_i = S_i[-1]$ and $S'_j$ ($j\neq i$) is the homotopy cokernel of the minimal left-$add(S_i[-1])$ approximation of $S_j$.\\
\end{definition}
\begin{example}
\indent The quiver here is $A_3$ straight orientation.\\
\begin{tikzcd}
I_1[-1]\arrow[rd]& &P_1\arrow[rd] & & P_3[1]\arrow[rd] & & S_2[1]\arrow[rd] & & I_1[1]\\
& P_2\arrow[ru]\arrow[rd]& &I_2\arrow[rd]\arrow[ru] & &P_2[1]\arrow[rd]\arrow[ru] & &I_2[1]\arrow[rd]\arrow[ru]\\
 P_3\arrow[ru]& &S_2\arrow[ru]& &I_1\arrow[ru] & &P_1[1]\arrow[ru] & &P_3[2]\\
\end{tikzcd}\
\indent $\{I_1, S_2, P_3\}$ is a simple-minded collection. When we do a forward mutation at $P_3$ we get $\{P_3[1], P_2, I_1\}$. When we do a forward mutation at $P_2$ now we get $\{S_2, P_2[1], P_1\}$. When we then do a forward mutation at $P_1$ we get $\{S_2, I_1, P_1[1]\}$.
\end{example}
\indent Now we need to introduce two more results that are crucial to Chapter \ref{C4}. Positive $c$-vectors are dimension vectors of elements of simple-minded collections. Such elements are all bricks. That is, all $c$-vectors are Schur. However we can indeed prove more. They are in fact real as well. In order to establish them we need two more lemmas.\\
\begin{lemma}
\cite{KQ15} (Prop 6.4)\label{KQ} Let $\Lambda$ be a hereditary algebra and $M,N$ be elements in a simple-minded collection of $D^b(\Lambda)$. Then $Ext^1(M,N)$ and $Ext^1(N,M)$ can not both be nonzero.
\end{lemma}
\begin{lemma}
\cite{BHIT15} (Cor 3.3.2) \label{BHIT} Let $Q$ be a quiver. Consider any maximal green sequence on any valued quiver $Q$. Then, at each step, the mutation is at a vertex of the mutated quiver $Q_0$ which is not the source of any arrow of infinite type.
\end{lemma}
\begin{lemma}
Let $k$ be an algebraically closed field. Let $\Lambda$ be a hereditary algebra over $k$. Then any $c$-vector $c$ that appears in any maximal green sequence is a real Schur root.
\end{lemma}
\begin{proof}
\indent Since the simples of $\Lambda$ are all exceptional if the lemma were incorrect then there must be some $c$-matrix in the maximal green sequence, $C$ such that all columns of $C$ are real Schur roots while one green mutation can somehow generate a root that isn't real. Here there can only be two cases, namely some mutation performed on $-v$ caused some $-w$ to be transformed into $-w'=-w-kv$ which isn't real, some mutation performed on $-v$ caused some $+w$ to be transformed into $w'=w-kv$ which isn't real. In the second case $w'$ may be positive or negative.\\
\indent For an arbitrary $c$-vector $v$ let $M_v$ be the brick such that $v$ is the dimension vector of $M_v$. In this case $\langle v,v\rangle=1-dim Ext^1(M_v,M_v)$. Hence $v$ being real is equivalent to $\langle v,v\rangle=1$.\\
\indent \textbf{Case 1}: Assume that some mutation performed on $-v$ caused some $-w$ to be transformed into $-w'=-w-kv$ which isn't real. $\langle w', w'\rangle = \langle w, w\rangle + k\langle v,w\rangle + k\langle w,v\rangle + k^2\langle v,v\rangle$. Since $Hom(M_v, M_w) = Hom(M_w, M_v) = 0$ due to $M_v$, $M_w$ being two elements in a simple-minded collection and $v, w$ are both real  $\langle w', w'\rangle = k^2+1-k\, dim Ext^1(M_v, M_w) - k\, dim Ext^1(M_w, M_v)$. Using properties of simple-minded collections $dim Ext^1(M_w, M_v)  = k$. Using Lemma \ref{KQ} we can see that $Ext^1(M_v, M_w)  = 0$, Hence $\langle w', w'\rangle = 1$. $w'$ is real.\\
\indent \textbf{Case 2}: Assume that some mutation performed on $-v$ caused some $w$ to be transformed into $-w'=w-kv$ which isn't real. Using properties of simple-minded collections it is obvious that $Ext^1(M_v, M_w) = Hom(M_v, M_w) = 0$. Regardless of whether $w'$ is positive or negative $\langle w', w'\rangle = \langle w, w\rangle - k\langle v,w\rangle - k\langle w,v\rangle + k^2\langle v,v\rangle = k^2+1 -k\,dim Hom(M_w, M_v) + k\,Ext^1(M_w, M_v)$. Using properties of simple-minded collections $dim Hom(M_w, M_v)  = k$. Using Lemma \ref{KQ} we can see that $Ext^1(M_w, M_v)  = 0$, Hence $\langle w', w'\rangle = 1$. $w'$ is real.\\
\indent The assumption has been refuted. Any $c$-vector $c$ that appears in any maximal green sequence is a real Schur root.
\end{proof}
\indent In order to prove Theorem \ref{C4T1} we need to first prove a lemma.
\begin{lemma}\label{C4L}
\indent If $-c_1, -c_2$ are negative $c$-vectors in $c$-matrix $C'$ in an maximal green sequence, $c_1$ and $c_2$ are dimension vectors of indecomposable modules $M_1$ and $M_2$. If $dim Ext^1(M_1, M_2) > 1$ then the mutation on $C'$ must not be done on $M_2$.
\end{lemma}
\begin{proof}
Assume that $-c_1$ is the $i$-th column and $-c_2$ is the $j$-th column. Using the definition of left mutations of simple-minded collections if $dim Ext^1(M_1, M_2) > 1$ then the mutation on $-c_2$ would cause $-c_1$ to be transformed into $-c_1-kc_2$ with $k>1$ because which could only happen if there are multiple edges from $i$ to $j$ \cite{KY12}. Due to Lemma \ref{BHIT} this was impossible.\\
\end{proof}
\subsection{$t$-structures}
\indent Here is the definition of $t-$structures.
\begin{definition}
A $t$-\textit{structure} on $D^b(\Lambda)$ is a pair $(D^{\leq 0},D^{\geq 0})$ such that the following holds.
\begin{enumerate}
\item For any $M\in D^b(\Lambda)$ there exists $M'\in D^{\leq 0}, M''\in D^{\geq 0}$ such that $M'\to M\to M''\to M'[1]$.
\item $D^{\leq 0}[1]\subseteq D^{\leq 0}$,  $D^{\geq 0}[1]\supseteq D^{\geq 0}$.
\item $(D^{\leq 0}[1], D^{\geq 0}) = 0$.
\end{enumerate}
\end{definition}
\begin{example}
Let $\Lambda$ be any finite dimensional algebra. The \textit{standard $t$-structure}\\ $(\cup_{m=0}^{\infty}\Lambda[m], \cup_{m=0}^{\infty}\Lambda[-m])$ is clearly a $t$-structure.
\end{example}
\indent Now let's define hearts which will be very useful for a crucial proof in Chapter \ref{C3}, namely the proof of Lemma \ref{lem:C3L1}.
\begin{definition}
The \textit{heart} of a $t$-structure $(D^{\leq 0},D^{\geq 0})$ is defined as $\mathcal{H} = D^{\leq 0}\cap D^{\geq 0}$.
\end{definition}
\begin{theorem}
\cite{BBD} Hearts of $t$-structures are Abelian categories. 
\end{theorem}
\begin{example}
Let $\Lambda$ be any finite dimensional algebra. The heart of the standard $t$-structure is $mod \Lambda$ itself which is of course Abelian.
\end{example}
\begin{definition}
\cite{EGNO}(Def 1.5.3) Given an object $X\in\catc$, then a Jordan-H\"older sequence or composition series for $X$ is a finite filtration, i.e. a finite sequence of subobject inclusions into $X$, starting with the zero objects $0=X_0\hookrightarrow X_1\hookrightarrow\cdots\hookrightarrow X_{n-1}\hookrightarrow X_n=X$ such that at each stage $i$ the quotient $X_i/X_{i-1}$ (i.e. the coimage of the monomorphism $X_{i-1}\hookrightarrow X_i$) is a simple object of $\catc$. If a Jordan-H\"older sequence for $X$ exists at all, then $X$ is said to be of finite length.
\end{definition}
\begin{definition}
An Abelian category is a \textit{length category} if all its objects have finite length.
\end{definition}
\begin{example}
Let $Q$ be the quiver $1\to 2$. The Abelian category $mod kQ$ is a length category because all its objects are finite direct sums of $P_2, I_1$ and $P_1$. $P_2$ and $I_1$ are simple and hence have length 1. $P_1$ has length 2. Hence any object in $mod kQ$ has finite length.
\end{example}
\indent Now let's introduce truncation functors associated with $t$-structures. 
\begin{lemma}
\cite{BBD}  Let $\catd$ be a triangulated category. Let $(D^{\leq 0},D^{\geq 0})$ be a $t$-structure on $\catd$. The inclusion ${\mathcal{D}^{\geq n} \rightarrow \mathcal{D}}$ has a left adjoint ${\tau_{\geq n}}$. Similarly, the inclusion ${\mathcal{D}^{\leq n} \rightarrow \mathcal{D}}$ has a right adjoint ${\tau_{\leq n}}$. These are called \textit{truncation functors}.
\end{lemma}
\begin{definition}
\cite{B07} A $t$-structure $(D^{\leq 0},D^{\geq 0})$ of the triangulated category $\catd$ is \textit{bounded} if $\catd = \cup_{i,j} (D^{\leq i}\cap D^{\geq j})$.
\end{definition}
%\begin{example}
%Let $\Lambda$ be any finite dimensional algebra. The standard $t$-structure is clearly bounded since $D^b(\Lambda) = \cup_i D^{\leq i}\cap D^{\geq i}$.
%\end{example}
%\indent $t$-structures can be mutated just like silting objects and simple-minded collections. In order to do so we first need to define torsion pairs in Abelian categories.\\
%\begin{definition}
%A \textit{torsion pair} $(\mathcal{T},\mathcal{F})$ in an Abelian category $\catc$ is a pair of two subcategories such that the following holds.
%\begin{enumerate}
%\item $Hom(\mathcal{T},\mathcal{F}) = 0$
%\item For any $M\in\catc\,\exists T\in\mathcal{T},\, F\in\mathcal{F}$ such that $0\to T\to M\to F\to 0$ is a short exact sequence.
%\item If for $M\in\catc$ and $Hom(M,\mathcal{F}) = 0$ then $M\in\mathcal{T}$.
%\item If for $M\in\catc$ and $Hom(\mathcal{T}, M) = 0$ then $M\in\mathcal{F}$.
%\end{enumerate}
%\end{definition}
%\begin{example}
%Let $\catc$ be $mod kQ$ with $Q$ being $1\to 2$ if we take $\mathcal{T} = add(P_2)$ and $\mathcal{F} = add(I_1)$ we can see that the pair $\cattf$ satisfies the conditions above and is hence a torsion pair in $\catc$.
%\end{example}
%\begin{example}
%Let $\catc$ be $mod kQ$ with $Q$ being $1\to 2$ straight orientation if we take $\mathcal{T} = add(P_1,I_1)$ and $\mathcal{F} = add(P_2)$ we can see that the pair $\cattf$ satisfies the conditions above and is hence a torsion pair in $\catc$.
%\end{example}
%\indent Now it is possible to define mutations of $t$-structures.
%\begin{definition}
%\cite{KY12}Let $\Lambda$ be a finite dimensional algebra, let $D^b(\Lambda)$ be the bounded derived category of $\Lambda$. Let $(\catc^{\leq 0},\catc^{\geq 0})$ be a $t$-structure of $D^b(\Lambda)$. Let $\cata$ be its heart. Let $\cattf$ be a torsion pair in $\cata$. The \textit{left mutation} or \textit{forward mutation} $\mu_i^+(\catc^{\leq 0},\catc^{\geq 0}) = (\catc'^{\leq 0},\catc^{\geq 0})$  where $\catc'^{\leq 0} = \{M\in\catc| H^m(M) = 0 \text{ for } m>0 \text{ and } H^0(M)\in\catt\}$, $\catc'^{\geq 0} = \{M\in\catc| H^m(M) = 0 \text{ for } m<-1 \text{ and } H^{-1}(M)\in\catf\}$. Similarly we can define \textit{right mutations} (or \textit{backward mutations}). 
%\end{definition}
\indent For more information about mutations of $t$-structures we recommend \cite{AI10}, \cite{BY13}, \cite{KQ15} and \cite{KY12}.
\subsection{Green sequences}
\indent The concepts we mentioned are related to each other due to the following result:
\begin{theorem}
\cite{KY12}(Thm 6.1, 7.12) Let $\Lambda$ be a finite-dimensional algebra over a field $k$. There are one-to-one
correspondences between
\begin{enumerate}
\item Equivalence classes of silting objects in $K^b(proj \Lambda)$.
\item Equivalence classes of simple-minded collections in $D^b(mod \Lambda)$.
\item Bounded $t$-structures of $D^b(mod \Lambda)$ with length heart.
\end{enumerate}
Moreover such correspondences are preserved under mutations.
\end{theorem}
\indent We will not explain the concepts of $K^b(proj \Lambda)$ and triangle equivalences because they are irrelevant to understanding of the problem. We refer interested readers to \cite{H88}. 
\begin{theorem}
\cite{H88}(3.3) If $\Lambda$ has finite global dimension $K^b(proj \Lambda)$ is triangle equivalent to $D^b(mod \Lambda)$.
\end{theorem}
\indent Since we have maximal green sequences it is reasonable to look at the generalization of this concept, namely $m$-maximal green sequences. In order to do so we need to define the general concept of green and red sequences. In principle any forward mutation is considered green and any backward mutation red.\\
\begin{definition}
\begin{enumerate}
\item Let $\Lambda$ be a finite dimensional algebra of finite global dimension, a mutation sequence in $D^b(\Lambda)$ is \textit{green} if it contains only forward mutations. 
\item Let $\Lambda$ be a finite dimensional algebra of finite global dimension, a mutation sequence in $D^b(\Lambda)$ is \textit{red} if it contains only backward mutations. 
\item Let $\Lambda$ be a finite dimensional algebra of finite global dimension, a mutation sequence in $D^b(\Lambda)$ is \textit{$k$-red} if it contains $k$ backward mutations. 
\item Let $\Lambda$ be a finite dimensional algebra of finite global dimension, a mutation sequence in $D^b(\Lambda)$ is \textit{$k$-green} if it contains $k$ forward mutations.
\end{enumerate}
\end{definition}
\indent Note that a $0$-red sequence is just a green one. A $0$-green sequence is just a red one. Now we can introduce $m$-maximal green sequences. For the purpose of the proof in Chapter \ref{C2} it is much better to use silting objects.\\
\begin{definition}
An \textit{$m$-maximal green sequence} is a green sequence of silting objects from $\Lambda$ to $\Lambda[m]$.
\end{definition}
\indent It is easy to see that a 1-maximal green sequence is just a maximal green sequence.\\
\begin{example}
\indent Again our example is $A_3$ straight orientation.\\
\begin{tikzcd}
I_1[-1]\arrow[rd]& &P_1\arrow[rd] & & P_3[1]\arrow[rd] & & S_2[1]\arrow[rd] & & I_1[1]\\
& P_2\arrow[ru]\arrow[rd]& &I_2\arrow[rd]\arrow[ru] & &P_2[1]\arrow[rd]\arrow[ru] & &I_2[1]\arrow[rd]\arrow[ru]\\
 P_3\arrow[ru]& &S_2\arrow[ru]& &I_1\arrow[ru] & &P_1[1]\arrow[ru] & &P_3[2]\\
\end{tikzcd}\\
\indent So $(P_1,P_2,P_3,P_1[1],P_2[1],P_3[1])$ is a 2-maximal green sequence, so is $(P_1,P_3,P_2,S_2,P_1[1],P_2[1],P_3[1])$ because they are both sequences of indecomposable objects forward mutations on which produce $\Lambda[2]$ from $\Lambda$. \\
\end{example}
\section{Tame quivers and tame hereditary algebras}
\indent In this subsection we will review the basics about tame hereditary algebras, the components of their Auslander-Reiten quivers and the components of Auslander-Reiten quivers of their bounded derived categories for they are crucial to Chapter \ref{C2}. For more details about tame algebras we would like to refer the readers to \cite{DR76}, \cite{R84} and \cite{SS06}.\\
\subsection{Tame quivers}
\begin{definition}
A \textit{tame algebra} is a $k$-algebra such that for each dimension vector there are finitely many 1-parameter families that parametrize all but finitely many indecomposable modules of the algebra.\\
\end{definition}
\begin{definition}
A \textit{tame quiver} is a quiver such that its path algebra is a tame algebra.\\
\end{definition}
\begin{example}
Here are some (connected) tame quivers, $\tilde{A_n}, \tilde{D_n}, \tilde{E_6}. \tilde{E_7}, \tilde{E_8}$. The orientation of the edges can be arbitrary as long as the quiver remains acyclic in the case of $\tilde{A_n}$.
\end{example}
$\begin{tikzcd}
\tilde{A_n} &    		&2\arrow[r]  &\cdots\arrow[r]    &i\arrow[rd]	 &\\
&1\arrow[rd]\arrow[ru]& 		  &  				&   		&n+1\\
&     				&i+1\arrow[r]&\cdots\arrow[r] 	&n\arrow[ru]& \\
\end{tikzcd}$\\
$\begin{tikzcd}
\tilde{D_n} &1\arrow[rd] &  		& 		     &		         		&					&n	\\
&		&3\arrow[r]&  4\arrow[r] & \cdots\arrow[r]           &n-1\arrow[rd]\arrow[ru] 	&\\
&2\arrow[ru]&		&   		    & 					& 					&n+1\\
\end{tikzcd}$\\
$\begin{tikzcd}
\tilde{E_6}& 1\arrow[r] & 2\arrow[r] & 3\arrow[r]\arrow[d] & 4\arrow[r] & 5\\
&		&		&  6\arrow[d] & 			& \\
&		&		&  7 & 			& \\
\end{tikzcd}$\\
$\begin{tikzcd}
\tilde{E_7}& 1\arrow[r] & 2\arrow[r] & 3\arrow[r]& 4\arrow[r]\arrow[d]  & 5\arrow[r] & 6\arrow[r] & 7\\
&		&		   &  		     & 	8		      & 		& 		&\\
\end{tikzcd}$\\
$\begin{tikzcd}
\tilde{E_8}\,\,\,\, 1\arrow[r] & 2\arrow[r] & 3\arrow[r]\arrow[d] & 4\arrow[r] & 5\arrow[r] & 6\arrow[r] & 7\arrow[r] & 8\\
	&		&  			9& 			& & & &\\
\end{tikzcd}$
\subsection{Standard stable tubes}
\indent In this subsection we are going to discuss \textit{standard stable tubes} because they are crucial to understanding of the proofs in Chapter \ref{C2}. Before that we first need several more concepts.\\
\begin{definition}
A quiver $Q$ is \textit{locally finite} if for any vertex $x\in Q_0$ the amount of vertices adjacent to $x$ is finite.
\end{definition}
\indent It is obvious that any finite quiver is locally finite.
\begin{definition}
\cite{ASS06} Let $\Gamma$ be a locally finite quiver without loops and $\tau$ be a bijection whose domain and codomain are both subsets of $\Gamma_0$. $(\Gamma,\tau)$ (and often simply $\Gamma$) is a \textit{translation quiver} if for every $x\in \Gamma_0$ such that $\tau x$ exists and for any $y\in\Gamma_0$ such that there exists at least one arrow from $y$ to $x$ the number of arrows from $y$ to $x$ is equal to the number of arrows from $\tau x$ to $y$.
\end{definition}
\begin{definition}
Let $(\Gamma, \tau)$ be a translation quiver. A point $x\in\Gamma_0$ is a \textit{projective point} if $\tau x$ is undefined. A point $x\in\Gamma_0$ is a \textit{injective point} if $\tau^{-1} x$ is undefined. 
\end{definition}
\indent Translation quivers are relevant to Auslander-Reiten theory. Using them we can have another level of abstraction when convenient.\\
\begin{lemma}
\cite{ASS06} The Auslander-Reiten quiver $\Gamma(mod \Lambda)$ of an algebra $\Lambda$ is a translation quiver, the translation $\tau$ being defined for all points $[M]$ such that $M$ is not a projective module by $\tau([M]):=[\tau M]$. 
\end{lemma}
\begin{definition}
Let $Q$ be a quiver. Let $(\catt,\tau)$ be a quiver with vertices of the form $(i,j)$ where $j > 0$ and $i\in [r]$. Here r + 1 is considered to be 1. There is an arrow from $(i,j)\to (i, j + 1)$ for any $i$ and an arrow from $(i,j)$ to $(i - 1, j - 1)$ for any $j>1$. Such a quiver is known as a \textit{stable tube of rank} $r$.
\end{definition}
%\indent To those who know what a \textit{translation quiver}
\begin{definition}
If a stable tube has rank 1 it is \textit{homogeneous}. Otherwise it is \textit{nonhomogeneous}.
\end{definition}
\indent I'm going to introduce one example of nonhomogeneous and homogeneous standard stable tubes each. For more details we recommend Chapter X of \cite{SS06}.\\
\begin{example}
\indent Here is a standard stable tube with rank 3.\\
\begin{tikzcd}
& \cdots\arrow[rd]& &\cdots\arrow[rd] & &\cdots\arrow[rd]&\\
M_{33}\arrow[rd]\arrow[ru]& &M_{13}\arrow[rd]\arrow[ru] & &M_{23}\arrow[rd]\arrow[ru] & & M_{33}\\
& M_{12}\arrow[ru]\arrow[rd]& &M_{22}\arrow[rd]\arrow[ru] & &M_{32}\arrow[rd]\arrow[ru] &\\
 M_1\arrow[ru]& &M_2\arrow[ru]& &M_3\arrow[ru] & & M_1\\
\end{tikzcd}
 $M_{ik}$ is rigid iff $k\leq 2$.\\
\indent Here $M_{ik}=\begin{tikzcd}M_{i+k-1}\\\cdots\\M_{i+1}\\M_i\end{tikzcd}$. we define the \textit{quasi-length} of $M_{ik}$ as $k$. The \textit{quasi-top} of the module is defined as $M_{i+k-1}$ and the \textit{quasi-socle} $M_i$.
\end{example}
Now let's see a homogeneous tube.\\
\begin{example}
\indent Here is a homogeneous standard stable tube.\\
\begin{tikzcd}
\cdots\arrow[d,bend left=50]\\
M_3\arrow[u,bend left=50]\arrow[d,bend left = 50]\\
M_2\arrow[u,bend left=50]\arrow[d,bend left = 50]\\
M\arrow[u, bend left = 50]\\
\end{tikzcd}
\indent Note that no module in this tube is rigid.\\
\indent Here $M_{k}=\begin{tikzcd}M\\\cdots\\M\end{tikzcd}$
\end{example}
\indent Now we need to introduce what it means for a stable tube to be \textit{standard}. Before doing so we need several new concepts.\\
\begin{definition}
\cite{SS06} Let $\catc$ be a component of the AR quiver $\Gamma(mod \Lambda)$ of some $k$-algebra $\Lambda$. For simplicity assume that $\Gamma(mod \Lambda)$ is without multiple arrows.
\begin{enumerate}
\item The \textit{path category} $k\catc$ of $\catc$ is the $k$-category with objects points in $\catc$ and morphisms from $x\in\catc_0$ to $y\in\catc_0$ the $k$-linear combinations of paths of $\catc$ from $x$ to $y$ with coefficients in $k$.
\item The ideal $M_{\catc}$ in the category $k\catc$ is defined as follows. To every non-projective point $x\in\catc_0$ corresponds a mesh in $\catc$ of the form $\begin{tikzcd}
 &y_1\arrow[ddr,"\alpha_1"]& \\
 &y_2\arrow[swap,dr,"\alpha_2"]& \\
 \tau x\arrow[uur,"\alpha'_1"]\arrow[swap,ur,"\alpha'_2"]\arrow[ddr,"\alpha'_t"] & &x\\
 &\cdots &\\
 &y_t \arrow[uur,"\alpha_t"]&\\
\end{tikzcd}$ and to this mesh we associate an element $m_x$ of $Hom_{k\catc}(\tau x,x)$ called the \textit{mesh element} defined by the formula $m_x = \Sigma_{i=1}^t \alpha'_i\alpha_i$. We denote by $M_{\catc}$ the ideal of $k\catc$ generated by all the mesh elements $m_x$ where $x$ ranges over all the non-projective points of $\catc$.
\item The \textit{mesh category} is the quotient $k$-category $k(\catc) = k\catc/M_{\catc}$.
\end{enumerate}
\end{definition}
\begin{definition}
\cite{SS06} Let $\catc$ be a component of the AR quiver $\Gamma(mod \Lambda)$ of some $k$-algebra $\Lambda$. $\catc$ is a \textit{standard component} of $\Gamma(mod \Lambda)$ if there exists an equivalence of $k$-categories $k(\catc)\approxeq ind\catc$ where $ind\catc$ is the full $k$-subcategory of $mod \Lambda$ whose objects are representatives of the isomorphism classes of the indecomposable modules in $\catc$.
\end{definition}
\begin{example} Here is the AR quiver of $\cata = mod kQ$ where $Q$ is the quiver $1\to 2$ with arrows labelled.
$\begin{tikzcd}
&P_1\arrow[rd,"b"] &\\
P_2\arrow[ru,"a"]& &I_1\\
\end{tikzcd}$ This AR quiver consists of only one component $\catc$ which is standard. Here is why. The path category has objects $P_2, P_1, I_1$ and morphisms are generated by identity morphisms, $a$ and $b$. Here $\catt(P_2,P_1) = \cata(P_2, P_1)$ and $\catt(P_1,I_1) = \cata(P_1, I_1)$. At the same time $\catt(P_2,I_1)$ is generated by $ba$ while $\cata(P_2,I_1) = 0$. The mesh ideal is generated by $ba$, hence the mesh category is equivalent to $mod kQ$. Hence $\catc$ is a standard component.
\end{example}
\subsection{Auslander-Reiten quivers of tame hereditary algebras}
\indent In this subsection we are going to discuss Auslander-Reiten quivers of basic tame hereditary algebras because information about them is slightly less well known.\\
\begin{theorem}
\indent \cite{DR76} The Auslander-Reiten quiver of a tame path algebra consists of three parts, the preprojectives, the preinjectives and the regulars.
\end{theorem}
\indent Before we can describe Auslander-Reiten quivers of tame path algebras we have to define several special translation quivers.\\
\begin{definition}
$\nn Q$ is defined as the quiver with vertices of the form $(i,j)$ where $j\in Q_0$ and $i\in\nn$. For any $i$ for any arrow from $j$ to $k$ there is an arrow from $(i,j)\to (i,k)$ and an arrow from $(i + 1,k)$ to $(i,j)$.
\end{definition}
\begin{definition}
$-\nn Q$ is defined as the quiver with vertices of the form $(i,j)$ where $j\in Q_0$ and $i\in -\nn$. For any $i$ for any arrow from $j$ to $k$ there is an arrow from $(i,j)\to (i,k)$ and an arrow from $(i,k)$ to $(i - 1,j)$.
\end{definition}
\begin{definition}
$\zz Q$ is defined as the quiver with vertices of the form $(i,j)$ where $j\in Q_0$ and $i\in\zz$. For any $i$ for any arrow from $j$ to $k$ there is an arrow from $(i,j)\to (i,k)$ and an arrow from $(i,k)$ to $(i - 1,j)$.
\end{definition}
\indent Note that the number of arrows from $i$ to $j$ in $Q$ is the same as the amount of arrows from $(a,i)$ to $(a,j)$ which is the same as the amount of arrows from $(a,j)$ to $(a-1,i)$ in $\zz Q$.\\
\indent Here are some basic properties of preprojective and preinjective components of AR quivers of basic tame hereditary algebras from \cite{ASS06} and \cite{SS06}.
\begin{theorem}\label{Tame}
\begin{enumerate}
\item The AR quiver of $kQ$ has one preprojective component which is isomorphic to $\mathbb{N}Q^{op}$
\item The AR quiver of $kQ$ has one preinjective component which is isomorphic to $-\mathbb{N}Q^{op}$.
\item All preprojective and preinjective modules in $kQ$ are rigid.
\item All but finitely many preprojectives and preinjectives are sincere.
\item There are infinitely many regular components, all of which are standard stable tubes $\mathbb{Z}A_{\infty}/(\tau^k)$.
\item All such tubes are pairwise orthogonal to each other. That is, if $M,N$ are in different tubes then $Hom (M,N) = Hom (N,M) = 0$.
\item All but at most three tubes have $k=1$. In this case we consider the component homogeneous.
\item All elements in a homogeneous tube are non-rigid, hence they and their shifts can not be summands of any silting object.
\item In a nonhomogeneous component $\mathbb{Z}A_{\infty}/(\tau^k)$ only indecomposables with quasi-length less than $k$ are rigid. In other words there are only finitely many rigid indecomposables in any nonhomogeneous component.
\item Only finitely many regular indecomposable modules are rigid. Hence only finitely many regular indecomposables and their shifts can appear in an $m$-maximal green sequence.
\end{enumerate}
\end{theorem}
\indent Now let's do an example of an AR quiver of a tame path algebra.\\
\begin{example}
The quiver is \begin{tikzcd}&2\arrow[d]&\\1\arrow[r]&5&3\arrow[l]\\&4\arrow[u]&\\\end{tikzcd}.
\indent Here is the preprojective component, $\mathcal{P}$.\\
\begin{tikzcd}
&P_1\arrow[rdd] & &\tau^{-1}P_1\arrow[rdd] &\cdots \\
&P_2\arrow[rd] & &\tau^{-1}P_2\arrow[rd] &\cdots\\
P_5\arrow[ruu]\arrow[ru]\arrow[rd]\arrow[rdd]& &\tau^{-1}P_5\arrow[ruu]\arrow[ru]\arrow[rd]\arrow[rdd] & &\tau^{-2}P_5\cdots\\
&P_3\arrow[ru] & &\tau^{-1}P_3\arrow[ru]  &\cdots\\
&P_4\arrow[ruu] & &\tau^{-1}P_4\arrow[ruu] &\cdots\\
\end{tikzcd}\\
Here is the preinjective component, $\mathcal{Q}$.\\
\begin{tikzcd}
\cdots&\tau I_1\arrow[rdd] & &I_1 &\\
\cdots&\tau I_2\arrow[rd] & &I_2 &\\
\cdots \tau I_5\arrow[ruu]\arrow[ru]\arrow[rd]\arrow[rdd]& & I_5\arrow[ruu]\arrow[ru]\arrow[rd]\arrow[rdd] & &\\
\cdots&\tau I_3\arrow[ru] & &I_3  &\\
\cdots&\tau I_4\arrow[ruu] & &I_4 &\\
\end{tikzcd}\\
\indent Here are the regular components. There are infinitely many homogeneous tubes and 3 nonhomogeneous ones. All objects in the homogeneous ones are non-rigid. The quasi-simple in the homogeneous tubes has dimension vector is (1,1,1,1,2). The quasi-simples in the three nonhomogeneous tubes have dimension vectors (1,1,0,0,1) and (0,0,1,1,1), (1,0,1,0,1) and (0,1,0,1,1), (1,0,0,1,1) and (0,1,1,0,1) respectively.\\
\end{example}
\indent Finally let's discuss Auslander-Reiten quivers of $D^b(kQ)$. For a tame quiver $Q$ there are infinitely many components of $D^b(kQ)$ consisting of shifts of preprojectives and preinjectives that are isomorphic to $\mathbb{Z}Q^{op}$. Let's label these components \textit{transjective}. The transjective component containing $\Lambda[m]$ is labelled $\mathcal{P}_m$.\\
\indent There are also infinitely many regular components. There are at most 3 nonhomogeneous tubes in $mod kQ[m]$ for any $m$. There are also infinitely many homogeneous tubes in $mod kQ[m]$ for any $m$. However since no module in a homogeneous tube is rigid they don't affect our problem.\\
\section{Wall-and-chamber structures}
\indent In this section we will discuss miscellaneous topics on the wall-and-chamber structures, namely picture groups and alternative definitions of maximal green sequences. This section is mostly relevant to Chapter \ref{C1} and Chapter \ref{C3}. For more details we recommend \cite{GHKK14}\cite{Mul15}\cite{BST17}\cite{BST18A}\cite{BST18B}\cite{IOTW15} and \cite{IT17}.\\
\subsection{Picture groups}\label{Picgr}
\indent We also need to use the concept of the picture groups in order to prove the formula in Chapter \ref{C1}. For more details about picture groups we recommend \cite{IT17}.\\
\indent Let's recall that for a quiver of finite type, any dimension vector of an indecomposable representation is referred to as a \textit{root}. Let $(-,-)$ be the standard Euclidean product. Let $D(\beta)\subseteq\mathbb{R}^n$, $D(\beta)= \{x\in\mathbb{R}^n: (x,\beta)=0,\ (x,\beta')\leq 0\text{ when }\beta'\subseteq\beta\}$. Here $\beta'\subseteq\beta$ means the unique indecomposable representation of dimension vector $\beta'$ is a subrepresentation of the unique indecomposable representation of dimension vector $\beta$. $D(\beta)$ for all these roots divide $\mathbb{R}^n$ into \textit{compartments} (or \textit{chambers}). The boundary of each compartment is the union of some $D(\beta)$ which we call \textit{walls}.\cite{IT17}\cite{IOTW4}\\
\begin{tikzpicture}
\draw [->] (-3,0) -- (3,0);
\node at (3.3,0) {$x$};
\node at (2,0.3) {$D(P_2)$};
\draw [->] (0,-3) -- (0,3);
\node at (0,3.3) {$y$};
\node at (-0.6,1.3) {$D(I_1)$};
\draw [-] (0,0) -- (3,-3);
\node at (1.2,-1.9) {$D(P_1)$};
\end{tikzpicture}
%\begin{tikzpicture}
%\draw (0,0) arc (90:130:2.5);
%\draw (0,0) arc (30:120:2.5);
%\end{tikzpicture}
\begin{example}
\indent The figure above is the wall-and-chamber structure of $Q:1\to 2$. $D(P_1), D(P_2)$\\$D(I_1)$ are the walls and there are 5 chambers. Note that $D(P_1)$ is only half of the line $x + y = 0$.
\end{example}
\begin{definition}
\cite{IT17} A \textit{picture group} of a cluster quiver of finite type $Q$ is a group $G(Q)=\langle S|R\rangle$ with $S$ in bijection with the set of real Schur roots (the generator for $\beta$ is $x(\beta)$) and $R$ the set of relations $x(\beta_i)x(\beta_j)=\Pi x(\gamma_k)$ with $\gamma_k$ running over all these real Schur roots which are linear combinations $\gamma_k = a_k\beta_i+b_k\beta_j$ with $a_k/b_k$ increasing (going from 0/1 where $\gamma_1=\beta_j$ to 1/0 where $\gamma_k=\beta_i$) for any pair $(\beta_i,\beta_j)$ such that they are Hom-orthogonal and $Ext(\beta_i,\beta_j)=0$.\\
\end{definition}
\indent Picture group elements can be used to denote mutations. $x_{\beta}$ denotes a green mutation on $c$-vector $-\beta$. $x^{-1}_{\beta}$ denotes a red mutation on $c$-vector $\beta$.\\
\indent Now we need to restrict the case to $A_n$ straight orientation. In this case due to Lemma \ref{CVA} the roots are in the form $\pm\beta_{ij}$ where $\beta_{ij}=e_j-e_i$ ($0<i<j<n$, $e_0$ is defined as the zero vector). The root $\beta_{ij}$ corresponds to the picture group generator $x_{ij}$ which we will define right now.\\
\indent We often simplify the notation of $x(\beta_{ij})$ to $x_{ij}$ which we use interchangeably with $x(\beta_{ij})$. The picture group for $A_n$ straight orientation is $G(A_n)=\{S|R\}$, $S=\{x_{ij}|0\leq i<j\leq n\}$, $R=\{x_{ij}x_{kl}=x_{kl}x_{ij}|[i,j]\cap[k,l]=\emptyset, [i,j]\text{ or }[k,l], $ i,j,k,l \\are distinct.\}$\cup\{x_{jk}x_{ij}=x_{ij}x_{ik}x_{jk}|0\leq i<j<k\leq n\}$.\\
\indent Igusa and Todorov proved \cite{IT17} that there exists a bijection between the set of maximal green sequences and the set $\mathcal{P}(c)$ of positive expressions of the Coxeter element of the picture group for any acyclic valued quiver of finite type which applies to $A_n$ straight orientation.\\
\subsection{Alternative definitions of maximal green sequences}
\indent In the following theorem by Kiyoshi Igusa multiple equivalent definition of maximal green sequences was introduced. The full version of Igusa's results includes more discussions about the wall-and-chamber structure which we will not discuss here. To understand more about the wall-and-chamber structure we suggest that the reader reads \cite{GHKK14}\cite{Mul15}\cite{BST17}\cite{BST18A}\cite{BST18B} and \cite{IOTW15}.
\begin{definition}
Let $\Lambda$ be a finite dimensional algebra. $\Lambda-$modules $M_1,\cdots,M_m$ are a \textit{backward Hom-orthogonal sequence} if $Hom_\Lambda(M_i,M_j)=0$ for $i>j$.
\end{definition}
\begin{example}
In the $Q:1\to 2$ example $\{P_2, P_1\}$ is a backward $Hom$-orthogonal sequence because $Hom(P_1, P_2) = 0$.
\end{example}
\begin{example}
In the $Q:1\to 2$ example $\{P_1, P_2\}$ is not a backward $Hom$-orthogonal sequence because $Hom(P_2, P_1) = k$.
\end{example}
\begin{definition}
Let $\Lambda$ be a finite dimensional algebra. A backward Hom-orthogonal sequence $M_1,\cdots,M_m$ is \textit{maximal} if no other modules can be inserted into the sequence preserving the property of backward Hom-orthogonality.
\end{definition}
\begin{example}
In the $Q:1\to 2$ example $\{P_2, P_1\}$ is not a maximal backward $Hom$-orthogonal sequence even though it is a backward $Hom$-orthogonal sequence because it can be extended to $P_2, P_1, I_1$ without losing backward $Hom$-orthogonality.
\end{example}
\begin{example}
In the $Q:1\to 2$ example $\{I_1, P_2\}$ is a maximal backward $Hom$-orthogonal sequence because $Hom(P_2, I_1) = 0$ and that there is no place to fit any other module in the sequence. For example $P_1$ can not be inserted before $P_2$ because $Hom(P_2, P_1)\neq 0$. At the same time it can not be inserted after $I_1$ because $Hom(P_1, I_1)\neq 0$. Hence the sequence can not be extended to include $P_1$.
\end{example}
\indent In fact $Q:1\to 2$ only has two maximal backward $Hom$-orthogonal sequences, namely $I_1,P_2$ and $P_2, P_1, I_1$. They are exactly the same as the two $c$-vectors in maximal green sequences of $Q$. Is this just a coincidence? No.
\begin{definition}
Let $\{M_1,\cdots, M_k\}$ be a fixed finite sequence of Schur objects in $mod \Lambda$. An \textit{Harder-Narasimhan (HN) filtration} with respect to $\{M_i\}$ aka an HN filtration of an object $X$ is k short exact sequences $0\to X_i\to X_{i-1}\to X_{i-1}/X_i$ such that $X_{i-1}/X_i\in\cate{M_i}$. 
\end{definition}
\begin{definition}
Let $\{M_1,\cdots, M_k\}$ be a fixed finite sequence of Schur objects in $mod \Lambda$. $\{M_1,\cdots, M_k\}$ is an \textit{finite Harder-Narasimhan (HN) system} if any $X\in mod \Lambda$ has a unique HN filtration with respect to $\{M_i\}$.
\end{definition}
\begin{example}
In the $Q:1\to 2$ example $\{I_1, P_2\}$ is a finite HN system because any module $X\in mod kQ$ has a unique HN filtration. In particular the unique HN filtration of $P_1$ is $0\to P_2\to P_1\to I_1\to 0$.
\end{example}
\begin{example}
In the $Q:1\to 2$ example $\{P_2, P_1, I_1\}$ is a finite HN system because any module $X\in mod kQ$ has a unique HN filtration.
\end{example}
\indent There are only two finite HN systems of $kQ$ where $Q:1\to 2$. They are exactly the same as the two maximal backward $Hom$-orthogonal sequences and the two maximal green sequences. This is in fact a general result.\\
\begin{theorem}
\cite{I17} Let $\Lambda$ be a finite dimensional hereditary algebra over a field $K$. Let $\beta_1,\cdots,\beta_m\in \mathbb{N}^n$ be any finite sequence of nonzero, nonnegative integer vectors. Then the following are equivalent.\label{thm:3}
\begin{enumerate}
%\item[(a)] There is a nonlinear stability function $Z_t:K_0\Lambda\to \mathbb{C}$ which is green and has exactly $m$ semistable pairs $(M_i,t_i)$ with $t_1<t_2<\cdots<t_m$ so that $\dim M_i=\beta_i$ for all $i$.%(making all pairs stable) 
%\item[(a)] There is a generic green path $\gamma:\mathbb{R}\to\mathbb{R}^n$ which crosses the walls $D(M_i)$, $i=1,\cdots,m$ in that order, and no other walls, so that $\dim M_i=\beta_i$ for all $i$.
\item[(a)] There exist $\Lambda$-modules $M_m,\cdots,M_1$ with $\dim M_i=\beta_i$ which form a finite Harder-Narasimhan system for $\Lambda$. 
\item[(b)] There exist a finite sequence of Schurian $\Lambda$-modules $\{M_1,\cdots,M_m\}$ with $\dim M_i=\beta_i$ such that $\{M_1,\cdots,M_m\}$ is a \textit{maximal backward Hom-orthogonal sequence}.
\item[(c)] There is a maximal green sequence for $\Lambda$ of length $m$ whose $i$th mutation is at the $c$-vector $\beta_i$. 
\end{enumerate}
\end{theorem}
\indent Such results have been generalized to the case of $m$-maximal green sequences in Chapter \ref{C3}.

%\section{Quiver folding}
%\indent In this section we will introduce the theory of quiver folding. Folding theory has been in folklore for a while. Despite not being used to prove anything in the thesis we are still going to discuss it here in details.\\
%\begin{definition}
%Let $B$ be an $n\times n$ exchange matrix and let $\rho\in S_n$ be a permutation. If $\rho(B)=B$ then $\rho$ is a \textit{symmetry} of $B$. The group of symmetries of $B$ is the \textit{Symmetry Group of} $B$ which we denote as $Sym\ B$. An exchange matrix with a non-trivial symmetry group is a \textit{symmetric exchange matrix}. Any nontrivial subgroup of $B$ is a \textit{Symmetry Subgroup of} $B$. The symmetry group of a valued quiver is defined as the symmetry group of the exchange matrix of the valued quiver.\\   
%\end{definition}
%\indent A symmetry subgroup $G$ acts on the extended exchange matrices $\tilde{B}$ in the obvious way, namely for some $\rho\in G$ $\rho \tilde{B}:=\rho(\tilde{B})$. We can also define right group actions of elements of $G$ on the set of $c$-vectors similarly, namely for any $v=(v_i)\in \mathbb{Z}^n$, $v\rho:=(v_{\rho(i)})$. Let $C(\mathcal{A})$ be the set of $c$-vectors of a cluster algebra of geometric type $\mathcal{A}(B)$. Orbits of $c\in C(\mathcal{A})$ is denoted as $Gc$. It is easy to see that all orbits are finite. $Fc := \Sigma_{c'\in Gc}\ c'$ is the \textit{folded version of} $c$.\\
%\indent Now we need to fold vertices first. For any nontrivial subgroup of $S_n$ let $n'$ be the set of orbits of the canonical group action of $S_n$ on $[n]$. Hence there exists maps from $[n]$ to $[n']$. Pick some surjection $f$ from $[n]$ to $[n']$ such that $f$ maps each orbit to one element of $[n']$. $f$ is a \textit{vertex folding map}. We sometimes abuse notations and identify $f(i)$ and $Gi$ when we do not need to specify $f$.\\
%\begin{definition}
%(1)For any valued quiver $Q$ and its symmetry subgroup $G$, the \textit{folded version of $Q$ with respect to $G$} is defined as below:\\
%For each valued arrow $i\overset{(d_{ij},d_{ji})}{\longrightarrow} j$ it is replaced by $Gi\overset{(d_{ij}|Gi|,d_{ji}|Gj|)}{\longrightarrow} Gj$.\\
%(2)For any symmetric exchange matrix $B$ and its symmetry subgroup $G$, the \textit{folded version of $B$ with respect to $G$} is defined as $\tilde{B}=(b'_{kl})$ where $b'_{GiGj}=b_{ij}|Gj|$. \\ \cite{Sal14}\cite{BHIT15}\\
%\end{definition}
%\begin{definition}
%For any symmetry subgroup $G$ of $B$ any extended exchange matrix $\tilde{B}=(B',C')$ such that $B'$ is symmetric is a \textit{symmetric extended exchange matrix with respect to} $G$ if for any $i\in [n]$ $|Gc_i|=|G_i|$ and for any $\rho\in G$, $Gc_{\rho(i)}=Gc_i$.\\
%\end{definition}
%\indent Any symmetric extended exchange matrix can be folded.\\
%\begin{definition}
%For any symmetry subgroup $G$ of $B$ any symmetric extended exchange matrix $\tilde{B}=(B',C')'$ with respect to $G$. Then for any vertex folding map $f$ $F\tilde{B}:=(FB',FC')'$ is the \textit{folded version of} $\tilde{B}$ with $FC'=(\tilde{c}_1,\cdots, \tilde{c}_n)$ defined below: $\tilde{c}_i=\Sigma_{g\in f^{-1}(i)} \tilde{c}_g/|Gg|$ where $\tilde{c}_{gi}:=\Sigma_{k\in f^{-1}(i)} c_{gk}$.\\
%\end{definition}
%\indent It is easy to see that if $c_i=e_i$ then $C=I_n$ then $C'=I_{n'}$ and if $C=-I_n$ then $C'=-I_{n'}$. Hence a framed valued quiver is folded into a framed valued quiver and a coframed valued quiver is folded into a coframed valued quiver. Also positive $c$-vectors are folded into positive ones and negative $c$-vectors are folded into negative ones.\\
%\begin{definition}
%For any symmetry subgroup $G$ of $B$ a mutation sequence $w=\Pi_{i=m}^1 \mu_{k_i}$ starting from an extended exchange matrix $\tilde{B}=(B',C')'$ with $B$ symmetric is \textit{symmetric with respect to } a symmetry subgroup $G$ if the following holds:\\
%1.$w$ is in the form $w=\Pi_{i=m}^1 \Pi_{j\in Gk_i} \mu_j$, which roughly means that vertices in any orbit is "mutated together".\\
%2.$\Pi_{i=m'}^1 \mu_{k_i} \tilde{B}$ is symmetric.\\
%\end{definition}
%\begin{definition}
%For any symmetry subgroup $G$ of a symmetric extended exchange matrix $\tilde{B}$ for any symmetric mutation sequence $w=\Pi_{i=m}^1 \Pi_{j\in Gk_i} \mu_j$, \textit{the folded version of $w$} is defined as $Gw:=\Pi_{i=m}^1 \mu'_{Gk_i}$.
%\end{definition}
%\indent It is easy to see that folding symmetric reddening sequences results in reddening sequences. Also folding symmetric green sequences results in green sequences. Folding symmetric maximal green sequences results in maximal green sequences.\\
%\begin{theorem}
%For any symmetry subgroup $G$ of a symmetric extended exchange matrix $\tilde{B}$ for any symmetric mutation sequence $w$, $F\circ w=Gw\circ F$.\\
%\end{theorem}
%\begin{proof}
%\indent Let's assume that the length of $Gw$ is 1. When the theorem has been proven in this particular case the rest is clear from induction. Hence let's assume $w=\Pi_{j\in Gi} \mu_j$ and $Gw=\mu'_{Gi}$. Note that $b_{jk}=0$ for any $j,k\in Gi$ since otherwise $j$ and $k$ would not be in the same orbit. It is also clear from symmetry that the set of vertices in $[n]$ that is a source of any valued arrow with some $j\in Gi$ its target is independent of the choice of $j$, which we denote as $P^{-}(Gi)$. Similarly we can define $P^+(Gi)$. The set of vertices in $[n]\backslash Gi$ that is not connected to any $j\in Gi$ is denoted as $I(Gi)$. Using the invariance lemmas it is easy to see that for any $j\in I(i)$ mutations at any element of $Gi$ does not affect the $j$-th row and the $j$-th column at all.\\
%\indent Assume that we do mutations on a symmetric extended exchange matrix $\tilde{B}=(B',C')'$. Let $B=(b_{jk})$, $C=(c_{jk})$. Hence $b'_{GjGk}=b_{ij}|Gj|$. $c'_{GjGk}=\Sigma_{l\in Gj k\in Gm}c_{lm}$.  Let $\tilde{C}=\mu'_{Gi}(C)=(c'_{j'k'})$ $w(\tilde{B})=(\tilde{b}_{jk})$ $w(\tilde{C})=(\tilde{c}_{jk})$ $F\tilde{B}=(\hat{B}',\hat{C}')'$ $F\circ w(\tilde{B})=(B_1',C_1')'$  $Gw\circ F(\tilde{B})=(B_2',C_2')'$. $B_1=(b^1_{jk})$ $C_1=(c^1_{jk})$ $B_2=(b^2_{jk})$ $C_2=(b^1_{jk})$.\\
%\indent Let's first calculate the left hand side. If $j$ or $k\in I(Gi)$ it is clear that $\tilde{b}_{jk}=b_{jk}$. If $j\in i$ or $k=i$ $\tilde{b}_{jk}=-b_{jk}$ due to the fact that $b_{jj'}=0$ for any $j,j'\in Gi$. Otherwise it is easy to see that $\tilde{b}_{jk}=b_{jk}+|Gi|sp(b_{ji},b_{ik})$ due to two facts: First of all, before a mutation at $l\in Gi$, the $l$-th row and column of the exchange matrix can not be affected by all previous mutations since between elements of $Gi$ there are no connections. Secondly for all $l\in Gi$, $b_{jl}$ and $b_{lk}$ are independent of $l$. Similarly, if $k\in Gi$ $\tilde{c}_{jk}=-c_{jk}$. Otherwise $\tilde{c}_{jk}=c_{jk}+\Sigma_{l\in Gi} sp(c_{jl}, b_{ik})$. Hence $b^1_{GjGk}=-b_{jk}|Gk|$ if $j$ or $k$ is in $Gi$. Otherwise $b^1_{GjGk}=b_{jk}|Gk|+|Gi||Gk|sp(b_{ji},b_{ik})$. If $k\in Gi$ $c^1_{GjGk}=-|Gk|\bar{c}_{GjGk}$. Otherwise $c^1_{GjGk}=\bar{c}_{GjGk}+\Sigma_{l\in Gi} sp(\bar{c}_{GjGl}, b_{ik})=\bar{c}_{GjGk}+|Gi|sp(\bar{c}_{GjGi},b_{ik})$.\\
%\indent Now let's calculate the right hand side. $\hat{b}_{GjGk}=|Gk|b_{jk}$. $\hat{c}_{GjGk}=|Gk|\bar{c}_{GjGk}$. If $j$ or $k\in Gi$ $b^2_{GjGk}=-b_{jk}|Gk|$. Otherwise $b^2_{GjGk}=b_{jk}|Gk|+sp(|Gi|b_{ji}, |Gk|b_{ik})=b_{jk}|Gk|+|Gi||Gk|sp(b_{ji}, b_{ik})$. If $j$ or $k\in Gi$ $c^2_{GjGk}=-|Gk|\bar{c}_{GjGk}$. Otherwise $c^2_{GjGk}=|Gk|\bar{c}_{GjGk}+sp(|Gi|\bar{c}_{GjGi},b_{ik}|Gk|)=|Gk|\bar{c}_{GjGk}+|Gi||Gk|sp(\bar{c}_{GjGi},b_{ik})$.\\
%\indent Hence $F\circ w=Gw\circ F$ has been proven.\\
%\end{proof}
