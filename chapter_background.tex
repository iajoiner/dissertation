\chapter{Background}\label{CB}
\section{Notations}
\indent In this paper $k$ is an algebraically closed field, all algebras will be finitely dimensional $k$-algebras. When the category I are discussing is clearly $\mathcal{C}$ $(M,N)$ will be an abbreviation of $Hom_{\mathcal{C}}(M,N)$. $(M[>0],N)$ is the union of $(M[k],N)$ for all $k>0$. If $\mathcal{P}, \mathcal{Q}$ are subcategories $D^b(\Lambda)$ then $(\mathcal{P}, \mathcal{Q}):=\bigcup\limits_{M\in\mathcal{P}}\bigcup\limits_{N\in\mathcal{Q}}(M,N)$ and $(\mathcal{P}[>0], \mathcal{Q}):=\bigcup\limits_{k>0}(\mathcal{P}[k], \mathcal{Q})$. Let $S$ be a set of objects in an additive category $\catc$. $add(S)$ is defined as the set of all elements of $\catc$ such that they are finite direct sums of objects in $S$.\\
\indent The definition of green and red are consistent with that of \cite{Kel11} and \cite{BDP13}. It is the exact opposite definition of green and red in \cite{BHIT15}.
\section{Quivers and path algebras}
\subsection{Quivers}
\begin{definition}
A \textit{quiver} $Q$ is a quadruple $(Q_0,Q_1,s,t)$ with $Q_0$ and $Q_1$ sets and $s,t$: $Q_1\rightarrow Q_0$. An element of $Q_0$ is a \textit{vertex} of $Q$. An element of $Q_1$ is an \textit{arrow} of $Q$. $s$ maps each arrow to its smyce and $t$ maps each arrow to its target.
\end{definition}
\indent Intuitively I can think of elements of $Q_1$ as oriented edges. Any arrow has a unique smyce and a unique target both of which are vertices. This is how I obtain the $s$ and $t$ maps. Unless necessary I generally omit the $s$ and $t$ and denote a quiver by $Q = (Q_0, Q_1)$.\\
\begin{example}
$\begin{tikzcd} 
1 \arrow[r] & 2 & 1 & 2 \arrow[l, shift right]\arrow[l, shift left]\arrow[r] & 3 & 1\arrow[r] & 2\arrow[r] & 3\\
\end{tikzcd}$\\
These three are quivers.
\end{example}
\indent Now I need to define subquivers.\\
\begin{definition}
A \textit{subquiver} $Q' = (Q'_0,Q'_1, s', t')$ in a quiver $Q = (Q_0,Q_1, s, t)$ is a quiver such that $Q'_0\subseteq Q_0$, $Q'_1\subseteq Q_1$, $s|_{Q'_1} = s'$ and $t|_{Q'_1} = t'$.
\end{definition}
\indent From now on I generally do not distinguish betIen $s$ and $s'$, $t$ and $t'$.\\
\indent Not all quivers are useful for the purpose of this paper. This is why I need to add restrictions. In order to do so I need to introduce several definitions.\\
\begin{definition}
An \textit{oriented cycle} in a quiver $Q$ is a subquiver $Q' = (Q'_0, Q'_1, s, t)$ such that $Q'_0 = \{v_0,v_1,\cdots, v_{k-1}\},Q'_1 = \{a_0,a_1,\cdots, a_{k-1}\}, s(a_i) = v_i, t(a_i) = v_{i+1}$. Here $v_k$ is defined as $v_0$.
\end{definition}
\begin{definition}
A \textit{$k$-cycle} is an oriented cycle with $k$ vertices.
\end{definition}
\begin{definition}
A \textit{loop} in a quiver $Q$ is an arrow from a vertex to itself, that is, a 1-cycle.
\end{definition}
\begin{definition}
A \textit{cluster quiver} is a quiver without loops or 2-cycles. 
\end{definition}
\indent In all but Chapter \ref{C3} and a part of Chapter \ref{C4} all quivers I discuss will be acyclic. Here is the definition of an acyclic quiver.\\
\begin{definition}
An \textit{acyclic quiver} is a quiver without any oriented cycles. 
\end{definition}
\subsection{Path Algebras}
\begin{definition}
A \textit{path} in a quiver $Q$ is a sequence of vertices $\{v_0,\cdots, v_k\}$ and a sequence of arrows $\{a_0,\cdots, a_{k-1}\}$ if $k>0$ such that $t(a_i) = v_{i+1}$, $ s(a_i) = v_i$ for any $i = 0,1,\cdots, k-1$. The smyce of the path is $v_0$ and the sink is $v_k$.
\end{definition}
\indent Paths with length 0 are known as \textit{trivial paths}. A trivial path only has a single vertex $v_0$ and no arrows at all. All other paths are uniquely determined by their arrows.\\
\indent Now I need to define multiplication of paths. In order to do so I need to define compatibility and concatenation.\\
\begin{definition}
Paths $v,w$ are \textit{compatible} if $t(v) = s(w)$.
\end{definition}
\begin{definition}
The \textit{concatenation} of compatible paths $v = \{a_0,\cdots, a_{k-1}\}$ and $w = \{b_0,\cdots, b_{l-1}\}$ is  $vw = \{a_0,\cdots, a_{k-1}, b_0,\cdots, b_{l-1}\}$.
\end{definition}
\begin{definition}
The \textit{path algebra} of a quiver $Q$ is a $k$-algebra generated by all the paths of the quiver. Multiplication of paths $v$ and $w$ is defined as the concatenation if they are compatible and 0 if they aren't, 
\end{definition}
\indent I only discuss path algebras of acyclic quivers in this thesis because my results are only about finite dimensional algebras.\\
\section{Mutations, mutation sequences and the associated permutation}
\indent In this section I will introduce mutations of quivers and matrices, different kinds of mutation sequences including green sequences, maximal green sequences, reddening sequences and loop sequences. I will also define the associated permutation. Results in this section are mostly used in Chapters \ref{C1} and \ref{C4}.\\
\subsection{Mutation of quivers}
\indent The concept of maximal green sequences has many different equivalent definitions. I will use a simple definition using quiver mutations in this subsection. Later I will introduce other definitions. Mutations of cluster quivers at vertex $k$ are defined in the following way:
\begin{enumerate}
\item For any pair of arrows $i\to k$ and $k\to j$ add an arrow $i\to j$.
\item Reverse all arrows starting from or ending up in $k$.
\item Delete all 2-cycles that are formed due to process (1) and (2).
\end{enumerate}
\begin{definition}
\begin{enumerate}
\item The \textit{framed quiver} $\hat{Q}$ of $Q$ is obtained from $Q$ by adding a vertex $i'$ and an arrow $i\rightarrow i'$ for every $i\in Q$.
\item The \textit{coframed quiver} $\breve{Q}$ of $Q$ is obtained from $Q$ by adding a vertex $i'$ and an arrow $i'\rightarrow i$ for every $i\in Q$.
\item An \textit{ice quiver} is a quiver $Q$ where a possibly empty set, $F\subseteq Q_0$, consists of vertices that can not mutate.
\end{enumerate}
\end{definition}
\indent An ice quiver $(Q,F)$ can not mutate at elements of $F$, so I call them \textit{frozen vertices}.
\begin{definition}
A non-frozen vertex $i$ is \textit{green} if and only if no arrow from a frozen vertex to $i$ exists. Otherwise it is \textit{red}.\cite{Kel11}
\end{definition}
$\begin{tikzcd}
1 \arrow[r] \arrow[green]{d} & 2\arrow[green]{d}\arrow[r,"\mu_1"]  & 1&2\arrow[l]\arrow[green]{d}\\
1' & 2'&1'\arrow[red]{u}& 2'\\
\end{tikzcd}$
\begin{definition}
A \textit{green sequence} is a sequence $\mathbf{i}=(i_1, i_2,\cdots, i_N)$ such that for all $1\leq t\leq N$ the vertex $i_t$ is green in the partially mutated ice quiver $\hat{Q}(\mathbf{i},t)=\mu_{i_{t-1}}\cdots\mu_2\mu_1(\hat{Q})$.
\end{definition}
\begin{definition}
A \textit{maximal green sequence} is a green sequence such that $\hat{Q}(\mathbf{i},N)$ does not have any green vertices.
\end{definition}
\begin{example} For quiver $1\to 2$ here is one of its two maximal green sequences.\\
 $\begin{tikzcd}
1 \arrow[r] \arrow[d] & 2\arrow [d]\arrow[r,"\mu_1"]  & 1&2\arrow[l]\arrow[d]\arrow[r,"\mu_2"]&1\arrow[r] & 2\\
1' & 2'&1'\arrow[u]& 2'&1'\arrow[u] & 2'\arrow[u]\\
\end{tikzcd}$
\end{example}
\indent I also need the definition of reddening sequences which are generalized versions of maximal green sequences in order to discuss the phenomenon of almost morphism finiteness in Chapter \ref{C2}.\\
\begin{definition}
A \textit{red-to-green sequence} or a \textit{reddening sequence}, is a sequence $\mathbf{i}=(i_1, i_2,\cdots, i_N)$ that transforms $\hat{Q}$ to a quiver $\hat{Q}(\mathbf{i},N) = \mu_{i_N}\cdots\mu_2\mu_1(\hat{Q})$ such that $\hat{Q}(\mathbf{i},N)$ does not have any green vertices.\cite{Mul15}\\
\end{definition}
\indent Now let's define a new concept, namely \textit{loop sequences} which is essential to the discussion about the permutation in Chapter \ref{C1}.\\
\begin{definition} 
A \textit{loop sequence} $w$ is a sequence of mutations $\mu_{i_k}\cdots\mu_{i_1}$ on an ice quiver $(Q,F)$ such that $\mu_{i_k}\cdots\mu_{i_1}(Q) = \rho(Q)$ for some permutation $\rho$.\\
\end{definition}
\subsection{Mutation of matrices}
\indent I can also use $c$\textit{-vectors} for this purpose. To do so I need to reinterpret mutations of cluster quivers in terms of mutations of matrices. I recall that cluster quivers correspond to \textit{exchange matrices} as defined below. For more details I recommend \cite{FZ01} and \cite{FZ06}.\\
\begin{definition}
\cite{FZ01} An \textit{exchange matrix} of a cluster quiver $Q$ with $n$ vertices is an $n\times n$ matrix such that $b_{ij}$ is the number of arrows from $i$ to $j$ minus the number of arrows from $j$ to $i$.
\end{definition}
\indent It is easy to see that exchange matrices of cluster quivers are always antisymmetric which is not true in the more general case of \textit{valued quivers} which I won't discuss in this paper. Moreover there is a 1-1 correspondence betIen antisymmetric exchange matrices and cluster quivers.\\
\indent Mutations of exchange matrices are defined here which exactly agree with mutations of cluster quivers.\\
\begin{definition}
\cite{FZ01} If I mutate an $n\times n$ exchange matrix $B = (b_{ij})$ at $k$ I obtain $B' = (b'_{ij})$ defined here.
$b'_{ij} = \begin{cases}
-b_{ij} & \text{if }i = k\text{ or }j = k\\
b_{ij} + b_{ik}|b_{kj}| & \text{if }b_{ik}b_{kj} > 0\\
b_{ij} & \text{in all other cases}
\end{cases}$
\end{definition}
\indent Each partially mutated ice quiver corresponds to an \textit{extended exchange matrix} defined below.\\
\begin{definition}
The \textit{extended exchange matrix} $B'$ corresponding to a partially mutated ice quiver $Q'$ is an $2n\times n$ matrix with the rows corresponding to vertices $\{1,2,\cdots, n, 1', 2',\cdots n'\}$ while the columns corresponds to the vertices $\{1,2,\cdots, n\}$. Here I use the number $n+i$ to represent $i'$. $b_{ij}$ is the number of arrows from $i$ to $j$ minus the number of arrows from $j$ to $i$.
\end{definition}
\indent An extended exchange matrix $B'$ has an upper and lower square submatrices, $B$ and $C$ respectively. The lower square matrix $C$ is known as the $C$\textit{-matrix}. Column vectors of an $C$-matrix are known as $c$\textit{-vectors}. A $c$-vector is positive if all its entries are non-negative and at least one is positive. A $c$-vector is negative if all its entries are non-positive and at least one is negative. Due to \cite{GHKK14} a $c$-vector is either positive or negative which is known as \textit{sign coherence}.\\
\indent A mutation on vertex $k$ is \textit{green} if the $c$-vector $c_k$ before the mutation is negative.  A mutation on vertex $k$ is \textit{red} if the $c$-vector $c_k$ before the mutation is positive. A \textit{maximal green sequence} is a mutation sequence from $C=-I_n$ to a permuted version of $I_n$ We can use a sequence of $c$-vectors to denote a maximal green sequence because we can use the $c$-vector corresponding to vertex $k$ to represent mutation at vertex $k$.\\
\subsection{Permutations}
\indent All reddening sequences have associated permutations. When comparing the quivers obtained from transforming the same framed quiver using two different reddening sequences, it is easy to see that they are just one permutation away from each other: If you do a correct permutation of vertices (that means both rows and columns together) you can transform one such matrix into another. In particular any quiver obtained by using a reddening sequence to transform a framed quiver is one permutation away from the coframed quiver.\\
\indent Here is the formal definition of such a permutation:\\
\begin{definition}
A \textit{permutation} from an ice quiver $(Q,F)$ to $(Q',F)$ is an isomorphism of quivers $Q\rightarrow Q'$ that preserve $F$.\cite{BDP13}\\
\end{definition}
\indent I have a result from \cite{BDP13} which helps us define the permutation:\\
\begin{theorem}
Let $Q$ be a cluster quiver and let $Q'$ be a quiver that is a result of a reddening sequence on $\hat{Q}$, then  $Q'$ equals to a permutation of $\breve{Q}$\cite{BDP13}\\
Due to the proposition, for a reddening sequence $\mathbf{i}=(i_1,\cdots, i_N)$, for some $\rho\in S_n$ I have $\mu_{i_N}\cdots\mu_{i_1}\hat{Q}=\rho\breve{Q}$.\\
\end{theorem}
\begin{definition}
The \textit{permutation of a reddening sequence} $\mathbf{i}$ is $\rho$ for which $\mu_{i_N}\cdots\mu_{i_1}\hat{Q}=\rho\breve{Q}$.\cite{GM14}\\
\end{definition}
\indent Here is one of the simplest examples of the concept of the permutation:\\
$\begin{tikzcd}
1 \arrow[r] \arrow[d] & 2\arrow [d]\arrow[r,"\mu_1"]  & 1&2\arrow[l]\arrow[d]\arrow[r,"\mu_2"]&1\arrow[r] & 2\\
1' \arrow[d,"\mu_2"]& 2'&1'\arrow[u]& 2'&1'\arrow[u] & 2'\arrow[u]\\
 1\arrow[d]\arrow[dr]&2\arrow[l]\arrow[r,"\mu_1"]&1\arrow[r]&2\arrow[dl]\arrow[r,"\mu_2"]&1&2\arrow[l]\arrow[u,"(12)"]\\
 1'&2'\arrow[u]&1'\arrow[u]&2'\arrow[ul]&1'\arrow[ur]&2'\arrow[ul]\\
\end{tikzcd}$\\
\indent It is obvious that the result of $\mu_2\mu_1$ and $\mu_2\mu_1\mu_2$ are not identical, though they can be transformed into each other by a single permutation on vertices.\\
\indent The associated permutation is not only intuitive when defined on reddening sequences. In the case of loop sequences they are even more intuitive. Using the $c$-vector theorem from \cite{ST12} I can see that the only nontrivial effect loop sequences can have on an extended matrix is a permutation. Hence loop sequences are equivalent to permutations on an extended matrix.\\
\indent I can also define the associated permutation of sequences using extended exchange matrices.\\
\begin{definition}
The \textit{matrix of a permutation}, $\sigma\in S_n$, is defined as the $n\times n$ matrix $P_\sigma=(\delta_{\sigma(i)j})$.\\
\end{definition}
\begin{definition}
(1)For an $m\times n$ matrix $M=(M_1,\cdots, M_n)$ and a permutation $\sigma\in S_n$, if $C=(M_{\sigma(1)},\cdots,M_{\sigma(n)})$ (or equivalently, $(c_{ij})=(m_{i\sigma(j))}$), I denote this as $C=\sigma(M)$.\\
(2)For an $n\times n$ matrix $A=(a_{ij})$ and a permutation $\sigma\in S_n$, if $D=(d_{ij})=(a_{\sigma(i)\sigma(j)})$, I denote this as $D=\tsig(A)$.\\
\end{definition}
\indent It is easy to see that $C=\sigma(M)$ if and only if $C=M\psiginv$. $D=\tsig(A)$ if and only if $D=\psig A\psiginv$.\\
\begin{definition}
For any loop sequence $w$ the permutation $\rho$ such that $w(\tilde{B})=\rho(\tilde{B})$ is defined as \textit{the associated permutation of the loop sequence $w$}.\\
\end{definition}
\indent In essence for all acyclic quivers, green-to-red sequences in general and maximal green sequences in particular do not have a natural definition of the permutation: The traditional one in essence is the permutation of an associated loop sequence: Take the reddening sequence and then do mutations at sinks only, go over all non-frozen vertices and return to the origin which constitutes the loop sequence I need.\\
\section{Bounded derived categories}
\indent In this section I will go over the basics about bounded derived categories, approximations, silting objects, simple-minded collections, torsion classes, $t$-structures and introduce the definition of numerous mutation sequences. This section mostly consists of background for chapters \ref{C3} and \ref{C4}.\\
\subsection{Bounded derived categories}
\indent In this subsection we need to use Auslander-Reiten Theory. However I'm not going to talk about the entire Auslander-Reiten theory even though some parts of it are crucial to the understanding of Chapter \ref{C2}. For Auslander-Reiten theory I refer the reader to Chapter IV of \cite{ASS06} and \cite{ARS}.\\
\indent I'm not going to talk about what triangulated categories and bounded derived categories are in details. For those who want to read about them I recommend Daniel Murfet's notes \cite{MurD1}\cite{MurD2}\cite{MurT1} for introduction and \cite{H88} for its application in the theory of finite dimensional algebras. In particular \cite{H88} is a good source for Auslander-Reiten theory in bounded derived categories which we will use extensively here.\\
\indent Let's recall that bounded derived categories $D^b(\Lambda)$ are obtained by identifying homotopic chain maps in the category of chain complexes $C({\Lambda})$ and then formally invert all quasi-isomorphisms through localization. In bounded derived categories of hereditary algebras the indecomposable objects are of the form $M[i]$ where $M$ is an indecomposable module and $i$ is the amount of shifts I perform. In bounded derived categories it is true that $M,N\in mod\Lambda$ $Hom_{D_b(\Lambda)}(M[i],N[j])=\begin{cases}
Ext_{\Lambda}^{j-i}(M,N) & \text{ if }j\geq i\\
0 & \text{ if }j<i
\end{cases}$.\\
\begin{example}
Let $Q$ be $\begin{tikzcd}1\arrow[r] & 2 \arrow[r] & 3\end{tikzcd}$. Here is the Auslander-Reiten quiver of $D^b(kQ)$.\\
\begin{tikzcd}[cramped,sep=small]
\cdots&I_1[-1]\arrow[rd]& &P_1\arrow[rd] & & P_3[1]\arrow[rd] & & S_2[1]\arrow[rd] & & I_1[1] & \cdots\\
 & \cdots & P_2\arrow[ru]\arrow[rd]& &I_2\arrow[rd]\arrow[ru] & &P_2[1]\arrow[rd]\arrow[ru] & &I_2[1]\arrow[rd]\arrow[ru] & \cdots\\
\cdots& P_3\arrow[ru]& &S_2\arrow[ru]& &I_1\arrow[ru] & &P_1[1]\arrow[ru] & &P_3[2] & \cdots\\
\end{tikzcd}
\end{example}
\subsection{Approximations}
\indent According to \cite{KY12} there are bijections betIen silting objects, $t$-structures, co-$t$-structures and simple-minded collections in a wide range of cases and such bijections respect mutations. In \cite{BY13} more bijections are mentioned. Here I only need to cover three of them, namely silting objects, simple-minded collections and $t$-structures. To understand their mutations I must first introduce the concept of approximations.\\
\begin{definition}
Let $\catc$ be a category and $\mathcal{X}$ be one of its subcategories. If $M\in Ob\catc, N\in Ob\mathcal{X}$, a morphism $f\in Hom_{\catc}(M,N)$ is a \textit{minimal left-$\mathcal{X}$ approximation} if for any $g\in End_{\catc} N$ such that $g\circ f = f$ $g$ is an isomorphism and for any $N'\in Ob\mathcal{X}$ for any $q\in Hom_{\catc}(M,N')$ I have $q$ factors through $f$.\\
\end{definition}
\begin{tikzcd}
M\arrow[r,"f"]\arrow[rd,"q"] & N\arrow[dashed,d,"l"]\\
 & N'\\
\end{tikzcd}
\begin{definition}
Let $\catc$ be a category and $\mathcal{X}$ be one of its subcategories. If $N\in Ob\catc, M\in Ob\mathcal{X}$, A morphism $f\in Hom_{\catc}(M,N)$ is a \textit{minimal right-$\mathcal{X}$ approximation} if for any $g\in End_{\catc} M$ such that $f\circ g = f$ $g$ is an isomorphism and for any $M'\in Ob\mathcal{X}$ for any $q\in Hom_{\catc}(M',N)$ I have $q$ factors through $f$.\\
\end{definition}
\begin{tikzcd}
M\arrow[r,"f"]& N\\
M'\arrow[ru,"q"]\arrow[u,dashed,"l"]& \\
\end{tikzcd}
\begin{example}
Let $\catc$ be $D^b(\Lambda)$ for some finite dimensional algebra $\Lambda$ and let $\mathcal{X}$ be one of its full subcategories. If $M\in\mathcal{X}$ then $1_M$ is both a minimal left-$\mathcal{X}$ approximation and a minimal right-$\mathcal{X}$ approximation.
\end{example}
\begin{example}
Let $Q$ be $A_2$ straight orientation. Let $\catc$ be $D^b(kQ)$. Let $M = P_2$ and $\mathcal{X} = add(P_1)$. The minimal left-$\mathcal{X}$ approximation is the canonical morphism $P_2\to P_1$ induced by the inclusion $P_2\to P_1$ in the module category.
\end{example}
\begin{example}
Let $Q$ be $A_2$ straight orientation. Let $\catc$ be $D^b(kQ)$. Let $M = P_2$ and $\mathcal{X} = add(P_1)$. The minimal right-$\mathcal{X}$ approximation is the zero morphism because there is no other morphism from $P_1$ to $P_2$.
\end{example}
\subsection{Silting objects}
\indent Now let's introduce silting objects. I can think of indecomposable summands of them as indecomposable projectives in an Abelian category.\\
\begin{definition}
Let $\Lambda$ be an algebra with $n$ primitive idempotents. A \textit{silting object} $T$ of $D^b(\Lambda)$ is an object such that $T$ has $n$ direct summands and $(T,T[m])=0$ for all $m>0$. A \textit{pre-silting object} is an object that only has to satisfy the second condition.\\
\end{definition}
\begin{example}
\indent Let's take $A_3$ straight orientation as an example.\\
\begin{tikzcd}
I_1[-1]\arrow[rd]& &P_1\arrow[rd] & & P_3[1]\arrow[rd] & & S_2[1]\arrow[rd] & & I_1[1]\\
& P_2\arrow[ru]\arrow[rd]& &I_2\arrow[rd]\arrow[ru] & &P_2[1]\arrow[rd]\arrow[ru] & &I_2[1]\arrow[rd]\arrow[ru]\\
 P_3\arrow[ru]& &S_2\arrow[ru]& &I_1\arrow[ru] & &P_1[1]\arrow[ru] & &P_3[2]\\
\end{tikzcd}\\
\indent $\Lambda[i]$ is a silting object for any $i$. $T_1=P_3[1]\oplus P_1[1] \oplus I_1[1]$ is also a silting object.
\end{example}
\indent Now that I already have the definition of silting objects I can discuss their mutations.\\
\begin{definition}
A \textit{forward mutation} on the direct summand $T_i$ of the silting object $T$ is $T'_i\oplus (T/T_i)$ where $T'_i$ is the cone/homotopy cokernel of the minimal left-$add (T/T_i)$ approximation of $T_i$.\\
A \textit{backward mutation} on the direct summand $T_i$ of the silting object $T$ is $T'_i\oplus (T/T_i)$ where $T'_i$ is homotopy kernel/ [-1] of the cone/ of the minimal right-$add (T/T_i)$ approximation of $T_i$.
\end{definition}
\begin{example}
\indent Again let's take $A_3$ straight orientation as an example.\\
\begin{tikzcd}
I_1[-1]\arrow[rd]& &P_1\arrow[rd] & & P_3[1]\arrow[rd] & & S_2[1]\arrow[rd] & & I_1[1]\\
& P_2\arrow[ru]\arrow[rd]& &I_2\arrow[rd]\arrow[ru] & &P_2[1]\arrow[rd]\arrow[ru] & &I_2[1]\arrow[rd]\arrow[ru]\\
 P_3\arrow[ru]& &S_2\arrow[ru]& &I_1\arrow[ru] & &P_1[1]\arrow[ru] & &P_3[2]\\
\end{tikzcd}\\
\indent $\Lambda$ is a silting object. When I do a forward mutation at $P_3$ I get $T'=S_2\oplus P_2\oplus P_1$. When I do a forward mutation at $P_1$ now I get $T''=S_2\oplus P_2\oplus P_1[1]$. When I do another forward mutation at $P_2$ I get $T'''=S_2\oplus P_3[1]\oplus P_1[1]$.
\end{example}
\subsection{Simple-minded collections}
\indent Now let's introduce simple-minded collections. They are simple objects in some Abelian category known as hearts of $t$-structures.\\
\begin{definition}
Let $\Lambda$ be an algebra with $n$ primitive idempotents. A \textit{simple-minded collection} $\{S_i\}_{i\in [n]}$ of $D^b(\Lambda)$ is an $n$-element set such that $(S_i[\geq 0],S_j)=0$ for all $i\neq j$, $(S_i[>0], S_i) = 0$ for all $i$, $(S_i,S_i)$ is a division algebra.\\
\end{definition}
\begin{example}
\indent As usual my example is $A_3$ straight orientation.\\
\begin{tikzcd}
I_1[-1]\arrow[rd]& &P_1\arrow[rd] & & P_3[1]\arrow[rd] & & S_2[1]\arrow[rd] & & I_1[1]\\
& P_2\arrow[ru]\arrow[rd]& &I_2\arrow[rd]\arrow[ru] & &P_2[1]\arrow[rd]\arrow[ru] & &I_2[1]\arrow[rd]\arrow[ru]\\
 P_3\arrow[ru]& &S_2\arrow[ru]& &I_1\arrow[ru] & &P_1[1]\arrow[ru] & &P_3[2]\\
\end{tikzcd}\\
\indent $\{I_1, S_2, P_3\}$ is a simple-minded collection. $\{P_3[1], P_2, I_1\}$ is also a simple-minded collection.
\end{example}
\begin{definition}
A \textit{forward mutation} on the element $S_i$ of the simple-minded collection $\{S_j\}$ is $\{S'_j\}$ where $S'_i = S_i[1]$ and $S'_j$ ($j\neq i$) is the cone/homotopy cokernel of the minimal left-$add(S_i)$ approximation of $S_j[-1]$.\\
A \textit{backward mutation} on the element $S_i$ of the simple-minded collection $\{S_j\}$ is $\{S'_j\}$ where $S'_i = S_i[-1]$ and $S'_j$ ($j\neq i$) is the cone/homotopy cokernel of the minimal left-$add(S_i[-1])$ approximation of $S_j$.\\
\end{definition}
\begin{example}
\indent The quiver here is $A_3$ straight orientation.\\
\begin{tikzcd}
I_1[-1]\arrow[rd]& &P_1\arrow[rd] & & P_3[1]\arrow[rd] & & S_2[1]\arrow[rd] & & I_1[1]\\
& P_2\arrow[ru]\arrow[rd]& &I_2\arrow[rd]\arrow[ru] & &P_2[1]\arrow[rd]\arrow[ru] & &I_2[1]\arrow[rd]\arrow[ru]\\
 P_3\arrow[ru]& &S_2\arrow[ru]& &I_1\arrow[ru] & &P_1[1]\arrow[ru] & &P_3[2]\\
\end{tikzcd}\
\indent $\{I_1, S_2, P_3\}$ is a simple-minded collection. When I do a forward mutation at $P_3$ I get $\{P_3[1], P_2, I_1\}$. When I do a forward mutation at $P_2$ now I get $\{S_2, P_2[1], P_1\}$. When I then do a forward mutation at $P_1$ I get $\{S_2, I_1, P_1[1]\}$.
\end{example}
\subsection{$t$-structures}
\indent Here is the definition of $t-$structures.
\begin{definition}
A $t$-\textit{structure} on $D^b(\Lambda)$ is a pair $(D^{\leq 0},D^{\geq 0})$ such that the following holds.
\begin{enumerate}
\item For any $M\in D^b(\Lambda)$ there exists $M'\in D^{\leq 0}, M''\in D^{\geq 0})$ such that $M'\to M\to M''\to M'[1]$.
\item $D^{\leq 0}[1]\subseteq D^{\leq 0}$,  $D^{\geq 0}[1]\supseteq D^{\geq 0}$.
\item $(D^{\leq 0}[1], D^{\geq 0}) = 0$
\end{enumerate}
\end{definition}
\begin{example}
Let $\Lambda$ be any finite dimensional algebra. $(\cup_{m=1}^{\infty}\Lambda[m], \cup_{m=0}^{\infty}\Lambda[-m])$ is clearly a $t$-structure.
\end{example}
\indent Now let's define hearts which will be very useful for a crucial proof in \ref{C3}, namely the proof of Lemma \ref{lem:C3L1}.
\begin{definition}
The \textit{heart} of a $t$-structure $(D^{\leq 0},D^{\geq 0})$ is defined as $\mathcal{H} = D^{\leq 0}\cap D^{\geq 0}$
\end{definition}
\begin{theorem}
\cite{BBD} Hearts of $t$-structures are Abelian categories. 
\end{theorem}
\indent $t$-structures can be mutated just like silting objects and simple-minded collections. In order to do so I first need to define torsion pairs in Abelian categories.\\
\begin{definition}
A \textit{torsion pair} $(\mathcal{T},\mathcal{F})$ in an Abelian category $\catc$ is a pair of two subcategories such that the following holds.
\begin{enumerate}
\item $Hom(\mathcal{T},\mathcal{F}) = 0$
\item For any $M\in\catc\,\exists T\in\mathcal{T},\, F\in\mathcal{F}$ such that $0\to T\to M\to F\to 0$ is a short exact sequence.
\item If for $M\in\catc$ $Hom(M,\mathcal{F}) = 0$, $M\in\mathcal{T}$.
\item If for $M\in\catc$ $Hom(\mathcal{T}, M) = 0$, $M\in\mathcal{F}$.
\end{enumerate}
\end{definition}
\begin{example}
Let $\catc$ be $mod kQ$ with $Q$ being $A_2$ straight orientation if I take $\mathcal{T} = add(P_2)$ and $\mathcal{F} = add(I_1)$ I can see that the pair $\cattf$ satisfies the conditions above and is hence a torsion pair in $\catc$.
\end{example}
\begin{example}
Let $\catc$ be $mod kQ$ with $Q$ being $A_2$ straight orientation if I take $\mathcal{T} = add(P_1,I_1)$ and $\mathcal{F} = add(P_2)$ I can see that the pair $\cattf$ satisfies the conditions above and is hence a torsion pair in $\catc$.
\end{example}
\indent Now it is possible to define mutations of $t$-structures.
\begin{definition}
\cite{KY12}Let $\Lambda$ be a finite dimensional algebra, let $D^b(\Lambda)$ be the bounded derived category of $\Lambda$. Let $(\catc^{\leq 0},\catc^{\geq 0})$ be a $t$-structure of $D^b(\Lambda)$. Let $\cata$ be its heart. Let $\cattf$ be a torsion pair in $\cata$. The \textit{left mutation} or \textit{forward mutation} $\mu_i^+(\catc^{\leq 0},\catc^{\geq 0}) = (\catc'^{\leq 0},\catc^{\geq 0})$  where $\catc'^{\leq 0} = \{M\in\catc| H^m(M) = 0 \text{ for } m>0 \text{ and } H^0(M)\in\catt\}$, $\catc'^{\geq 0} = \{M\in\catc| H^m(M) = 0 \text{ for } m<-1 \text{ and } H^{-1}(M)\in\catf\}$. Similarly we can define \textit{right mutations} (or \textit{backward mutations}). 
\end{definition}
\subsection{Green sequences}
\indent Since I have maximal green sequences it is reasonable to look at the generalization of this concept, namely $m$-maximal green sequences. In order to do so I need to define the general concept of green and red sequences. In principle any forward mutation is considered green and any backward mutation red.\\
\begin{definition}
\begin{enumerate}
\item Let $\Lambda$ be a finite dimensional algebra of finite global dimension, a mutation sequence in $D^b(\Lambda)$ is \textit{green} if it contains only forward mutations. 
\item Let $\Lambda$ be a finite dimensional algebra of finite global dimension, a mutation sequence in $D^b(\Lambda)$ is \textit{red} if it contains only backward mutations. 
\item Let $\Lambda$ be a finite dimensional algebra of finite global dimension, a mutation sequence in $D^b(\Lambda)$ is \textit{$k$-red} if it contains $k$ backward mutations. 
\item Let $\Lambda$ be a finite dimensional algebra of finite global dimension, a mutation sequence in $D^b(\Lambda)$ is \textit{$k$-green} if it contains $k$ forward mutations.
\end{enumerate}
\end{definition}
\indent Note that a $0$-red sequence is just a green one. A $0$-green sequence is just a red one. Now I can introduce $m$-maximal green sequences. For the purpose of the proof in Chapter \ref{C2} it is much better to use silting objects.\\
\begin{definition}
An \textit{$m$-maximal green sequence} is a green sequence of silting objects from $\Lambda$ to $\Lambda[m]$.
\end{definition}
\indent It is easy to see that a 1-maximal green sequence is just a maximal green sequence.\\
\begin{example}
\indent Again my example is $A_3$ straight orientation.\\
\begin{tikzcd}
I_1[-1]\arrow[rd]& &P_1\arrow[rd] & & P_3[1]\arrow[rd] & & S_2[1]\arrow[rd] & & I_1[1]\\
& P_2\arrow[ru]\arrow[rd]& &I_2\arrow[rd]\arrow[ru] & &P_2[1]\arrow[rd]\arrow[ru] & &I_2[1]\arrow[rd]\arrow[ru]\\
 P_3\arrow[ru]& &S_2\arrow[ru]& &I_1\arrow[ru] & &P_1[1]\arrow[ru] & &P_3[2]\\
\end{tikzcd}\\
\indent So $(P_1,P_2,P_3,P_1[1],P_2[1],P_3[1])$ is a 2-maximal green sequence, so is $(P_1,P_3,P_2,S_2,P_1[1],P_2[1],P_3[1])$ because they are both sequences of indecomposable objects forward mutations on which produce $\Lambda[2]$ from $\Lambda$. \\
\end{example}
\section{Tame quivers and tame hereditary algebras}
\indent In this subsection I will review the basics about tame hereditary algebras, the components of their Auslander-Reiten quivers and the components of Auslander-Reiten quivers of their bounded derived categories for they are crucial to Chapter \ref{C2}. For more details about tame algebras I would like to refer the readers to \cite{DR76}, \cite{R84} and \cite{SS06}.\\
\subsection{Tame quivers}
\begin{definition}
A \textit{tame algebra} is a $k$-algebra such that for each dimension there are finitely many 1-parameter families that parametrize all but finitely many indecomposable modules of the algebra.\\
\end{definition}
\begin{definition}
A \textit{tame quiver} is a quiver such that its path algebra is a tame algebra.\\
\end{definition}
\begin{example}
Here are all the (connected) tame quivers, $\tilde{A_n}, \tilde{D_n}, \tilde{E_6}. \tilde{E_7}, \tilde{E_8}$.
\end{example}
$\begin{tikzcd}
\tilde{A_n} &    		&2\arrow[r]  &\cdots\arrow[r]    &i\arrow[rd]	 &\\
&1\arrow[rd]\arrow[ru]& 		  &  				&   		&n+1\\
&     				&i+1\arrow[r]&\cdots\arrow[r] 	&n\arrow[ru]& \\
\end{tikzcd}$\\
$\begin{tikzcd}
\tilde{D_n} &1\arrow[rd] &  		& 		     &		         		&					&n	\\
&		&3\arrow[r]&  4\arrow[r] & \cdots\arrow[r]           &n-1\arrow[rd]\arrow[ru] 	&\\
&2\arrow[ru]&		&   		    & 					& 					&n+1\\
\end{tikzcd}$\\
$\begin{tikzcd}
\tilde{E_6}& 1\arrow[r] & 2\arrow[r] & 3\arrow[r]\arrow[d] & 4\arrow[r] & 5\\
&		&		&  6\arrow[d] & 			& \\
&		&		&  7 & 			& \\
\end{tikzcd}$\\
$\begin{tikzcd}
\tilde{E_7}& 1\arrow[r] & 2\arrow[r] & 3\arrow[r]& 4\arrow[r]\arrow[d]  & 5\arrow[r] & 6\arrow[r] & 7\\
&		&		   &  		     & 	8		      & 		& 		&\\
\end{tikzcd}$\\
$\begin{tikzcd}
\tilde{E_8}\,\,\,\, 1\arrow[r] & 2\arrow[r] & 3\arrow[r]\arrow[d] & 4\arrow[r] & 5\arrow[r] & 6\arrow[r] & 7\arrow[r] & 8\\
	&		&  			9& 			& & & &\\
\end{tikzcd}$
\subsection{Auslander-Reiten quivers of tame hereditary algebras}
\indent In this subsection I'm going to discuss Auslander-Reiten quivers of basic tame hereditary algebras because information about them is slightly less well known.\\
\begin{theorem}
\indent The Auslander-Reiten quiver of a tame path algebra consists of three parts, the preprojectives, the preinjectives and the regulars.
\end{theorem}
\indent Here are some basic properties of preprojective and preinjective components of AR quivers of basic tame hereditary algebras.
\begin{theorem}
\begin{enumerate}
\item The AR quiver of $kQ$ has one preprojective component which is isomorphic to $\mathbb{N}Q^{op}$
\item The AR quiver of $kQ$ has one preinjective component which is isomorphic to $-\mathbb{N}Q^{op}$.
\item All preprojective and preinjective modules in $kQ$ are rigid.
\item All but finitely many preprojectives and preinjectives are sincere.
\item There are infinitely many regular components, all of which are standard tubes $\mathbb{Z}A_{\infty}/(\tau^k)$.
\item All but at most three tubes have $k=1$. In this case I consider the component homogeneous.
\item All elements in a homogeneous tube are non-rigid, hence they and their shifts can not be summands of any silting object.
\item In a nonhomogeneous component $\mathbb{Z}A_{\infty}/(\tau^k)$ only indecomposables with quasi-length less than $k$ are rigid. In other words there are only finitely many rigid indecomposables in any nonhomogeneous component.
\item Only finitely many regular indecomposable modules are rigid. Hence only finitely many regular indecomposables and their shifts can appear in an $m$-maximal green sequence.
\end{enumerate}
\end{theorem}
\indent I'm going to introduce one example of nonhomogeneous and homogeneous standard stable tubes each. For more details I recommend Chapter X of \cite{SS06}.\\
\begin{example}
\indent Here is a standard stable tube with rank 3.\\
\begin{tikzcd}
& \cdots\arrow[rd]& &\cdots\arrow[rd] & &\cdots\arrow[rd]&\\
M_{33}\arrow[rd]\arrow[ru]& &M_{13}\arrow[rd]\arrow[ru] & &M_{23}\arrow[rd]\arrow[ru] & & M_{33}\\
& M_{12}\arrow[ru]\arrow[rd]& &M_{22}\arrow[rd]\arrow[ru] & &M_{32}\arrow[rd]\arrow[ru] &\\
 M_1\arrow[ru]& &M_2\arrow[ru]& &M_3\arrow[ru] & & M_1\\
\end{tikzcd}
 $M_{ik}$ is rigid iff $k\leq 2$.\\
\indent Here $M_{ik}=\begin{tikzcd}M_{i+k-1}\\\cdots\\M_{i+1}\\M_i\end{tikzcd}$. I define the \textit{quasi-length} of $M_{ik}$ as $k$.
\end{example}
Now let's see a homogeneous tube.\\
\begin{example}
\indent Here is a homogeneous standard stable tube.\\
\begin{tikzcd}
\cdots\arrow[d,bend left=50]\\
M_3\arrow[u,bend left=50]\arrow[d,bend left = 50]\\
M_2\arrow[u,bend left=50]\arrow[d,bend left = 50]\\
M\arrow[u, bend left = 50]\\
\end{tikzcd}
\indent Note that no module in this tube is rigid.\\
\indent Here $M_{k}=\begin{tikzcd}M\\\cdots\\M\end{tikzcd}$
\end{example}
\indent Now let's do an example of an AR quiver of a tame path algebra.\\
\begin{example}
The quiver is \begin{tikzcd}&2\arrow[d]&\\1\arrow[r]&5&3\arrow[l]\\&4\arrow[u]&\\\end{tikzcd}.
\indent Here is the preprojective component, $\mathcal{P}$.\\
\begin{tikzcd}
&P_1\arrow[rdd] & &\tau^{-1}P_1\arrow[rdd] &\cdots \\
&P_2\arrow[rd] & &\tau^{-1}P_2\arrow[rd] &\cdots\\
P_5\arrow[ruu]\arrow[ru]\arrow[rd]\arrow[rdd]& &\tau^{-1}P_5\arrow[ruu]\arrow[ru]\arrow[rd]\arrow[rdd] & &\tau^{-2}P_5\cdots\\
&P_3\arrow[ru] & &\tau^{-1}P_3\arrow[ru]  &\cdots\\
&P_4\arrow[ruu] & &\tau^{-1}P_4\arrow[ruu] &\cdots\\
\end{tikzcd}\\
Here is the preinjective component, $\mathcal{Q}$.\\
\begin{tikzcd}
\cdots&\tau I_1\arrow[rdd] & &I_1 &\\
\cdots&\tau I_2\arrow[rd] & &I_2 &\\
\cdots \tau I_5\arrow[ruu]\arrow[ru]\arrow[rd]\arrow[rdd]& & I_5\arrow[ruu]\arrow[ru]\arrow[rd]\arrow[rdd] & &\\
\cdots&\tau I_3\arrow[ru] & &I_3  &\\
\cdots&\tau I_4\arrow[ruu] & &I_4 &\\
\end{tikzcd}\\
\indent Here are the regular components. There are infinitely many homogeneous tubes and 3 nonhomogeneous ones. All objects in the homogeneous ones are non-rigid. The quasi-simple in the homogeneous tubes has dimension vector is (1,1,1,1,2). The quasi-simples in the three nonhomogeneous tubes have dimension vectors (1,1,0,0,1) and (0,0,1,1,1), (1,0,1,0,1) and (0,1,0,1,1), (1,0,0,1,1) and (0,1,1,0,1) respectively.\\
\end{example}
\indent Finally let's discuss Auslander-Reiten quivers of $D^b(kQ)$. For a tame quiver $Q$ there are infinitely many components of $D^b(kQ)$ consisting of shifts of preprojectives and preinjectives that are isomorphic to $\mathbb{Z}Q^{op}$. Let's label these components \textit{transjective}. The transjective component containing $\Lambda[m]$ is labelled $\mathcal{P}_m$.\\
\indent There are also infinitely many regular components. There are at most 3 nonhomogeneous tubes in $mod kQ[m]$ for any $m$. There are also infinitely many homogeneous tubes in $mod kQ[m]$ for any $m$. However since no module in a homogeneous tube is rigid they don't affect my problem.\\
\section{Wall-and-chamber Structures}
\indent In this section I will discuss the basics of the wall-and-chamber structure, picture groups and alternative definitions of maximal green sequences. This section is mostly relevant to chapters \ref{C1} and \ref{C3}.\\
\subsection{Picture groups}
\indent I also need to use the concept of the picture groups in order to prove the formula below.\\
\indent For a quiver of finite type, any dimension vector of an indecomposable representation, which I also refer to as a \textit{root}. Let $D(\beta)\subseteq\mathbb{R}^n$, $D(\beta)= \{x\in\mathbb{R}^n: <x,\beta>=0,\ <x,\beta'>\leq 0\text{ when }\beta'\subseteq\beta\}$. Here $\beta'\subseteq\beta$ means the unique indecomposable representation of dimension vector $\beta'$ is a subrepresentation of the unique indecomposable representation of dimension vector $\beta$. $D(\beta)$ for all these roots divide $\mathbb{R}^n$ into \textit{compartments}. The boundary of each compartment is the union of some $D(\beta)$ which I call \textit{walls}.\cite{IT17}\cite{IOTW4} Sometimes my abuse notation and use the root $\beta$ to mean the wall $D(\beta)$ when the meaning is clear. I also use the notation $+\beta$ to mean the wall $\beta$ is a part of the boundary of a compartment $\mathcal{U}$ and for any point $x\in\mathcal{U}$, $<x,\beta>\ >0$. Similarly I have the notation $-\beta$. For example $+\beta-\beta'$ means that $\beta$ and $\beta'$ are parts of the boundary of a compartment $\mathcal{U}$ and for any point $x\in\mathcal{U}$, $<x,\beta>\ >0$ and $<x,\beta'>\ <0$.\\
\indent In $A_n$ in particular since all indecomposable representations are thin, the roots are $\beta_{ij}=e_j-e_i$ ($0<i<j<n$, $e_0$ is defined as the zero vector).\\
\begin{definition}
A \textit{picture group} of a cluster quiver of finite type $Q$ is a group $G(Q)=<S|R>$ with $S$ in bijection with the set of real Schur roots (the generator for $\beta$ is $x(\beta)$) and $R$ the set of relations $x(\beta_i)x(\beta_j)=\Pi x(\gamma_k)$ with $\gamma_k$ running over all these real Schur roots which are linear combinations $\gamma_k = a_k\beta_i+b_k\beta_j$ with $a_k/b_k$ increasing (going from 0/1 where $\gamma_1=\beta_j$ to 1/0 where $\gamma_k=\beta_i$) for any pair $(\beta_i,\beta_j)$ such that they are Hom-orthogonal and $Ext(\beta_i,\beta_j)=0$. \cite{IT17}\\
\end{definition}
\indent Note that for quiver $A_n$ all roots are real and Schur hence a real Schur root is just a root. Also I often simplify the notation of $x(\beta_{ij})$ to $x_{ij}$ which I use interchangeably with $x(\beta_{ij})$. The picture group for $A_n$ straight orientation is $G(A_n)=\{S|R\}$, $S=\{x_{ij}|0\leq i<j\leq n\}$, $R=\{x_{ij}x_{kl}=x_{kl}x_{ij}|[i,j]\cap[k,l]=\emptyset, [i,j]\text{ or }[k,l], $ i,j,k,l \\are distinct.\}$\cup\{x_{jk}x_{ij}=x_{ij}x_{ik}x_{jk}|0\leq i<j<k\leq n\}$.\\
\indent Todorov proved with Igusa \cite{IT17} that there exists a bijection betIen the set of maximal green sequences and the set $\mathcal{P}(c)$ of positive expressions of the Coxeter element of the picture group for any acyclic valued quiver of finite type which applies to $A_n$ straight orientation.\\
\subsection{Alternative definitions of maximal green sequences}
\indent In the following theorem by Kiyoshi Igusa multiple equivalent definition of maximal green sequences was introduced. To understand more about the wall-and-chamber structure I suggest that the reader reads \cite{IOTW15}, \cite{GHKK14} or \cite{BST17}.
\begin{theorem}
\cite{I17} Let $\Lambda$ be a finite dimensional hereditary algebra over a field $K$. Let $\beta_1,\cdots,\beta_m\in \mathbb{N}^n$ be any finite sequence of nonzero, nonnegative integer vectors. Then the following are equivalent.\label{thm:3}
\begin{enumerate}
\item[(a)] There is a nonlinear stability function $Z_t:K_0\Lambda\to \mathbb{C}$ which is green and has exactly $m$ semistable pairs $(M_i,t_i)$ with $t_1<t_2<\cdots<t_m$ so that $\dim M_i=\beta_i$ for all $i$.%(making all pairs stable) 
\item[(b)] There is a generic green path $\gamma:\mathbb{R}\to\mathbb{R}^n$ which crosses the walls $D(M_i)$, $i=1,\cdots,m$ in that order, and no other walls, so that $\dim M_i=\beta_i$ for all $i$.
\item[(c)] There exist $\Lambda$-modules $M_1,\cdots,M_m$ with $\dim M_i=\beta_i$ which form a finite Harder-Narasimhan system for $\Lambda$. 
\item[(d)] There exist Schurian $\Lambda$-modules $M_1,\cdots,M_m$ with $\dim M_i=\beta_i$ so that \begin{enumerate}
\item[(1)] $Hom_\Lambda(M_i,M_j)=0$ for $i<j$.
\item[(2)] No other modules can be inserted into the sequence preserving (1).
\end{enumerate}
\item[(e)] There is a maximal green sequence for $\Lambda$ of length $m$ whose $i$th mutation is at the $c$-vector $\beta_i$. 
\end{enumerate}
\end{theorem}
\section{Quiver folding}
\indent In this section we will introduce the theory of quiver folding. Folding theory has been in folklore for a while. Since it will be useful for proving a result in Chapter \ref{C4} I'm going to discuss it here in details.\\
\begin{definition}
Let $B$ be an $n\times n$ exchange matrix and let $\rho\in S_n$ be a permutation. If $\rho(B)=B$ then $\rho$ is a \textit{symmetry} of $B$. The group of symmetries of $B$ is the \textit{Symmetry Group of} $B$ which I denote as $Sym\ B$. An exchange matrix with a non-trivial symmetry group is a \textit{symmetric exchange matrix}. Any nontrivial subgroup of $B$ is a \textit{Symmetry Subgroup of} $B$. The symmetry group of a valued quiver is defined as the symmetry group of the exchange matrix of the valued quiver.\\   
\end{definition}
\indent A symmetry subgroup $G$ acts on the extended exchange matrices $\tilde{B}$ in the obvious way, namely for some $\rho\in G$ $\rho \tilde{B}:=\rho(\tilde{B})$. I can also define right group actions of elements of $G$ on the set of $c$-vectors similarly, namely for any $v=(v_i)\in \mathbb{Z}^n$, $v\rho:=(v_{\rho(i)})$. Let $C(\mathcal{A})$ be the set of $c$-vectors of a cluster algebra of geometric type $\mathcal{A}(B)$. Orbits of $c\in C(\mathcal{A})$ is denoted as $Gc$. It is easy to see that all orbits are finite. $Fc := \Sigma_{c'\in Gc}\ c'$ is the \textit{folded version of} $c$.\\
\indent Now I need to fold vertices first. For any nontrivial subgroup of $S_n$ let $n'$ be the set of orbits of the canonical group action of $S_n$ on $[n]$. Hence there exists maps from $[n]$ to $[n']$. Pick some surjection $f$ from $[n]$ to $[n']$ such that $f$ maps each orbit to one element of $[n']$. $f$ is a \textit{vertices folding map}. I sometimes abuse notations and identify $f(i)$ and $Gi$ when I do not need to specify $f$.\\
\begin{definition}
(1)For any valued quiver $Q$ and its symmetry subgroup $G$, the \textit{folded version of $Q$ with respect to $G$} is defined as below:\\
For each valued arrow $i\overset{(d_{ij},d_{ji})}{\longrightarrow} j$ it is replaced by $Gi\overset{(d_{ij}|Gi|,d_{ji}|Gj|)}{\longrightarrow} Gj$.\\
(2)For any symmetric exchange matrix $B$ and its symmetry subgroup $G$, the \textit{folded version of $B$ with respect to $G$} is defined as $\tilde{B}=(b'_{kl})$ where $b'_{GiGj}=b_{ij}|Gj|$. \\ \cite{Sal14}\cite{BHIT15}\\
\end{definition}
\begin{definition}
For any symmetry subgroup $G$ of $B$ any extended exchange matrix $\tilde{B}=(B',C')$ such that $B'$ is symmetric is a \textit{symmetric extended exchange matrix with respect to} $G$ if for any $i\in [n]$ $|Gc_i|=|G_i|$ and for any $\rho\in G$, $Gc_{\rho(i)}=Gc_i$.\\
\end{definition}
\indent Any symmetric extended exchange matrix can be folded.\\
\begin{definition}
For any symmetry subgroup $G$ of $B$ any symmetric extended exchange matrix $\tilde{B}=(B',C')'$ with respect to $G$. Then for any vertices folding map $f$ $F\tilde{B}:=(FB',FC')'$ is the \textit{folded version of} $\tilde{B}$ with $FC'=(\tilde{c}_1,\cdots, \tilde{c}_n)$ defined below: $\tilde{c}_i=\Sigma_{g\in f^{-1}(i)} \tilde{c}_g/|Gg|$ where $\tilde{c}_{gi}:=\Sigma_{k\in f^{-1}(i)} c_{gk}$.\\
\end{definition}
\indent It is easy to see that if $c_i=e_i$ then $C=I_n$ then $C'=I_{n'}$ and if $C=-I_n$ then $C'=-I_{n'}$. Hence a framed valued quiver is folded into a framed valued quiver and a coframed valued quiver is folded into a coframed valued quiver. Also positive $c$-vectors are folded into positive ones and negative $c$-vectors are folded into negative ones.\\
\begin{definition}
For any symmetry subgroup $G$ of $B$ a mutation sequence $w=\Pi_{i=m}^1 \mu_{k_i}$ starting from an extended exchange matrix $\tilde{B}=(B',C')'$ with $B$ symmetric is \textit{symmetric with respect to } a symmetry subgroup $G$ if the following holds:\\
1.$w$ is in the form $w=\Pi_{i=m}^1 \Pi_{j\in Gk_i} \mu_j$, which roughly means that vertices in any orbit is "mutated together".\\
2.$\Pi_{i=m'}^1 \mu_{k_i} \tilde{B}$ is symmetric.\\
\end{definition}
\begin{definition}
For any symmetry subgroup $G$ of a symmetric extended exchange matrix $\tilde{B}$ for any symmetric mutation sequence $w=\Pi_{i=m}^1 \Pi_{j\in Gk_i} \mu_j$, \textit{the folded version of $w$} is defined as $Gw:=\Pi_{i=m}^1 \mu'_{Gk_i}$.
\end{definition}
\indent It is easy to see that folding symmetric reddening sequences results in reddening sequences. Also folding symmetric green sequences results in green sequences. Folding symmetric maximal green sequences results in maximal green sequences.\\
\begin{theorem}
For any symmetry subgroup $G$ of a symmetric extended exchange matrix $\tilde{B}$ for any symmetric mutation sequence $w$, $F\circ w=Gw\circ F$.\\
\end{theorem}
\begin{proof}
\indent Let's assume that the length of $Gw$ is 1. When the theorem has been proven in this particular case the rest is clear from induction. Hence let's assume $w=\Pi_{j\in Gi} \mu_j$ and $Gw=\mu'_{Gi}$. Note that $b_{jk}=0$ for any $j,k\in Gi$ since otherwise $j$ and $k$ would not be in the same orbit. It is also clear from symmetry that the set of vertices in $[n]]$ that is a smyce of any valued arrow with some $j\in Gi$ its target is independent of the choice of $j$, which I denote as $P^{-}(Gi)$. Similarly I can define $P^+(Gi)$. The set of vertices in $[n]\backslash Gi$ that is not connected to any $j\in Gi$ is denoted as $I(Gi)$. Using the invariance lemmas it is easy to see that for any $j\in I(i)$ mutations at any element of $Gi$ does not affect the $j$-th row and the $j$-th column at all.\\
\indent Assume that I do mutations on a symmetric extended exchange matrix $\tilde{B}=(B',C')'$. Let $B=(b_{jk})$, $C=(c_{jk})$. Hence $b'_{GjGk}=b_{ij}|Gj|$. $c'_{GjGk}=\Sigma_{l\in Gj k\in Gm}c_{lm}$.  Let $\tilde{C}=\mu'_{Gi}(C)=(c'_{j'k'})$ $w(\tilde{B})=(\tilde{b}_{jk})$ $w(\tilde{C})=(\tilde{c}_{jk})$ $F\tilde{B}=(\hat{B}',\hat{C}')'$ $F\circ w(\tilde{B})=(B_1',C_1')'$  $Gw\circ F(\tilde{B})=(B_2',C_2')'$. $B_1=(b^1_{jk})$ $C_1=(c^1_{jk})$ $B_2=(b^2_{jk})$ $C_2=(b^1_{jk})$.\\
\indent Let's first calculate the left hand side. If $j$ or $k\in I(Gi)$ it is clear that $\tilde{b}_{jk}=b_{jk}$. If $j\in i$ or $k=i$ $\tilde{b}_{jk}=-b_{jk}$ due to the fact that $b_{jj'}=0$ for any $j,j'\in Gi$. Otherwise it is easy to see that $\tilde{b}_{jk}=b_{jk}+|Gi|sp(b_{ji},b_{ik})$ due to two facts: First of all, before a mutation at $l\in Gi$, the $l$-th row and column of the exchange matrix can not be affected by all previous mutations since betIen elements of $Gi$ there are no connections. Secondly for all $l\in Gi$, $b_{jl}$ and $b_{lk}$ are independent of $l$. Similarly, if $k\in Gi$ $\tilde{c}_{jk}=-c_{jk}$. Otherwise $\tilde{c}_{jk}=c_{jk}+\Sigma_{l\in Gi} sp(c_{jl}, b_{ik})$. Hence $b^1_{GjGk}=-b_{jk}|Gk|$ if $j$ or $k$ is in $Gi$. Otherwise $b^1_{GjGk}=b_{jk}|Gk|+|Gi||Gk|sp(b_{ji},b_{ik})$. If $k\in Gi$ $c^1_{GjGk}=-|Gk|\bar{c}_{GjGk}$. Otherwise $c^1_{GjGk}=\bar{c}_{GjGk}+\Sigma_{l\in Gi} sp(\bar{c}_{GjGl}, b_{ik})=\bar{c}_{GjGk}+|Gi|sp(\bar{c}_{GjGi},b_{ik})$.\\
\indent Now let's calculate the right hand side. $\hat{b}_{GjGk}=|Gk|b_{jk}$. $\hat{c}_{GjGk}=|Gk|\bar{c}_{GjGk}$. If $j$ or $k\in Gi$ $b^2_{GjGk}=-b_{jk}|Gk|$. Otherwise $b^2_{GjGk}=b_{jk}|Gk|+sp(|Gi|b_{ji}, |Gk|b_{ik})=b_{jk}|Gk|+|Gi||Gk|sp(b_{ji}, b_{ik})$. If $j$ or $k\in Gi$ $c^2_{GjGk}=-|Gk|\bar{c}_{GjGk}$. Otherwise $c^2_{GjGk}=|Gk|\bar{c}_{GjGk}+sp(|Gi|\bar{c}_{GjGi},b_{ik}|Gk|)=|Gk|\bar{c}_{GjGk}+|Gi||Gk|sp(\bar{c}_{GjGi},b_{ik})$.\\
\indent Hence $F\circ w=Gw\circ F$ has been proven.\\
\end{proof}
