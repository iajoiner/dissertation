\documentclass{brandeis-dissertation}
\usepackage{graphicx}
\usepackage{tikz-cd}
\usepackage{hyperref}
\usepackage{amsmath}
\usepackage{amssymb}
\usepackage{amsthm}
\usepackage{amsfonts}
\newtheorem{theorem}{Theorem}[section]
\newtheorem{lemma}[theorem]{Lemma}
\newtheorem{corollary}[theorem]{Corollary}
\newtheorem{conjecture}[theorem]{Conjecture}
\theoremstyle{definition}
\newtheorem{definition}[theorem]{Definition}
\newtheorem{example}[theorem]{Example}
\newtheorem{xca}[theorem]{Exercise}
\theoremstyle{remark}
\newtheorem{remark}[theorem]{Remark}
\newcommand{\cata}{\mathcal{A}}
\newcommand{\catc}{\mathcal{C}}
\newcommand{\catd}{\mathcal{D}}
\newcommand{\catt}{\mathcal{T}}
\newcommand{\cate}{\mathcal{E}}
\newcommand{\catf}{\mathcal{F}}
\newcommand{\cattf}{(\mathcal{T},\mathcal{F})}
\newcommand{\fz}{\tilde{\mathbb{Z}}}
\newcommand{\nn}{\mathbb{N}}
\newcommand{\zz}{\mathbb{Z}}
\newcommand{\sig}{\sigma}
\newcommand{\tsig}{\tilde{\sigma}}
\newcommand{\siginv}{{\sigma}^{-1}}
\newcommand{\psig}{P_{\sigma}}
\newcommand{\psiginv}{{P_{\sigma}}^{-1}}
\newcommand{\eem}[2]{\begin{bmatrix}#1\\#2\end{bmatrix}}
\newcommand{\abovearrow}[1]{\overset{#1}{\longrightarrow}}
\newcommand{\hos}{$Hom^{\leq 0}$-orthogonal sequence}
%Triangulated categories
\newcommand{\triang}[4]{#1\to #2\to #3\to #4}%triangle, math mode required
\newcommand{\tmtriang}[4]{$#1\to #2\to #3\to #4$}%triangle, text mode required
\newcommand{\triangwm}[7]{#1\overset{#2}{\to} #3\overset{#4}{\to} #5\overset{#6}{\to} #7}%triangle with morphisms, math mode required, some morphisms can be omitted
\newcommand{\tmtriangwm}[7]{$#1\overset{#2}{\to} #3\overset{#4}{\to} #5\overset{#6}{\to} #7$}%triangle with morphisms, text mode required, some morphisms can be omitted
\newcommand{\rightdoublearrow}{\arrow[r, shift right=0.6ex]\arrow[r, shift left=0.6ex]}
\newcommand{\leftquadruplearrow}{\arrow[l, shift right=1.2ex]\arrow[l, shift left=0.4ex]\arrow[l, shift right=0.4ex]\arrow[l, shift left=1.2ex]}
\newcommand{\righttwicedoublearrow}{\arrow[rr, shift right=0.6ex]\arrow[rr, shift left=0.6ex]}
\newcommand{\lefttwicedoublearrow}{\arrow[ll, shift right=0.6ex]\arrow[ll, shift left=0.6ex]}
\newcommand{\lefttwicetriplearrow}{\arrow[ll, shift right=0.8ex]\arrow[ll]\arrow[ll, shift left=0.8ex]}
%    Absolute value notation
\newcommand{\abs}[1]{\lvert#1\rvert}
% a_ig_j element
\newcommand{\ag}[2]{a_{#1}g_{#2}}

%    Blank box placeholder for figures (to avoid requiring any
%    particular graphics capabilities for printing this document).
\newcommand{\blankbox}[2]{%
  \parbox{\columnwidth}{\centering
%    Set fboxsep to 0 so that the actual size of the box will match the
%    given measurements more closely.
    \setlength{\fboxsep}{0pt}%
    \fbox{\raisebox{0pt}[#2]{\hspace{#1}}}%
  }%
}
%\numberwithin{equation}{section}
\title{Four problems related to maximal green sequences}
\author{Ying Zhou}
\department{Department of Mathematics}
\advisor{Kiyoshi Igusa, Department of Mathematics}
\reader{Olivier Bernardi, Department of Mathematics}
\reader{Gordana Todorov, Department of Mathematics, Northeastern University}
\dean{Eric Chasalow}
\graduationyear{2019}
\graduationmonth{May}
\acknowledgments{
\indent First of all I would like to thank my advisor, Kiyoshi Igusa for his constant patience, dedication and support. I also would like to thank Gordana Todorov for being my de facto second advisor. I would like to thank Otto Kerner, Hugh Thomas, Mike Prest and Nathan Reading for their guidance. I also would like to thank Alex Martsinkovsky, Ralf Schiffler, Thomas Brustle, Dan Zacharia, Graham Leuschke, Shiping Liu and Nathan Reading for their hospitality during my visits.\\
\indent I would like to thank my wonderful colleagues who worked and work in the same field as me, such as Charles Paquette, Hipolito Treffinger, Dylan Rupel, Steve Hermes, Alexander Garver, Kaveh Mousavand, Jacob Matherne, Emily Barnard, Shijie Zhu, Van Nguyen, Job Rock, Eric Hanson, Matt Garcia, Moses Kim and PJ Apruzzese. I also would like to thank Jordan Tirrell, Devin Murray, Joshua Eike, Daoyuan Han, Shahriar Mirzadeh, Tarakaram Gollamudi, Aaron Berkowitz, Nick Wadleigh, Cristobal Lemus-Vidales, Qing Liu, Anurag Rao and Mishel Skenderi. My life at Brandeis would have been boring without you all.\\
\indent Special thanks go to Susan Parker, Becci Torrey and Keith Merrill for being such great course coordinators. I also would like to thank Alex Mitchell, Yuze He, Daniel Johnston, Prabu Gugagantha and many other undergrads I have taught, helped and chatted with for the great time we spent together.\\ %I also would like to thank Janet Ledda, Anthony Bottaro, Catherine Broderick and Emily Palmer for their help.\\
\indent Finally I would like to thank my family and other friends.}
\thesisabstract{Maximal green sequences were invented by Bernhard Keller and have a lot of applications in cluster algebra and particle physics. In this dissertation I will discuss four separate topics related to maximal green sequences. First of all I will discuss the problem of associated permutations of mutation sequences and establish a formula for the associated permutation in the case of $A_n$ straight orientation which answers a question by Muller. Secondly I will introduce the concept of $m$-maximal green sequences and discuss the problem of $m$-maximal green sequence-finiteness of path algebras of tame quivers which generalizes a result by Brustle, Dupont and Perotin. Then I will introduce two alternative definitions of $m$-maximal green sequences of hereditary algebras and show that they are both equivalent to the known ones which extends a result by Igusa. Finally I will discuss the problem of maximal green sequences in quivers with multiple edges and fully describe maximal green sequences of quivers with multiple edges which is the first step towards a proof of the conjecture that all acyclic quivers have finitely many maximal green sequences.}
\begin{document}
%\signaturepage*
\thesisfront
%\input{chapter_intro}
\chapter{Background}\label{CB}
\section{Notations and conventions}
\indent In this paper $k$ is an algebraically closed field, all algebras will be finitely dimensional $k$-algebras. All modules are assumed to be right modules. The symbol $[n]$ is defined as the set $\{1,2,\cdots, n\}$ which is consistent with how it is usually used in cluster theory.\\
\indent When the category we are discussing is clearly $\mathcal{C}$ then $(M,N)$ will be an abbreviation of $Hom_{\mathcal{C}}(M,N)$. $(M[>0],N)$ is the union of $(M[k],N)$ for all $k>0$. If $\mathcal{P}, \mathcal{Q}$ are subcategories of $D^b(\Lambda)$ then $(\mathcal{P}, \mathcal{Q}):=\bigcup\limits_{M\in\mathcal{P}}\bigcup\limits_{N\in\mathcal{Q}}(M,N)$ and $(\mathcal{P}[>0], \mathcal{Q}):=\bigcup\limits_{k>0}(\mathcal{P}[k], \mathcal{Q})$. Let $S$ be a set of objects in an additive category $\catc$. $add(S)$ is defined as the set of all elements of $\catc$ such that they are summands of finite direct sums of objects in $S$. $\mathcal{E}(X)$ is the extension closure of $X$.\\
\indent The definition of green and red (vertices, mutations, sequences) are consistent with that of \cite{Kel11} and \cite{BDP13}. It is the exact opposite definition of green and red in \cite{BHIT15}.\\
%\indent If $m,n\in\mathbb{R}$ then $sp(m,n):=\begin{cases}
%mn & \text{if }m>0,n>0\\
%-mn & \text{if }m<0,n<0\\
%0 & \text{if } mn\leq 0\end{cases}$\\
\section{Quivers and path algebras}
\subsection{Quivers}
\begin{definition}
A \textit{quiver} $Q$ is a quadruple $(Q_0,Q_1,s,t)$ with $Q_0$ and $Q_1$ sets and $s,t$: $Q_1\rightarrow Q_0$. An element of $Q_0$ is a \textit{vertex} of $Q$. An element of $Q_1$ is an \textit{arrow} of $Q$. The map $s$ maps each arrow to its source and $t$ maps each arrow to its target.
\end{definition}
\indent Intuitively we can think of elements of $Q_1$ as oriented edges. Any arrow has a unique source and a unique target both of which are vertices. This is how we obtain the $s$ and $t$ maps. Unless necessary we generally omit the $s$ and $t$ and denote a quiver by $Q = (Q_0, Q_1)$.\\
\begin{example}
$\begin{tikzcd} 
1 \arrow[r] & 2 & 1 & 2 \arrow[l, shift right]\arrow[l, shift left]\arrow[r] & 3 & 1\arrow[r] & 2\arrow[r] & 3\\
\end{tikzcd}$\\
These three are quivers.
\end{example}
\indent Now we need to define subquivers.\\
\begin{definition}
A \textit{subquiver} $Q' = (Q'_0,Q'_1, s', t')$ in a quiver $Q = (Q_0,Q_1, s, t)$ is a quiver such that $Q'_0\subseteq Q_0$, $Q'_1\subseteq Q_1$, $s|_{Q'_1} = s'$ and $t|_{Q'_1} = t'$.
\end{definition}
\indent From now on we generally do not distinguish between $s$ and $s'$, $t$ and $t'$.\\
\indent Not all quivers are useful for the purpose of this paper. This is why we need to add restrictions. In order to do so we need to introduce several definitions.\\
\begin{definition}
An \textit{oriented cycle} in a quiver $Q$ is a subquiver $Q' = (Q'_0, Q'_1, s, t)$ such that $Q'_0 = \{v_0,v_1,\cdots, v_{k-1}\},Q'_1 = \{a_0,a_1,\cdots, a_{k-1}\}, s(a_i) = v_i$ and $t(a_i) = v_{i+1}$. Here $v_k$ is defined as $v_0$.
\end{definition}
\begin{definition}
A \textit{$k$-cycle} is an oriented cycle with $k$ vertices.
\end{definition}
\begin{definition}
A \textit{loop} in a quiver $Q$ is an arrow from a vertex to itself, that is, a 1-cycle.
\end{definition}
\begin{definition}
A \textit{cluster quiver} is a quiver without loops or 2-cycles. 
\end{definition}
\indent In all but Chapter \ref{C3} and a part of Chapter \ref{C4} all quivers we discuss will be acyclic. Here is the definition of an acyclic quiver.\\
\begin{definition}
An \textit{acyclic quiver} is a quiver without any oriented cycles. 
\end{definition}
\subsection{Path Algebras}
\begin{definition}
A \textit{path} in a quiver $Q$ is a sequence of vertices $\{v_0,\cdots, v_k\}$ and a sequence of arrows $\{a_0,\cdots, a_{k-1}\}$ if $k>0$ such that $t(a_i) = v_{i+1}$, $ s(a_i) = v_i$ for any $i = 0,1,\cdots, k-1$. The source of the path is $v_0$ and the sink is $v_k$.
\end{definition}
\indent Paths with length 0 are known as \textit{trivial paths}. A trivial path only has a single vertex $v_0$ and no arrows at all. All other paths are uniquely determined by their arrows.\\
\indent Now we need to define multiplication of paths. In order to do so we need to define compatibility and concatenation.\\
\begin{definition}
Paths $v,w$ are \textit{compatible} if $t(v) = s(w)$.
\end{definition}
\begin{definition}
The \textit{concatenation} of compatible paths $v = \{a_0,\cdots, a_{k-1}\}$ and $w = \{b_0,\cdots, b_{l-1}\}$ is  $vw = \{a_0,\cdots, a_{k-1}, b_0,\cdots, b_{l-1}\}$.
\end{definition}
\begin{definition}
The \textit{path algebra} of a quiver $Q$ is a $k$-algebra generated by all the paths of the quiver. Multiplication of paths $v$ and $w$ is defined as the concatenation if they are compatible and 0 if they aren't, 
\end{definition}
\indent From a homological point of view path algebras are very nice, namely they are \textit{hereditary}. In other words their global dimensions are at most one.
\begin{theorem}
The path algebra $kQ$ of any acyclic quiver $Q$ is hereditary. That is, for any $M,N\in mod kQ$ for all $k>1$ it is true that $Ext^k(M,N)=0$.
\end{theorem}
\indent We mostly only discuss path algebras of acyclic quivers in this thesis because our results are only about finite dimensional algebras.\\
\subsection{Euler matrices and the Euler-Ringel form}
\indent Now let's define \textit{Euler matrices} which will be very useful in the understanding of Chapter \ref{C4}.\\
\begin{definition}
Let $Q$ be an acyclic quiver. The \textit{Euler matrix of } $Q$ is defined as the matrix $E = (e_{ij})$ where $e_{ij} = \begin{cases}
1 & \text{if } i=j\\
-k & \text{if there are }k\text{ arrows from }i\text{ to }j\\
0 & \text{in all other cases}\\
\end{cases}$
\end{definition}
\begin{example}
The Euler matrix $E$ of the quiver $Q:1\to 2$ is $E=\begin{bmatrix}1 & -1\\0 & 1\\\end{bmatrix}$.
\end{example}
\begin{example}
The Euler matrix $E$ of the quiver $Q:\begin{tikzcd}1\arrow[r] & 2\rightdoublearrow & 3\end{tikzcd}$ is $E=\begin{bmatrix}1 & -1 & 0\\0 & 1 & -2 \\0 & 0 & 1\\\end{bmatrix}$.
\end{example}
\indent Using the Euler matrix we can define \textit{Euler-Ringel forms}.\\
\begin{definition}
Let $Q$ be an acyclic quiver and $E$ be its Euler matrix. The Euler-Ringel form of $Q$ (or $kQ$) is defined as $\langle x,y\rangle_Q:=x^tEy$.
\end{definition}
\indent When which $Q$ we are talking about isn't ambiguous we can just use $\langle x,y\rangle$ to refer to the Euler-Ringel form of $Q$.
\subsection{Modules in hereditary algebras}
\indent Now let's review some basic concepts about modules. In particular we will review the concepts of bricks and stones.\\
\begin{definition}
\indent An indecomposable module $M$ over an algebra $\Lambda$ is \textit{Schur} or a \textit{brick} if its endomorphism ring $End M$ is a division algebra. In particular if $\Lambda$ is a $k$-algebra where $k$ is an algebraically closed field then a brick is an indecomposable module such that $End M=k$.
\end{definition}
\begin{definition}
\indent An indecomposable module $M$ over an algebra $\Lambda$ is \textit{rigid}, \textit{exceptional} or a \textit{stone} if $Ext^1 (M,M)=0$.
\end{definition}
\indent Here is a well-known result about Schur and rigid modules in hereditary algebras.
\begin{theorem}
\indent \cite{ASS06}Let $k$ be an algebraically closed field, let $\Lambda$ be a finite-dimensional $k$-algebra. Then any rigid $\Lambda-$module $M$ is Schur.
\end{theorem}
\subsection{Dimension vectors and roots}
\indent The concept of roots originated from Lie theory. Here we will introduce some basic terminologies. Basically a lot of concepts we define using modules can also be defined using their \textit{dimension vectors} which we will define here.\\
\begin{definition}
The \textit{dimension vector} of a module $M$ in a finite dimensional $k$- algebra $\Lambda$ with $n$ indecomposable idempotents $e_1,\cdots, e_n$ is defined as $c_M:=\{c_1,\cdots, c_n\}$ where $c_i:= dim_k Me_i$.
\end{definition}
\indent Now let's define real roots and imaginary roots. Before that we first need to define sign coherence in vectors.\\
\begin{definition}
A vector $c$ in $\fz^n$ is \textit{sign coherent} if it is nonzero and all its entries are either all nonpositive or nonnegative.
\end{definition}
\indent The entries of a sign coherent vector are either all nonnegative or all nonpositive. In the former case we say it is \textit{positive}. In the latter case we say it is \textit{negative}.\\
\begin{definition}
Let $Q$ be an acyclic quiver with $n$ vertices. $c\in \nn^n$ is a \textit{real root} if it is sign coherent and $\langle c, c\rangle_Q = 1$.
\end{definition}
\begin{definition}
Let $Q$ be an acyclic quiver with $n$ vertices. $c\in \nn^n$ is an \textit{imaginary root} if it is sign coherent and $\langle c, c\rangle_Q \leq 0$.
\end{definition}
\section{Mutations, mutation sequences and the associated permutation}
\indent In this section we will introduce mutations of quivers and matrices, different kinds of mutation sequences including green sequences, maximal green sequences, reddening sequences and loop sequences. We will also define the associated permutation. Results in this section are mostly used in Chapters \ref{C1} and \ref{C4}.\\
\subsection{Mutation of quivers}
\indent The concept of maximal green sequences has many different equivalent definitions. We will use a simple definition using quiver mutations in this subsection. Later we will introduce other definitions.\\
\begin{definition}
Let $Q$ be a cluster quiver. \textit{Mutation} of $Q$ at vertex $k$ is defined in the following way:
\begin{enumerate}
\item For any pair of arrows $i\to k$ and $k\to j$ add an arrow $i\to j$.
\item Reverse all arrows starting from or ending up in $k$.
\item Delete all 2-cycles that are formed due to process (1) and (2).
\end{enumerate}
\end{definition}
\begin{definition}
\begin{enumerate}
\item The \textit{framed quiver} $\hat{Q}$ of $Q$ is obtained from $Q$ by adding a vertex $i'$ and an arrow $i\rightarrow i'$ for every $i\in Q$.
\item The \textit{coframed quiver} $\breve{Q}$ of $Q$ is obtained from $Q$ by adding a vertex $i'$ and an arrow $i'\rightarrow i$ for every $i\in Q$.
\item An \textit{ice quiver} is a quiver $Q$ where a possibly empty set, $F\subseteq Q_0$, consists of vertices that are not allowed to mutate.
\end{enumerate}
\end{definition}
\indent An ice quiver $(Q,F)$ can not mutate at elements of $F$, so we call them \textit{frozen vertices}.
\begin{definition}
A non-frozen vertex $i$ is \textit{green} if and only if no arrow from a frozen vertex to $i$ exists. Otherwise it is \textit{red}.\cite{Kel11}
\end{definition}
\begin{example}
\indent In this graph below we did a mutation at 1 from the framed quiver of $Q: 1\to 2$. After the mutation the vertex changed from being green to being red.\\
$\begin{tikzcd}
1 \arrow[r] \arrow[green]{d} & 2\arrow[green]{d}\arrow[r,"\mu_1", shift right=3.5ex]  & 1&2\arrow[l]\arrow[green]{d}\\
1' & 2'&1'\arrow[red]{u}& 2'\\
\end{tikzcd}$
\end{example}
\begin{definition}
A \textit{green sequence} is a sequence $\mathbf{i}=(i_1, i_2,\cdots, i_N)$ such that for all $1\leq t\leq N$ the vertex $i_t$ is green in the partially mutated ice quiver $\hat{Q}(\mathbf{i},t)=\mu_{i_{t-1}}\cdots\mu_2\mu_1(\hat{Q})$.
\end{definition}
\begin{definition}
A \textit{maximal green sequence} is a green sequence such that $\hat{Q}(\mathbf{i},N)$ does not have any green vertices.
\end{definition}
\begin{example} For quiver $1\to 2$ here is one of its two maximal green sequences.\\
 $\begin{tikzcd}
1 \arrow[r] \arrow[d,green] & 2\arrow [d,green]\arrow[r,"\mu_1", shift right=3.5ex]  & 1&2\arrow[l]\arrow[d,green]\arrow[r,"\mu_2", shift right=3.5ex]&1\arrow[r] & 2\\
1' & 2'&1'\arrow[u,red]& 2'&1'\arrow[u,red] & 2'\arrow[u,red]\\
\end{tikzcd}$
\end{example}
\indent We also need the definition of reddening sequences which are generalized versions of maximal green sequences in order to discuss the phenomenon of almost morphism finiteness in Chapter \ref{C2}.\\
\begin{definition}
A \textit{red-to-green sequence} or a \textit{reddening sequence}, is a sequence $\mathbf{i}=(i_1, i_2,\cdots, i_N)$ that transforms $\hat{Q}$ to a quiver $\hat{Q}(\mathbf{i},N) = \mu_{i_N}\cdots\mu_2\mu_1(\hat{Q})$ such that $\hat{Q}(\mathbf{i},N)$ does not have any green vertices.\cite{Mul15}\\
\end{definition}
\subsection{Mutation of matrices}
\indent We can also use $c$\textit{-vectors} for this purpose. To do so we need to reinterpret mutations of cluster quivers in terms of mutations of matrices. We recall that cluster quivers correspond to \textit{exchange matrices} as defined below. For more details we recommend \cite{FZ01} and \cite{FZ06}.\\
\begin{definition}
\cite{FZ01} An \textit{exchange matrix} of a cluster quiver $Q$ with $n$ vertices is an $n\times n$ matrix such that $b_{ij}$ is the number of arrows from $i$ to $j$ minus the number of arrows from $j$ to $i$.
\end{definition}
\indent It is easy to see that exchange matrices of cluster quivers are always antisymmetric which is not true in the more general case of \textit{valued quivers} which we won't discuss in this paper. Moreover there is a 1-1 correspondence between antisymmetric exchange matrices and cluster quivers.\\
\indent Mutations of exchange matrices are defined here which exactly agree with mutations of cluster quivers.\\
\begin{definition}
\cite{FZ01} If we mutate an $n\times n$ exchange matrix $B = (b_{ij})$ at $k$ we obtain $B' = (b'_{ij})$ defined here.
$b'_{ij} = \begin{cases}
-b_{ij} & \text{if }i = k\text{ or }j = k\\
b_{ij} + b_{ik}|b_{kj}| & \text{if }b_{ik}b_{kj} > 0\\
b_{ij} & \text{in all other cases}
\end{cases}$
\end{definition}
\indent Each partially mutated ice quiver corresponds to an \textit{extended exchange matrix} defined below.\\
\begin{definition}
The \textit{extended exchange matrix} $B'$ corresponding to a partially mutated ice quiver $Q'$ is an $2n\times n$ matrix with the rows corresponding to vertices $\{1,2,\cdots, n, 1', 2',\cdots n'\}$ while the columns corresponds to the vertices $\{1,2,\cdots, n\}$. Here we use the number $n+i$ to represent $i'$. Here again $b_{ij}$ is the number of arrows from $i$ to $j$ minus the number of arrows from $j$ to $i$.
\end{definition}
\indent An extended exchange matrix $B'$ has an upper and lower square submatrices, $B$ and $C$ respectively. The lower square matrix $C$ is known as the $C$\textit{-matrix}. Column vectors of an $C$-matrix are known as $c$\textit{-vectors}. A $c$-vector is positive if all its entries are non-negative and at least one is positive. A $c$-vector is negative if all its entries are non-positive and at least one is negative. Due to \cite{GHKK14} a $c$-vector is either positive or negative which is known as \textit{sign coherence}.\\
\indent A mutation on vertex $k$ is \textit{green} if the $c$-vector $c_k$ before the mutation is negative.  A mutation on vertex $k$ is \textit{red} if the $c$-vector $c_k$ before the mutation is positive. A \textit{maximal green sequence} is a mutation sequence from $C=-I_n$ to a permuted version of $I_n$ We can use a sequence of $c$-vectors to denote a maximal green sequence because we can use the $c$-vector corresponding to vertex $k$ to represent mutation at vertex $k$ which is possible since all $C$-matrices are invertible.\\
\subsection{Permutations}
\indent All reddening sequences have associated permutations. When comparing the quivers obtained from transforming the same framed quiver using two different reddening sequences, it is easy to see that they are just one permutation away from each other: If you do a correct permutation of vertices (that means both rows and columns together) you can transform one such matrix into another. In particular any quiver obtained by using a reddening sequence to transform a framed quiver is one permutation away from the coframed quiver.\\
\indent In this subsection if $Q$ is a cluster quiver then $\hat{Q}, \breve{Q}$ are the framed and coframed quiver associated with cluster quiver $Q$ respectively.\\
\indent Here is the formal definition of such a permutation:\\
\begin{definition}
\cite{BDP13} A \textit{permutation} from an ice quiver $(Q,F)$ to $(Q',F)$ is an isomorphism of quivers $Q\rightarrow Q'$ that preserve $F$.\\
\end{definition}
\indent We have a result from \cite{BDP13} which helps us define the permutation:\\
\begin{theorem}
\cite{BDP13} Let $Q$ be a cluster quiver and let $Q'$ be a quiver that is a result of a reddening sequence on $\hat{Q}$, then  $Q'$ equals to a permutation of $\breve{Q}$. That is, for a reddening sequence $\mathbf{i}=(i_1,\cdots, i_N)$, for some $\rho\in S_n$ we have $\mu_{i_N}\cdots\mu_{i_1}\hat{Q}=\rho\breve{Q}$.\\
\end{theorem}
\begin{definition}
\cite{GM14} The \textit{permutation of a reddening sequence} $\mathbf{i}$ is $\rho$ for which $\mu_{i_N}\cdots\mu_{i_1}\hat{Q}=\rho\breve{Q}$.\\
\end{definition}
\indent Here is one of the simplest examples of the concept of the permutation:\\
$\begin{tikzcd}
1 \arrow[r] \arrow[d,green] & 2\arrow [d,green]\arrow[r,"\mu_1", shift right=3.5ex]  & 1&2\arrow[l]\arrow[d,green]\arrow[r,"\mu_2", shift right=3.5ex]&1\arrow[r] & 2\\
1' \arrow[d,"\mu_2", shift left=4.5ex]& 2'&1'\arrow[u,red]& 2'&1'\arrow[u,red] & 2'\arrow[u,red]\\
 1\arrow[dr,green]\arrow[d,green]&2\arrow[l]\arrow[r,"\mu_1", shift right=3.5ex]&1\arrow[r]&2\arrow[dl,green]\arrow[r,"\mu_2", shift right=3.5ex]&1&2\arrow[l]\arrow[u,"(12)", shift left=4.5ex]\\
 1'&2'\arrow[u,red]&1'\arrow[u,red]&2'\arrow[ul,red]&1'\arrow[ur,red]&2'\arrow[ul,red]\\
\end{tikzcd}$\\
\indent It is obvious that the result of $\mu_2\mu_1$ and $\mu_2\mu_1\mu_2$ are not identical, though they can be transformed into each other by a single permutation on vertices.\\
\indent We can also define the associated permutation of sequences using extended exchange matrices.\\
\begin{definition}
The \textit{matrix of a permutation}, $\sigma\in S_n$, is defined as the $n\times n$ matrix $P_\sigma=(\delta_{\sigma(i)j})$.\\
\end{definition}
\begin{definition}
(1)For an $m\times n$ matrix $M=(M_1,\cdots, M_n)$ and a permutation $\sigma\in S_n$, if $C=(M_{\sigma(1)},\cdots,M_{\sigma(n)})$ (or equivalently, $(c_{ij})=(m_{i\sigma(j))}$), we denote this as $C=\sigma(M)$.\\
(2)For an $n\times n$ matrix $A=(a_{ij})$ and a permutation $\sigma\in S_n$, if $D=(d_{ij})=(a_{\sigma(i)\sigma(j)})$, we denote this as $D=\tsig(A)$.\\
\end{definition}
\indent It is easy to see that $C=\sigma(M)$ if and only if $C=M\psiginv$. $D=\tsig(A)$ if and only if $D=\psig A\psiginv$.\\
\indent Now let's define a new concept, namely \textit{loop sequences} which is essential to the discussion about the permutation in Chapter \ref{C1}.\\
\begin{definition} 
A \textit{loop sequence} $w$ is a sequence of mutations $\mu_{i_k}\cdots\mu_{i_1}$ on an ice quiver $(Q,F)$ such that $\mu_{i_k}\cdots\mu_{i_1}(Q) = \rho(Q)$ for some permutation $\rho$.\\
\end{definition}
\begin{example}
\indent Let $Q$ be $1\to 2$. Here is the loop sequence (1,2,1,2,1) with associated permutation $(12)$.\\
$\begin{tikzcd}
1 \arrow[r] \arrow[d,green] & 2\arrow [d,green]\arrow[r,"\mu_1", shift right=3.5ex]  & 1&2\arrow[l]\arrow[d,green]\arrow[r,"\mu_2", shift right=3.5ex]&1\arrow[r] & 2\\
1' \arrow[d,"(12)", shift left=4.5ex]& 2'&1'\arrow[u,red]& 2'&1'\arrow[u,red] & 2'\arrow[u,red]\arrow[d,"\mu_1", shift right=4.5ex]\\
 1\arrow[dr,green]&2\arrow[l]\arrow[dl,green]&1\arrow[r]\arrow[l,"\mu_1", shift left=3.5ex]&2\arrow[dl,green]\arrow[d,green]&1\arrow[l,"\mu_2", shift left=3.5ex]\arrow[d,green]&2\arrow[l]\\
 1'&2'&1'&2'\arrow[ul,red]&1'\arrow[ur,red]&2'\arrow[u,red]\\
\end{tikzcd}$
\end{example}
\begin{definition}
For any loop sequence $w$ the permutation $\rho$ such that $w(\tilde{B})=\rho(\tilde{B})$ is defined as \textit{the associated permutation of the loop sequence $w$}.\\
\end{definition}
\indent In essence for all acyclic quivers, green-to-red sequences in general and maximal green sequences in particular do not have a natural definition of the permutation: The traditional one in essence is the permutation of an associated loop sequence: Take the reddening sequence and then do mutations at sinks only, go over all non-frozen vertices and return to the origin which constitutes the loop sequence we need.\\
\section{Green sequences in bounded derived categories}
\indent In this section we will go over the basics about bounded derived categories, approximations, silting objects, simple-minded collections, torsion classes, $t$-structures and introduce the definition of numerous mutation sequences. This section mostly consists of background for Chapter \ref{C3} and Chapter \ref{C4}.\\
\subsection{Triangulated categories and bounded derived categories}
\indent In this subsection we need to use Auslander-Reiten Theory. However I'm not going to talk about the entire Auslander-Reiten theory even though some parts of it are crucial to the understanding of Chapter \ref{C2}. For Auslander-Reiten theory we refer the reader to Chapter IV of \cite{ASS06} and \cite{ARS}.\\
\indent I'm not going to talk about what triangulated categories and bounded derived categories are in details. For those who want to read about them we recommend Daniel Murfet's notes \cite{MurD1}\cite{MurD2}\cite{MurT1} for introduction and \cite{H88} for its application in the theory of finite dimensional algebras. In particular \cite{H88} is a good source for Auslander-Reiten theory in bounded derived categories which we will use extensively here.\\
\indent Let's recall some basic facts about triangulated categories that we will use a lot in this paper. In a triangulated category $\catt$ there exists an automorphism $[1]$ known as the \textit{translation functor}. $[n]:=([1])^n$ for any integer $n$. The \textit{cone} or \textit{homotopy cokernel} of a morphism $A\overset{f}{\to}B\in\catt$ is some $C\in\catt$ such that there exists $g,h\in\catt$ such that $\triangwm{A}{f}{B}{g}{C}{h}{A[1]}$ is a distinguished triangle.The \textit{homotopy kernel} of a morphism $A\overset{f}{\to}B\in\catt$ is some $C\in\catt$ such that there exists $g,h\in\catt$ such that $\triangwm{C}{g}{A}{f}{B}{h}{C[1]}$ is a distinguished triangle. Using axioms of triangulated categories any morphism $A\overset{f}{\to}B\in\catt$ has a homotopy kernel and a homotopy cokernel.\\
\indent Let's recall that bounded derived categories $D^b(\Lambda)$ are obtained by identifying homotopic chain maps in the category of chain complexes $C({\Lambda})$ and then formally invert all quasi-isomorphisms through localization. In bounded derived categories of hereditary algebras the indecomposable objects are of the form $M[i]$ where $M$ is an indecomposable module and $i$ is the amount of shifts we perform. In bounded derived categories it is true that $M,N\in mod\Lambda$ $Hom_{D^b(\Lambda)}(M[i],N[j])=\begin{cases}
Ext_{\Lambda}^{j-i}(M,N) & \text{ if }j\geq i\\
0 & \text{ if }j<i
\end{cases}$.\\
\begin{example}
Let $Q$ be $\begin{tikzcd}1\arrow[r] & 2 \arrow[r] & 3\end{tikzcd}$. Here is the Auslander-Reiten quiver of $D^b(kQ)$.\\
\begin{tikzcd}[cramped,sep=small]
\cdots&I_1[-1]\arrow[rd]& &P_1\arrow[rd] & & P_3[1]\arrow[rd] & & S_2[1]\arrow[rd] & & I_1[1] & \cdots\\
 & \cdots & P_2\arrow[ru]\arrow[rd]& &I_2\arrow[rd]\arrow[ru] & &P_2[1]\arrow[rd]\arrow[ru] & &I_2[1]\arrow[rd]\arrow[ru] & \cdots\\
\cdots& P_3\arrow[ru]& &S_2\arrow[ru]& &I_1\arrow[ru] & &P_1[1]\arrow[ru] & &P_3[2] & \cdots\\
\end{tikzcd}
\end{example}
\subsection{Approximations}
\indent According to \cite{KY12} there are bijections between silting objects, $t$-structures, co-$t$-structures and simple-minded collections in a wide range of cases and such bijections respect mutations. In \cite{BY13} more bijections are mentioned. Here we only need to cover three of them, namely silting objects, simple-minded collections and $t$-structures. To understand their mutations we must first introduce the concept of approximations.\\
\begin{definition}
Let $\catc$ be a category and $\mathcal{X}$ be one of its subcategories. Let $M\in Ob\catc, N\in Ob\mathcal{X}$ and $f\in Hom_{\catc}(M,N)$.
\begin{enumerate}
\item $f$ is a \textit{left-$\mathcal{X}$ approximation} if for any $N'\in Ob\mathcal{X}$ and for any $q\in Hom_{\catc}(M,N')$ we have $q$ factors through $f$.
\item $f$ is \textit{left minimal} if for any $g\in End_{\catc} N$ such that $g\circ f = f$ the morphism $g$ is an isomorphism.
\item $f$ is a \textit{minimal left-$\mathcal{X}$ approximation} if it is both left minimal and is a left-$\mathcal{X}$ approximation.
\end{enumerate}
\end{definition}
\begin{tikzcd}
M\arrow[r,"f"]\arrow[rd,"q"] & N\arrow[dashed,d,"l"]\\
 & N'\\
\end{tikzcd}
\begin{definition}
Let $\catc$ be a category and $\mathcal{X}$ be one of its subcategories. Let $M\in Ob\catc, N\in Ob\mathcal{X}$ and $f\in Hom_{\catc}(M,N)$.
\begin{enumerate}
\item $f$ is a \textit{right-$\mathcal{X}$ approximation} if for any $M'\in Ob\mathcal{X}$ and for any $q\in Hom_{\catc}(M',N)$ we have $q$ factors through $f$.
\item $f$ is \textit{right minimal} if for any $g\in End_{\catc} M$ such that $f\circ g = f$ the morphism $g$ is an isomorphism.
\item $f$ is a \textit{minimal right-$\mathcal{X}$ approximation} if it is both right minimal and is a right-$\mathcal{X}$ approximation.
\end{enumerate}
\end{definition}
\begin{tikzcd}
M\arrow[r,"f"]& N\\
M'\arrow[ru,"q"]\arrow[u,dashed,"l"]& \\
\end{tikzcd}
\begin{example}
Let $\catc$ be $D^b(\Lambda)$ for some finite dimensional algebra $\Lambda$ and let $\mathcal{X}$ be one of its full subcategories. If $M\in\mathcal{X}$ then $1_M$ is both a minimal left-$\mathcal{X}$ approximation and a minimal right-$\mathcal{X}$ approximation.
\end{example}
\begin{example}
Let $Q$ be $1\to 2$. Let $\catc$ be $D^b(kQ)$. Let $M = P_2$ and $\mathcal{X} = add(P_1)$. The minimal left-$\mathcal{X}$ approximation is the canonical morphism $P_2\to P_1$ induced by the inclusion $P_2\to P_1$ in the module category.
\end{example}
\begin{example}
Let $Q$ be $1\to 2$. Let $\catc$ be $D^b(kQ)$. Let $M = P_2$ and $\mathcal{X} = add(P_1)$. The minimal right-$\mathcal{X}$ approximation is the zero morphism because there is no other morphism from $P_1$ to $P_2$.
\end{example}
\subsection{Silting objects}
\indent Now let's introduce silting objects.\\ %We can think of indecomposable summands of them as indecomposable projectives in an Abelian category.\\
\begin{definition}
Let $\Lambda$ be an algebra with $n$ primitive idempotents. A \textit{silting object} $T$ of $D^b(\Lambda)$ is an object such that $T$ has $n$ direct summands and $(T,T[m])=0$ for all $m>0$. A \textit{pre-silting object} is an object that only has to satisfy the second condition.\\
\end{definition}
\begin{example}
\indent Let's take $A_3$ straight orientation as an example.\\
\begin{tikzcd}
I_1[-1]\arrow[rd]& &P_1\arrow[rd] & & P_3[1]\arrow[rd] & & S_2[1]\arrow[rd] & & I_1[1]\\
& P_2\arrow[ru]\arrow[rd]& &I_2\arrow[rd]\arrow[ru] & &P_2[1]\arrow[rd]\arrow[ru] & &I_2[1]\arrow[rd]\arrow[ru]\\
 P_3\arrow[ru]& &S_2\arrow[ru]& &I_1\arrow[ru] & &P_1[1]\arrow[ru] & &P_3[2]\\
\end{tikzcd}\\
\indent $\Lambda[i]$ is a silting object for any $i$. $T_1=P_3\oplus P_1 \oplus I_1[1]$ is also a silting object.
\end{example}
\indent Now that we already have the definition of silting objects we can discuss their mutations.\\
\begin{definition}
A \textit{forward mutation} on the direct summand $T_i$ of the silting object $T$ is $T'_i\oplus (T/T_i)$ where $T'_i$ is the homotopy cokernel of the minimal left-$add (T/T_i)$ approximation of $T_i$.\\
A \textit{backward mutation} on the direct summand $T_i$ of the silting object $T$ is $T'_i\oplus (T/T_i)$ where $T'_i$ is homotopy kernel of the minimal right-$add (T/T_i)$ approximation of $T_i$.
\end{definition}
\begin{example}
\indent Again let's take $A_3$ straight orientation as an example.\\
\begin{tikzcd}
I_1[-1]\arrow[rd]& &P_1\arrow[rd] & & P_3[1]\arrow[rd] & & S_2[1]\arrow[rd] & & I_1[1]\\
& P_2\arrow[ru]\arrow[rd]& &I_2\arrow[rd]\arrow[ru] & &P_2[1]\arrow[rd]\arrow[ru] & &I_2[1]\arrow[rd]\arrow[ru]\\
 P_3\arrow[ru]& &S_2\arrow[ru]& &I_1\arrow[ru] & &P_1[1]\arrow[ru] & &P_3[2]\\
\end{tikzcd}\\
\indent $\Lambda$ is a silting object. When we do a forward mutation at $P_3$ we get $T'=S_2\oplus P_2\oplus P_1$. When we do a forward mutation at $P_1$ now we get $T''=S_2\oplus P_2\oplus P_1[1]$. When we do another forward mutation at $P_2$ we get $T'''=S_2\oplus P_3[1]\oplus P_1[1]$.
\end{example}
\subsection{Simple-minded collections}
\indent Now let's introduce simple-minded collections. They are simple objects in some Abelian category known as hearts of $t$-structures.\\
\begin{definition}
Let $\Lambda$ be an algebra with $n$ primitive idempotents. A \textit{simple-minded collection} $\{S_i\}_{i\in [n]}$ of $D^b(\Lambda)$ is an $n$-element set such that $(S_i[\geq 0],S_j)=0$ for all $i\neq j$, $(S_i[>0], S_i) = 0$ for all $i$ and $(S_i,S_i)$ is a division algebra.\\
\end{definition}
\begin{example}
\indent As usual our example is $A_3$ straight orientation.\\
\begin{tikzcd}
I_1[-1]\arrow[rd]& &P_1\arrow[rd] & & P_3[1]\arrow[rd] & & S_2[1]\arrow[rd] & & I_1[1]\\
& P_2\arrow[ru]\arrow[rd]& &I_2\arrow[rd]\arrow[ru] & &P_2[1]\arrow[rd]\arrow[ru] & &I_2[1]\arrow[rd]\arrow[ru]\\
 P_3\arrow[ru]& &S_2\arrow[ru]& &I_1\arrow[ru] & &P_1[1]\arrow[ru] & &P_3[2]\\
\end{tikzcd}\\
\indent $\{I_1, S_2, P_3\}$ is a simple-minded collection. $\{P_3[1], P_2, I_1\}$ is also a simple-minded collection.
\end{example}
\begin{definition}
A \textit{forward mutation} on the element $S_i$ of the simple-minded collection $\{S_j\}$ is $\{S'_j\}$ where $S'_i = S_i[1]$ and $S'_j$ ($j\neq i$) is the homotopy cokernel of the minimal left-$add(S_i)$ approximation of $S_j[-1]$.\\
A \textit{backward mutation} on the element $S_i$ of the simple-minded collection $\{S_j\}$ is $\{S'_j\}$ where $S'_i = S_i[-1]$ and $S'_j$ ($j\neq i$) is the homotopy cokernel of the minimal left-$add(S_i[-1])$ approximation of $S_j$.\\
\end{definition}
\begin{example}
\indent The quiver here is $A_3$ straight orientation.\\
\begin{tikzcd}
I_1[-1]\arrow[rd]& &P_1\arrow[rd] & & P_3[1]\arrow[rd] & & S_2[1]\arrow[rd] & & I_1[1]\\
& P_2\arrow[ru]\arrow[rd]& &I_2\arrow[rd]\arrow[ru] & &P_2[1]\arrow[rd]\arrow[ru] & &I_2[1]\arrow[rd]\arrow[ru]\\
 P_3\arrow[ru]& &S_2\arrow[ru]& &I_1\arrow[ru] & &P_1[1]\arrow[ru] & &P_3[2]\\
\end{tikzcd}\
\indent $\{I_1, S_2, P_3\}$ is a simple-minded collection. When we do a forward mutation at $P_3$ we get $\{P_3[1], P_2, I_1\}$. When we do a forward mutation at $P_2$ now we get $\{S_2, P_2[1], P_1\}$. When we then do a forward mutation at $P_1$ we get $\{S_2, I_1, P_1[1]\}$.
\end{example}
\indent Now we need to introduce two more results that are crucial to Chapter \ref{C4}. Positive $c$-vectors are dimension vectors of elements of simple-minded collections. Such elements are all bricks. That is, all $c$-vectors are Schur. However we can indeed prove more. They are in fact real as well.\\
\begin{lemma}
Let $k$ be an algebraically closed field. Let $\Lambda$ be a hereditary algebra over $k$. Then any $c$-vector $c$ that appears in any MGS is a real Schur root.
\end{lemma}
\begin{proof}
\indent Since the simples of $\Lambda$ are all exceptional if the lemma were incorrect then there must be some $c$-matrix in the MGS, $C$ such that all columns of $C$ are real Schur roots while one green mutation can somehow generate a root that isn't real. Here there can only be two cases, namely some mutation performed on $-v$ caused some $-w$ to be transformed into $-w'=-w-kv$ which isn't real, some mutation performed on $-v$ caused some $+w$ to be transformed into $w'=w-kv$ which isn't real. In the second case $w'$ may be positive or negative.\\
\indent For an arbitrary $c$-vector $v$ let $M_v$ be the brick such that $v$ is the dimension vector of $M_v$. In this case $\langle v,v\rangle=1-dim Ext^1(M_v,M_v)$. Hence $v$ being real is equivalent to $\langle v,v\rangle=1$.\\
\indent \textbf{Case 1}: Assume that some mutation performed on $-v$ caused some $-w$ to be transformed into $-w'=-w-kv$ which isn't real. $\langle w', w'\rangle = \langle w, w\rangle + k\langle v,w\rangle + k\langle w,v\rangle + k^2\langle v,v\rangle$. Since $Hom(M_v, M_w) = Hom(M_w, M_v) = 0$ due to $M_v$, $M_w$ being two elements in a simple-minded collection and $v, w$ are both real  $\langle w', w'\rangle = k^2+1-k\, dim Ext^1(M_v, M_w) - k\, dim Ext^1(M_w, M_v)$. Using properties of simple-minded collections $dim Ext^1(M_w, M_v)  = k$. Using Prop 6.4 in \cite{KQ15} we can see that $Ext^1(M_v, M_w)  = 0$, Hence $\langle w', w'\rangle = 1$. $w'$ is real.\\
\indent \textbf{Case 2}: Assume that some mutation performed on $-v$ caused some $w$ to be transformed into $-w'=w-kv$ which isn't real. Using properties of simple-minded collections it is obvious that $Ext^1(M_v, M_w) = Hom(M_v, M_w) = 0$. Regardless of whether $w'$ is positive or negative $\langle w', w'\rangle = \langle w, w\rangle - k\langle v,w\rangle - k\langle w,v\rangle + k^2\langle v,v\rangle = k^2+1 -k\,dim Hom(M_w, M_v) + k\,Ext^1(M_w, M_v)$. Using properties of simple-minded collections $dim Hom(M_w, M_v)  = k$. Using Prop 6.4 in \cite{KQ15} we can see that $Ext^1(M_w, M_v)  = 0$, Hence $\langle w', w'\rangle = 1$. $w'$ is real.\\
\indent The assumption has been refuted. Any $c$-vector $c$ that appears in any MGS is a real Schur root.
\end{proof}
\indent In order to prove \ref{C4T1} we need to first prove a lemma.
\begin{lemma}\label{C4L}
\indent If $-c_1, -c_2$ are negative $c$-vectors in $C$-matrix $C'$ in an MGS, $c_1$ and $c_2$ are dimension vectors of indecomposable modules $M_1$ and $M_2$. If $dim Ext^1(M_1, M_2) > 1$ then the mutation on $C'$ must not be done on $M_2$.
\end{lemma}
\begin{proof}
Assume that $-c_1$ is the $i$-th column and $-c_2$ is the $j$-th colu,n. According to \cite{KY12} using the definition of left mutations of simple-minded collections if $dim Ext^1(M_1, M_2) > 1$ then the mutation on $-c_2$ would cause $-c_1$ to be transformed into $-c_1-kc_2$ with $k>1$ because which could only happen if there are multiple edges from $i$ to $j$. Due to \cite{BHIT15} this was impossible.\\
\end{proof}
\subsection{$t$-structures}
\indent Here is the definition of $t-$structures.
\begin{definition}
A $t$-\textit{structure} on $D^b(\Lambda)$ is a pair $(D^{\leq 0},D^{\geq 0})$ such that the following holds.
\begin{enumerate}
\item For any $M\in D^b(\Lambda)$ there exists $M'\in D^{\leq 0}, M''\in D^{\geq 0}$ such that $M'\to M\to M''\to M'[1]$.
\item $D^{\leq 0}[1]\subseteq D^{\leq 0}$,  $D^{\geq 0}[1]\supseteq D^{\geq 0}$.
\item $(D^{\leq 0}[1], D^{\geq 0}) = 0$
\end{enumerate}
\end{definition}
\begin{example}
Let $\Lambda$ be any finite dimensional algebra. The \textit{standard $t$-structure}\\ $(\cup_{m=0}^{\infty}\Lambda[m], \cup_{m=0}^{\infty}\Lambda[-m])$ is clearly a $t$-structure.
\end{example}
\indent Now let's define hearts which will be very useful for a crucial proof in Chapter \ref{C3}, namely the proof of Lemma \ref{lem:C3L1}.
\begin{definition}
The \textit{heart} of a $t$-structure $(D^{\leq 0},D^{\geq 0})$ is defined as $\mathcal{H} = D^{\leq 0}\cap D^{\geq 0}$
\end{definition}
\begin{theorem}
\cite{BBD} Hearts of $t$-structures are Abelian categories. 
\end{theorem}
\begin{example}
Let $\Lambda$ be any finite dimensional algebra. The heart of the standard $t$-structure is $mod \Lambda$ itself which is of course Abelian.
\end{example}
\indent Now let's introduce truncation functors associated with $t$-structures. 
\begin{lemma}
\cite{BBD}  Let $\catd$ be a triangulated category. Let $(D^{\leq 0},D^{\geq 0})$ be a $t$-structure on $\catd$. The inclusion ${\mathcal{D}^{\geq n} \rightarrow \mathcal{D}}$ has a left adjoint ${\tau_{\geq n}}$. Similarly, the inclusion ${\mathcal{D}^{\leq n} \rightarrow \mathcal{D}}$ has a right adjoint ${\tau_{\leq n}}$. These are called \textit{truncation functors}.
\end{lemma}
%\begin{definition}
%\cite{B07} A $t$-structure $(D^{\leq 0},D^{\geq 0})$ of the triangulated category $\catd$ is \textit{bounded} if $\catd = \cup_{i,j} D^{\leq i}\cap D^{\geq j}$.
%\end{definition}
%\begin{example}
%Let $\Lambda$ be any finite dimensional algebra. The standard $t$-structure is clearly bounded since $D^b(\Lambda) = \cup_i D^{\leq i}\cap D^{\geq i}$.
%\end{example}
%\indent $t$-structures can be mutated just like silting objects and simple-minded collections. In order to do so we first need to define torsion pairs in Abelian categories.\\
%\begin{definition}
%A \textit{torsion pair} $(\mathcal{T},\mathcal{F})$ in an Abelian category $\catc$ is a pair of two subcategories such that the following holds.
%\begin{enumerate}
%\item $Hom(\mathcal{T},\mathcal{F}) = 0$
%\item For any $M\in\catc\,\exists T\in\mathcal{T},\, F\in\mathcal{F}$ such that $0\to T\to M\to F\to 0$ is a short exact sequence.
%\item If for $M\in\catc$ and $Hom(M,\mathcal{F}) = 0$ then $M\in\mathcal{T}$.
%\item If for $M\in\catc$ and $Hom(\mathcal{T}, M) = 0$ then $M\in\mathcal{F}$.
%\end{enumerate}
%\end{definition}
%\begin{example}
%Let $\catc$ be $mod kQ$ with $Q$ being $1\to 2$ if we take $\mathcal{T} = add(P_2)$ and $\mathcal{F} = add(I_1)$ we can see that the pair $\cattf$ satisfies the conditions above and is hence a torsion pair in $\catc$.
%\end{example}
%\begin{example}
%Let $\catc$ be $mod kQ$ with $Q$ being $1\to 2$ straight orientation if we take $\mathcal{T} = add(P_1,I_1)$ and $\mathcal{F} = add(P_2)$ we can see that the pair $\cattf$ satisfies the conditions above and is hence a torsion pair in $\catc$.
%\end{example}
%\indent Now it is possible to define mutations of $t$-structures.
%\begin{definition}
%\cite{KY12}Let $\Lambda$ be a finite dimensional algebra, let $D^b(\Lambda)$ be the bounded derived category of $\Lambda$. Let $(\catc^{\leq 0},\catc^{\geq 0})$ be a $t$-structure of $D^b(\Lambda)$. Let $\cata$ be its heart. Let $\cattf$ be a torsion pair in $\cata$. The \textit{left mutation} or \textit{forward mutation} $\mu_i^+(\catc^{\leq 0},\catc^{\geq 0}) = (\catc'^{\leq 0},\catc^{\geq 0})$  where $\catc'^{\leq 0} = \{M\in\catc| H^m(M) = 0 \text{ for } m>0 \text{ and } H^0(M)\in\catt\}$, $\catc'^{\geq 0} = \{M\in\catc| H^m(M) = 0 \text{ for } m<-1 \text{ and } H^{-1}(M)\in\catf\}$. Similarly we can define \textit{right mutations} (or \textit{backward mutations}). 
%\end{definition}
\indent For more information about mutations of $t$-structures we recommend \cite{AI10}, \cite{BY13}, \cite{KQ15} and \cite{KY12}.
\subsection{Green sequences}
\indent Since we have maximal green sequences it is reasonable to look at the generalization of this concept, namely $m$-maximal green sequences. In order to do so we need to define the general concept of green and red sequences. In principle any forward mutation is considered green and any backward mutation red.\\
\begin{definition}
\begin{enumerate}
\item Let $\Lambda$ be a finite dimensional algebra of finite global dimension, a mutation sequence in $D^b(\Lambda)$ is \textit{green} if it contains only forward mutations. 
\item Let $\Lambda$ be a finite dimensional algebra of finite global dimension, a mutation sequence in $D^b(\Lambda)$ is \textit{red} if it contains only backward mutations. 
\item Let $\Lambda$ be a finite dimensional algebra of finite global dimension, a mutation sequence in $D^b(\Lambda)$ is \textit{$k$-red} if it contains $k$ backward mutations. 
\item Let $\Lambda$ be a finite dimensional algebra of finite global dimension, a mutation sequence in $D^b(\Lambda)$ is \textit{$k$-green} if it contains $k$ forward mutations.
\end{enumerate}
\end{definition}
\indent Note that a $0$-red sequence is just a green one. A $0$-green sequence is just a red one. Now we can introduce $m$-maximal green sequences. For the purpose of the proof in Chapter \ref{C2} it is much better to use silting objects.\\
\begin{definition}
An \textit{$m$-maximal green sequence} is a green sequence of silting objects from $\Lambda$ to $\Lambda[m]$.
\end{definition}
\indent It is easy to see that a 1-maximal green sequence is just a maximal green sequence.\\
\begin{example}
\indent Again our example is $A_3$ straight orientation.\\
\begin{tikzcd}
I_1[-1]\arrow[rd]& &P_1\arrow[rd] & & P_3[1]\arrow[rd] & & S_2[1]\arrow[rd] & & I_1[1]\\
& P_2\arrow[ru]\arrow[rd]& &I_2\arrow[rd]\arrow[ru] & &P_2[1]\arrow[rd]\arrow[ru] & &I_2[1]\arrow[rd]\arrow[ru]\\
 P_3\arrow[ru]& &S_2\arrow[ru]& &I_1\arrow[ru] & &P_1[1]\arrow[ru] & &P_3[2]\\
\end{tikzcd}\\
\indent So $(P_1,P_2,P_3,P_1[1],P_2[1],P_3[1])$ is a 2-maximal green sequence, so is $(P_1,P_3,P_2,S_2,P_1[1],P_2[1],P_3[1])$ because they are both sequences of indecomposable objects forward mutations on which produce $\Lambda[2]$ from $\Lambda$. \\
\end{example}
\section{Tame quivers and tame hereditary algebras}
\indent In this subsection we will review the basics about tame hereditary algebras, the components of their Auslander-Reiten quivers and the components of Auslander-Reiten quivers of their bounded derived categories for they are crucial to Chapter \ref{C2}. For more details about tame algebras we would like to refer the readers to \cite{DR76}, \cite{R84} and \cite{SS06}.\\
\subsection{Tame quivers}
\begin{definition}
A \textit{tame algebra} is a $k$-algebra such that for each dimension vector there are finitely many 1-parameter families that parametrize all but finitely many indecomposable modules of the algebra.\\
\end{definition}
\begin{definition}
A \textit{tame quiver} is a quiver such that its path algebra is a tame algebra.\\
\end{definition}
\begin{example}
Here are all the (connected) tame quivers, $\tilde{A_n}, \tilde{D_n}, \tilde{E_6}. \tilde{E_7}, \tilde{E_8}$. The orientation of the edges can be arbitrary as long as the quiver remains acyclic in the case of $\tilde{A_n}$.
\end{example}
$\begin{tikzcd}
\tilde{A_n} &    		&2\arrow[r]  &\cdots\arrow[r]    &i\arrow[rd]	 &\\
&1\arrow[rd]\arrow[ru]& 		  &  				&   		&n+1\\
&     				&i+1\arrow[r]&\cdots\arrow[r] 	&n\arrow[ru]& \\
\end{tikzcd}$\\
$\begin{tikzcd}
\tilde{D_n} &1\arrow[rd] &  		& 		     &		         		&					&n	\\
&		&3\arrow[r]&  4\arrow[r] & \cdots\arrow[r]           &n-1\arrow[rd]\arrow[ru] 	&\\
&2\arrow[ru]&		&   		    & 					& 					&n+1\\
\end{tikzcd}$\\
$\begin{tikzcd}
\tilde{E_6}& 1\arrow[r] & 2\arrow[r] & 3\arrow[r]\arrow[d] & 4\arrow[r] & 5\\
&		&		&  6\arrow[d] & 			& \\
&		&		&  7 & 			& \\
\end{tikzcd}$\\
$\begin{tikzcd}
\tilde{E_7}& 1\arrow[r] & 2\arrow[r] & 3\arrow[r]& 4\arrow[r]\arrow[d]  & 5\arrow[r] & 6\arrow[r] & 7\\
&		&		   &  		     & 	8		      & 		& 		&\\
\end{tikzcd}$\\
$\begin{tikzcd}
\tilde{E_8}\,\,\,\, 1\arrow[r] & 2\arrow[r] & 3\arrow[r]\arrow[d] & 4\arrow[r] & 5\arrow[r] & 6\arrow[r] & 7\arrow[r] & 8\\
	&		&  			9& 			& & & &\\
\end{tikzcd}$
\subsection{Auslander-Reiten quivers of tame hereditary algebras}
\indent In this subsection I'm going to discuss Auslander-Reiten quivers of basic tame hereditary algebras because information about them is slightly less well known.\\
\begin{theorem}
\indent \cite{DR76} The Auslander-Reiten quiver of a tame path algebra consists of three parts, the preprojectives, the preinjectives and the regulars.
\end{theorem}
\indent Here are some basic properties of preprojective and preinjective components of AR quivers of basic tame hereditary algebras.
\begin{theorem}
\begin{enumerate}
\item The AR quiver of $kQ$ has one preprojective component which is isomorphic to $\mathbb{N}Q^{op}$
\item The AR quiver of $kQ$ has one preinjective component which is isomorphic to $-\mathbb{N}Q^{op}$.
\item All preprojective and preinjective modules in $kQ$ are rigid.
\item All but finitely many preprojectives and preinjectives are sincere.
\item There are infinitely many regular components, all of which are standard tubes $\mathbb{Z}A_{\infty}/(\tau^k)$.
\item All but at most three tubes have $k=1$. In this case we consider the component homogeneous.
\item All elements in a homogeneous tube are non-rigid, hence they and their shifts can not be summands of any silting object.
\item In a nonhomogeneous component $\mathbb{Z}A_{\infty}/(\tau^k)$ only indecomposables with quasi-length less than $k$ are rigid. In other words there are only finitely many rigid indecomposables in any nonhomogeneous component.
\item Only finitely many regular indecomposable modules are rigid. Hence only finitely many regular indecomposables and their shifts can appear in an $m$-maximal green sequence.
\end{enumerate}
\end{theorem}
\indent I'm going to introduce one example of nonhomogeneous and homogeneous standard stable tubes each. For more details we recommend Chapter X of \cite{SS06}.\\
\begin{example}
\indent Here is a standard stable tube with rank 3.\\
\begin{tikzcd}
& \cdots\arrow[rd]& &\cdots\arrow[rd] & &\cdots\arrow[rd]&\\
M_{33}\arrow[rd]\arrow[ru]& &M_{13}\arrow[rd]\arrow[ru] & &M_{23}\arrow[rd]\arrow[ru] & & M_{33}\\
& M_{12}\arrow[ru]\arrow[rd]& &M_{22}\arrow[rd]\arrow[ru] & &M_{32}\arrow[rd]\arrow[ru] &\\
 M_1\arrow[ru]& &M_2\arrow[ru]& &M_3\arrow[ru] & & M_1\\
\end{tikzcd}
 $M_{ik}$ is rigid iff $k\leq 2$.\\
\indent Here $M_{ik}=\begin{tikzcd}M_{i+k-1}\\\cdots\\M_{i+1}\\M_i\end{tikzcd}$. we define the \textit{quasi-length} of $M_{ik}$ as $k$.
\end{example}
Now let's see a homogeneous tube.\\
\begin{example}
\indent Here is a homogeneous standard stable tube.\\
\begin{tikzcd}
\cdots\arrow[d,bend left=50]\\
M_3\arrow[u,bend left=50]\arrow[d,bend left = 50]\\
M_2\arrow[u,bend left=50]\arrow[d,bend left = 50]\\
M\arrow[u, bend left = 50]\\
\end{tikzcd}
\indent Note that no module in this tube is rigid.\\
\indent Here $M_{k}=\begin{tikzcd}M\\\cdots\\M\end{tikzcd}$
\end{example}
\indent Now let's do an example of an AR quiver of a tame path algebra.\\
\begin{example}
The quiver is \begin{tikzcd}&2\arrow[d]&\\1\arrow[r]&5&3\arrow[l]\\&4\arrow[u]&\\\end{tikzcd}.
\indent Here is the preprojective component, $\mathcal{P}$.\\
\begin{tikzcd}
&P_1\arrow[rdd] & &\tau^{-1}P_1\arrow[rdd] &\cdots \\
&P_2\arrow[rd] & &\tau^{-1}P_2\arrow[rd] &\cdots\\
P_5\arrow[ruu]\arrow[ru]\arrow[rd]\arrow[rdd]& &\tau^{-1}P_5\arrow[ruu]\arrow[ru]\arrow[rd]\arrow[rdd] & &\tau^{-2}P_5\cdots\\
&P_3\arrow[ru] & &\tau^{-1}P_3\arrow[ru]  &\cdots\\
&P_4\arrow[ruu] & &\tau^{-1}P_4\arrow[ruu] &\cdots\\
\end{tikzcd}\\
Here is the preinjective component, $\mathcal{Q}$.\\
\begin{tikzcd}
\cdots&\tau I_1\arrow[rdd] & &I_1 &\\
\cdots&\tau I_2\arrow[rd] & &I_2 &\\
\cdots \tau I_5\arrow[ruu]\arrow[ru]\arrow[rd]\arrow[rdd]& & I_5\arrow[ruu]\arrow[ru]\arrow[rd]\arrow[rdd] & &\\
\cdots&\tau I_3\arrow[ru] & &I_3  &\\
\cdots&\tau I_4\arrow[ruu] & &I_4 &\\
\end{tikzcd}\\
\indent Here are the regular components. There are infinitely many homogeneous tubes and 3 nonhomogeneous ones. All objects in the homogeneous ones are non-rigid. The quasi-simple in the homogeneous tubes has dimension vector is (1,1,1,1,2). The quasi-simples in the three nonhomogeneous tubes have dimension vectors (1,1,0,0,1) and (0,0,1,1,1), (1,0,1,0,1) and (0,1,0,1,1), (1,0,0,1,1) and (0,1,1,0,1) respectively.\\
\end{example}
\indent Finally let's discuss Auslander-Reiten quivers of $D^b(kQ)$. For a tame quiver $Q$ there are infinitely many components of $D^b(kQ)$ consisting of shifts of preprojectives and preinjectives that are isomorphic to $\mathbb{Z}Q^{op}$. Let's label these components \textit{transjective}. The transjective component containing $\Lambda[m]$ is labelled $\mathcal{P}_m$.\\
\indent There are also infinitely many regular components. There are at most 3 nonhomogeneous tubes in $mod kQ[m]$ for any $m$. There are also infinitely many homogeneous tubes in $mod kQ[m]$ for any $m$. However since no module in a homogeneous tube is rigid they don't affect our problem.\\
\section{Wall-and-chamber structures}
\indent In this section we will discuss miscellaneous topics on the wall-and-chamber structures, namely picture groups and alternative definitions of maximal green sequences. This section is mostly relevant to Chapter \ref{C1} and Chapter \ref{C3}. For more details we recommend \cite{GHKK14}\cite{Mul15}\cite{BST17}\cite{BST18A}\cite{BST18B}\cite{IOTW15} and \cite{IT17}.\\
\subsection{Picture groups}\label{Picgr}
\indent We also need to use the concept of the picture groups in order to prove the formula in Chapter \ref{C1}. For more details about picture groups we recommend \cite{IT17}.\\
\indent For a quiver of finite type, any dimension vector of an indecomposable representation is referred to as a \textit{root}. Let $D(\beta)\subseteq\mathbb{R}^n$, $D(\beta)= \{x\in\mathbb{R}^n: \langle x,\beta\rangle=0,\ \langle x,\beta'\rangle\leq 0\text{ when }\beta'\subseteq\beta\}$. Here $\beta'\subseteq\beta$ means the unique indecomposable representation of dimension vector $\beta'$ is a subrepresentation of the unique indecomposable representation of dimension vector $\beta$. $D(\beta)$ for all these roots divide $\mathbb{R}^n$ into \textit{compartments}. The boundary of each compartment is the union of some $D(\beta)$ which we call \textit{walls}.\cite{IT17}\cite{IOTW4} Sometimes we abuse notations and use the root $\beta$ to mean the wall $D(\beta)$ when the meaning is clear. We also use the notation $+\beta$ to mean the wall $\beta$ is a part of the boundary of a compartment $\mathcal{U}$ and for any point $x\in\mathcal{U}$, $\langle x,\beta\rangle\ >0$. Similarly we have the notation $-\beta$. For example $+\beta-\beta'$ means that $\beta$ and $\beta'$ are parts of the boundary of a compartment $\mathcal{U}$ and for any point $x\in\mathcal{U}$, $\langle x,\beta\rangle\ >0$ and $\langle x,\beta'\rangle\ <0$.\\
\begin{definition}
\cite{IT17} A \textit{picture group} of a cluster quiver of finite type $Q$ is a group $G(Q)=\langle S|R\rangle$ with $S$ in bijection with the set of real Schur roots (the generator for $\beta$ is $x(\beta)$) and $R$ the set of relations $x(\beta_i)x(\beta_j)=\Pi x(\gamma_k)$ with $\gamma_k$ running over all these real Schur roots which are linear combinations $\gamma_k = a_k\beta_i+b_k\beta_j$ with $a_k/b_k$ increasing (going from 0/1 where $\gamma_1=\beta_j$ to 1/0 where $\gamma_k=\beta_i$) for any pair $(\beta_i,\beta_j)$ such that they are Hom-orthogonal and $Ext(\beta_i,\beta_j)=0$.\\
\end{definition}
\indent Now we need to restrict the case to $A_n$ straight orientation, namely quivers of the form $1\to 2\to 3\to\cdots\to n$.
\begin{definition}
Let $Q$ be an acyclic quiver. A representation $(\{V_i\}_{i\in Q_0}, \{\phi_a\}_{a\in Q_1})$ of $Q$ is \textit{thin} if $dim(V_i) \leq 1$ for all $i\in Q_0$.
\end{definition}
\indent In $A_n$ in particular since all indecomposable representations are thin, the roots are $\beta_{ij}=e_j-e_i$ ($0<i<j<n$, $e_0$ is defined as the zero vector). The root $\beta_{ij}$ corresponds to the picture group generator $x_{ij}$ which we will define right now.\\
\indent Note that for quiver $A_n$ all roots are real and Schur hence a real Schur root is just a root. Also we often simplify the notation of $x(\beta_{ij})$ to $x_{ij}$ which we use interchangeably with $x(\beta_{ij})$. The picture group for $A_n$ straight orientation is $G(A_n)=\{S|R\}$, $S=\{x_{ij}|0\leq i<j\leq n\}$, $R=\{x_{ij}x_{kl}=x_{kl}x_{ij}|[i,j]\cap[k,l]=\emptyset, [i,j]\text{ or }[k,l], $ i,j,k,l \\are distinct.\}$\cup\{x_{jk}x_{ij}=x_{ij}x_{ik}x_{jk}|0\leq i<j<k\leq n\}$.\\
\indent Igusa and Todorov proved \cite{IT17} that there exists a bijection between the set of maximal green sequences and the set $\mathcal{P}(c)$ of positive expressions of the Coxeter element of the picture group for any acyclic valued quiver of finite type which applies to $A_n$ straight orientation.\\
\subsection{Alternative definitions of maximal green sequences}
\indent In the following theorem by Kiyoshi Igusa multiple equivalent definition of maximal green sequences was introduced. The full version of Igusa's results includes more discussions about the wall-and-chamber structure which we will not discuss here. To understand more about the wall-and-chamber structure we suggest that the reader reads \cite{GHKK14}\cite{Mul15}\cite{BST17}\cite{BST18A}\cite{BST18B} and \cite{IOTW15}.
\begin{definition}
Let $\Lambda$ be a finite dimensional algebra. $\Lambda-$modules $M_1,\cdots,M_m$ are a \textit{backward Hom-orthogonal sequence} if $Hom_\Lambda(M_i,M_j)=0$ for $i>j$.
\end{definition}
\begin{example}
In the $Q:1\to 2$ example $\{P_2, P_1\}$ is a backward $Hom$-orthogonal sequence because $Hom(P_1, P_2) = 0$.
\end{example}
\begin{example}
In the $Q:1\to 2$ example $\{P_1, P_2\}$ is not a backward $Hom$-orthogonal sequence because $Hom(P_2, P_1) = k$.
\end{example}
\begin{definition}
Let $\Lambda$ be a finite dimensional algebra. A backward Hom-orthogonal sequence $M_1,\cdots,M_m$ is \textit{maximal} if no other modules can be inserted into the sequence preserving the property of backward Hom-orthogonality.
\end{definition}
\begin{example}
In the $Q:1\to 2$ example $\{P_2, P_1\}$ is not a maximal backward $Hom$-orthogonal sequence even though it is a backward $Hom$-orthogonal sequence because it can be extended to $P_2, P_1, I_1$ without losing backward $Hom$-orthogonality.
\end{example}
\begin{example}
In the $Q:1\to 2$ example $\{I_1, P_2\}$ is a maximal backward $Hom$-orthogonal sequence because $Hom(P_2, I_1) = 0$ and that there is no place to fit any other module in the sequence. For example $P_1$ can not be inserted before $P_2$ because $Hom(P_2, P_1)\neq 0$. At the same time it can not be inserted after $I_1$ because $Hom(P_1, I_1)\neq 0$. Hence the sequence can not be extended to include $P_1$.
\end{example}
\indent In fact $Q:1\to 2$ only has two maximal backward $Hom$-orthogonal sequences, namely $I_1,P_2$ and $P_2, P_1, I_1$. They are exactly the same as the two $c$-vectors in maximal green sequences of $Q$. Is this just a coincidence? No.
\begin{definition}
Let $\{M_1,\cdots, M_k\}$ be a fixed finite sequence of Schur objects in $mod \Lambda$. An \textit{Harder-Narasimhan (HN) filtration} with respect to $\{M_i\}$ aka an HN filtration of an object $X$ is k short exact sequences $0\to X_i\to X_{i-1}\to X_{i-1}/X_i$ such that $X_{i-1}/X_i\in\cate{M_i}$. 
\end{definition}
\begin{definition}
Let $\{M_1,\cdots, M_k\}$ be a fixed finite sequence of Schur objects in $mod \Lambda$. $\{M_1,\cdots, M_k\}$ is an \textit{finite Harder-Narasimhan (HN) system} if any $X\in mod \Lambda$ has a unique HN filtration with respect to $\{M_i\}$.
\end{definition}
\begin{example}
In the $Q:1\to 2$ example $\{I_1, P_2\}$ is a finite HN system because any module $X\in mod kQ$ has a unique HN filtration. In particular the unique HN filtration of $P_1$ is $0\to P_2\to P_1\to I_1\to 0$.
\end{example}
\begin{example}
In the $Q:1\to 2$ example $\{P_2, P_1, I_1\}$ is a finite HN system because any module $X\in mod kQ$ has a unique HN filtration.
\end{example}
\indent There are only two finite HN systems of $kQ$ where $Q:1\to 2$. They are exactly the same as the two maximal backward $Hom$-orthogonal sequences and the two maximal green sequences. This is in fact a general result.\\
\begin{theorem}
\cite{I17} Let $\Lambda$ be a finite dimensional hereditary algebra over a field $K$. Let $\beta_1,\cdots,\beta_m\in \mathbb{N}^n$ be any finite sequence of nonzero, nonnegative integer vectors. Then the following are equivalent.\label{thm:3}
\begin{enumerate}
%\item[(a)] There is a nonlinear stability function $Z_t:K_0\Lambda\to \mathbb{C}$ which is green and has exactly $m$ semistable pairs $(M_i,t_i)$ with $t_1<t_2<\cdots<t_m$ so that $\dim M_i=\beta_i$ for all $i$.%(making all pairs stable) 
%\item[(b)] There is a generic green path $\gamma:\mathbb{R}\to\mathbb{R}^n$ which crosses the walls $D(M_i)$, $i=1,\cdots,m$ in that order, and no other walls, so that $\dim M_i=\beta_i$ for all $i$.
\item[(a)] There exist $\Lambda$-modules $M_m,\cdots,M_1$ with $\dim M_i=\beta_i$ which form a finite Harder-Narasimhan system for $\Lambda$. 
\item[(b)] There exist a finite sequence of Schurian $\Lambda$-modules $\{M_1,\cdots,M_m\}$ with $\dim M_i=\beta_i$ such that $\{M_1,\cdots,M_m\}$ is a \textit{maximal backward Hom-orthogonal sequence}.
\item[(c)] There is a maximal green sequence for $\Lambda$ of length $m$ whose $i$th mutation is at the $c$-vector $\beta_i$. 
\end{enumerate}
\end{theorem}
\indent Such results have been generalized to the case of $m$-maximal green sequences in Chapter \ref{C3}.

%\section{Quiver folding}
%\indent In this section we will introduce the theory of quiver folding. Folding theory has been in folklore for a while. Despite not being used to prove anything in the thesis we are still going to discuss it here in details.\\
%\begin{definition}
%Let $B$ be an $n\times n$ exchange matrix and let $\rho\in S_n$ be a permutation. If $\rho(B)=B$ then $\rho$ is a \textit{symmetry} of $B$. The group of symmetries of $B$ is the \textit{Symmetry Group of} $B$ which we denote as $Sym\ B$. An exchange matrix with a non-trivial symmetry group is a \textit{symmetric exchange matrix}. Any nontrivial subgroup of $B$ is a \textit{Symmetry Subgroup of} $B$. The symmetry group of a valued quiver is defined as the symmetry group of the exchange matrix of the valued quiver.\\   
%\end{definition}
%\indent A symmetry subgroup $G$ acts on the extended exchange matrices $\tilde{B}$ in the obvious way, namely for some $\rho\in G$ $\rho \tilde{B}:=\rho(\tilde{B})$. We can also define right group actions of elements of $G$ on the set of $c$-vectors similarly, namely for any $v=(v_i)\in \mathbb{Z}^n$, $v\rho:=(v_{\rho(i)})$. Let $C(\mathcal{A})$ be the set of $c$-vectors of a cluster algebra of geometric type $\mathcal{A}(B)$. Orbits of $c\in C(\mathcal{A})$ is denoted as $Gc$. It is easy to see that all orbits are finite. $Fc := \Sigma_{c'\in Gc}\ c'$ is the \textit{folded version of} $c$.\\
%\indent Now we need to fold vertices first. For any nontrivial subgroup of $S_n$ let $n'$ be the set of orbits of the canonical group action of $S_n$ on $[n]$. Hence there exists maps from $[n]$ to $[n']$. Pick some surjection $f$ from $[n]$ to $[n']$ such that $f$ maps each orbit to one element of $[n']$. $f$ is a \textit{vertex folding map}. We sometimes abuse notations and identify $f(i)$ and $Gi$ when we do not need to specify $f$.\\
%\begin{definition}
%(1)For any valued quiver $Q$ and its symmetry subgroup $G$, the \textit{folded version of $Q$ with respect to $G$} is defined as below:\\
%For each valued arrow $i\overset{(d_{ij},d_{ji})}{\longrightarrow} j$ it is replaced by $Gi\overset{(d_{ij}|Gi|,d_{ji}|Gj|)}{\longrightarrow} Gj$.\\
%(2)For any symmetric exchange matrix $B$ and its symmetry subgroup $G$, the \textit{folded version of $B$ with respect to $G$} is defined as $\tilde{B}=(b'_{kl})$ where $b'_{GiGj}=b_{ij}|Gj|$. \\ \cite{Sal14}\cite{BHIT15}\\
%\end{definition}
%\begin{definition}
%For any symmetry subgroup $G$ of $B$ any extended exchange matrix $\tilde{B}=(B',C')$ such that $B'$ is symmetric is a \textit{symmetric extended exchange matrix with respect to} $G$ if for any $i\in [n]$ $|Gc_i|=|G_i|$ and for any $\rho\in G$, $Gc_{\rho(i)}=Gc_i$.\\
%\end{definition}
%\indent Any symmetric extended exchange matrix can be folded.\\
%\begin{definition}
%For any symmetry subgroup $G$ of $B$ any symmetric extended exchange matrix $\tilde{B}=(B',C')'$ with respect to $G$. Then for any vertex folding map $f$ $F\tilde{B}:=(FB',FC')'$ is the \textit{folded version of} $\tilde{B}$ with $FC'=(\tilde{c}_1,\cdots, \tilde{c}_n)$ defined below: $\tilde{c}_i=\Sigma_{g\in f^{-1}(i)} \tilde{c}_g/|Gg|$ where $\tilde{c}_{gi}:=\Sigma_{k\in f^{-1}(i)} c_{gk}$.\\
%\end{definition}
%\indent It is easy to see that if $c_i=e_i$ then $C=I_n$ then $C'=I_{n'}$ and if $C=-I_n$ then $C'=-I_{n'}$. Hence a framed valued quiver is folded into a framed valued quiver and a coframed valued quiver is folded into a coframed valued quiver. Also positive $c$-vectors are folded into positive ones and negative $c$-vectors are folded into negative ones.\\
%\begin{definition}
%For any symmetry subgroup $G$ of $B$ a mutation sequence $w=\Pi_{i=m}^1 \mu_{k_i}$ starting from an extended exchange matrix $\tilde{B}=(B',C')'$ with $B$ symmetric is \textit{symmetric with respect to } a symmetry subgroup $G$ if the following holds:\\
%1.$w$ is in the form $w=\Pi_{i=m}^1 \Pi_{j\in Gk_i} \mu_j$, which roughly means that vertices in any orbit is "mutated together".\\
%2.$\Pi_{i=m'}^1 \mu_{k_i} \tilde{B}$ is symmetric.\\
%\end{definition}
%\begin{definition}
%For any symmetry subgroup $G$ of a symmetric extended exchange matrix $\tilde{B}$ for any symmetric mutation sequence $w=\Pi_{i=m}^1 \Pi_{j\in Gk_i} \mu_j$, \textit{the folded version of $w$} is defined as $Gw:=\Pi_{i=m}^1 \mu'_{Gk_i}$.
%\end{definition}
%\indent It is easy to see that folding symmetric reddening sequences results in reddening sequences. Also folding symmetric green sequences results in green sequences. Folding symmetric maximal green sequences results in maximal green sequences.\\
%\begin{theorem}
%For any symmetry subgroup $G$ of a symmetric extended exchange matrix $\tilde{B}$ for any symmetric mutation sequence $w$, $F\circ w=Gw\circ F$.\\
%\end{theorem}
%\begin{proof}
%\indent Let's assume that the length of $Gw$ is 1. When the theorem has been proven in this particular case the rest is clear from induction. Hence let's assume $w=\Pi_{j\in Gi} \mu_j$ and $Gw=\mu'_{Gi}$. Note that $b_{jk}=0$ for any $j,k\in Gi$ since otherwise $j$ and $k$ would not be in the same orbit. It is also clear from symmetry that the set of vertices in $[n]$ that is a source of any valued arrow with some $j\in Gi$ its target is independent of the choice of $j$, which we denote as $P^{-}(Gi)$. Similarly we can define $P^+(Gi)$. The set of vertices in $[n]\backslash Gi$ that is not connected to any $j\in Gi$ is denoted as $I(Gi)$. Using the invariance lemmas it is easy to see that for any $j\in I(i)$ mutations at any element of $Gi$ does not affect the $j$-th row and the $j$-th column at all.\\
%\indent Assume that we do mutations on a symmetric extended exchange matrix $\tilde{B}=(B',C')'$. Let $B=(b_{jk})$, $C=(c_{jk})$. Hence $b'_{GjGk}=b_{ij}|Gj|$. $c'_{GjGk}=\Sigma_{l\in Gj k\in Gm}c_{lm}$.  Let $\tilde{C}=\mu'_{Gi}(C)=(c'_{j'k'})$ $w(\tilde{B})=(\tilde{b}_{jk})$ $w(\tilde{C})=(\tilde{c}_{jk})$ $F\tilde{B}=(\hat{B}',\hat{C}')'$ $F\circ w(\tilde{B})=(B_1',C_1')'$  $Gw\circ F(\tilde{B})=(B_2',C_2')'$. $B_1=(b^1_{jk})$ $C_1=(c^1_{jk})$ $B_2=(b^2_{jk})$ $C_2=(b^1_{jk})$.\\
%\indent Let's first calculate the left hand side. If $j$ or $k\in I(Gi)$ it is clear that $\tilde{b}_{jk}=b_{jk}$. If $j\in i$ or $k=i$ $\tilde{b}_{jk}=-b_{jk}$ due to the fact that $b_{jj'}=0$ for any $j,j'\in Gi$. Otherwise it is easy to see that $\tilde{b}_{jk}=b_{jk}+|Gi|sp(b_{ji},b_{ik})$ due to two facts: First of all, before a mutation at $l\in Gi$, the $l$-th row and column of the exchange matrix can not be affected by all previous mutations since between elements of $Gi$ there are no connections. Secondly for all $l\in Gi$, $b_{jl}$ and $b_{lk}$ are independent of $l$. Similarly, if $k\in Gi$ $\tilde{c}_{jk}=-c_{jk}$. Otherwise $\tilde{c}_{jk}=c_{jk}+\Sigma_{l\in Gi} sp(c_{jl}, b_{ik})$. Hence $b^1_{GjGk}=-b_{jk}|Gk|$ if $j$ or $k$ is in $Gi$. Otherwise $b^1_{GjGk}=b_{jk}|Gk|+|Gi||Gk|sp(b_{ji},b_{ik})$. If $k\in Gi$ $c^1_{GjGk}=-|Gk|\bar{c}_{GjGk}$. Otherwise $c^1_{GjGk}=\bar{c}_{GjGk}+\Sigma_{l\in Gi} sp(\bar{c}_{GjGl}, b_{ik})=\bar{c}_{GjGk}+|Gi|sp(\bar{c}_{GjGi},b_{ik})$.\\
%\indent Now let's calculate the right hand side. $\hat{b}_{GjGk}=|Gk|b_{jk}$. $\hat{c}_{GjGk}=|Gk|\bar{c}_{GjGk}$. If $j$ or $k\in Gi$ $b^2_{GjGk}=-b_{jk}|Gk|$. Otherwise $b^2_{GjGk}=b_{jk}|Gk|+sp(|Gi|b_{ji}, |Gk|b_{ik})=b_{jk}|Gk|+|Gi||Gk|sp(b_{ji}, b_{ik})$. If $j$ or $k\in Gi$ $c^2_{GjGk}=-|Gk|\bar{c}_{GjGk}$. Otherwise $c^2_{GjGk}=|Gk|\bar{c}_{GjGk}+sp(|Gi|\bar{c}_{GjGi},b_{ik}|Gk|)=|Gk|\bar{c}_{GjGk}+|Gi||Gk|sp(\bar{c}_{GjGi},b_{ik})$.\\
%\indent Hence $F\circ w=Gw\circ F$ has been proven.\\
%\end{proof}

\chapter{Permutation}\label{C1}
\section{The general theory of permutations}
\subsection{Mutation systems}
\indent Let's define a natural setting of the theory of permutations which is completely combinatorial. Let $[n]=\{1,2,\cdots,n\}$.  A \textit{mutation graph} is defined as a connected $n$-regular graph without loops. Let $T=(T_0,T_1)$ be a mutation graph. A \textit{signed edge} of $T$ is a triple $(a,h,t)$ where $a\in T_1$, $h$ and $t$ are the two endpoints of $a$, defined to be the head and tail of the signed edge respectively. For every signed edge $a^{ht}$ there is its inverse signed edge $a^{th}$. Let $\tilde{T_1}$ be the union of all signed edges of $T$. A \textit{walk} is a path in $\tilde{T_1}$ such that the sources and targets of each signed edge are compatible. Walks on $T$ are in the form  $w=\Pi a_k^{i_{k-1} i_k}$.\\ 
\indent For each vertex $x\in T_0$ we associate a set $N(x)$ which is the set that contains all vertices adjacent to $x$. Note that $|N(x)|=n$. For each $a^{ht}\in\tilde{T_1}$ we associate a bijection $f_{a^{ht}}: N(x)\to N(y)$ such that $a^{ht}$ and $a^{th}$ are inverses of each other for any $a\in T_1$. This bijection is called \textit{mutation} as per \cite{FZ06}. The set of all $f_a$ is denoted $A$, the \textit{set of mutations}. The tuple $(T,A)$ is called a \textit{mutation system}. We can also define a natural bijection $f_w:N(x)\to N(y)$ associated with each walk $w=\Pi a_k^{i_{k-1} i_k}$, namely $f_w=f_{a_k}^{i_{k-1}i_k}\cdots f_{a_1}^{i_0i_1}$.\\
\indent Now let's define a bijection $j_x:[n]\to N(x)$ for each $x\in T_0$. This bijection is called the \textit{fixed ordering} of the seed $N(x)$. Let $J$ be the set of all fixed orderings which we call a \textit{fixed ordering set}. The tuple $(T,A,J)$ is called a \textit{ordered mutation system}. Now for each bijection $j'_x:[n]\to N(x)$ we can define its \textit{associated permutation relative to $J$} below:\\
\begin{definition}
For any $(T,A,J)$ for any $x\in T_0$ for any bijection $g:[n]\to N(x)$ the \textit{associated permutation relative to $J$} is defined as $\rho(g)=j_x^{-1}g$. \\
\end{definition}
\indent Now we can define what is the associated permutation of a mutation relative to $J$.\\
\begin{definition}
For any $(T,A,J)$ for any $x,y\in T_0$ for any mutation $f_a:N(x)\to N(y)$ the \textit{associated permutation relative to $J$} is defined as $\rho(f_a)=j_y^{-1}f_aj_x$.
\end{definition}
\indent In other words, $\rho(f_a)$ is the permutation such that the following diagram commutes:\\
$\begin{tikzcd}
{[n]} \arrow[r,"\rho(f_a)"] \arrow[d,"j_x"] & {[n]}\arrow[d,"j_y"]\\
N(x) \arrow[r,"f_a"] & N(y)\\
\end{tikzcd}$\\
\indent Now we can define what it means to be the associated permutation relative to $J$ of any walk $p=a_k^{i_k}\cdots a_1^{i_1}$, namely $\rho(p)=j_y^{-1}f_pj_x$. It is easy to see from the diagram below that $\rho(p)=\rho(a_k)^{i_k}\cdots \rho(a_1)^{i_1}$.\\
$\begin{tikzcd}
{[n]} \arrow[r,"\rho(f_{a_1})^{i_1}"] \arrow[d,"j_{x_1}"] & {[n]}\arrow[r,"\rho(f_{a_2})^{i_2}"]\arrow[d,"j_{x_2}"] & \cdots {[n]}\arrow[d,"j_{x_k}"]\\
N(x_1) \arrow[r,"f_{a_1}^{i_1}"] & N(x_2)\arrow[r,"f_{a_2}^{i_2}"]  & \cdots N(x_k)\\
\end{tikzcd}$\\
\indent We can also discuss the relation between the permutation of a walk $p$ relative to different fixed orderings.\\
\begin{theorem}
(Change of fixed ordering formula) For any mutation system $(T,A)$ for any fixed ordering set $J_1.J_2$ for any $x,y\in T_0$ for any walk $p:x\to y$, let $\rho_1(p),\rho_2(p)$ be the associated permutation of $p$ relative to $J_1,J_2$ respectively. Let $\tau_x=j_{1x}^{-1}j_{2x}$, $\tau_y=j_{1y}^{-1}j_{2y}$. Then $\rho_2(p)=\tau_y^{-1}\rho_1(p)\tau_x=j_{2y}^{-1}j_{1y}\rho_1(p)j_{1x}^{-1}j_{2x}$.\\
\end{theorem}
\indent The theorem can be verified easily by the diagram below.\\
$\begin{tikzcd}
{[n]} \arrow[r,"\rho_2(p)"]\arrow[d,"\tau_x"]\arrow[dd,bend right = 50,"j_{2x}"] & {[n]}\arrow[d,"\tau_y"]\arrow[dd,bend left = 50,"j_{2y}"] \\
{[n]} \arrow[r,"\rho_1(p)"] \arrow[d,"j_{1x}"] & {[n]}\arrow[d,"j_{1y}"]\\
N(x) \arrow[r,"f_a"] & N(y)\\
\end{tikzcd}$\\
\indent We can see that in essence the permutation of a reddening or loop sequence is just special cases of permutations of walks.\\
\indent It is obvious that any $(T,A,J)$ induces a map $g:\tilde{T_1}\to [n]$ that assigns a \textit{fixed position} to each signed edge. We call the map $g$ the \textit{fixed position map}. It is also obvious that for any walk $w: x\rightarrow y$, we only need to fix $j_x, j_y$ to have a fixed $\rho(w)$. $\rho(w)$ is independent of $j_z$ for any $z\neq x, y$. Hence the definition of $\rho(w)$ can be done with any arbitrary choice of $j_z$ for any $z\neq x,y$. In fact even not defining them is also fine since we can just define $\rho(w)$ as $j_y^{-1}f_wj_x$.\\
\section{The associated permutation in $A_n$}
\indent When the quiver is $A_n$ straight orientation we can make much stronger claims. In fact there is a canonical permutation of any mutation sequence. Using the notations in Section \ref{Picgr} the formula of the associated permutation of reddening sequences and loop sequences is given as the following.\\
\begin{theorem}
In $A_n$ straight orientation, the permutation associated with a picture group element that transforms the framed quiver into the coframed quiver or the framed quiver itself is $\rho(\prod_{k}x_{i_kj_k}^{\delta_k}) = (\prod_{k}(i_k+1,j_k))^{-1}$. Here $\delta_k\in\{+,-\}$.\\
\end{theorem}
\indent This formula works for any maximal green, reddening, loop sequences that starts from and ends up in the framed quiver. It also extended the definition of an associated permutation to the set of arbitrary finite sequences of mutations in $A_n$ straight orientation. One interesting property of $A_n$ is that the associated permutation of a mutation only depends on the $c$-vector but not which cluster-tilting object on which the mutation is conducted. This is a highly nontrivial fact: The associated permutation in the general case seems way less regular.\\
\subsection{Forbidden Pairs-of-Walls Lemmas}
\indent Since we use picture groups and related structures to prove the theorem, we need to examine what kind of pairs of walls can not exist in any compartment. Before doing so we first need to discuss notations. Sometimes we abuse notations and use the root $\beta$ to mean the wall $D(\beta)$ when the meaning is clear. We also use the notation $+\beta$ to mean the wall $\beta$ is a part of the boundary of a compartment $\mathcal{U}$ and for any point $x\in\mathcal{U}$, $\langle x,\beta\rangle\ >0$. Similarly we have the notation $-\beta$. For example $+\beta-\beta'$ means that $\beta$ and $\beta'$ are parts of the boundary of a compartment $\mathcal{U}$ and for any point $x\in\mathcal{U}$, $\langle x,\beta\rangle\ >0$ and $\langle x,\beta'\rangle\ <0$.\\
\begin{lemma}
(First Forbidden Pairs-of-Walls Lemma) For any compartment for any short exact sequence of roots $0\rightarrow s\rightarrow r\rightarrow q\rightarrow 0$ it is impossible to have pairs of such walls: $+s-r$ or $-q+r$.\\
(Second Forbidden Pairs-of-Walls Lemma) For quiver $A_n$ for any compartment for any short exact sequence of roots $0\rightarrow s\rightarrow r\rightarrow q\rightarrow 0$ it is impossible to have pairs of such walls: $-s-r$, $-r-q$, $+s+r$, or $+q+r$.\\
\end{lemma}
\indent The first lemma is obvious since in either case $\langle x,s\rangle\ >0$ but the root $R$ is stable which is impossible. As for the second lemma, the reason $-s-r$ can not appear is that in $A_n$ when you cross $D(s)$ you are either going to have $+s-r$ or $+s-(r+s)$. The former is impossible due to Forbidden Pairs-of-Walls Lemma. The latter is impossible due to c-vector theorem and the fact that the sum of a root and any of its subroots is no longer a root any more in $A_n$. The reason the other three cases can not happen is almost identical.\\
\subsection{Proof of the formula}
\indent The basic idea in proving the theorem is below:\\
\indent Since picture group elements freely commute with permutations, what we want to prove can be reduced to  $\rho(\prod_{k}(i_k+1,j_k)x_{i_kj_k})=id$. This property can further be reduced to proving that for all $k$, $(i_k+1,j_k)x_{i_kj_k}$ in some sense does not permute the c-vectors. This in turn can be reduced to $(i_k+1,j_k)x_{i_kj_k}$ as an operation on extended exchange matrices maintain certain properties defined below:\\
\begin{definition}
An $n\times n$ matrix $M\in M_n(\mathbb{Z})$ is \textit{standard} if the following holds:\\
1. The diagonal entries are all nonzero.\\
2. All positive entries can only exist on the diagonal or above. and all negative entries can only exist on the diagonal or below.\\
3. All columns are in the form $\pm\beta_{ij}$.\\
\end{definition}
It is easy to see that all columns of the form $-\beta_{ij}$ has to be the $(i+1)$-th column and all columns of the form $\beta_{ij}$ has to be the $j$-th column since all other positions violate either Axiom 1 or 2. It is also trivial that the only results of a permutation of columns of $\pm I_n$ that are standard are $\pm I_n$ themselves.\\
\begin{example}
\indent Here are several examples:\\\\
$\begin{bmatrix}
1 & 1\\
0 & -1\\
\end{bmatrix}$ and
$\begin{bmatrix}
1 & 0\\
-1 & -1\\
\end{bmatrix}$ are standard matrices because all three axioms hold.\\\\
$\begin{bmatrix}
0 & -1\\
-1 & 0\\
\end{bmatrix}$ and
$\begin{bmatrix}
-1 & -1\\
1 & 0\\
\end{bmatrix}$ are not standard matrices since axioms 1 and 2 are violated.\\
\end{example}
%Now let's define regularity of an operation which is important for a technicality in the proof below:\\
%\begin{definition}
%An operation $(i_k+1,j_k)x_{i_kj_k}$ from a standard matrix to another standard matrix is \textit{regular} if it preserves the sign of all columns when $j_k>i_k+1$ and if it only changes the sign of the $j_k$-th column and preserve the sign of all other columns when $j_k=i_k+1$.\\
%\end{definition}
\indent The lemma to be proven that can almost immediately lead to the theorem is stated below:\\
\begin{lemma}
In $A_n$ straight orientation, $(i_k+1,j_k)x_{i_kj_k}$ or $(i_k+1,j_k)x_{i_kj_k}^{-1}$ transforms a standard matrix $\tilde{B}_{k-1}$ into a standard matrix $\tilde{B}_k$.\\
\end{lemma}
\begin{proof}
\indent We will only prove in the green case since the red case is almost identical to the green one. In this proof $i_k$ is simplified as $i$ and $j_k$ is simplified as $j$.\\
\indent Case 1: If $j - i  = 1$. Here we have a simple root and the associated permutation $(i+1,j)$ is identity. Hence the proof reduces to $x_{ij}$ transforms a standard matrix to another standard one. $x_{ij}$ merely flips the $j$-th column from $-e_j$ to $e_j$ and may lengthen some $-\beta_{li}$ to $-\beta_{lj}$ for $l<i$ and shorten some $\beta_{il}$ to $\beta_{jl}$ for $l>j$ without changing which column they are in, but no other operation happens or violations of First Forbidden Pairs-of-Wall Theorem ($-q+r$) will happen, hence the resulting matrix is still standard.\\
\indent Case 2: If $j - i > 1$. Here we have an extra generator and the associated permutation is not identity. Now let's discuss what $(i+1,j) x_{ij}$ actually does on each column:\\
\indent a) $l\leq i$. Due to $c$-vector theorem \cite{IOTW15} all c-vectors have to be $\pm\beta_{ab}$ for some $0\leq a<b\leq n$. So the only change that can ever happen is that $-\beta_{li}$ may be lengthen to $-\beta_{lj}$. Neither of these cause a $C$-matrix to violate standardness.\\
\indent b) $i + 1 < l < j$. Again due to $c$-vector theorem in \cite{IOTW15} the only plausible situation is $\beta_{il}$ was transformed into $-\beta_{lj}$. But this constitutes a $-q+r$ situation which violates the Forbidden Pairs-of-Wall Lemma.\\
\indent c) $l = j$. We notice several facts:\\
\indent (1).The $c$-vector $c_j$ can not be negative. If this is the case we have a compartment with two walls $D(\beta_{ij})$ and $D(\beta_{j-1,m})$. This can not happen since these two roots can not be both stable due to \cite{ST12} when $m>j$ or the $+r+s$ situation appears when $m=j$. Hence we can assume that the $j$-th column is $\beta_{mj}$ for some $m<j$.\\
\indent (2). It is true that $m$ can not be less than $i$. Otherwise we have a $+s-r$ situation which violates the First Forbidden Pairs-of-Wall Lemma.\\
\indent (3). Also it is impossible for $\beta_{mj}$ to remain itself after doing $x_{ij}$ since otherwise we have a $-s-r$ situation which violates the Second Forbidden Pairs-of-Wall Lemma.\\
\indent Hence the $j$-th column is positive, $m>i$ and $-\beta_{ij}$ actually add to the $j$-th column. So after $(i+1,j)x_{ij}$ is performed the $i$-th column is $-\beta_{im}$ and the $j$-th column is $\beta_{ij}$.\\
\indent d) $l>j$. It is true that $\beta_{il}$ can be shortened to $\beta_{jl}$. Other than that, the only plausible case is $l = j + 1$ and the $k$th c-vector is $-\beta_{lm}$ for some $m>l$. In this case we can nor let $-\beta_{ij}$ add to the $l$-th column since otherwise $-q+r$ will be created after the mutation which violates the Forbidden Pairs-of-Wall Lemma.\\
\indent Using similar methods we can see that $(i+1, j)x_{ij}$ transforms a standard matrix into another one.\\
\end{proof}
\indent The theorem can be proven below:\\
\begin{proof}
The identity matrix $I_n$ is standard. Since $(i_k+1,j_k)x_{i_kj_k}$ or $(i_k+1,j_k)x_{i_kj_k}^{-1}$ transforms a standard matrix into another standard for all $k$, the result of transforming a standard matrix, $I_n$ by $\prod_{k}(i_k+1,j_k)x_{i_kj_k}^{\delta_k}$ is standard. Since we get a permutation of columns of $\pm I_n$ at the end of this transformation and any permutation of columns of $\pm I_n$ that result in a standard matrix has to be the trivial permutation, $\rho(\prod_{k}(i_k+1,j_k)x_{i_kj_k}^{\delta_k})=id$, hence the formula is correct.\\
\end{proof}
\subsection{The formula of associated permutation for any mutation sequences}
\indent Due to the theorem we can extend the definition of associated permutations to any arbitrary mutation sequence in $A_n$ straight orientation which reduces to the existing definitions of the associated permutation of reddening and loop sequences due to the theorem above.
\begin{definition}
In $A_n$ straight orientation, the \textit{associated permutation of a mutation sequence} in correspondence to the picture group element $\prod_{k}x_{i_kj_k}^{\delta_k}$ acting on a $c$-matrix with associated permutation $\sigma$ is defined as $\rho(\prod_{k}x_{i_kj_k}^{\delta_k}) = \sigma(\prod_{k}(i_k+1,j_k))^{-1}\sigma^{-1}$. Here $\delta_k\in\{+,-\}$.\\
\end{definition}
\indent In particular any mutation at a vertex with $c$-vector $\pm\beta_{ij}$ has an associated permutation $\sigma(i+1,j)\sigma^{-1}$ with $\sigma$ the permutation of the $c$-matrix before the mutation. In the special case when $i+1=j$ which is when $\beta_{ij}$ is a simple root the associated permutation is trivial.\\ 
\chapter{Tame path algebras are green sequence-finite}\label{C2}
\section{Introduction}
 In Br\"ustle-Dupont-P\'erotin \cite{BDP13} and the paper by Br\"ustle, Hermes, Igusa and Todorov \cite{BHIT15} it is proven that there are finitely maximal green sequences when the quiver is of finite, tame type or the quiver is mutation equivalent to a quiver of finite or tame type. Furthermore in \cite{BHIT15} it is proven that any tame quiver has finitely many $k$-reddening sequences.\\
\indent When we restrict our attention to the case where the algebra is basic, connected and hereditary it is a path algebra of a quiver \cite{ASS06}. In this chapter when we say an $m$-maximal green sequence of an algebra we mean an $m$-maximal green sequence of its path algebra. Here is the main theorem we have proven.\\
\begin{theorem}
Any tame quiver has finitely many $m$-maximal green sequences.\label{T:C3T}
\end{theorem}
\indent To prove this theorem we only need to prove that only finitely many indecomposable objects can appear as summands of silting objects that can appear in $m$-maximal green sequences of tame quivers $Q$. Since all indecomposable objects of a basic tame path algebra have to be transjective or regular, only finitely many rigid regular objects between $\Lambda$ and $\Lambda[m]$ in $D_b(\Lambda)$, namely the modules on the nonhomogeneous tubes $\mathbb{Z}A_\infty/\langle\tau^k\rangle$ with no repeating composition factors and their shifts. Hence the problem is reduced to proving that only finitely many indecomposable transjective objects between $\Lambda$ and $\Lambda[m]$ can appear in $m$-maximal green sequences.\\
\indent To prove this theorem we need two lemmas.\\
\begin{lemma}\label{def:C3L1}
For a tame quiver $Q$ any silting object in $D^b(kQ)$ contains at most $n-2$ regular summands. In other words, at least 2 summands have to be transjective. 
\end{lemma}
\begin{lemma}\label{def:C3L2}
For a tame quiver $Q$ there is a uniform bound, depending only on $Q$ and $m$, on the transjective degree of any transjective summand in any silting object in any $m$-maximal green sequence $D^b(kQ)$.
\end{lemma}
\indent It is easy to see why Lemma \ref{def:C3L2} implies the theorem. Here the \textit{transjective degree} of an indecomposable transjective object $\tau^iP_j[k]$ is defined as $deg(\tau^iP_j[k])=i$. The \textit{maximal transjective degree} and \textit{minimal transjective degree} of a silting object are defined as the highest/lowest transjective degree of its indecomposable transjective summands respectively.\\
\indent In Section 2 we prove Lemma \ref{def:C3L1}. In Section 3 we prove Lemma \ref{def:C3L2}. In Section 4 we further generalize the theorem to arbitrary finite mutation sequences with finitely many forward/green or backward/red mutations.\\
\section{Proof of Lemma \ref{def:C3L1}}
\indent To prove Lemma \ref{def:C3L1} we need to understand regular components of Auslander-Reiten quivers of $D^b(kQ)$ for tame quivers. Regular components of Auslander-Reiten quivers of tame path algebras are all standard stable tubes with at most three tubes nonhomogeneous (see \cite{DR76} and Chapter X of \cite{SS06}). Note that no object on a homogeneous tube is rigid so no object there can appear in a silting object of $D^b(kQ)$. Hence we only need to discuss the nonhomogeneous tubes.\\
\indent It is easy to see that in an indecomposable object in a standard stable tube $\mathcal{T}$ of size $n$, $M$ and any of its shifts can not be in the same pre-silting object.\\ 
\begin{definition}
If $\{M_i\}_{i\in I}$ are a family of indecomposable objects of $D^b(kQ)$ and $\Pi_{i\in I}M_i[n_i]$ is not pre-silting for any $\{n_i\}_{i\in I}$ We say that $\{M_i\}_{i\in I}$ is \textit{silting-incompatible}. Otherwise we say that it is \textit{silting-compatible}.
\end{definition}
\indent From now on in this proof we identify $[n]$ with $\mathbb{Z}/n\mathbb{Z}$ and hence will no longer differentiate between $0$ and $n$ which we usually denote as $n$. It is also clear that it makes sense to define a cyclic order on $[n]$.\\
\begin{definition}
Let $a,b\in [n]$. The \textit{interval} $[a,b]$ is defined as the following:
$[a,b]:=\begin{cases}
\{x|a\leq x\leq b, x\in [n]\} & \text{ if } a\leq b\\
\{x|a\leq x \leq n, x\in [n] \text{ or }1\leq x \leq b, x\in [n]\} & \text{ if } a > b\\
\end{cases}$
\end{definition}
\begin{definition}
Let $a_0, a_1\cdots, a_{k-1}$ be elements of $[n]$. Define $a_k$ The proposition $P(a_0, a_1,\cdots, a_{k-1})$ is defined as the statement that all $a_i$ are distinct and that for any $i\in\{0,1,\cdots, k-1\}$ for any $l\neq i, i+1$ it is true that $a_l$ is not in the interval $[a_i, a_{i+1}]$.
\end{definition}
\indent For example $P(1,2,3)$ and $P(4,1,2)$ hold in $[4]$ while $P(1,2,1)$, $P(2,1,3)$ and $P(4,3,1)$ do not.\\
\begin{definition}
Let $M_i$ be the quasi-simples of the tube such that $\tau M_i=M_{i-1}$. a regular module in $D^b(kQ)$ \textit{regular sincere} if its composition series contain all quasi-simples.
\end{definition}
\indent No indecomposable regular sincere modules or their shifts can appear as summands in any silting object because they are not rigid. (See Corollary X.2.7 of \cite{SS06}).As for the remaining $n(n-1)$ indecomposable regular modules that are actually rigid we can unambiguously label them as $M_{ij}$ if the quasi-top and quasi-socle of the object are $M_j$ and $M_i$ respectively. Note that $M_i=M_{ii}$. It is clear that $\tau M_{ij}=M_{i-1,j-1}$ and $\tau^{-1} M_{ij}=M_{i+1,j+1}$.\\
\indent Now let's prove two easy lemmas on what can not appear in a pre-silting object in a regular component of the Auslander-Reiten quiver of $D^b(kQ)$.\\
\begin{lemma}
\begin{enumerate}\label{lem:C3LN}
\item If $M$ and $N$ are regular modules in a nonhomogeneous tube in the Auslander-Reiten quiver of $kQ$. If $Hom(M,N)\neq 0$ and $Ext^1(N,M)\neq 0$, then $M$ and $N$ are silting-incompatible.\\
\item If $M_1,\cdots, M_k$ are regular modules in a nonhomogeneous tube in the Auslander-Reiten quiver of $kQ$. If $Ext^1(M_i,M_{i+1})\neq 0$ for any $1\leq i<k$  and $Ext^1(M_k,M_1)\neq 0$, then $\{M_i\}$  is silting-incompatible.\\
\end{enumerate}
\end{lemma}
\begin{proof}
\indent For (1) since $Hom(M,N)\neq 0$ if $i>j$ we have $Ext^{i-j}(M[i],N[j])\neq 0$ . Since $Ext^1(N,M)\neq 0$ if $i\leq j$ it is true that $Ext^{j-i+1}(N[j],M[i])\neq 0$ . Hence $M[i]\oplus N[j]$ is not pre-silting for any arbitrary $i$ and $j$.\\
\indent For (2) for arbitrary $n_1,\cdots n_k$ use the argument above it is easy to see that if $\oplus_{i=1}^kM_i[n_i]$ is pre-silting, then $n_2>n_1$, $n_3>n_2$, $\cdots, n_1>n_k$ which is impossible. Hence $\{M_i\}$  is silting-incompatible.
\end{proof}
\begin{lemma}\label{def:C3L3}
Any pre-silting object in a standard stable tube of size $n$ contains at most $n-1$ summands.
\end{lemma}
\indent To prove this lemma we need the following lemma.
\begin{lemma}\label{lem:C3L4}
Any pre-silting object in a standard stable tube of size $n$ can not be regular sincere.
\end{lemma}
\begin{proof}
\indent Assume that a pre-silting object $T=\oplus_{i=1}^k T_i$ in a standard stable tube of size $n$ is regular sincere. Let $T_{l_1},\cdots, T_{l_m}$ be a minimal set of indecomposable summands of $T$ such that their direct sum $T'=\oplus_{i=1}^m T_{l_i}$ is regular sincere. Note that if $M_{ij}$ and $M_{kl}$ are both summands of $T$, $P(i,k,l,j)$ holds $M_{kl}$ and $M_{ij}$ can not both be summands of $T'$ due to minimality. If $m=1$ then $T'$ is a regular sincere indecomposable regular object which contradicts the fact that $T'$ is pre-silting. If $m>1$ without loss of generality assume that $T_{l_1}=M_{1p}$ for some $p\neq n$. Any indecomposable object with its quasi-socle $M_i, 1\leq i\leq p$ can not be a summand of $T'$ either due to silting incompatibility or minimality. Hence there has to be a summand of $T'$ with its quasi-socle $p+1$. Repeat this procedure it's easy to see that $T'=\oplus_{i=1}^m M_{(t_{i-1}+1)t_i}$ with $t_0=t_m=n$. In this case by Lemma \ref{lem:C3LN}(2) the object can not be pre-silting.
\end{proof}
\indent Now we can prove Lemma \ref{def:C3L3}.
\begin{proof}
\indent Since any pre-silting object in a standard stable tube of size $n$ can not be regular sincere, without loss of generality it is a pre-silting object in the exact subcategory of $\mathcal{T}$ closed under extensions such that $M_1,\cdots M_{n-1}$ are the only simple objects. It is easy to see using the condition that the tube is standard stable which is a result of Theorem \ref{Tame}(5) that the category $add(\{M_{ij}\}_{1\leq i < j\leq n-1})$ is isomorphic to the module category of $kA_{n-1}$ with straight orientation and as a result any pre-silting object with all indecomposable summands in it or its shifts has at most $n-1$ summands. 
\end{proof}
\indent Finally we can prove Lemma \ref{def:C3L1}.
\begin{proof}
\indent Due to Lemma \ref{def:C3L3} and \cite{DR76} there are at most $n-2$ regular components in $D^b(kQ)$ when $Q$ is a tame quiver. This is true for each type so this is true for all tame quivers.
\end{proof}
\section{Proof of Lemma \ref{def:C3L2}}
\indent To prove Lemma \ref{def:C3L2} we need to rephrase an argument in \cite{BDP13} using degrees.
\begin{lemma}
(\cite{BDP13}, Lemma 10.1) Let $H$ be a representation-infinite connected hereditary algebra. Then there exists $N\geq 0$ such that for any $k\geq N$, for any projective $H$-module $P$, the $H$-modules $\tau^{-k}P$ and $\tau^kP[1]$ are sincere.
\end{lemma}
\begin{lemma}
(\cite{BDP13}) Let $Q$ be a tame quiver and $M_1,M_2$ two transjective modules of $kQ$. If $\{M_1,M_2\}$ is silting-compatible, then $|deg(M_1)-deg(M_2)|\leq N$ 
\end{lemma}
\begin{proof}
If $k-l>N$ we need to prove that $\tau^kP_a$ and $\tau^lP_b$ are silting-incompatible. If $i\leq j$ $Ext^{j-i+1}(\tau^lP_b[j],\tau^kP_a[i])=Ext^1(\tau^lP_b,\tau^kP_a)=Hom(\tau^{k-1}P_a,\tau^l P_b)=Hom(P_a,\tau^{l-k+1}P_b)\neq 0$ since $\tau^{l-k+1}P_b$ is a sincere preprojective module. If $i>j$ $Ext^{i-j}(\tau^kP_a[i],\tau^lP_b[j])=Ext^1(\tau^kP_a[1],\tau^lP_b)=Hom(\tau^{l-1}P_a,\tau^k P_b[1])=Hom(P_a,\tau^{k-l+1}P_b[1])\neq 0$ since $\tau^{k-l+1}P_b[1]$ is a sincere preinjective module. Hence $\tau^kP_a$ and $\tau^lP_b$ are silting-incompatible. Exchange the objects if $k-l<-N$. Hence the lemma has been proven.
\end{proof}
\indent Now we can prove Lemma \ref{def:C3L2} following a modified version of the argument in \cite{BDP13}.
\begin{proof}
\indent We only need to prove that there is a lower bound of minimal transjective degrees of silting objects that can appear in $m$-maximal green sequences. Assume that $\tau_kP_i[j]$ is in a silting object in an $m$-maximal green sequence of $kQ$. Note that due to Lemma \ref{def:C3L1} there are at least 2 transjective components in any silting object in $D^b(kQ)$. Note that each mutation on a transjective object $T$ in $\mathcal{P}_i$ can result in a transjective object in $\mathcal{P}_{i+1}$, a transjective object in $\mathcal{P}_i$ with degree less than or equal to $deg(T)$ or a regular object in $\mathcal{R}_i$. Each mutation on a regular object $T'$ in $\mathcal{R}_i$ can result in an object of $\mathcal{R}_i$, an object of $\mathcal{P}_{i+1}$ or an object of $\mathcal{R}_{i+1}$. Let $L$ be the minimal transjective degree of a silting object. No green mutation within a component or green mutation from a regular component to another one can increase $L$. All other green mutations may increase $L$ by at most $N$. However there are only $n$ summands of a silting object, $m+1$ transjective components and $m$ regular components so the amount of mutations that can increase $L$ is finite. To reach $\Lambda[m]$ which is of degree 0 $L$ has to be at least $-2mnN$. As a result no indecomposable transjective object in any silting object in an $m$-maximal green sequence can have a degree less than $-2mnN$. Similarly silting objects in $m$-maximal green sequences can not have maximal transjective degree higher than $2mnN$ or it can not start from $\Lambda$.
\end{proof}
\section{Almost morphism finiteness}
\indent Using the same method we can prove a stronger result.
\begin{theorem}\label{C3T2}
If $Q$ is a Dynkin or tame quiver and $T_1$, $T_2$ are silting objects of $D^b(kQ)$ then there are finitely many $k$-red and finitely many $k$-green mutation sequences from $T_1$ to $T_2$ for any $k$.
\end{theorem}
\indent Note that we only need to prove that part of the statement about $k$-red sequences. To prove the theorem we first need to prove the following lemma which is a generalization of Lemma 4.4.2 in \cite{BHIT15}.
\begin{lemma}\label{C3L4}
\begin{enumerate}
\item Any $k$-red sequence from $T_1$ to $T_2$ can go through any silting object at most $r+1$ times.
\item Any $k$-green sequence from $T_1$ to $T_2$ can go through any silting object at most $r+1$ times.
\end{enumerate}
\end{lemma}
\begin{proof}
We only need to prove (1). It is clear from the definition of mutations that a green sequence can go through any silting object at most once. (See \cite{BY13} and \cite{KY12} for more details.) Let's define a \textit{green arm} of a mutation sequence as a maximal subsequence of the mutation sequence that is green. Similarly we can define what is a \textit{red arm}. Assume that an $k$-red sequence $\{T_i\}$ has $n_r$ red arms and $n_g$ green arms. $n_r\leq k$. $n_g\leq n_r+1$. Let $n_1$ be the number of red arms of length 1 and $n_2$ the number of red arms of length at least 2. It's clear that $n_r=n_1+n_2$ and $n_1+2n_2\leq k$. Note that any silting object on a red arm of length 1 is on a green arm. Hence $\{T_i\}$ can go through any silting object at most $n_g+n_2\leq n_1+2n_2+1\leq k+1$ times.
\end{proof}
\indent It is easy to see that the bounds established in the lemma are optimal. Now we can prove the theorem. Note that the lemma above implies that in the Euclidean case if we can prove that for any $k$ if there are finitely many rigid objects that ca$k$-redn appear as summands of silting objects in $k$-red sequences Theorem \ref{C3T2} will been proven. 
\begin{proof}
\indent As we said above we will only prove the part about $k$-red sequences. Assume that all indecomposable summands of $T_1$ and $T_2$ are between $\Lambda[i]$ and $\Lambda[j]$. Since there are only $k$ red mutations, all indecomposable summands that appear in $k$-red sequences from $T_1$ to $T_2$ have to be between $\Lambda[i-k]$ and $\Lambda[j+k]$.\\ 
\indent If $Q$ is Dynkin there are only finitely many indecomposable objects between $\Lambda[i-k]$ and $\Lambda[j+k]$ and hence only finitely many silting objects can exist on an $k$-red sequence. Due to Lemma \ref{C3L4} there are finitely many $k$-red sequences. \\
\indent From now on we assume that $Q$ is Euclidean. There are only finitely many regular rigid indecomposable objects between $\Lambda[i-k]$ and $\Lambda[j+k]$ so the problem has been reduced to proving that only finitely many transjective indecomposable components can appear in silting objects in $k$-red sequences.\\
\indent Let the minimal degree of $T_2$ be $L$. Note that a red mutation can increase the minimal degree of a silting object by at most $N$. Use an argument similar to that one used to prove Theorem \ref{C3T} we can prove that no indecomposable transjective object with degree less than $L-2nN(2k+j-i)-kN$ can appear in any $k$-red sequences from $T_1$ to $T_2$. Similarly let the maximal degree of $T_1$ be $U$. No indecomposable transjective object with degree less than $U+2nN(2k+j-i)+kN$ can appear in any $k$-red sequence from $T_1$ to $T_2$. Hence there are only finitely many indecomposable transjective objects can appear in any $k$-red sequence from $T_1$ to $T_2$ and the theorem is proven.
\end{proof}
\indent Note that the bounds of transjective degrees in the proofs of Theorem \ref{T:C3T} and Theorem \ref{C3T2} above are very crude. In the future we will try to find better bounds.\\
\indent Finally let's define a new term to characterize finite dimensional algebras that satisfy the conditions of Theorem \ref{C3T2}.\\
\begin{definition}
\begin{enumerate}
\item A finite dimensional algebra $\Lambda$ of finite global dimension such that it has finitely many $k$-red sequences from any silting object $T_1$ to any silting object $T_2$ for any $k$ is \textit{almost morphism finite}.
\item A finite dimensional algebra $\Lambda$ of finite global dimension such that it has finitely many green sequences from any silting object $T_1$ to any silting object $T_2$ for any $m$ is \textit{green sequence finite}.
\end{enumerate}
\end{definition}
\indent Hence we can rephrase Theorem \ref{C3T2} as the following:
\begin{theorem}
If $\Lambda$ is the path algebra of a quiver of finite or tame type, then $\Lambda$ is almost morphism finite.
\end{theorem}
\indent Note that the condition of an algebra being almost morphism finite is stronger than the condition that it is green sequence finite which is stronger than the condition that there are finitely many $m$-maximal green sequences for any $m$. An almost morphism finite algebra has finitely many $k$-red sequences for any $k$ hence it has finitely many green-to-red sequences with $k$ red mutations.\\
\chapter{Two alternative definitions of $m$-maximal green sequences}\label{C3}
\section{Introduction}
\indent In Chapter \ref{CB} we introduced a result by Igusa, namely Theorem \ref{thm:3} namely there are new alternative definitions of maximal green sequences. How the result can possibly be generalized to $m$-maximal green sequences in general is an interesting question that we have mostly solved.\\
\begin{theorem}\label{C3T}
%\indent (Theorem \ref{C3TB}) Let $\Lambda$ be a finite dimensional hereditary algebra. Let $(C^{\leq 0}, C^{\geq 0}), (C'^{\leq 0}, C'^{\geq 0})$ be two $t$-structures such that there exists at least one green sequence from $(C^{\leq 0}, C^{\geq 0})$ to $(C'^{\leq 0}, C'^{\geq 0})$. Let $\mathcal{T} = C^{\leq 0}\cap C'^{\geq 0}$.  Let $M_1,\cdots, M_n$ be a finite sequence in $\mathcal{T}$. The following are equivalent.
%\begin{theorem}\label{C3TB}
%\indent Let $\Lambda$ be a finite dimensional hereditary algebra. Let $(C^{\leq 0}, C^{\geq 0}), (C'^{\leq 0}, C'^{\geq 0})$ be two $t$-structures such that there exists at least one green sequence from $(C^{\leq 0}, C^{\geq 0})$ to $(C'^{\leq 0}, C'^{\geq 0})$. Let $\mathcal{T} = C^{\leq 0}\cap C'^{\geq 0}$.  Let $M_1,\cdots, M_n$ be a finite sequence in $\mathcal{T}$. The following are equivalent.
\indent (Theorem \ref{C3TB}) Let $\Lambda$ be a finite dimensional hereditary algebra. Let $\mathcal{T} = add(\cup_{i=0}^{m-1} (mod\,\Lambda)[i])$. Let $\{M_1,\cdots, M_n\}$ be a finite sequence of nonzero objects in $\mathcal{T}$. The following are equivalent:
\begin{enumerate}
\item The sequence is a maximal sequence of backward $Hom^{\leq 0}$-orthogonal Schurian objects $\{M_n\}$ in $\mathcal{T}$.
\item The sequence is a finite sequence in $\mathcal{T}$ that forms a finite HN system for $\mathcal{T}$.
\item The sequence is a sequence of simples from the simple-minded collection $\{S_1,\cdots, S_n\}$ to  $\{S_1[m],\cdots, S_n[m]\}$. that is, it is an $m$-maximal green sequence..
\end{enumerate}
\end{theorem}

%\begin{enumerate}
%\item The sequence is a maximal sequence of backward $Hom^{\leq 0}$-orthogonal Schurian objects $\{M_n\}$ on $\mathcal{T}$.
%\item The sequence is a finite $c$-green sequence on $\mathcal{T}$.
%\item The sequence is a green sequence from a simple-minded collection $\{X_i\}$ to another one $\{Y_i\}$.
%\end{enumerate}
%\end{theorem}
\indent In Section 2 we will discuss an alternative definition of the stability condition on module categoris. In Section 3 we will discuss maximal backward-$Hom^{\leq 0}$ orthogonal sequences. In Section 4 we will discuss Harder-Narasimhan filtrations. In Section 5 we will establish the fact that the two alternative definitions are equivalent to the original ones.\\
\section{Alternative definition of the stability condition on module categories}
\indent This section is not used in the rest of the chapter. However it does provide new definitions of stability and semistability that are previously underdiscussed. Moreover the ideas in this section are related to $Hom^{\leq 0}$-backward orthogonal sequences.\\
\indent In general in a triangulated category the concepts of monomorphisms and epimorphisms are less important because there are almost no nontrivial ones. Instead the concept of homotopy kernels and homotopy cokernels are much more important.\\
\indent Before we can generalize the idea of a maximal green sequence we first need to generalize the idea of a stability condition without always relying on monomorphisms and epimorphisms.\\
\begin{theorem}
If $\Lambda$ is a finite dimensional hereditary algebra for an indecomposable module $M$ in a module category $mod \Lambda$ for a stability condition $\phi$ the following are equivalent:
\begin{enumerate}
\item $M$ is stable. That is, for any proper submodule $N$ of $M$ it is true that $\phi(N)<\phi(M)$.
\item For any proper quotient module $N$ of $M$ it is true that $\phi(M)<\phi(N)$.
\item For any indecomposable stable module $N\not\cong M$ such that $(M,N)\neq 0$ it is true that $\phi(M)<\phi(N)$.
\item For any indecomposable stable module $N\not\cong M$ such that $(N,M)\neq 0$ it is true that $\phi(N)<\phi(M)$.
\end{enumerate}
\end{theorem}
\begin{proof}
\indent (1)$\to$(2) If $N$ is a quotient module of $M$, we have the short exact sequence $0\to\sum_{i=1}^kR_i\to M\to N\to 0$. Since $M$ is stable $\phi(\sum_{i=1}^kR_i)<\phi(M)$ hence $\phi(N)>\phi(M)$.\\
\indent (2)$\to$(1) This proof is analogous to the proof of (1)$\to$(2).\\
\indent (1),(2)$\to$(3) If $M$ is a submodule of $N$ or $N$ is a quotient module of $M$ then due to (1) and (2) the statement is trivially true. Assume that there exists neither monomorphisms nor epimorphisms from $M$ to $N$. Assume that there exists $0\neq f\in(M,N)$ it is easy to see that $Im f\in mod\Lambda$. Take one of its indecomposable summand, $L$. It is easy to see that $L$ is a proper submodule of $N$ and a proper quotient module of $M$ at the same time. Since $M,N$ are stable we have $\phi(M)<\phi(L)<\phi(N)$.\\
\indent (1),(2)$\to$(4) This proof is analogous to the proof of (1),(2)$\to$(3).\\
\indent (4)$\to$(1) Use induction. If $M$ is simple it is of course stable. Hence (4)$\to$(1) is trivially true in the case of simples. Otherwise assume that (4)$\to$(1) is already true for all indecomposable modules with dimension less than $dim(M)$. If $M$ satisfies condition (4) then for any of its stable submodule $N$ we already have $\phi(N)<\phi(M)$ so we only need to focus on the non-stable ones. Assume that $L$ is one of its minimal non-stable indecomposable proper submodules such that $\phi(L)\geq\phi(M)$. By induction since $\lnot$(1) $\to$ $\lnot$(4) holds for $L$ there has to be an indecomposable stable module $N\not\cong L$ such that $(N,L)\neq 0$ and $\phi(N)\geq \phi(L)$. Hence $\phi(N)\geq\phi(L)\geq\phi(M)$ and $(N,M)\neq 0$. Hence $M$ does not satisfy condition $(4)$ and we have reached a contradiction. As a result (4)$\to$(1) is proven.\\
\indent (3)$\to$(2) This proof is analogous to the proof of (4)$\to$(1). Use induction. If $M$ is simple (2) of course holds. Hence (3)$\to$(2) is trivially true in the case of simples. Otherwise assume that (3)$\to$(2) is already true for all indecomposable modules with dimension less than $dim(M)$. If $M$ satisfies condition (3) then for any of its stable quotient submodule $N$ we have $\phi(M)<\phi(N)$ so we only need to focus on the non-stable ones. Assume that $L$ is one of its minimal non-stable indecomposable proper quotient modules such that $\phi(M)\geq\phi(L)$. By induction since $\lnot$(2) $\to$ $\lnot$(3) holds for $L$ there has to be an indecomposable stable module $N\not\cong L$ such that $(L,N)\neq 0$ and $\phi(L)\geq \phi(N)$. Hence $\phi(M)\geq\phi(L)\geq\phi(N)$ and $(M,N)\neq 0$. Hence $M$ does not satisfy condition $(3)$ and we have reached a contradiction. As a result (3)$\to$(2) is proven.\\
\end{proof}
\indent We can obtain a similar result in the case of semistability.\\
\begin{theorem}
If $\Lambda$ is a finite dimensional algebra for an indecomposable module $M$ in a module category $mod \Lambda$ for a stability condition $\phi$ the following are equivalent:
\begin{enumerate}
\item $M$ is semistable.
\item For any proper quotient module $N$ of $M$ it is true that $\phi(M)\leq\phi(N)$.
\item For any indecomposable stable module $N$ such that $(M,N)\neq 0$ it is true that $\phi(M)\leq\phi(N)$.
\item For any indecomposable stable module $N$ such that $(N,M)\neq 0$ it is true that $\phi(N)\leq\phi(M)$.
\end{enumerate}
\end{theorem}
\section{Maximal backward $Hom^{\leq 0}$ orthogonal sequences}
\indent In order to discuss maximal backward $Hom^{\leq 0}$ orthogonal sequences we first need to define them.\\
\begin{definition}
$M_1,M_2,\cdots, M_k$ is a \textit{backward $Hom^{\leq 0}$ orthogonal sequence} of Schur objects if $(M_i[\geq 0], M_j) = 0$ for all $i>j$ and all $M_i$s are non-zero.
\end{definition}
\begin{definition}
$M_1,M_2,\cdots, M_k\in \mathcal{T}$ is a \textit{maximal backward $Hom^{\leq 0}$ orthogonal sequence} of Schur objects on $\mathcal{T}$ if $(M_i[\geq 0], M_j) = 0$ for all $i>j$, all $M_i$ are Schur and that for any other Schur object $M'\in \mathcal{T}$ if it is inserted anywhere in the sequence it will no longer be backward $Hom^{\leq 0}$ orthogonal.
\end{definition}
\indent Now let's prove a crucial lemma.\\
%\indent Our goal is to prove that if $M_1,M_2,\cdots, M_k$ is a maximal $Hom^{\leq 0}$-backward orthogonal sequence of Schur objects we can have $E_0(M_1,\cdots, M_k)=\mathcal{T}$.
\begin{lemma}\label{lem:C3L1}
If $H_1, H_2$ are two hearts of $t$-structures $(C^{\leq 0}, C^{\geq 0})$ and $(C'^{\leq 0}, C'^{\geq 0})$ respectively, there exists a maximal backward $Hom^{\leq 0}$-orthogonal sequence from $H_1$ to $H_2$ then the first term of the sequence has to be a simple of $H_1$ that is not in $H_2$.
\end{lemma}
\begin{proof}
\indent Let's first assume that $M\in H_1[l]$ with $l\geq 0$ is the first term of the maximal backward $Hom^{\leq 0}$-orthogonal sequence. Let the truncation functors of $(C^{\leq 0}, C^{\geq 0})$ be $\tau_{\geq n}$ and $\tau_{\leq n}$ respectively. There exists some simple $S\in H_1$ such that $S[l]$ is a subobject of $M$ in $H_1[l]$ which is Abelian. If there exists no non-initial term $N$ in the sequence such that $(N,S) \neq 0$ (note that it is impossible to have $(N[i],S)\neq 0$ for positive $i$ due to $N=\tau_{\leq 0}N$) then the sequence is not maximal because $S$ can be inserted before $M$. Hence we assume that such an $N$ exists, In this case $(N[l],M)= 0$ or the sequence would have no longer been backward $Hom^{\leq 0}$-orthogonal. Let $N' = \tau_{\geq 0}N$ and $N'' = \tau_{<0} N$ . So we have the canonical triangle $N''\to N\to N'\to N''[1]$. Due to $(N[l],M) = 0$ it is obvious that $(N'[l],M) = 0$. Note that $M, N'[l], S[l]\in H_1[l]$. $(N'[l],S[l]) = 0$ since $S[l]$ is a subobject of $M$ in an Abelian category. As a result $(N',S) = 0$ and hence $(N,S) = 0$ which contradicts the assumption that $(N,S)\neq 0$.\\
\indent Now let's assume that $M\in \tau_{\geq -l}$ but not $\tau_{\geq -l+1}$. Let $M' = \tau_{\leq -l}M$. Let $S[l]$ be a subobject of $M'$ in $H_1[l]$ which is Abelian. Note that $(S[l],M) = (S[l],M')\neq 0$ because $(S[l], M/M'[-1]) = 0$ since $M/M'[-1]\in \tau_{\geq -l+1}$. If there exists no non-initial term $N$ in the sequence such that $(N,S) \neq 0$ (note that again it is impossible to have $(N[i],S)\neq 0$ for positive $i$ due to $N=\tau_{\leq 0}N$) then the sequence is not maximal because $S$ can be inserted before $M$. Hence we assume that such an $N$ exists, In this case $(N[l],M)= 0$ or the sequence would have no longer been backward $Hom_{\leq 0}$-orthogonal. Let $N' = \tau_{\geq 0}N$ and $N'' = \tau_{<0} N$ . So we have the canonical triangle $N''\to N\to N'\to N''[1]$. Due to $(N[l],M) = 0$ it is obvious that $(N'[l],M) = 0$ and $(N'[l],M')=0$. Note that $M', N'[l], S[l]\in H_1[l]$. $(N'[l],S[l]) = 0$ since $S[l]$ is a subobject of $M'$ in an Abelian category. As a result $(N',S) = 0$ and hence $(N,S) = 0$ which contradicts the assumption that $(N,S)\neq 0$.\\
\end{proof}
%%%%%%%%%%%%%%%%%%%%%%%%%%%%%%%%%%%%%%%%%%%%%%%
\section{Harder-Narasimhan filtration}
\indent Now we need to introduce Harder-Narasimhan (HN) filtrations.\\
\begin{definition}
Let $\catt$ be a subcategory of a triangulated category. $M_1,\cdots, M_k\in \catt$ are a finite sequence of nonzero objects.  An \textit{HN filtration of object $X\in\catt$ with respect to $\{M_i\}$} aka an HN filtration of an object $X$ is the following diagram:\\
$\begin{tikzcd}
0 = X_k\arrow[rr] &                               & X_{k-1}\arrow[ld]\arrow[rr]&                & X_{k-2}\arrow[ld]& \cdots & X_0 = X\arrow[ld]\arrow[l,leftarrow]\\
                            & X_{k-1}/X_k\arrow[lu, dashrightarrow] &        & X_{k-2}/X_{k-1}\arrow[lu, dashrightarrow] & \cdots & X_0/X_1&
\end{tikzcd}$
where $X_{i-1}/X_i$ is a self-extension of $M_i$. 
\end{definition}
\indent If an object has an HN filtration with respect to $\{M_i\}_{i\in[N]}$ then it makes sense to define its \textit{lowest and highest indices} with respect to the filtration.\\
\begin{definition}
Let $\catt$ be a subcategory of a triangulated category. $M_1,\cdots, M_k\in \catt$ are a finite sequence of nonzero objects. If some object $X\in\catt$ has an HN filtration $\begin{tikzcd}
0 = X_k\arrow[rr] &                               & X_{k-1}\arrow[ld]\arrow[rr]&                & X_{k-2}\arrow[ld]& \cdots & X_0 = X\arrow[ld]\arrow[l,leftarrow]\\
                            & X_{k-1}/X_k\arrow[lu, dashrightarrow] &        & X_{k-2}/X_{k-1}\arrow[lu, dashrightarrow] & \cdots & X_0/X_1&
\end{tikzcd}$ with respect to $M_1,\cdots, M_k$ then we can define the following:
\begin{enumerate}
\item The \textit{lowest index} in the filtration is defined as the smallest $i$ such that $X_{i-1}/X_i\neq 0$.
\item The \textit{highest index} in the filtration is defined as the largest $i$ such that $X_{i-1}/X_i\neq 0$.
\end{enumerate}
\end{definition}
\indent If $X$ has a unique HN filtration with respect to $M_1,\cdots, M_k$ then the lowest and highest indices of $X$ are only dependent on $M_1,\cdots, M_k$. In this case we can refer to them as $l_X$ and $h_X$ without any ambiguity.\\
\indent We may sometimes abuse notations and refer to the unique HN filtration of some $X$ as $0\to X_{h_X-1}\to\cdots\to X_{l_X} = X$.\\
\begin{definition}
Let $\catt$ be a subcategory of a triangulated category. $M_1,\cdots, M_k\in \catt$ are a finite sequence of nonzero objects. $M_1,\cdots, M_k$ form a \textit{finite HN system for $\catt$} if any object $X\in\catt$ has a unique HN filtration with respect to $M_1,\cdots, M_k$.
\end{definition}
%\indent The fact that a category accepts a unique Harder-Narasimhan (HN) filtration is a very strong condition. In this case we can define the \textit{degree} of any object in $D^b(\Lambda)$ as the composition length of the object when decomposed using the HN filtration.\\
\begin{lemma}\label{lem:C3L3}
Let $\Lambda$ be a finite dimensional hereditary algebra. Let $\mathcal{T} = add(\cup_{i=0}^{m-1} (mod\,\Lambda)[i])$. Let $M_1,\cdots, M_n$ be a finite sequence of nonzero objects in $\mathcal{T}$. If any object $Y$ in $\catt$ accept a unique HN filtration $0\to Y_m\to\cdots\to Y_1=Y$ with $Y_i/Y_{i+1}\in \mathcal{E}(M_i)$, the following holds.
\begin{enumerate}
%\item For any $i$ $M_i\in mod\, \Lambda[k]$ for a certain $k$.
\item For any $i$ it is true that $M_i$ is indecomposable.
\item For any $i$ it is true that $M_i$ is Schur.
\end{enumerate}
\end{lemma}
\begin{proof}
%\indent For (1) assume that $M_i = \oplus_{k}M_i^k$ such that $M_i^k\in\, mod\, \Lambda[k]$. Define the generalized dimension vector of $M_i$ as $dim\, M_i\, =\, \Sigma_k (-1)^k\,dim\,M_i^k$. Any object $X$ that only have $i$-th entry in its HN filtration has to satisfy $dim\,X = k\,dim\,M_i$ for some nonnegative $k$.  If not all entries of $dim\, M_i$ are non-negative then $M_i^k = 0$ for all odd $k$ for otherwise the unique HN filtration of $M_i^k$ can not only have the $i$-th entry which violates the condition. Similarly if not all entries of $dim\, M_i$ are non-positive then $M_i^k = 0$ for all even $k$. If $dim\, M_i = 0$ then all $M_i^k$ has to be 0 because otherwise $dim\,M_i^k$ could never be a multiple of $dim\,M_i$. Hence $dim\, M_i$ has to be either a positive vector or a negative vector. Without loss of generality let's assume that it is a positive one. In this case $M_i^k = 0$ for all odd $k$. If $M_i^k \neq 0$ and $M_i^l\neq 0$ then $dim\,M_i^k\,<\,dim\,M_i$. This can not happen either.\\
\indent For (1) assume that $M_i$ is decomposable. Then $M_i = A\oplus B$ with $A\neq 0$ and $B\neq 0$. In this case $A$ and $B$ have nontrivial HN filtrations and adding them up we should obtain an HN filtration for $M_i$ that isn't the canonical one which contradicts the fact that $M_i$ has a unique HN filtration.\\
\indent For (2) assume that $M_i$ is not Schur. Without loss of generality we can assume that $M_i\in mod\Lambda$. Then there exists $f\in End(M_i)$ such that $Im f\neq M_i$. Take an indecomposable direct summand of $Im f$, $Q$. Since $Q$ is a quotient module of $M_i$ and $\Lambda$ is hereditary $Ext^1(M_i/Q, M_i)\to Ext^1(M_i/Q, Q) \neq 0$ is a surjection. We have the following diagram.\\
$\begin{tikzcd}
0\arrow[r] & M_i\arrow[r]\arrow[d] & X\arrow[r]\arrow[d] & M_i/Q \arrow[r]\arrow[d,equal] & 0\\
0\arrow[r] & Q \arrow[r] & M_i\arrow[r] & M_i/Q \arrow[r] & 0\\
\end{tikzcd}$\\
\indent Due to $0\to M_i\to Q\oplus X\to M_i\to 0$ being a short exact sequence with is a consequence of a pushout diagram on the left and $M_i\rightarrowtail X$ being monomorphic, $Q\oplus X$ has two HN-filtrations, one containing the $i-$th entry only while the other definitely contain what is not in the $i$-th entry because $Q$ can not only have the $i$-th entry.\\
\end{proof}
%\begin{lemma}\label{lem:C3L3}
%Let $(C^{\leq 0}, C^{\geq 0}), (C'^{\leq 0}, C'^{\geq 0})$ be two $t$-structures such that there exists at least one green sequence from $(C^{\leq 0}, C^{\geq 0})$ to $(C'^{\leq 0}, C'^{\geq 0})$. Let $\mathcal{T} = C^{\leq 0}\cap C'^{\geq 0}$.  If any object $Y$ in $\catt$ accept a unique HN filtration $0\to Y_m\to\cdots\to Y_1=Y$ with $Y_i/Y_{i+1}\in \mathcal{E}(M_i)$, the following holds.
%\begin{enumerate}
%\item For any $i$ $M_i\in mod\, \Lambda[k]$ for a certain $k$.
%\item For any $i$ $M_i$ is indecomposable.
%\item For any $i$ $M_i$ is Schur.
%\end{enumerate}
%\end{lemma}
%\indent Now we need to restrict the situation to $\mathcal{T} = C^{\leq 0}\cap C'^{\geq 0}$ for $t$-structures $(C^{\leq 0}, C^{\geq 0})$ and $(C'^{\leq 0}, C'^{\geq 0})$.
%\begin{proof}
%\indent (1), (2) and (3) are all trivially true because $\catt$ is closed under direct summands and extensions.
%\end{proof}
\begin{lemma}\label{lem:C3L2}
Let $\Lambda$ be a finite dimensional hereditary algebra. Let $\mathcal{T} = add(\cup_{i=0}^{m-1} (mod\,\Lambda)[i])$. Let $M_1,\cdots, M_n$ be a finite sequence of nonzero objects in $\mathcal{T}$. If any object $Y$ in $D^b(\Lambda)$ accept a unique HN filtration $0\to Y_N\to\cdots\to Y_1=Y$ with $Y_i/Y_{i+1}\in \mathcal{E}(M_i)$, the following holds.
\begin{enumerate}
\item For any $i>j$ it is true that $Hom(M_i,M_j) = 0$.
\item If $M_j = M_i[1]$ then $j>i$. Moreover if $Y$ is any object, the lowest nonzero entry of the HN filtration of $Y$ has index $l_Y$, can not be higher than the highest nonzero entry of the HN filtration of $Y[1]$ then $l_Y\leq h_{Y[1]}$.
\item If $Y_i\neq 0$ for some $i\in[N]$ then $Hom(Y_i, Y)\neq 0$ .
\item If $Y/Y_i\neq 0$ for some $1<i\leq N$ then $Hom(Y, Y/Y_i)\neq 0$ .
\item For any $i>j$ it is true that $Hom(M_i[1],M_j) = 0$.
\item If $M_j = M_i[m]$ where $m>0$ then $j>i$. Moreover if $Y$ is any object and the lowest nonzero entry of the HN filtration of $Y$ has index can not be higher than the highest nonzero entry of the HN filtration of $Y[m]$ for any positive $m$. 
\item $Hom(M_i[m],M_j) = 0$ where $m>0$ for any $i>j$.
\end{enumerate}
\end{lemma}
\begin{proof}
\indent (1) is true because otherwise we have completely different HN filtrations for $M_i\oplus M_j$, namely $\triangwm{M_i}{(1,0)^t}{M_i\oplus M_j}{(0,1)}{M_j}{0}{M_i[1]}$ and $\triangwm{M_i}{(1,f)^t}{M_i\oplus M_j}{(f,-1)}{M_j}{0}{M_i[1]}$ where $f$ is a nontrivial morphism from $M_i$ to $M_j$.\\
\indent (2) is true because otherwise $M_i\to 0\to M_i[1]\to M_i[1]$ will be an HN filtration of 0 and hence there will be at least two HN filtrations of 0. Similarly if the HN filtration of $Y$ is strictly before the HN filtration of $Y[1]$ $0\to Y\to Y[1]\to 0$ will be an HN filtration of 0.\\
\indent Let $l \geq i$ be any number such that $Y_l/Y_{l+1}\neq 0$. Since $Y_i\neq 0$ such $l$ must exist. (3) is true because $\triang{Y_i}{Y}{Y/Y_i}{Y_i[1]}$ is a triangle. If $Hom(Y_i,Y)= 0$ the triangle splits and $Y/Y_i = Y\oplus Y_i[1]$. Since $Y_i[1]$ has a unique HN filtration $Y/Y_i$ has two HN filtrations, one with the $l$-th entry 0 and one with a nontrivial $l$-th entry. \\
\indent Let $l < i$ be any number such that $Y_l/Y_{l+1}\neq 0$. Since $Y/Y_i\neq 0$ such $l$ must exist. (4) is true because $\triang{Y_i}{Y}{Y/Y_i}{Y_i[1]}$ is a triangle. If $Hom(Y,Y/Y_i)= 0$ the triangle splits and $Y_i = Y\oplus Y_i[-1]$. Since $Y_i[-1]$ has a unique HN filtration $Y_i$ has two HN filtrations, one with the $l$-th entry 0 and one with a nontrivial $l$-th entry. \\
\indent Now let's prove (5). Assume that $i>j$ and $Hom(M_i[1],M_j) \neq 0$. Since $Hom(M_i,M_j[-1])\neq 0$ and that $M_j\in\catt$ it is clear that $M_j[-1]\in\catt$ and hence has a unique HN filtration. Let's first assume that the highest term of the HN filtration of $M_j[-1]$ is a self-extension of $M_i$. If this is not the case assume that the highest term is a self-extension of $M_{i'}$. If $i'\leq j$ then it is clear that $Hom(M_i,M_j[-1])=0$ since the Hom from $M_i$ to all terms in the HN filtration of $M_j[-1]$ is 0 due to (1). Hence $i'>j$ and we can simply use $i'$ instead of $i$ since $Hom(M_{i'},M_j[-1])\neq 0$. So we can indeed assume that the highest entry of the HN filtration of $M_j[-1]$ is a self-extension of $M_i$. Let such an entry be $X_i$. Since $Hom(M_i[1], M_j)\neq 0$ and $M_i\in\catt$ we can see that $M_i[1]\in\catt$. As a result $X_i[1]\in\catt$. Let $h$ be the highest nontrivial index of the HN filtration of $X_i[1]$ then $h$ is not higher than $j$ or the highest nontrivial index of $M_j[-1]/X_i$ both of which are lower than $i$ due to (4) since $M_j[-1]\to M_j[-1]/X_i\to X_i[1]\to M_j$ is a triangle. Apply (4) again to $X_i[1]$ as a self-extension of $M_i[1]$ the highest nontrivial index of the HN filtration of $M_i[1]$ is lower than $i$. Apply (2) to $M_i$ and we can reach a contradiction.\\
\indent As for (6), since (2) is already proven let's assume that the result has been proven for all positive integers below $m$ and use induction. For the first claim it is clear that $M_i[1]$ can not be any $M_l$ or the induction hypothesis would be violated. It is also clear that $M_i[1]\in\catt$ since $M_i, M_i[m]\in\catt$. Take the lowest nonzero entry $Y_k$ of the HN filtration of $M_i[1]$. If $k=i$ then $Hom(M_i[1],M_i)\neq 0$ which can not happen. If $k>i$ then we can apply the induction hypothesis to the HN filtration of $M_i[1]$ and $M_i[m]$ and show that this is false. Hence $k<i$. In this case $Hom(M_i[1],Y_k)\neq 0$. Hence $Hom(M_i[1],M_k)\neq 0$ which is impossible due to (5). Now let's prove the second claim. Here since $Y,Y[m]\in\catt$ so does $Y[1]$. Let the lowest entry of the HN filtration of $Y[1]$ be $N_k\in\mathcal{E}(M_k)$ and let $0\to Y_j\to\cdots Y_l = Y$ be the HN filtration of $Y$. It is easy to see that $k<l$ due to the induction hypothesis applied to $Y[1]$ and $Y[k]$. Hence $Hom(Y[1],M_k)\neq 0$. Hence for some $h>k$ we have $Hom(M_h[1],M_k)\neq 0$ which is impossible due to (5).\\
\indent Finally we need to prove (7). Assume that the result is true for any positive integer below $m$ which is legit because (5) is already proven. Assume that $i>j$ and $Hom(M_i[m],M_j) \neq 0$. Since $Hom(M_i, M_j[-m])\neq 0$ and $M_j\in\catt$ it is clear that $M_j[-m]\in\catt$. Let's first assume that the highest entry of the HN filtration of $M_j[-m]$ is a self-extension of $M_i$. If this is not the case assume that the highest term is a self-extension of $M_{i'}$. If $i'\leq j$ then it is clear that $Hom(M_i,M_j[-m])=0$ since the Hom from $M_i$ to all terms in the HN filtration of $M_j[-m]$ is 0 due to (1). Hence $i'>j$ and we can simply use $i'$ instead of $i$ since $Hom(M_{i'},M_j[-m])\neq 0$. So we can indeed assume that the highest entry of the HN filtration of $M_j[-m]$ is a self-extension of $M_i$ and let such an entry be $X_i$. Since $Hom(M_i[1], M_j[1-m])\neq 0$ and $M_i\in\catt$ we can see that $M_i[1]\in\catt$. As a result $X_i[1]\in\catt$. Let $h$ be the highest nontrivial index of the HN filtration of $X_i[1]$ then $h$ is no higher than the highest nontrivial index of $M_j[-m]/X_i$ or $Hom(M_h,M_j[1-m])\neq 0$ in which case $h\leq j$ by induction since $M_j[-m]\to M_j[-m]/X_i\to X_i[1]\to M_j[1-m]$ is a triangle. Hence in both cases $h<i$. Apply (4) again to $X_i[1]$ as a self-extension of $M_i[1]$ the highest nontrivial index of the HN filtration of $M_i[1]$ is lower than $i$. Apply (2) to $M_i$ and we can reach a contradiction.\\
\end{proof}
%\indent Here Lemma 4.4.2 only applies if the unique HN filtration exists for all objects $Y\in D^b(\Lambda)$. When restricted to the case of actual green sequences most of the arguments remain  the same. However truncation functors need to be used in some cases.\\
%\begin{lemma}\label{lem:C3L4}
%Let $(C^{\leq 0}, C^{\geq 0}), (C'^{\leq 0}, C'^{\geq 0})$ be two $t$-structures such that there exists at least one green sequence from $(C^{\leq 0}, C^{\geq 0})$ to $(C'^{\leq 0}, C'^{\geq 0})$. Let $\mathcal{T} = C^{\leq 0}\cap C'^{\geq 0}$. Let $M_1,\cdots, M_N$ be nonzero objects of $\catt$. If any object $Y$ in $\catt$ accept a unique HN filtration $0\to Y_N\to\cdots\to Y_1=Y$ with $Y_i/Y_{i+1}\in \mathcal{E}(M_i)$, the following holds.
%\begin{enumerate}
%\item $Hom(M_i,M_j) = 0$ for any $i>j$.
%\item If $M_j = M_i[1]$ then $j>i$. Moreover if $Y$ is any object such that $Y[1]\in\catt$ and the lowest nonzero entry of the HN filtration of $Y$ has index can not be higher than the highest nonzero entry of the HN filtration of $Y[1]$. 
%\item If $Y_i\neq 0$ $Hom(Y_i, Y)\neq 0$ for any $i$.
%\item If $Y/Y_i\neq 0$ $Hom(Y, Y/Y_i)\neq 0$ for any $i>1$.
%\item If $Hom(M_i[1],M_j) = 0$ for any $i>j$.
%\item If $M_j = M_i[m]$ where $m>0$ then $j>i$. Moreover if $Y$ is any object and the lowest nonzero entry of the HN filtration of $Y$ has index can not be higher than the highest nonzero entry of the HN filtration of $Y[m]$ for any positive $m$. 
%\item $Hom(M_i[m],M_j) = 0$ where $m>0$ for any $i>j$.
%\end{enumerate}
%\end{lemma}
%\begin{proof}
%\indent (1) is true for the same reason why Lemma \ref{lem:C3L2} (1) is true. In particular $M_i\oplus M_j$ is in $\catt$ because $C'^{\leq 0}$ and $C^{\geq 0}$ are both closed under extensions.\\
%\indent The proof of(2)-(4) are the same as the proof of Lemma \ref{lem:C3L2} (2)-(4).\\
%\indent For (5) we must consider situations where numerous objects we mentioned are not actually in $\catt$. Assume that $i>j$ and $Hom(M_i[1],M_j) \neq 0$. Let $\tilde{M_i}[1]:=\tau'^{\geq 0}(M_i[1])$ and $\bar{M_i}[1]:=\tau'^{< 0}(M_i[1])$. $\bar{M_i}[1] \to M_i[1]\to \tilde{M_i}[1]\to \bar{M_i}[2]$. Since $Hom(\bar{M_i}[1], M_j) = Hom(\bar{M_i}[2],M_j) = 0$ $Hom(\tilde{M_i}[1],M_j) \neq 0$. Let $\tilde{M_j}[-1]:=\tau^{\leq 0}(M_j[-1])$ and $\bar{M_j}[-1]:=\tau^{> 0}(M_j[-1])$. $\tilde{M_j}[-1] \to M_j[-1]\to \bar{M_j}[-1]\to \tilde{M_j}$. Since $Hom(\tilde{M_i},\bar{M_j}[-1]) = Hom(\tilde{M_i},\bar{M_j}[-2]) = 0$ we have $Hom(\tilde{M_i},\tilde{M_j}[-1])\neq 0$.\\
%\indent Let's first assume that the highest term of the HN filtration of $\tilde{M_j}[-1]$ is a self-extension of $M_i$. If this is not the case assume that the highest term is a self-extension of $M_{i'}$. If $i'\leq j$ then it is clear that $Hom(M_i,\tilde{M_j}[-1])=0$ since the Hom from $M_i$ to all terms in the HN filtration of $\tilde{M_j}[-1]$ is 0 due to (1). Hence $i'>j$. Let $\tilde{M_i'}[1]:=\tau'^{\geq 0}(M_i
%[1])$ and $\bar{M_i'}[1]:=\tau'^{< 0}(M_i'[1])$. $\bar{M_{i'}}[1] \to M_{i'}[1]\to \tilde{M_{i'}}[1]\to \bar{M_{i'}}[2]$. Since $Hom(\bar{M_{i'}}[1], M_j) = Hom(\bar{M_{i'}}[2],M_j) = 0$ $Hom(M_{i'}[1],M_j) = Hom(\tilde{M_{i'}}[1],M_j) \neq 0$.  We can simply use $i'$ instead of $i$ since $Hom(M_{i'},\tilde{M_j}[-1])\neq 0$. \\
%\indent So we can indeed assume that the highest entry of the HN filtration of $\tilde{M_j}[-1]$ is a self-extension of $M_i$. Let such an entry be $X_i$. Let $\tilde{X_i}[1]:=\tau'^{\geq 0}(X_i[1])$ and $\bar{X_i}[1]:=\tau'^{< 0}(X_i[1])$. Let $h$ be the highest nontrivial index of the HN filtration of $\tilde{X_i}[1]$ then $h$ is not higher than $j$ or the highest nontrivial index of $\tilde{M_j}[-1]/X_i$ both of which are lower than $i$ due to (4) since $\tilde{M_j}[-1]\to \tilde{M_j}[-1]/X_i\to X_i[1]\to 
%tilde{M_j}$ is a triangle. Apply (4) again to $X_i[1]$ as a self-extension of $M_i[1]$ the highest nontrivial index of the HN filtration of $M_i[1]$ is lower than $i$. Apply (2) to $M_i$ and we can reach a contradiction.\\
%\indent For (6) since if $Y, Y[m]\in\catt$ so is $Y[1]$ the argument in the proof of Lemma \ref{lem:C3L2} (6) does not need to be changed.\\
%\indent For (7) just like for (5) we need to use truncation functors.\\
%\end{proof}
\section{Equivalence of the definitions}
\begin{theorem}\label{C3TB}
%\indent Let $\Lambda$ be a finite dimensional hereditary algebra. Let $(C^{\leq 0}, C^{\geq 0}), (C'^{\leq 0}, C'^{\geq 0})$ be two $t$-structures such that there exists at least one green sequence from $(C^{\leq 0}, C^{\geq 0})$ to $(C'^{\leq 0}, C'^{\geq 0})$. Let $\mathcal{T} = C^{\leq 0}\cap C'^{\geq 0}$.  Let $M_1,\cdots, M_n$ be a finite sequence in $\mathcal{T}$. The following are equivalent.
\indent Let $\Lambda$ be a finite dimensional hereditary algebra. Let $\mathcal{T} = add(\cup_{i=0}^{m-1} (mod\,\Lambda)[i])$. Let $M_1,\cdots, M_n$ be a finite sequence of nonzero objects in $\mathcal{T}$. The following are equivalent:
\begin{enumerate}
\item The sequence is a maximal backward $Hom^{\leq 0}$-orthogonal sequence of Schurian objects $\{M_n\}$ in $\mathcal{T}$.
\item The sequence is a finite sequence in $\mathcal{T}$ that forms a finite HN system for $\mathcal{T}$.
\item The sequence is a sequence of simples from the simple-minded collection $\{S_1,\cdots, S_n\}$ to  $\{S_1[m],\cdots, S_n[m]\}$, that is, it is an $m$-maximal green sequence.
\end{enumerate}
\end{theorem}
\begin{proof}
\indent (1)$\to$(3) By applying Lemma \ref{lem:C3L1} repeatedly it is easy to see that any maximal backward $Hom^{\leq 0}$-orthogonal sequence of Schurian objects $\{M_n\}$ on $\mathcal{T}$ is also a sequence of simples in hearts of $t$-structures related to each other by a finite sequence of forward mutations. Hence (1) implies (3).\\
\indent (2)$\to$(1) Schurness has been proved in Lemma \ref{lem:C3L3}. Backward $Hom^{\leq 0}$ orthogonality has already been proven in Lemma \ref{lem:C3L2}. Maximality holds because of Lemma \ref{lem:C3L2}(3) and (4) due to the reasoning below. Since if the sequence is not maximal then there exists some $M$.such that it can be inserted in the backward $Hom^{\leq 0}$ orthogonal sequence. However such an $M$ must have a unique HN filtration. Hence there exists some $i\leq j$ such that $(M_j, M)\neq 0$ and $(M, M_i)\neq 0$. In this case $M$ can not be inserted in the backward $Hom^{\leq 0}$ orthogonal sequence which proves its maximality.\\
\indent (3)$\to$(2) This is obvious because using truncation functors we can easily show that any object in $\catt$ can be written uniquely as an HN filtration. 
\end{proof}
\chapter{Quivers with multiple edges}\label{C4}
\section{Introduction}
\indent Maximal green sequences (MGSs) were invented by Bernhard Keller \cite{Kel11}. Brustle-Dupont-Perotin \cite{BDP13} and the paper by the first author together with Brustle, Hermes and Todorov \cite{BHIT15} have proven that there are finitely maximal green sequences when the quiver is of finite, tame type or the quiver is mutation equivalent to a quiver of finite or tame types. Furthermore in \cite{BHIT15} it is proven that any tame quiver has finitely many $k$-reddening sequences.\\
\indent However the situation is still pretty much uncharted in the wild case other than cases where the quiver has three vertices which was proven in  \cite{BDP13} which contains a proof highly dependent on the quiver only having three vertices. Despite the fact that the wild case is still unknown in general we can indeed solve it for many easy cases. For example for quivers such as the $k$-Kronecker quiver and $\begin{tikzcd}1\arrow[r, shift right=0.6ex]\arrow[r, shift left=0.6ex] & 2\arrow[r, shift right=0.6ex]\arrow[r, shift left=0.6ex] & 3\end{tikzcd}$ things are really simple due to Lemma \ref{BHIT}.\\
\indent In this chapter we will generalize the results and introduce three theorems that can significantly simplify understanding of maximal green sequences in simply-laced quivers with multiple edges.\\
\indent We can completely describe MGSs of ME-ful quivers using MGSs of their ME-free versions.\\
\begin{theorem}
(Theorem \ref{C4T1B}) MGSs of an acyclic quiver $Q$ are a subset of the set of $Q$-ME-free MGSs of its ME-free version, $Q'$.\label{C4T1}
\end{theorem}
\begin{theorem}
(Theorem \ref{C4T3B}) Let $Q$ be an ME-ful acyclic quiver and $Q'$ be its ME-free version. The MGSs of $Q$ are exactly the $Q$-ME-free MGSs $(C_0,C_1,\cdots C_m)$ of $Q'$ such that for any multiple edge from $i$ to $j$ in $Q$ for any $C$-matrix $C_i$ in the MGS such that there exists a negative $c$-vector with support containing $i$ the mutation on $C_i$ in the MGS isn't done on any negative $c$-vector with support containing $j$.\label{C4T3}
\end{theorem}
\indent In other words to understand MGSs of an acyclic quiver $Q$ we only need to understand the MGSs of its ME-free version which makes multiple edges largely irrelevant in understanding MGSs of acyclic quivers.\\
\indent We can obtain the following crucial corollaries in the acyclic case:
\begin{corollary}\label{C4C}
(Corollary \ref{C4CB})The following statements are true:
\begin{enumerate}
\item The number of maximal green sequences of a quiver $Q$ is no greater than that of its ME-free version.
\item All quivers with an MGS-finite ME-free version must themselves be MGS-finite.
\item No minimally MGS-infinite quiver can contain multiple edges.
\item Any two ME-equivalent quivers are MGS-equivalent to each other.
\end{enumerate}
\end{corollary}
\indent If the quiver isn't necessarily acyclic we still have the following result:
\begin{theorem}
(Theorem \ref{C4T2B})Assume that ($\tilde{Q},\breve{Q})$ are $k$-partition of $Q$ for some $k>1$ any MGS of $Q$ is an MGS of $\tilde{Q}\cup\breve{Q}$.\label{C4T2}
\end{theorem}
\indent In Section 2 we will discuss MGS-finiteness in general. In Section 3 we will prove Theorems \ref{C4T1} and \ref{C4T3}. In Section 4 we will prove Theorem \ref{C4T2}.
\section{MGS-finiteness}
\indent In this section let's review the basics about what kind of quivers have finitely many maximal green sequences.
\begin{definition}
A quiver $Q$ is \textit{MGS-finite} if $Q$ has finitely many maximal green sequences. Any quiver that isn't MGS-finite is \textit{MGS-infinite}.
\end{definition}
\indent Here are some results that are either already known or easily proven about MGS-finiteness of quivers.
\begin{theorem}
\cite{BDP13}Any acyclic quiver $Q$ of finite type or tame type as well as any acyclic quiver $Q$ of wild type with three vertices are MGS-finite.
\end{theorem}
\begin{theorem}
\cite{BHIT15} (Thm 2)\label{BHIT2} If the quiver $Q$ is mutation equivalent to an acyclic quiver of tame type, then Q has only finitely many maximal green sequences.
\end{theorem}
\begin{theorem}
Any quiver $Q$ mutation equivalent to an acyclic quiver of finite or tame type is MGS-finite.
\end{theorem}
\begin{proof}
Due to Theorem \ref{BHIT2} the result is already proven in the mutation-equivalent to tame type case. For the mutation-equivalent to finite type case using the Rotation Lemma in \cite{BHIT15} it is obvious that any MGS in such a quiver must be an $k$-reddening sequence of an acyclic quiver of finite type for a fixed $k$. There are only finitely many such sequences because a $k$-reddening sequence can only repeat a cluster $k+1$ times due to Lemma \ref{C3L4} and in an acyclic quiver of finite type there are only finitely many cluster-tilting objects and hence finitely many clusters.
\end{proof}
\begin{lemma}
If $Q$ is a quiver that isn't connected, $Q^1$, $Q^2$, $\cdots$ $Q^n$ are its connected components. Each $Q^i$ is MGS-finite if and only if $Q$ is MGS-finite.
\end{lemma}
\begin{proof}
\indent Any MGS of $Q$ is essentially formed from taking an MGS $w_i$ of $Q^i$ for each $i$ and then put these mutations together such that the order of elements in each $w_i$ is preserved.\\ 
\indent Since we can obtain all MGSs of $Q^i$ by deleting all $c$-vectors not supported on $Q^i_0$ from all MGSs of $Q$ it is easy to see that if $Q$ is MGS-finite so is $Q^i$ for any $i$.\\
\indent On the other hand if all $Q^i$s are MGS-finite it is easy to see that so is $Q$ because the set of admissible $c$-vectors of $Q$ is the union of admissible $c$-vectors in MGSs of $Q^i$ all of which are finite.\\
\end{proof}
%\indent Using quiver folding it is easy to show that we only need to consider the simply-laced case.\\
\indent There is also an unrelated result about MGS-finiteness we proved which we will include here.
\begin{definition}
A quiver is of \textit{finite green mutation type} if there are finitely many exchange matrices along its maximal green sequences.
\end{definition}
It is easy to see that any quiver that has finitely many maximal green sequences is of finite green mutation type.\\
\begin{lemma}\label{CF1}
If the coframed quiver $\breve{Q}$ of a quiver $Q$ is of finite green mutation type, $Q$ has finitely many maximal green sequences.\\
\end{lemma}
\begin{proof}
For a quiver $Q$ with $|Q_0|=n$, let $Q'=\breve{Q}$ be its coframed quiver and $Q''$ be the coframed quiver of $Q'$. Let's label the extra vertices of $Q'$ as $1',\cdots, n'$. Note that any maximal green sequence $w=(w_1,\cdots, w_k)$ of $Q$ can be extended into a maximal green sequence of $Q'$, $w'=(w_1,\cdots, w_k, 1', 2',\cdots, n')$. Note that any extended exchange matrix that appears in any maximal green sequence of $Q$ is an exchange matrix in some maximal green sequence of $Q'$. Since $Q'$ is of finite green mutation type, there are only finitely many exchange matrices in all maximal green sequences of $Q'$. Hence there are only finitely many extended exchange matrices in any maximal green sequence of $Q$. Since extended exchange matrices can not be repeated in a maximal green sequence, $Q$ has finitely many maximal green sequences.\\ 
\end{proof}
\indent Here is an easy corollary of the lemma above:\\
\begin{corollary}\label{CF2}
If all quivers are of finite green mutation type, all quivers have finitely many maximal green sequences.\\
\end{corollary}
\indent Lemma \ref{CF1} and Corollary \ref{CF2} also hold for valued quivers which we won't discuss in this paper. The proofs don't change when generalized to valued quivers.\\ 
\section{The acyclic case}
\indent Now we need some basic definitions in order to describe and prove the results.
\begin{definition}
A quiver with at least one multiple edge is \textit{ME-ful}. Otherwise it is \textit{ME-free}.
\end{definition}
\begin{definition}
A \textit{multiple edges-free (ME-free)} version of a quiver $Q$ is produced by removing all multiple edges from $Q$ while retaining single edges and vertices.
\end{definition}
\indent For example the ME-free version of the $m$-Kronecker quiver for any $m$ is the quiver $A_1\times A_1$, namely the quiver with two vertices and no arrows.\\
\indent In this section we will use the fact that a path in the semi-invariant picture of $Q$ is also a path in the semi-invariant picture of its ME-free version, $Q'$. Since the definition of whether a path is green and generic differ in semi-invariant pictures of different quivers we will use the concept of \textit{strong genetic green paths} to exclude problematic cases.
\begin{definition}
Let $Q$ be an ME-ful quiver, $Q'$ be its ME-free version. A path in the semi-invariant pictures of $Q$ and $Q'$ is \textit{strong generic green} if it is a generic green path in both pictures.
\end{definition}
\begin{definition}
Let $Q$ be an ME-ful quiver.
\begin{enumerate}
\item A $c$-vector in $Q$ is \textit{ME-free} if it is ME-free if considered as a dimension vector of $Q$. Any $c$-vector in $Q$ that isn't ME-free is \textit{ME-ful}.
\item An MGS in $Q$ is \textit{ME-free} if all its $c$-vectors are ME-free. An MGS of $Q$ that isn't ME-free is \textit{ME-ful}.
\item A generic green path in the semi-invariant picture of $Q$ is \textit{ME-free} if it crosses no wall corresponding to an ME-ful $c$-vector. A generic green path in the semi-invariant picture of $Q$ that isn't ME-free is \textit{ME-ful}.
\item A module of $kQ$ is \textit{ME-free/ME-ful} if its $c$-vector is ME-free/ME-ful.
\end{enumerate}
\end{definition}
\indent Note that if an MGS is ME-free all $c$-vectors in all $c$-matrices in it including those that aren't mutated must be ME-free.\\
\indent If $Q$ is an ME-ful quiver and $Q'$ is its ME-free version it does not technically make sense to discuss ME-fulness of any module of $kQ'$. Here we are going to use the same definition we used in defining ME-fulness of vectors and MGSs of $Q$.
\begin{definition}
Let $Q$ be an ME-ful quiver and let $Q'$ be its ME-free version.
\begin{enumerate}
\item A $c$-vector in $Q'$ is \textit{Q-ME-free} if it is ME-free if considered as a dimension vector of $Q$. Any $c$-vector in $Q'$ that isn't $Q$-ME-free is \textit{Q-ME-ful}.
\item An MGS in $Q'$ is \textit{Q-ME-free} if all its $c$-vectors are $Q$-ME-free. An MGS of $Q'$ that isn't $Q$-ME-free is \textit{Q-ME-ful}.
\item A generic green path in the semi-invariant picture of $Q'$ is \textit{Q-ME-free} if it crosses no wall corresponding to a $Q$-ME-ful $c$-vector. A generic green path in the semi-invariant picture of $Q'$ that isn't $Q$-ME-free is \textit{Q-ME-ful}.
\item A strongly generic green path in the semi-invariant picture of $Q'$ is \textit{strongly $Q$-ME-free} if it is $Q$-ME-free and does not cross any wall corresponding to a $Q$-ME-ful $c$-vector in the semi-invariant picture of $Q$. A generic green path in the semi-invariant picture of $Q'$ that isn't strongly $Q$-ME-free is \textit{weakly Q-ME-ful}.
\item A module of $kQ'$ is \textit{Q-ME-free/Q-ME-ful} if its $c$-vector is $Q$-ME-free/$Q$-ME-ful.
\end{enumerate}
\end{definition}
\indent We will sometimes abuse the notations and use the term $Q$-ME-free for $c$-vectors/MGSs of $Q$. In this case they are just ME-free $c$-vectors/MGSs.
\begin{definition}
If $Q$ and $Q'$ have the same number of vertices, a GS $w$ of $kQ$ \textit{is equivalent to} a GS $w'$ of $kQ'$ if $w$ and $w'$ mutates on the same sequence of $c$-vectors and start from the same $c$-matrix up to permutations. 
\end{definition}
\indent Using the equivalence it makes sense to identify certain MGSs of $Q$ and $Q'$. It is in this sense that we claim and prove that all MGSs of an ME-ful quiver $Q$ are MGSs of its ME-free version, $Q'$.\\
\indent In order to state a corollary we also need three more definitions.
\begin{definition}
The \textit{skeleton} of a quiver $Q$ is produced by replacing all multiple edges from $Q$ by single edges with the sources and targets unchanged.\\
\end{definition}
\indent For example the ME-free version of the $m$-Kronecker quiver for any $m$ is the quiver $A_2$.
\begin{definition}
$Q$ and $Q'$ are quivers. If they have the same ME-free version and the same skeleton then they are \textit{ME-equivalent}.
\end{definition}
\begin{definition}
If every MGS of $Q$ corresponds to some MGS of $Q'$ and vice versa then $Q$ and $Q'$ are MGS-equivalent. 
\end{definition}
%\begin{definition}
%Let $S$ be a subset of $[n]$. A $c$-vector is \textit{$S$-free} if the support of the vector does not. A GS that isn't $S$-free is \textit{$S$-ful}. A module is \textit{$S$-free/$S$-ful} if its $c$-vector is $S$-free/$S$-ful.
%\end{definition}
\begin{lemma}
\indent Let $Q$ be a quiver and $Q'$ be its ME-free version. The following holds:\label{L2}
\begin{enumerate}
\item The set of $Q$-ME-free $c$-vectors of $Q$ and $Q'$ coincide.
\item If $Q$ is an ME-ful quiver then for any positive $Q$-ME-ful vector $c\in\mathbb{R}^n$ then $\langle M,M\rangle_{kQ} - \langle M,M\rangle_{kQ'} \leq -2$.
\item If $Q$ is an ME-ful quiver. Then any of the $Q$-ME-ful $c$-vectors can not be a dimension vector of an exceptional module for $Q'$. Any of the $Q$-ME-ful $c$-vectors of $Q'$ can not be a dimension vector of an exceptional module for $Q$.
\end{enumerate}
\end{lemma}
\begin{proof}
\indent For (1) let the Euler matrices of $Q, Q''$ be $E = e_{ij}, E' = (e'_{ij})$ respectively. Then we have $\langle c,c\rangle_{kQ} = \langle c,c\rangle_{kQ'}$ because whenever $e_{ij}, e'_{ij}$ differ $c_i = 0$ or $c_j = 0$ leaving the term related to $(i,j)$ being 0. Hence the set of $Q$-ME-free $c$-vectors of $Q, Q'$ corresponding to exceptional modules coincide.\\
\indent For (2) assume that such a vector $c$ exists. We have $\langle c,c\rangle_{kQ} = \langle c,c\rangle_{kQ'} = 1$. However the Euler matrix $E = (e_{ij})$ of $Q$ and the Euler matrix $E' = (e'_{ij})$ of $Q'$ differ in the sense that there exists some pair $(i,j)\in [n]$ such that $c_i\neq 0, c_j> 0$ and $0 = e'_{ij} > -2 \geq e_{ij}$. Since for any $k,l\in [n]$ we have $e'_{kl}\geq e_{kl}$ it is easy to see that $\langle c,c\rangle_{kQ'} > \langle c,c\rangle_{kQ}$ and that $\langle M,M\rangle_{kQ} - \langle M,M\rangle_{kQ'} \leq -2$.\\
\indent (3) is a consequence of (2) since $\langle M,M\rangle_{kQ}$ and $\langle M,M\rangle_{kQ'} $ can not both be 1.
\end{proof}
\begin{lemma}
Let $Q$ be an ME-ful quiver. Any MGS of an ME-ful quiver $Q$ must not contain any $Q$-ME-ful $c$-vector of $Q'$ or any vector $c$ which is an imaginary root of $Q'$.
\end{lemma}
\begin{proof}
\indent Due to Lemma \ref{L2}(3) we only need to prove the second part. In that case $\langle c,c\rangle_{kQ} \leq \langle c,c\rangle_{kQ'} < 1$. Hence $c$ is not a $c$-vector of $Q$.
%\indent Using an argument similar to that of the lemma above any MGS of $Q$ can not share any $Q$-ME-ful $c$-vector $Q'$ has. As a result any generic green path inducing a $Q$-ME-ful MGS of $Q'$ induces an infinite reddening sequence in $kQ$.\\
%\indent Since any generic green path inducing an infinite green sequence of $Q'$ has to cross at least one wall corresponding to an indecomposable non-rigid module $M$ it has to cross at least one wall corresponding to an indecomposable non-rigid module in $Q$ since $\langle c,c\rangle_{kQ} < \langle c,c\rangle_{kQ'}$ holds for all $Q$-ME-ful $c$-vectors. At the same time the wall stands because by adding arrows with zero maps we can see $M$ as indecomposable non-rigid modules of $kQ'$ and conditions for $M$ to be stable is the same in $kQ$ and $kQ'$. As a result any generic green path inducing an infinite green sequence of $Q'$ induces an infinite green sequence in $kQ$.\\
%\indent Hence the only possible strongly generic green paths inducing MGSs in $kQ$ must be from $Q$-ME-free green sequences in $kQ'$.
\end{proof}
%\indent Note that the set of generic green paths in the semi-invariant picture of $Q'$ corresponding to an $Q$-ME-free MGS of $Q'$ is a superset of the set of $Q$-ME-free generic green paths in the semi-invariant picture of $Q'$ because the latter requires that a generic green path that does not cross any $Q$-ME-ful wall in the semi-invariant picture of $Q'$ do not cross any $Q$-ME-ful wall in the semi-invariant picture of $Q$ as well.
%\begin{lemma}
%Let $Q$ be an ME-ful quiver and let $Q'$ be its ME-free version. Then any MGS of $Q$ must be a $Q$-ME-free MGS of $Q'$.
%\end{lemma}
\indent Now we can easily establish the following theorem.
\begin{theorem}
MGSs of an acyclic quiver $Q$ are a subset of the set of $Q$-ME-free MGSs of its ME-free version, namely $Q'$.\label{C4T1B}
\end{theorem}
\begin{proof}
\indent If the statement is incorrect along an MGS of $Q$ pick the first $C$-matrix that isn't shared by $Q'$ assuming that such an MGS exists. \\
\indent In this case either at least one $c$-vector is $Q$-ME-ful or none is. If some $c$-vector is $Q$-ME-ful it must be formed by extending one $Q$-ME-free exceptional module by another $Q$-ME-free exceptional module in $Q$ (i.e. $dim Ext_{kQ}(A,B) = 1$ because it can not be larger due to Lemma \ref{C4L}. Let's label the indecomposable module formed by the extension $M$. We need $Ext_{kQ}(B,A) = 0$ so that $\langle M,M\rangle_{kQ} = 1$) while in $Q'$  there are no such extensions (i.e. $dim Ext_{kQ'}(A,B) = 0$). However this is impossible because $A, B$ are rigid, $Hom$-orthogonal and indecomposable because $\langle M,M\rangle_{kQ} - \langle M,M\rangle_{kQ'} \leq -2$ which causes $\langle M,M\rangle_{kQ'}$ to be at least 3 which is impossible because $\langle M,M\rangle_{kQ'} = \langle A,A\rangle_{kQ'} + \langle A,B\rangle_{kQ'} + \langle B,A\rangle_{kQ'} + \langle B,B\rangle_{kQ'} = 2 - dim Ext_{kQ'}(A,B) - dim Ext_{kQ'}(B,A)$ is at most 2.\\
\indent If no $c$-vector is $Q$-ME-ful then in $kQ$, $kQ'$ the relevant $Hom$ and $Ext$ groups shouldn't differ because neither of them involve the multiple edges that are absent in $kQ'$ . As a result that can't happen either.\\
\indent Hence the $C$-matrices corresponding to $Q, Q'$ in the MGS are all the same. Any MGS of $Q$ must be an MGS of $Q'$ with the same $C$-matrices. Since all the $C$-matrices of the two quivers are the same they have the same associated permutation.\\
\end{proof}
\indent Now we can prove a stronger result.\\
\begin{theorem}\label{C4T3B}
Let $Q$ be an ME-ful acyclic quiver and $Q'$ be its ME-free version. The MGSs of $Q$ are exactly the $Q$-ME-free MGSs $(C_0,C_1,\cdots C_m)$ of $Q'$ such that for any multiple edge from $i$ to $j$ in $Q$ for any $C$-matrix $C_i$ in the MGS such that there exists a negative $c$-vector with support containing $i$ it is true that the mutation on $C_i$ in the MGS isn't done on any negative $c$-vector with support containing $j$.
\end{theorem}
\begin{proof}
Let's compare $\langle M, N \rangle_{kQ}$ and $\langle M, N \rangle_{kQ'}$. They differ if and only if there exists some multiple edge from $i$ to $j$ such that $i$ is in the support of $M$ and $j$ is in the support of $N$. In this case since $Hom_{kQ}(M,N) = Hom_{kQ'}(M,N)  = Ext_{kQ'}(M,N) = 0$ it is true that $dim Ext_{kQ}(M,N) > 0$. Repeating the argument in Theorem \ref{C4T1B} we can show that this is the only possible scenario for a $Q$-ME-free MGSs of $Q'$ to not be identical to an MGS in $Q$.
\end{proof}
\begin{corollary}\label{C4CB}
The following statements are true:
\begin{enumerate}
\item The number of maximal green sequences of a quiver $Q$ is no greater than that of its ME-free version.
\item All quivers with an MGS-finite ME-free version must themselves be MGS-finite.
\item No minimally MGS-infinite quiver can contain multiple edges.
\item Any two ME-equivalent quivers are MGS-equivalent to each other.
\end{enumerate}
\end{corollary}
\begin{proof}
\indent Only (4) needs to be proven even though it is still obvious. For ME-equivalent quivers $Q$ and $Q'$ the conditions of Theorem \ref{C4T3B} are identical which is why the number of MGS are identical.
\end{proof}
\begin{example}
The maximal green sequences of $Q: \begin{tikzcd}
1\righttwicedoublearrow\arrow[rd] &  & 3\\
 & 2\arrow[ur]
\end{tikzcd}$ are maximal green sequences of its ME-free version $Q': 1\to 2\to 3$ that has no $c$-vector with support containing $\{1,3\}$ and satisfies the conditions in Theorem \ref{C4T3B} with respect to the arrow $\begin{tikzcd} 1\rightdoublearrow & 3\end{tikzcd}$. It's easy to see that $Q$ is MGS-finite. In fact it has 3 MGSs.
\end{example}
\begin{example}
The maximal green sequences of $Q: \begin{tikzcd}[cramped, sep=small]
1\arrow[r] & 2 \rightdoublearrow&  3 \arrow[r] & 4\\
\end{tikzcd}$ are some maximal green sequences of its ME-free version $Q':1\to 2$\quad$3\to 4$ % \begin{tikzcd}
%1\arrow[r] & 2&  3 \arrow[r] & 4\\
%\end{tikzcd}$  
that has no $c$-vector with support containing $\{2,3\}$ and satisfies the conditions in Theorem \ref{C4T3B} with respect to the arrow $\begin{tikzcd} 2\rightdoublearrow & 3\end{tikzcd}$. It's easy to see that $Q$ is MGS-finite because $A_2$ is.
\end{example}
\indent Now we can provide a much shorter proof to the fact that all acyclic quivers with three vertices are MGS-finite which was originally proven in \cite{BDP13}.
\begin{corollary}
Any acyclic quiver with at most three vertices is MGS-finite.
\end{corollary}
\begin{proof}
Due to the theorem we only need to show that any ME-free acyclic quiver with at most three vertices is MGS-finite. Such a quiver is either of finite or tame type and is hence MGS-finite.
\end{proof}
\section{The general case}
\indent In the general case the theorem above isn't correct. We can show that using the following counterexample. The quiver $Q$ here is $\begin{tikzcd}
1\arrow[rd] &  & 3\lefttwicedoublearrow\\
 & 2\arrow[ur]
\end{tikzcd}$.\\
$\begin{bmatrix} 
0 &1 & -2\\
-1 & 0 & 1\\
2 & -1 & 0\\
-1 & 0 & 0\\
0 & -1 & 0\\
0 & 0 & -1\\
\end{bmatrix}\overset{\mu_2}{\to}\begin{bmatrix} 
0 &-1 & -1\\
1 & 0 & -1\\
1 & 1 & 0\\
-1 & 0 & 0\\
-1 & 1 & 0\\
0 & 0 & -1\\
\end{bmatrix}\overset{\mu_1}{\to}\begin{bmatrix} 
0 &1 & 1\\
-1 & 0 & -1\\
-1 & 1 & 0\\
1 & -1 & -1\\
1 & 0 & -1\\
0 & 0 & -1\\
\end{bmatrix}\overset{\mu_3}{\to}\begin{bmatrix} 
0 &2 & -1\\
-2 & 0 & 1\\
1 & -1 & 0\\
0 & -1 & 1\\
0 & 0 & 1\\
-1 & 0 & 1\\
\end{bmatrix}\overset{\mu_1}{\to}\begin{bmatrix} 
0 &-2 & 1\\
2 & 0 & -1\\
-1 & 1 & 0\\
0 & -1 & 1\\
0 & 0 & 1\\
1 & 0 & 0\\
\end{bmatrix}\overset{\mu_2}{\to}\begin{bmatrix} 
0 &2 & -1\\
-2 & 0 & 1\\
1 & -1 & 0\\
0 & 1 & 0\\
0 & 0 & 1\\
1 & 0 & 0\\
\end{bmatrix}$\\
\indent Here we have a maximal green sequence with at least one ME-full $c$-vector. Moreover it is easy to see that if we replace the double edge by triple edge and obtain $Q': \begin{tikzcd}
1\arrow[rd] &  & 3\lefttwicetriplearrow\\
 & 2\arrow[ur]
\end{tikzcd}$\ (2,1,3,1,2) is not an MGS of the quiver $Q'$ nor is it an MGS of the ME-free version or skeleton of $Q$.\\
\indent However we can still perform quiver cutting in more limited situations. Let's first introduce a concept.
\begin{definition}
A $k$\textit{-edge} is a tuple $(i,j)$ where $i,j\in [n]$ and $k|b_{ij}, k|b_{ji}$.
\end{definition}
\begin{definition}
Let $Q$ be a quiver possibly having oriented cycles, let $k$ be an integer greater than 1. Assume that $Q_0$ = $\tilde{Q}_0 + \breve{Q}_0$, $P = Q]_{\tilde{Q}_0}, R = Q]_{\breve{Q}_0}$. If for all $i\in \tilde{Q}_0, j\in  \breve{Q}_0$ $k|b_{ij}$ and $k|b_{ji}$ we say $Q$ is $k$-\textit{partible} and $(P, R)$ is a $k$\textit{-partition} of $Q$.
\end{definition}
\begin{theorem}
Assume that ($\tilde{Q},\breve{Q})$ are $k$-partition of $Q$ for some $k>1$ then any MGS of $Q$ is an MGS of $\tilde{Q}\cup\breve{Q}$.\label{C4T2B}
\end{theorem}
\begin{proof}
The property that for any $i\in Q_1$ and $j\in Q_2$ $k|c_{ij}$ is preserved by mutation. Hence any mutation that cause any $c$-vector to cross bot has to violate the Sink before Source Theorem in \cite{BHIT15}.
\end{proof}
\begin{corollary}
Under the conditions of the theorem above, if $\tilde{Q}$ and $\breve{Q}$ are MGS-finite then so is $Q$.
\end{corollary}
\begin{proof}
If $\tilde{Q}$ and $\breve{Q}$ are MGS-finite so is $\tilde{Q}\cup\breve{Q}$. As a result so is $Q$ due to the theorem.
\end{proof}
\begin{example}
$Q:\begin{tikzcd}
1\arrow[rd] &                                             &                   &5\arrow[dd]\\
                 &2\arrow[dl]\rightdoublearrow & 4\arrow[ur] &\\
3\arrow[uu] &                                            &                  &6\arrow[ul]\\
\end{tikzcd}$ is a quiver with oriented cycles. Due to the theorem we can cut the $\begin{tikzcd}2\rightdoublearrow & 4\end{tikzcd}$ arrow. After cutting this arrow it is easy to see that $Q$ is MGS-finite.
\end{example}
\begin{example}
$Q:\begin{tikzcd}
 &       2\rightdoublearrow                                      &  3\arrow[rd]&\\
1\arrow[ru]                 &    &  &4\arrow[dl]\\
 &     6\arrow[ul]                                       & 5\leftquadruplearrow                 &\\
\end{tikzcd}$ is another quiver with oriented cycles.  Due to the theorem we can cut the $\begin{tikzcd}2\rightdoublearrow & 3\end{tikzcd}$ and  $\begin{tikzcd}6 & 5\leftquadruplearrow\end{tikzcd}$ arrows. After cutting these arrows it is easy to see that $Q$ is MGS-finite.
\end{example}
%\begin{}
\bibliographystyle{amsplain}
\begin{thebibliography}{10}
\bibitem{AI10} Takuma Aihara and Osamu Iyama, \textit{Silting mutation in triangulated categories}, J London Math Soc (2012) 85 (3): 633-668.
\bibitem{ARS} Maurice Auslander, Idun Reiten and Sverre O. Smalo, \textit{Representation Theory of Artin Algebras}, Cambridge University Press, Aug 21, 1997.
\bibitem{ASS06} Ibrahim Assem, Daniel Simson and Andrzej Skowronski, \textit{Elements of the Representation Theory of Associative Algebras, Volume 1, Techniques of Representation Theory}, London Mathematical Society Student Texts, 2006.
\bibitem{BBD} A.A. Beilinson, J. Bernstein, Pierre Deligne, \textit{Analyse et topologie sur les espaces singuliers}, Ast\'erisque 100, 1983. (in French)
\bibitem{B07} Tom Bridgeland, \textit{Stability Conditions on Triangulated Categories}, Annals of Mathematics, vol. 166, no. 2, 2007, pp. 317-345. JSTOR, \href{www.jstor.org/stable/20160065}{www.jstor.org/stable/20160065}.
\bibitem{BDP13} Thomas Br\"ustle, Gr\'{e}goire Dupont and Matthieu P\'{e}rotin, \textit{On Maximal Green Sequences},  Int Math Res Notices (2014), 4547-4586.
\bibitem{BHIT15} Thomas Br\"ustle, Stephen Hermes, Kiyoshi Igusa and Gordana Todorov, \textit{Semi-invariant pictures and two conjectures on maximal green sequences}, J Algebra {\bf 473} (2017): 80-109.
\bibitem{BST17} Thomas Br\"ustle, David Smith and Hipolito Treffinger, \textit{Stability conditions, $\tau$-tilting Theory and Maximal Green Sequences}, \href{https://arxiv.org/abs/1705.08227}{arXiv:1705.08227 [math.RT]}, 2017.
\bibitem{BST18A} Thomas Br\"ustle, David Smith and Hipolito Treffinger, \textit{Wall and Chamber Structure for finite-dimensional Algebras}\href{https://arxiv.org/abs/1805.01880}{arXiv:1805.01880 [math.RT]}, 2018.
\bibitem{BST18B} Thomas Br\"ustle, David Smith and Hipolito Treffinger, \textit{Stability Conditions and Maximal Green Sequences in Abelian Categories}, \href{https://arxiv.org/abs/1805.04382}{arXiv:1805.04382 [math.RT]}, 2018.
\bibitem{BY13} Thomas Br\"ustle and Dong Yang, \textit{Ordered Exchange Graphs}, Advances in Representation Theory of Algebras, 135--193, EMS Ser. Congr. Rep., Eur. Math. Soc., Z�rich, 2013.
\bibitem{DR76} Vlastimil Dlab and Claus Michael Ringel, \textit{Indecomposable representations of graphs and algebras}, Memoirs of AMS, Vol. 173, 1976. 
\bibitem{FST11} Anna Felikson, Michael Shapiro, Pavel Tumarkin, \textit{Cluster algebras and triangulated orbifolds}, Advances in Mathematics, Volume 231, Issue 5, (2012) 2953-3002.
\bibitem{FZ01} Sergey Fomin and Andrei Zelevinsky, \textit{Cluster algebras I: Foundations}, J. Amer. Math. Soc. 15 (2002), 497-529.
\bibitem{FZ06} Sergey Fomin and Andrei Zelevinsky, \textit{Cluster algebras IV: Coefficients}, Compositio Math. 143 (2007) 112-164.
%\bibitem{Gab72} Peter Gabriel, \textit{Unzerlegbare Darstellungen. I}, Manuscripta Mathematica 6: 71�103, 1972. (in German)
\bibitem{GHKK14} Mark Gross, Paul Hacking, Sean Keel and Maxim Kontsevich, \textit{
Canonical bases for cluster algebras}, J. Amer. Math. Soc. 31 (2018), 497-608.
\bibitem{GM14} Alexander Garver and Gregg Musiker, \textit{On Maximal Green Sequences For Type $A$ Quivers}, G. J Algebr Comb (2017) 45: 553-599.
\bibitem{H88} Dieter Happel, \textit{Triangulated Categories in the Representation of Finite Dimensional Algebras}, London Mathematical Society Lecture Note Series, Cambridge: Cambridge University Press, 1988.
.%\bibitem{Igu14} Kiyoshi Igusa, \textit{Notes on picture groups and maximal green sequences}, 2014
\bibitem{IOTW15} Kiyoshi Igusa, Kent Orr, Gordana Todorov and Jerzy Weyman, \textit{Modulated semi-invariants},  \href{http://arxiv.org/abs/1507.03051}{arXiv:1507.03051 [math.RT]}.
\bibitem{IOTW4} Kiyoshi Igusa, Kent Orr, Gordana Todorov and Jerzy Weyman, \textit{Picture groups of finite type and cohomology in type $A_n$}, unpublished preprint, 2014.
%\bibitem{IT14} Kiyoshi Igusa and Gordana Todorov, \textit{Picture groups and maximal green sequences}, unpublished preprint 2014.
\bibitem{I17} Kiyoshi Igusa, \textit{Linearity of stability conditions}, \href{https://arxiv.org/abs/1706.06986}{arXiv:1706.06986}.
\bibitem{IT17} Kiyoshi Igusa and Gordana Todorov, \textit{Picture groups and maximal green sequences}, unpublished preprint 2014.
\bibitem{IY06} Osamu Iyama and Yuji Yoshino, \textit{Mutation in triangulated categories and rigid Cohen-Macaulay modules}, Y. Invent. math. (2008) 172: 117-168.
\bibitem{IZ17} Kiyoshi Igusa and Ying Zhou, \textit{Tame Hereditary Algebras have finitely many m-Maximal Green Sequences}, \href{https://arxiv.org/abs/1706.09118}{arXiv:1706.09118}.
\bibitem{Kel11} Bernhard Keller, \textit{Quiver mutation and quantum dilogarithm identities}, Representations of Algebras and Related Topics, Editors A. Skowronski and K. Yamagata, EMS Series of Congress Reports, European Mathematical Society (2011): 85-116.
\bibitem{KQ15} Alastair King, Yu Qiu, \textit{Exchange graphs and Ext quivers}, Advances in Mathematics, Volume 285, 1106-1154, 2015.
\bibitem{KY12} Steffen Koenig and Dong Yang, \textit{Silting objects, simple-minded collections, t-structures and co-t-structures for finite-dimensional algebras}, Doc. Math. 19 (2014), 403-438. 
\bibitem{Mul15} Gregory Muller, \textit{The existence of a maximal green sequence is not invariant under quiver mutation}, Combinatorics, Volume 23, Issue 2 (2016) :P2.47. 
\bibitem{MurD1} Daniel Murfet, \href{http://therisingsea.org/notes/DerivedCategories.pdf}{Derived Categories Part I}.
\bibitem{MurD2} Daniel Murfet, \href{http://therisingsea.org/notes/DerivedCategoriesPart2.pdf}{Derived Categories Part II}.
\bibitem{MurT1} Daniel Murfet, \href{http://therisingsea.org/notes/TriangulatedCategories.pdf}{Triangulated Categories Part I}.
%\bibitem{NR15} Tomoki Nakanishi and Dylan Rupel, \textit{Companion cluster algebras to a generalized cluster algebra}, \href{http://arxiv.org/abs/1504.06758}{arXiv:1504.06758 [math.RA]}, 2015.
\bibitem{NZ11} Tomoki Nakanishi and Andrei Zelevinsky, \textit{On tropical dualities in cluster algebras}, Contemporary Mathematics 565, 217-226.
\bibitem{R84} Claus Michael Ringel, \textit{Tame Algebras and Integral Quadratic Forms}, Volume 1099 of \textit{Lecture Notes in Mathematics}, Springer-Verlag, 1984
\bibitem{Sal14} Ibrahim Saleh, \textit{Exchange Maps of Cluster Algebras}, \href{http://arxiv.org/abs/1011.0894v3}{arXiv:1011.0894 [math.RT]}, 2014.
\bibitem{SS06} Daniel Simson and Andrzej Skowronski, \textit{Elements of the Representation Theory of Associative Algebras, Volume 2, Tubes and Concealed Algebras of Euclidean Type}, London Mathematical Society Student Texts, 2006.
\bibitem{ST12} David Speyer, Hugh Thomas, \textit{Acyclic cluster algebras revisited}, In: Buan A., Reiten I., Solberg �. (eds) \textit{Algebras, Quivers and Representations. Abel Symposia}, vol 8. Springer, Berlin, Heidelberg.
\end{thebibliography}

\end{document}