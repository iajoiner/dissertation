\chapter{Quivers with multiple edges}\label{C4}
\section{Introduction}
\indent Maximal green sequences (MGSs) were invented by Bernhard Keller \cite{Kel11}. Brustle-Dupont-Perotin \cite{BDP13} and the paper by the first author together with Brustle, Hermes and Todorov \cite{BHIT15} have proven that there are finitely maximal green sequences when the quiver is of finite, tame type or the quiver is mutation equivalent to a quiver of finite or tame types. Furthermore in \cite{BHIT15} it is proven that any tame quiver has finitely many $k$-reddening sequences.\\
\indent However the situation is still pretty much uncharted in the wild case other than cases where the quiver has three vertices which was proven in  \cite{BDP13} which contains a proof highly dependent on the quiver only having three vertices. Despite the fact that the wild case is still unknown in general we can indeed solve it for many easy cases. For example for quivers such as the $k$-Kronecker quiver and $\begin{tikzcd}1\arrow[r, shift right=0.6ex]\arrow[r, shift left=0.6ex] & 2\arrow[r, shift right=0.6ex]\arrow[r, shift left=0.6ex] & 3\end{tikzcd}$ things are really simple due to the Target before Source Theorem in \cite{BHIT15}.\\
\indent In this chapter we will generalize the results and introduce three theorems that can significantly simplify understanding of maximal green sequences in simply-laced quivers with multiple edges.\\
\indent We can completely describe MGSs of ME-ful quivers using MGSs of their ME-free versions.\\
\begin{theorem}
(Theorem \ref{C4T1B}) MGSs of an acyclic quiver $Q$ are a subset of the set of $Q$-ME-free MGSs of its ME-free version, $Q'$.\label{C4T1}
\end{theorem}
\begin{theorem}
(Theorem \ref{C4T3B}) MGSs of an ME-ful acyclic quiver $Q$ are exactly the $Q$-ME-free MGSs of its ME-free version, $Q'$ such that for any multiple edge from $i$ to $j$ in $Q$ if there exists any negative $c$-vector with support containing $i$ no mutation is done on any negative $c$-vector with support containing $j$.\label{C4T3}
\end{theorem}
\indent In other words to understand MGSs of an acyclic quiver $Q$ we only need to understand the MGSs of its ME-free version which makes multiple edges largely irrelevant in understanding MGSs of acyclic quivers.
\indent We can obtain the following crucial corollaries in the acyclic case:
\begin{corollary}\label{C4C}
(Corollary \ref{C4CB})The following statements are true:
\begin{enumerate}
\item The amount of maximal green sequences of a quiver $Q$ is no greater than that of its ME-free version.
\item All quivers with an MGS-finite ME-free version must themselves be MGS-finite.
\item No minimally MGS-infinite quiver can contain multiple edges.
\item Any two ME-equivalent quivers are MGS-equivalent to each other.
\end{enumerate}
\end{corollary}
\indent If the quiver isn't necessarily acyclic we still have the following result:
\begin{theorem}
(Theorem \ref{C4T2B})Assume that ($\tilde{Q},\breve{Q})$ are $k$-partition of $Q$ for some $k>1$ any MGS of $Q$ is an MGS of $\tilde{Q}\cup\breve{Q}$.\label{C4T2}
\end{theorem}
\indent In Section 2 we will provide the background required to understand the rest of the paper. In Section 3 we will prove Theorems \ref{C4T1} and \ref{C4T3}. In Section 4 we will prove Theorem \ref{C4T2}.
\section{MGS-finiteness}
\indent In this section let's review the basics about what kind of quivers have finitely many maximal green sequences.
\begin{definition}
A quiver $Q$ is \textit{MGS-finite} if $Q$ has finitely many maximal green sequences. Any quiver that isn't MGS-finite is \textit{MGS-infinite}.
\end{definition}
\indent Here are some results that are either already known or easily proven about MGS-finiteness of quivers.
\begin{theorem}
\cite{BDP13}Any acyclic quiver $Q$ of finite type or tame type as well as any acyclic quiver $Q$ of wild type with three vertices are MGS-finite.
\end{theorem}
\begin{theorem}
Any quiver $Q$ mutation equivalent to an acyclic quiver of finite or tame type is MGS-finite.
\end{theorem}
\begin{proof}
Due to \cite{BHIT15} the result is already proven in the mutation-equivalent to tame type case. For the mutation-equivalent to finite type case using the Rotation Lemma in \cite{BHIT15} it is obvious that any MGS in such a quiver must be an $k$-reddening sequence of an acyclic quiver of finite type for a fixed $k$. There are only finitely many such sequences because a $k$-reddening sequence can only repeat a cluster $k+1$ times as shown in \cite{BHIT15} and \cite{IZ17} and in an acyclic quiver of finite type there are only finitely many cluster-tilting objects and hence clusters.
\end{proof}
\begin{lemma}
If $Q$ is a quiver that isn't connected, $Q^1$, $Q^2$, $\cdots$ $Q^n$ are its connected components. Each $Q^i$ is MGS-finite if and only if $Q$ is MGS-finite.
\end{lemma}
\begin{proof}
\indent Any MGS of $Q$ is essentially formed from taking an MGS $w_i$ of $Q^i$ for each $i$ and then put these mutations together such that the order of elements in each $w_i$ is preserved.\\ 
\indent Since we can obtain all MGSs of $Q^i$ by deleting all $c$-vectors not supported on $Q^i_0$ from all MGSs of $Q$ it is easy to see that if $Q$ is MGS-finite so is $Q^i$ for any $i$.\\
\indent On the other hand if all $Q^i$s are MGS-finite it is easy to see that so is $Q$ because the set of admissible $c$-vectors of $Q$ is the union of admissible $c$-vectors in MGSs of $Q^i$ all of which are finite.\\
\end{proof}
\indent Using quiver folding it is easy to show that we only need to consider the simply-laced case.\\
\indent There is also an unrelated result about MGS-finiteness I proved which I will include here.
\begin{definition}
A valued quiver is of \textit{finite green mutation type} if there are finitely many exchange matrices along its maximal green sequences.
\end{definition}
It is easy to see that any valued-quiver that has finitely many maximal green sequences is of finite green mutation type.\\
\begin{lemma}
If the coframed quiver $\breve{Q}$ of a valued quiver $Q$ is of finite green mutation type, $Q$ has finitely many maximal green sequences.\\
\end{lemma}
\begin{proof}
For a valued quiver $Q$ with $|Q_0|=n$, let $Q'=\breve{Q}$ be its coframed quiver and $Q''$ be the coframed quiver of $Q'$. Let's label the extra vertices of $Q'$ as $1',\cdots, n'$. Note that any maximal green sequence $w=(w_1,\cdots, w_k)$ of $Q$ can be extended into a maximal green sequence of $Q'$, $w'=(w_1,\cdots, w_k, 1', 2',\cdots, n')$. Note that any extended exchange matrix that appears in any maximal green sequence of $Q$ is an exchange matrix in some maximal green sequence of $Q'$. Since $Q'$ is of finite green mutation type, there are only finitely many exchange matrices in all maximal green sequences of $Q'$. Hence there are only finitely many extended exchange matrices in any maximal green sequence of $Q$. Since extended exchange matrices can not be repeated in a maximal green sequence, $Q$ has finitely many maximal green sequences.\\ 
\end{proof}
\indent Here is an easy corollary of the lemma above:\\
\begin{corollary}
If all valued quivers are of finite green mutation type, all valued quivers have finitely many maximal green sequences.\\
\end{corollary}
\section{The acyclic case}
\begin{definition}
A \textit{multiple edges-free (ME-free)} version of a quiver $Q$ is produced by removing all multiple edges from $Q$ while retaining single edges and vertices.
\end{definition}
\indent For example the ME-free version of the $m$-Kronecker quiver for any $m$ is the quiver $A_1\times A_1$, namely the quiver with two vertices and no arrows.\\
\indent In this section we will use the fact that a path in the semi-invariant picture of $Q$ is also a path in the semi-invariant picture of its ME-free version, $Q'$. Since the definition of whether a path is green and generic differ in semi-invariant pictures of different quivers we will use the concept of \textit{strong genetic green paths} to exclude problematic cases.
\begin{definition}
Let $Q$ be an ME-ful quiver, $Q'$ be its ME-free version. A path in the semi-invariant pictures of $Q$ and $Q'$ is \textit{strong generic green} if it is a generic green path in both pictures.
\end{definition}
\begin{definition}
Let $Q$ be an ME-ful quiver.
\begin{enumerate}
\item A $c$-vector in $Q$ is \textit{ME-free} if for it is ME-free if considered as a dimensioHowevern vector of $Q$. Any $c$-vector in $Q$ that isn't ME-free is \textit{ME-ful}.
\item An MGS in $Q$ is \textit{ME-free} if all its $c$-vectors are ME-free. An MGS of $Q$ that isn't ME-free is \textit{ME-ful}.
\item A generic green path in the semi-invariant picture of $Q$ is \textit{ME-free} if it crosses no wall corresponding to an ME-ful $c$-vector. A generic green path in the semi-invariant picture of $Q$ that isn't ME-free is \textit{ME-ful}.
\item A module of $kQ$ is \textit{ME-free/ME-ful} if its $c$-vector is ME-free/ME-ful.
\end{enumerate}
\end{definition}
\indent Note that if an MGS is ME-free all $c$-vectors in all $c$-matrices in it including those that aren't mutated must be ME-free.\\
\indent If $Q$ is an ME-ful quiver and $Q'$ is its ME-free version it does not technically make sense to discuss ME-fulness of any module of $kQ'$. Here we are going to use the same definition we used in defining ME-fulness of vectors and MGSs of $Q$.
\begin{definition}
Let $Q$ be an ME-ful quiver and let $Q'$ be its ME-free version.
\begin{enumerate}
\item A $c$-vector in $Q'$ is \textit{Q-ME-free} if for it is ME-free if considered as a dimension vector of $Q$. Any $c$-vector in $Q'$ that isn't $Q$-ME-free is \textit{Q-ME-ful}.
\item An MGS in $Q'$ is \textit{Q-ME-free} if all its $c$-vectors are $Q$-ME-free. An MGS of $Q'$ that isn't $Q$-ME-free is \textit{Q-ME-ful}.
\item A generic green path in the semi-invariant picture of $Q'$ is \textit{Q-ME-free} if it crosses no wall corresponding to a $Q$-ME-ful $c$-vector. A generic green path in the semi-invariant picture of $Q'$ that isn't $Q$-ME-free is \textit{Q-ME-ful}.
\item A strongly generic green path in the semi-invariant picture of $Q'$ is \textit{strongly $Q$-ME-free} if it is $Q$-ME-free and does not cross any wall corresponding to a $Q$-ME-ful $c$-vector in the semi-invariant picture of $Q$. A generic green path in the semi-invariant picture of $Q'$ that isn't strongly $Q$-ME-free is \textit{weakly Q-ME-ful}.
\item A module of $kQ'$ is \textit{Q-ME-free/Q-ME-ful} if its $c$-vector is $Q$-ME-free/$Q$-ME-ful.
\end{enumerate}
\end{definition}
\indent We will sometimes abuse the notations and use the term $Q$-ME-free for $c$-vectors/MGSs of $Q$. In this case they are just ME-free $c$-vectors/MGSs.
\begin{definition}
If $Q$ and $Q'$ have the same number of vertices, a GS $w$ of $kQ$ \textit{is equivalent to} a GS $w'$ of $kQ'$ if $w$ and $w'$ mutates on the same sequence of $c$-vectors and start from the same $c$-matrix up to permutations. 
\end{definition}
\indent Using the equivalence it makes sense to identify certain MGSs of $Q$ and $Q'$. It is in this sense that we claim and prove that all MGSs of an ME-ful quiver $Q$ are MGSs of its ME-free version, $Q'$.
\indent In order to state a corollary we also need three more definitions.
\begin{definition}
The \textit{skeleton} of a quiver $Q$ is produced by replacing all multiple edges from $Q$ by single edges with the sources and targets unchanged.
\end{definition}
\indent For example the ME-free version of the $m$-Kronecker quiver for any $m$ is the quiver $A_2$.
\begin{definition}
$Q$ and $Q'$ are quivers. If they have the same ME-free version and the same skeleton then they are \textit{ME-equivalent}.
\end{definition}
\begin{definition}
If every MGS of $Q$ corresponds to some MGS of $Q'$ and vice versa then $Q$ and $Q'$ are MGS-equivalent. 
\end{definition}
%\begin{definition}
%Let $S$ be a subset of $[n]$. A $c$-vector is \textit{$S$-free} if the support of the vector does not. A GS that isn't $S$-free is \textit{$S$-ful}. A module is \textit{$S$-free/$S$-ful} if its $c$-vector is $S$-free/$S$-ful.
%\end{definition}
\begin{lemma}
\indent Let $Q$ be a quiver and $Q'$ be its ME-free version. The following holds:\label{L2}
\begin{enumerate}
\item The set of $Q$-ME-free $c$-vectors of $Q$ and $Q'$ coincide.
\item If $Q$ is an ME-ful quiver then for any positive $Q$-ME-ful vector $c\in\mathbb{R}^n$ $\langle M,M\rangle_{kQ} - \langle M,M\rangle_{kQ'} \leq -2$.
\item If $Q$ is an ME-ful quiver. Then any of the $Q$-ME-ful $c$-vectors can not be a dimension vector of an indecomposable rigid module for $Q'$. Any of the $Q$-ME-ful $c$-vectors of $Q'$ can not be a dimension vector of an indecomposable rigid module for $Q$.
\end{enumerate}
\end{lemma}
\begin{proof}
\indent For (1).Let the Euler matrices of $Q, Q''$ be $E = e_{ij}, E' = (e'_{ij})$ respectively. $\langle c,c\rangle_{kQ} = \langle c,c\rangle_{kQ'}$ because whenever $e_{ij}, e'_{ij}$ differ $c_i = 0$ or $c_j = 0$ leaving the term related to $(i,j)$ 0. Hence the set of $Q$-ME-free $c$-vectors of $Q, Q'$ corresponding to indecomposable rigid modules coincide.\\
\indent For (2) Assume that such a vector, $c$ exists. $\langle c,c\rangle_{kQ} = \langle c,c\rangle_{kQ'} = 1$. However the Euler matrix $E = (e_{ij})$ of $Q$ and the Euler matrix $E' = (e'_{ij})$ of $Q'$ differ in the sense that there exists some pair $(i,j)\in [n]$ such that $c_i\neq 0, c_j> 0$ and $0 = e'_{ij} > -2 \geq e_{ij}$. Since for any $k,l\in [n]$ $e'_{kl}\geq e_{kl}$ it is easy to see that $\langle c,c\rangle_{kQ'} > \langle c,c\rangle_{kQ}$ and that $\langle M,M\rangle_{kQ} - \langle M,M\rangle_{kQ'} \leq -2$.\\
\indent (3) is a consequence of (2) since $\langle M,M\rangle_{kQ}$ and $\langle M,M\rangle_{kQ'} $ can not both be 1.
\end{proof}
\begin{lemma}
Let $Q$ be an ME-ful quiver. Any MGS of a ME-ful quiver $Q$ must not contain any $Q$-ME-ful $c$-vector of $Q'$ or any vector $c$ such that the indecomposable module $M$ with $dim(M) = c$ aren't real Schur.
\end{lemma}
\begin{proof}
\indent Due to \ref{L2}(3) we only need to prove the second part. In that case $\langle c,c\rangle_{kQ} \leq \langle c,c\rangle_{kQ'} < 1$. Hence $c$ is not a $c$-vector of $Q$.
%\indent Using an argument similar to that of the lemma above any MGS of $Q$ can not share any $Q$-ME-ful $c$-vector $Q'$ has. As a result any generic green path inducing a $Q$-ME-ful MGS of $Q'$ induces an infinite reddening sequence in $kQ$.\\
%\indent Since any generic green path inducing an infinite green sequence of $Q'$ has to cross at least one wall corresponding to an indecomposable non-rigid module $M$ it has to cross at least one wall corresponding to an indecomposable non-rigid module in $Q$ since $\langle c,c\rangle_{kQ} < \langle c,c\rangle_{kQ'}$ holds for all $Q$-ME-ful $c$-vectors. At the same time the wall stands because by adding arrows with zero maps we can see $M$ as indecomposable non-rigid modules of $kQ'$ and conditions for $M$ to be stable is the same in $kQ$ and $kQ'$. As a result any generic green path inducing an infinite green sequence of $Q'$ induces an infinite green sequence in $kQ$.\\
%\indent Hence the only possible strongly generic green paths inducing MGSs in $kQ$ must be from $Q$-ME-free green sequences in $kQ'$.
\end{proof}
%\indent Note that the set of generic green paths in the semi-invariant picture of $Q'$ corresponding to an $Q$-ME-free MGS of $Q'$ is a superset of the set of $Q$-ME-free generic green paths in the semi-invariant picture of $Q'$ because the latter requires that a generic green path that does not cross any $Q$-ME-ful wall in the semi-invariant picture of $Q'$ do not cross any $Q$-ME-ful wall in the semi-invariant picture of $Q$ as well.
%\begin{lemma}
%Let $Q$ be an ME-ful quiver and let $Q'$ be its ME-free version. Then any MGS of $Q$ must be a $Q$-ME-free MGS of $Q'$.
%\end{lemma}
\indent In order to prove \ref{C4T1} we need to first prove a lemma.
\begin{lemma}\label{C4L}
\indent If $-c_1, -c_2$ are negative $c$-vectors in $C$-matrix $C'$ in an MGS, $c_1$ and $c_2$ are dimension vectors of indecomposable modules $M_1$ and $M_2$. If $dim Ext^1(M_1, M_2) > 1$ then the mutation on $C'$ must not be done on $M_2$.
\end{lemma}
\begin{proof}
Assume that $-c_1$ is the $i$-th column and $-c_2$ is the $j$-th colu,n. According to \cite{KY12} using the definition of left mutations of simple-minded collections if $dim Ext^1(M_1, M_2) > 1$ then the mutation on $-c_2$ would cause $-c_1$ to be transformed into $-c_1-kc_2$ with $k>1$ because which could only happen if there are multiple edges from $i$ to $j$. Due to \cite{BHIT15} this was impossible.\\
\end{proof}
\indent Now we can easily establish the following theorem.
\begin{theorem}
MGSs of an acyclic quiver $Q$ are a subset of the set of $Q$-ME-free MGSs of its ME-free version, $Q'$.\label{C4T1B}
\end{theorem}
\begin{proof}
\indent If the statement is incorrect along an MGS of $Q$ pick the first $C$-matrix that isn't shared by $Q'$ assuming that such an MGS exists. \\
\indent In this case either at least one $c$-vector is $Q$-ME-ful or none is. If some $c$-vector is $Q$-ME-ful it must be formed by extending one $Q$-ME-free indecomposable rigid module by another $Q$-ME-free indecomposable rigid module in $Q$ (i.e. $dim Ext_{kQ}(A,B) = 1$ because it can not be larger due to Lemma \ref{C4L}. Let's label the indecomposable module formed by the extension $M$. We need $Ext_{kQ}(B,A) = 0$ so that $\langle M,M\rangle_{kQ} = 1$) while in $Q'$  there are no such extensions (i.e. $dim Ext_{kQ'}(A,B) = 0$). However this is impossible because $A, B$ are rigid, $Hom$-orthogonal and indecomposable because $\langle M,M\rangle_{kQ} - \langle M,M\rangle_{kQ'} \leq -2$ which causes $\langle M,M\rangle_{kQ'}$ to be at least 3 which is impossible because $\langle M,M\rangle_{kQ'} = \langle A,A\rangle_{kQ'} + \langle A,B\rangle_{kQ'} + \langle B,A\rangle_{kQ'} + \langle B,B\rangle_{kQ'} = 2 - dim Ext_{kQ'}(A,B) - dim Ext_{kQ'}(B,A)$ is at most 2.\\
\indent If no $c$-vector is $Q$-ME-ful then in $kQ$, $kQ'$ the relevant $Hom$ and $Ext$ groups shouldn't differ because neither of them involve the multiple edges that are absent in $kQ'$ . As a result that can't happen either.\\
\indent Hence the $C$-matrices corresponding to $Q, Q'$ in the MGS are all the same. Any MGS of $Q$ must be an MGS of $Q'$ with the same $C$-matrices. Since all the $C$-matrices of the two quivers are the same they have the same associated permutation.\\
\end{proof}
\indent Now we can prove a stronger result.\\
\begin{theorem}\label{C4T3B}
MGSs of an ME-ful acyclic quiver $Q$ are exactly the $Q$-ME-free MGSs of its ME-free version, $Q'$ such that for any multiple edge from $i$ to $j$ in $Q$ if there exists any negative $c$-vector with support containing $i$ no mutation is done on any negative $c$-vector with support containing $j$.
\end{theorem}
\begin{proof}
Let's compare $\langle M, N \rangle_{kQ}$ and $\langle M, N \rangle_{kQ'}$. They differ if and only if there exists some multiple edge from $i$ to $j$ such that $i$ is in the support of $M$ and $j$ is in the support of $N$. In this case since $Hom_{kQ}(M,N) = Hom_{kQ'}(M,N)  = Ext_{kQ'}(M,N) = 0$ $dim Ext_{kQ}(M,N) > 0$. Repeating the argument in \ref{C4T3B} we can show that this is the only possible scenario for a $Q$-ME-free MGSs of $Q'$ to not be identical to an MGS in $Q$.
\end{proof}
\begin{corollary}\label{C4CB}
The following statements are true:
\begin{enumerate}
\item The amount of maximal green sequences of a quiver $Q$ is no greater than that of its ME-free version.
\item All quivers with an MGS-finite ME-free version must themselves be MGS-finite.
\item No minimally MGS-infinite quiver can contain multiple edges.
\item Any two ME-equivalent quivers are MGS-equivalent to each other.
\end{enumerate}
\end{corollary}
\begin{proof}
\indent Only (4) needs to be proven even though it is still obvious. For ME-equivalent quivers $Q$ and $Q'$ the conditions of \ref{C4T3B} are identical which is why the amount of MGS are identical.
\end{proof}
\begin{example}
The maximal green sequences of $Q: \begin{tikzcd}
1\righttwicedoublearrow\arrow[rd] &  & 3\\
 & 2\arrow[ur]
\end{tikzcd}$ are some maximal green sequences of its ME-free version $Q': 1\to 2\to 3$ that has no $c$-vector with support containing $\{1,3\}$ that remain ME-free in $Q$. It's easy to see that $Q$ is MGS-finite. In fact it has 3 MGSs.
\end{example}
\begin{example}
The maximal green sequences of $Q: \begin{tikzcd}
1\arrow[r] & 2 \rightdoublearrow&  3 \arrow[r] & 4\\
\end{tikzcd}$ are some maximal green sequences of its ME-free version $Q': \begin{tikzcd}
1\arrow[r] & 2&  3 \arrow[r] & 4\\
\end{tikzcd}$ that has no $c$-vector with support containing $\{2,3\}$ that remain ME-free in $Q$. It's easy to see that $Q$ is MGS-finite because $A_2$ is.
\end{example}
\indent Now we can provide a much shorter proof to the fact that all acyclic quivers with three vertices are MGS-finite which was originally proven in \cite{BDP13}.
\begin{corollary}
Any acyclic quiver with at most three vertices is MGS-finite.
\end{corollary}
\begin{proof}
Due to the theorem we only need to show that any ME-free acyclic quiver with at most three vertices is MGS-finite. Such a quiver is either of finite or tame type and is hence MGS-finite.
\end{proof}
\section{The general case}
\indent In the general case the theorem above isn't correct. We can show that using the following counterexample. The quiver $Q$ here is $\begin{tikzcd}
1\arrow[rd] &  & 3\lefttwicedoublearrow\\
 & 2\arrow[ur]
\end{tikzcd}$.\\
$\begin{bmatrix} 
0 &1 & -2\\
-1 & 0 & 1\\
2 & -1 & 0\\
1 & 0 & 0\\
0 & 1 & 0\\
0 & 0 & 1\\
\end{bmatrix}\overset{\mu_2}{\to}\begin{bmatrix} 
0 &-1 & -1\\
1 & 0 & -1\\
1 & 1 & 0\\
1 & 0 & 0\\
0 & -1 & 1\\
0 & 0 & 1\\
\end{bmatrix}\overset{\mu_3}{\to}\begin{bmatrix} 
0 &-1 & 1\\
1 & 0 & 1\\
-1 & -1 & 0\\
1 & 0 & 0\\
1 & 0 & -1\\
1 & 1 & -1\\
\end{bmatrix}\overset{\mu_1}{\to}\begin{bmatrix} 
0 &1 & -1\\
-1 & 0 & 2\\
1 & -2 & 0\\
-1 & 0 & 1\\
-1 & 0 & 0\\
-1 & 1 & 0\\
\end{bmatrix}\overset{\mu_3}{\to}\begin{bmatrix} 
0 &-1 & 1\\
1 & 0 & -2\\
-1 & 2 & 0\\
0 & 0 & -1\\
-1 & 0 & 0\\
-1 & 1 & 0\\
\end{bmatrix}\overset{\mu_2}{\to}\begin{bmatrix} 
0 &1 & -1\\
-1 & 0 & 2\\
1 & -2 & 0\\
0 & 0 & -1\\
-1 & 0 & 0\\
0 & -1 & 0\\
\end{bmatrix}$\\
\indent Here we have a maximal green sequence with at least one ME-full $c$-vector. Moreover it is easy to see that if we replace the double edge by triple edge and obtain $Q': \begin{tikzcd}
1\arrow[rd] &  & 3\lefttwicetriplearrow\\
 & 2\arrow[ur]
\end{tikzcd}$\ (2,3,1,3,2) is not a $c$-vector of the quiver $Q'$.\\
\indent However we can still perform quiver cutting in more limited situations. Let's first introduce a concept.
\begin{definition}
A $k$\textit{-edge} is a tuple $(i,j)$ where $i,j\in [n]$ and $k|b_{ij}, k|b_{ji}$.
\end{definition}
\begin{definition}
Let $Q$ be a quiver possibly having oriented cycles, let $k$ be an integer greater than 1. Assume that $Q_0$ = $\tilde{Q}_0 + \breve{Q}_0$, $P = Q]_{\tilde{Q}_0}, R = Q]_{\breve{Q}_0}$. If for all $i\in \tilde{Q}_0, j\in  \breve{Q}_0$ $k|b_{ij}$ and $k|b_{ji}$ we say $Q$ is $k$-\textit{partible} and $(\tilde{Q}, \breve{Q})$ is a $k$\textit{-partition} of $Q$.
\end{definition}
\begin{theorem}
Assume that ($\tilde{Q},\breve{Q})$ are $k$-partition of $Q$ for some $k>1$ any MGS of $Q$ is an MGS of $\tilde{Q}\cup\breve{Q}$.\label{C4T2B}
\end{theorem}
\begin{proof}
The property that for any $i\in Q_1$ and $j\in Q_2$ $k|c_{ij}$ is preserved by mutation. Hence any mutation that cause any $c$-vector to cross bot has to violate the Sink before Source Theorem.
\end{proof}
\begin{corollary}
Under the conditions of the theorem above, if $\tilde{Q}$ and $\breve{Q}$ are MGS-finite so is $Q$.
\end{corollary}
\begin{proof}
If $\tilde{Q}$ and $\breve{Q}$ are MGS-finite so is $\tilde{Q}\cup\breve{Q}$. As a result so is $Q$ due to the theorem.
\end{proof}
\begin{example}
$Q:\begin{tikzcd}
1\arrow[rd] &                                             &                   &5\arrow[dd]\\
                 &2\arrow[dl]\rightdoublearrow & 4\arrow[ur] &\\
3\arrow[uu] &                                            &                  &6\arrow[ul]\\
\end{tikzcd}$ is a quiver with oriented cycles. Due to the theorem we can cut the $\begin{tikzcd}2\rightdoublearrow & 4\end{tikzcd}$ arrow. After cutting this arrow it is easy to see that $Q$ is MGS-finite.
\end{example}
\begin{example}
$Q:\begin{tikzcd}
 &       2\rightdoublearrow                                      &  3\arrow[rd]&\\
1\arrow[ru]                 &    &  &4\arrow[dl]\\
 &     6\arrow[ul]                                       & 5\leftquadruplearrow                 &\\
\end{tikzcd}$ is another quiver with oriented cycles.  Due to the theorem we can cut the $\begin{tikzcd}2\rightdoublearrow & 3\end{tikzcd}$ and  $\begin{tikzcd}6 & 5\leftquadruplearrow\end{tikzcd}$ arrows. After cutting these arrows it is easy to see that $Q$ is MGS-finite.
\end{example}