\chapter{Permutation}\label{C1}
\section{The general theory of permutations}
\subsection{Mutation systems}
\indent Let's define a natural setting of the theory of permutations which is completely combinatorial. Let $[n]=\{1,2,\cdots,n\}$.  A \textit{mutation graph} is defined as a connected $n$-regular graph without loops. Let $T=(T_0,T_1)$ be a mutation graph. A \textit{signed edge} of $T$ is a triple $(a,h,t)$ where $a\in T_1$, $h$ and $t$ are the two endpoints of $a$, defined to be the head and tail of the signed edge respectively. Let $\tilde{T_1}$ be the union of all signed edges of $T$. A \textit{walk} is a path in $\tilde{T_1}$ such that the smyces and targets of each signed edge are compatible. Walks on $T$ are in the form  $w=\Pi a_k^{i_{k-1} i_k}$.\\ 
\indent For each vertex $x\in T_0$ I associate a set $N(x)$ which is the set that contains all vertices adjacent to $x$. Note that $|N(x)|=n$. For each $a_{ht}\in\tilde{T_1}$ I associate a bijection $f_{a_{ht}}: N(x)\to N(y)$ such that $a_{ht}$ and $a_{th}$ are inverses of each other for any $a\in T_1$. This bijection is called \textit{mutation} as per \cite{FZ4}. The set of all $f_a$ is denoted $A$, the \textit{set of mutations}. The tuple $(T,A)$ is called a \textit{mutation system}. I can also define a natural bijection $f_w:N(x)\to N(y)$ associated with each walk $w=\Pi a_k^{i_{k-1} i_k}$, namely $f_w=f_{a_k}^{i_{k-1}i_k}\cdots f_{a_1}^{i_0i_1}$.\\
\indent Now let's define a bijection $j_x:[n]\to N(x)$ for each $x\in T_0$. This bijection is called the \textit{fixed ordering} of the seed $N(x)$. Let $J$ be the set of all fixed orderings which I call a \textit{fixed ordering set}. The tuple $(T,A,J)$ is called a \textit{ordered mutation system}. Now for each bijection $j'_x:[n]\to N(x)$ I can define its \textit{associated permutation relative to $J$} below:\\
\begin{definition}
For any $(T,A,J)$ for any $x\in T_0$ for any bijection $g:[n]\to N(x)$ the \textit{associated permutation relative to $J$} is defined as $\rho(g)=j_x^{-1}g$. \\
\end{definition}
\indent Now I can define what is the associated permutation of a mutation relative to $J$.\\
\begin{definition}
For any $(T,A,J)$ for any $x,y\in T_0$ for any mutation $f_a:N(x)\to N(y)$ the \textit{associated permutation relative to $J$} is defined as $\rho(f_a)=j_y^{-1}f_aj_x$.
\end{definition}
\indent In other words, $\rho(f_a)$ is the permutation such that the following diagram commutes:\\
$\begin{tikzcd}
{[n]} \arrow[r,"\rho(f_a)"] \arrow[d,"j_x"] & {[n]}\arrow[d,"j_y"]\\
N(x) \arrow[r,"f_a"] & N(y)\\
\end{tikzcd}$\\
\indent Now I can define what it means to be the associated permutation relative to $J$ of any walk $p=a_k^{i_k}\cdots a_1^{i_1}$, namely $\rho(p)=j_y^{-1}f_pj_x$. It is easy to see from the diagram below that $\rho(p)=\rho(a_k)^{i_k}\cdots \rho(a_1)^{i_1}$.\\
$\begin{tikzcd}
{[n]} \arrow[r,"\rho(f_{a_1})^{i_1}"] \arrow[d,"j_{x_1}"] & {[n]}\arrow[r,"\rho(f_{a_2})^{i_2}"]\arrow[d,"j_{x_2}"] & \cdots {[n]}\arrow[d,"j_{x_k}"]\\
N(x_1) \arrow[r,"f_{a_1}^{i_1}"] & N(x_2)\arrow[r,"f_{a_2}^{i_2}"]  & \cdots N(x_k)\\
\end{tikzcd}$\\
\indent I can also discuss the relation betIen the permutation of a walk $p$ relative to different fixed orderings.\\
\begin{theorem}
(Change of fixed ordering formula) For any mutation system $(T,A)$ for any fixed ordering set $J_1.J_2$ for any $x,y\in T_0$ for any walk $p:x\to y$, let $\rho_1(p),\rho_2(p)$ be the associated permutation of $p$ relative to $J_1,J_2$ respectively. Let $\tau_x=j_{1x}^{-1}j_{2x}$, $\tau_y=j_{1y}^{-1}j_{2y}$. Then $\rho_2(p)=\tau_y^{-1}\rho_1(p)\tau_x=j_{2y}^{-1}j_{1y}\rho_1(p)j_{1x}^{-1}j_{2x}$.\\
\end{theorem}
\indent The theorem can be verified easily by the diagram below.\\
$\begin{tikzcd}
{[n]} \arrow[r,"\rho_2(p)"]\arrow[d,"\tau_x"]\arrow[dd,bend right = 50,"j_{2x}"] & {[n]}\arrow[d,"\tau_y"]\arrow[dd,bend left = 50,"j_{2y}"] \\
{[n]} \arrow[r,"\rho_1(p)"] \arrow[d,"j_{1x}"] & {[n]}\arrow[d,"j_{1y}"]\\
N(x) \arrow[r,"f_a"] & N(y)\\
\end{tikzcd}$\\
\indent I can see that in essence the permutation of a reddening or loop sequence is just special cases of permutations of walks.\\
\indent It is obvious that any $(T,A,J)$ induces a map $g:\tilde{T_1}\to [n]$ that assigns a \textit{fixed position} to each signed edge. I call the map $g$ the \textit{fixed position map}. It is also obvious that for any walk $w: x\rightarrow y$, I only need to fix $j_x, j_y$ to have a fixed $\rho(w)$. $\rho(w)$ is independent of $j_z$ for any $z\neq x, y$. Hence the definition of $\rho(w)$ can be done with any arbitrary choice of $j_z$ for any $z\neq x,y$. In fact even not defining them is also fine since I can just define $\rho(w)$ as $j_y^{-1}f_wj_x$.\\
\section{The associated permutation in $A_n$}
\indent When the quiver is $A_n$ straight orientation I can make much stronger claims. In fact there is a canonical permutation of any mutation sequence. The formula of the associated permutation of reddening sequences and loop sequences is given as the following.\\
\begin{theorem}
In $A_n$ straight orientation, the permutation associated with a picture group element that transforms the framed quiver into the coframed quiver or the framed quiver itself is $\rho(\prod_{k}x_{i_kj_k}^{\delta_k}) = (\prod_{k}(i_k+1,j_k))^{-1}$. Here $\delta_k\in\{+,-\}$.\\
\end{theorem}
\indent This formula works for any maximal green, reddening, loop sequences that starts from and ends up in the framed quiver. It also extended the definition of an associated permutation to the set of arbitrary finite sequences of mutations in $A_n$ straight orientation. One interesting property of $A_n$ is that the associated permutation of a mutation only depends on the $c$-vector but not which cluster-tilting object on which the mutation is conducted. This is a highly nontrivial fact: The associated permutation in the general case seems way less regular.\\
\subsection{Forbidden Pairs-of-Walls Lemmas}
\indent Since I use picture groups and related structures to prove the theorem, I need to examine what kind of pairs of walls can not exist in any compartment.\\
\begin{lemma}
(First Forbidden Pairs-of-Walls Lemma) For any compartment for any short exact sequence of roots $0\rightarrow s\rightarrow r\rightarrow q\rightarrow 0$ it is impossible to have pairs of such walls: $+s-r$ or $-q+r$.\\
(Second Forbidden Pairs-of-Walls Lemma) For quiver $A_n$ for any compartment for any short exact sequence of roots $0\rightarrow s\rightarrow r\rightarrow q\rightarrow 0$ it is impossible to have pairs of such walls: $-s-r$, $-r-q$, $+s+r$, or $+q+r$.\\
\end{lemma}
\indent The first lemma is obvious since in either case $<x,s>\ >0$ but the root $R$ is stable which is impossible. As for the second lemma, the reason $-s-r$ can not appear is that in $A_n$ when you cross $D(s)$ you are either going to have $+s-r$ or $+s-(r+s)$. The former is impossible due to Forbidden Pairs-of-Walls Lemma. The latter is impossible due to c-vector theorem and the fact that the sum of a root and any of its subroots is no longer a root any more in $A_n$. The reason the other three cases can not happen is almost identical.\\
\subsection{Proof of the formula}
\indent The basic idea in proving the theorem is below:\\
\indent Since picture group elements freely commute with permutations, what I want to prove can be reduced to  $\rho(\prod_{k}(i_k+1,j_k)x_{i_kj_k})=id$. This property can further be reduced to proving that for all $k$, $(i_k+1,j_k)x_{i_kj_k}$ in some sense does not permute the c-vectors. This in turn can be reduced to $(i_k+1,j_k)x_{i_kj_k}$ as an operation on extended exchange matrices maintain certain properties defined below:\\
\begin{definition}
An $n\times n$ matrix $M\in M_n(\mathbb{Z})$ is \textit{standard} if the following holds:\\
1. The diagonal entries are all nonzero.\\
2. All positive entries can only exist on the diagonal or above. and all negative entries can only exist on the diagonal or below.\\
3. All columns are in the form $\pm\beta_{ij}$.\\
\end{definition}
It is easy to see that all columns of the form $-\beta_{ij}$ has to be the $(i+1)$-th column and all columns of the form $\beta_{ij}$ has to be the $j$-th column since all other positions violate either Axiom 1 or 2. It is also trivial that the only results of a permutation of columns of $\pm I_n$ that are standard are $\pm I_n$ themselves.\\
\begin{example}
\indent Here are several examples:\\\\
$\begin{bmatrix}
1 & 1\\
0 & -1\\
\end{bmatrix}$ and
$\begin{bmatrix}
1 & 0\\
-1 & -1\\
\end{bmatrix}$ are standard matrices because all three axioms hold.\\\\
$\begin{bmatrix}
0 & -1\\
-1 & 0\\
\end{bmatrix}$ and
$\begin{bmatrix}
-1 & -1\\
1 & 0\\
\end{bmatrix}$ are not standard matrices since axioms 1 and 2 are violated.\\
\end{example}
%Now let's define regularity of an operation which is important for a technicality in the proof below:\\
%\begin{definition}
%An operation $(i_k+1,j_k)x_{i_kj_k}$ from a standard matrix to another standard matrix is \textit{regular} if it preserves the sign of all columns when $j_k>i_k+1$ and if it only changes the sign of the $j_k$-th column and preserve the sign of all other columns when $j_k=i_k+1$.\\
%\end{definition}
\indent The lemma to be proven that can almost immediately lead to the theorem is stated below:\\
\begin{lemma}
In $A_n$ straight orientation, $(i_k+1,j_k)x_{i_kj_k}$ or $(i_k+1,j_k)x_{i_kj_k}^{-1}$ transforms a standard matrix $\tilde{B}_{k-1}$ into a standard matrix $\tilde{B}_k$.\\
\end{lemma}
\begin{proof}
\indent I will only prove in the green case since the red case is almost identical to the green one. In this proof $i_k$ is simplified as $i$ and $j_k$ is simplified as $j$.\\
\indent Case 1: If $j - i  = 1$. Here I have a simple root and the associated permutation $(i+1,j)$ is identity. Hence the proof reduces to $x_{ij}$ transforms a standard matrix to another standard one. $x_{ij}$ merely flips the $j$-th column from $-e_j$ to $e_j$ and may lengthen some $-\beta_{li}$ to $-\beta_{lj}$ for $l<i$ and shorten some $\beta_{il}$ to $\beta_{jl}$ for $l>j$ without changing which column they are in, but no other operation happens or violations of First Forbidden Pairs-of-Wall Theorem ($-q+r$) will happen, hence the resulting matrix is still standard.\\
\indent Case 2: If $j - i > 1$. Here I have an extra generator and the associated permutation is not identity. Now let's discuss what $(i+1,j) x_{ij}$ actually does on each column:\\
\indent a) $l\leq i$. Due to $c$-vector theorem \cite{IOTW3} all c-vectors have to be $\pm\beta_{ab}$ for some $0\leq a<b\leq n$. So the only change that can ever happen is that $-\beta_{li}$ may be lengthen to $-\beta_{lj}$. Neither of these cause a $C$-matrix to violate standardness.\\
\indent b) $i + 1 < l < j$. Again due to $c$-vector theorem in \cite{IOTW3} the only plausible situation is $\beta_{il}$ was transformed into $-\beta_{lj}$. But this constitutes a $-q+r$ situation which violates the Forbidden Pairs-of-Wall Lemma.\\
\indent c) $l = j$. I notice several facts:\\
\indent (1).The $c$-vector $c_j$ can not be negative. If this is the case I have a compartment with two walls $D(\beta_{ij})$ and $D(\beta_{j-1,m})$. This can not happen since these two roots can not be both stable due to \cite{ST12} when $m>j$ or the $+r+s$ situation appears when $m=j$. Hence I can assume that the $j$-th column is $\beta_{mj}$ for some $m<j$.\\
\indent (2). $m$ can not be less than $i$. Otherwise I have a $+s-r$ situation which violates the First Forbidden Pairs-of-Wall Lemma.\\
\indent (3). Also it is impossible for $\beta_{mj}$ to remain itself after doing $x_{ij}$ since otherwise I have a $-s-r$ situation which violates the Second Forbidden Pairs-of-Wall Lemma.\\
\indent Hence the $j$-th column is positive, $m>i$ and $-\beta_{ij}$ actually add to the $j$-th column. So after $(i+1,j)x_{ij}$ is performed the $i$-th column is $-\beta_{im}$ and the $j$-th column is $\beta_{ij}$.\\
\indent d) $l>j$. $\beta_{il}$ can be shortened to $\beta_{jl}$. Other than that, the only plausible case is $l = j + 1$ and the $k$th c-vector is $-\beta_{lm}$ for some $m>l$. In this case I can nor let $-\beta_{ij}$ add to the $l$-th column since otherwise $-q+r$ will be created after the mutation which violates the Forbidden Pairs-of-Wall Lemma.\\
\indent Using similar methods I can see that $(i+1, j)x_{ij}$ transforms a standard matrix into another one.\\
\end{proof}
\indent The theorem can be proven below:\\
\begin{proof}
The identity matrix $I_n$ is standard. Since $(i_k+1,j_k)x_{i_kj_k}$ or $(i_k+1,j_k)x_{i_kj_k}^{-1}$ transforms a standard matrix into another standard for all $k$, the result of transforming a standard matrix, $I_n$ by $\prod_{k}(i_k+1,j_k)x_{i_kj_k}^{\delta_k}$ is standard. Since I get a permutation of columns of $\pm I_n$ at the end of this transformation and any permutation of columns of $\pm I_n$ that result in a standard matrix has to be the trivial permutation, $\rho(\prod_{k}(i_k+1,j_k)x_{i_kj_k}^{\delta_k})=id$, hence the formula is correct.\\
\end{proof}
\subsection{The formula of associated permutation for any mutation sequences}
\indent Due to the theorem I can extend the definition of associated permutations to any arbitrary mutation sequence in $A_n$ straight orientation which reduces to the existing definitions of the associated permutation of reddening and loop sequences due to the theorem above.
\begin{definition}
In $A_n$ straight orientation, the \textit{associated permutation of a mutation sequence} in correspondence to the picture group element $\prod_{k}x_{i_kj_k}^{\delta_k}$ acting on a $c$-matrix with associated permutation $\sigma$ is defined as $\rho(\prod_{k}x_{i_kj_k}^{\delta_k}) = \sigma(\prod_{k}(i_k+1,j_k))^{-1}\sigma^{-1}$. Here $\delta_k\in\{+,-\}$.\\
\end{definition}
\indent In particular any mutation at a vertex with $c$-vector $\pm\beta_{ij}$ has an associated permutation $\sigma(i+1,j)\sigma^{-1}$ with $\sigma$ the permutation of the $c$-matrix before the mutation. In the special case when $i+1=j$ which is when $\beta_{ij}$ is a simple root the associated permutation is trivial.\\ 